% \iffalse meta-comment
% 
% Copyright (C) 2023 by Marcos Bujosa <mbujosab@ucm.es>
% -----------------------------------------------------
% 
% This file may be distributed and/or modified under the
% conditions of the LaTeX Project Public License, either
% version 1.3 of this license or (at your option) any later
% version. The latest version of this license is in:
% 
% http://www.latex-project.org/lppl.txt
% 
% and version 1.3 or later is part of all distributions of
% LaTeX version 2005/12/01 or later.
% 
% \fi
%
% \iffalse
%<package>\NeedsTeXFormat{LaTeX2e}[2005/12/01]
%<package>\ProvidesPackage{nacal}
%<package>    [2023/01/05 v1.0
%<package>    [2023/09/22 v1.3 Notacion Asociativa para un Curso de Álgebra Lineal]
%<package>\RequirePackage{amsmath,amssymb}
%<package>\RequirePackage{xcolor}
%<package>\RequirePackage{xparse}
%<package>\RequirePackage{xspace}
%<package>\RequirePackage{leftidx}
%<package>\RequirePackage{pict2e}
%<package>\RequirePackage{esvect} %\RequirePackage[b]{esvect} %\RequirePackage[e]{esvect}
%<package>%\RequirePackage{stackengine} % para PSpanNew
%<package>\providecommand{\eng}[2]{#1}
%<package>\DeclareOption{esp}{\renewcommand{\eng}[2]{#1}}
%<package>\DeclareOption{eng}{\renewcommand{\eng}[2]{#2}}
%<package>\DeclareOption*{\PackageWarning{nacal}{Unknown ‘\CurrentOption’}}
%<package>\ProcessOptions\relax%
%<*driver>
\documentclass{ltxdoc}
\usepackage[dvipsnames,svgnames]{xcolor}
\usepackage{nacal}
\usepackage[utf8]{inputenc}
\usepackage[T1]{fontenc}
\usepackage[spanish]{babel}
\usepackage[colorlinks=true,linkcolor=DarkGreen,urlcolor=Maroon]{hyperref}
%\usepackage{fullpage}
\usepackage[twoside, a4paper, hmargin=2.8cm, vmargin=2.2cm, includeheadfoot]{geometry}
\EnableCrossrefs
\CodelineIndex
\RecordChanges
% \OnlyDescription
\setcounter{tocdepth}{4}
\begin{document}
  \DocInput{nacal.dtx}
\end{document}
%</driver>
% \fi
%
% 
% \CharacterTable
%  {Upper-case    \A\B\C\D\E\F\G\H\I\J\K\L\M\N\O\P\Q\R\S\T\U\V\W\X\Y\Z
%   Lower-case    \a\b\c\d\e\f\g\h\i\j\k\l\m\n\o\p\q\r\s\t\u\v\w\x\y\z
%   Digits        \0\1\2\3\4\5\6\7\8\9
%   Exclamation   \!     Double quote  \"     Hash (number) \#
%   Dollar        \$     Percent       \%     Ampersand     \&
%   Acute accent  \'     Left paren    \(     Right paren   \)
%   Asterisk      \*     Plus          \+     Comma         \,
%   Minus         \-     Point         \.     Solidus       \/
%   Colon         \:     Semicolon     \;     Less than     \<
%   Equals        \=     Greater than  \>     Question mark \?
%   Commercial at \@     Left bracket  \[     Backslash     \\
%   Right bracket \]     Circumflex    \^     Underscore    \_
%   Grave accent  \`     Left brace    \{     Vertical bar  \|
%   Right brace   \}     Tilde         \~}
% 
%
% \changes{v1.0}{2022/12/04}{Versión inicial}
% \changes{v1.1}{2023/01/14}{Añadidos más comandos y reimplementación de los subíndices de matrices}
% \changes{v1.2}{2023/01/14}{Inclusión opcional índices en matriz por vector y vector por matriz}
% \changes{v1.3}{2023/09/22}{Inclusión de notación para Econometría}
%
% \GetFileInfo{nacal.sty}
%
% \DoNotIndex{\left,\right }
% \DoNotIndex{\\!,\\\}
% \DoNotIndex{\scriptscriptstyle,\times}
% \DoNotIndex{\ensuremath}
% \DoNotIndex{\IfBooleanTF}
% \DoNotIndex{\IfNoValueTF}
% \DoNotIndex{\NewDocumentCommand}
% \DoNotIndex{\xspace}
%
% 
% \title{El paquete \textsf{nacal}\thanks{Este documento
%   corresponde a \textsf{nacal}~\fileversion, fecha \filedate.}}
% \author{Marcos Bujosa \\ \texttt{mbujosab@ucm.es}}
% 
% \maketitle
% 
% \begin{abstract}
% Paquete que define los macros empleados al escribir el libro
%   \href{https://github.com/mbujosab/CursoDeAlgebraLineal}{Un Curso de
%     Álgebra Lineal}
%   (\url{https://github.com/mbujosab/CursoDeAlgebraLineal}) con Notación Asociativa (NAcAL).
% \end{abstract}
% 
% \tableofcontents
% 
% \section*{Introducción}
% 
% Para el Curso de Álgebra Lineal con Notación Asociativa he creado
% multitud de macros que definen la notación empleada en el material
% docente (libro, transparencias, ejercicios, ¿vídeos?).
% 
% \section{Uso}
% 
% \subsection{Conjuntos de números}
% 
% \DescribeMacro{\Nn}
% \DescribeMacro{\Zz}
% \DescribeMacro{\Rr}
% \DescribeMacro{\Kk}
% \DescribeMacro{\Cc}
% Respecto a estos comandos, véase el párrafo explicativo de la
% Sección~\ref{sec:TransfElem}
% 
% Los comandos \cs{Nn}, \cs{Zz}, \cs{Rr}, \cs{Kk} y \cs{Cc} no tienen argumentos
% y denotan el conjunto de números naturales, de números enteros, de números reales y
% números complejos respectivamente
% \begin{center}
%   |\Nn \Zz \Rr \Kk \Cc|
%   \hspace{1.2cm}
%   \fbox{$\Nn\ \Zz\ \Rr\ \Kk\ \Cc$}
% \end{center}
% \bigskip
% 
% \DescribeMacro{\N}
% \DescribeMacro{\Z}
% \DescribeMacro{\R}
% \DescribeMacro{\K}
% \DescribeMacro{\CC}
% Los comandos \cs{N}, \cs{Z}, \cs{R}, \cs{K}, \cs{CC},  tienen 1 argumento
% opcional correspondiente a un superíndice
% \begin{center}
%   |\N \N[5]|
%   \hspace{1.2cm}
%   \fbox{$\N$} \fbox{$\N[5]$}
% \end{center}
% \begin{center}
%   |\Z \Z[3]|
%   \hspace{1.2cm}
%   \fbox{$\Z$} \fbox{$\Z[3]$}
% \end{center}
% \begin{center}
%   |\R \R[(\R[n])]|
%   \hspace{1.2cm}
%   \fbox{$\R$}
%   \fbox{$\R[(\R[n])]$}
% \end{center}
% \begin{center}
%   |\K \K[(\R[n])]|
%   \hspace{1.2cm}
%   \fbox{$\K$}
%   \fbox{$\K[(\R[n])]$}
% \end{center}
% \begin{center}
%   |\Cc \CC[n]|
%   \hspace{1.2cm}
%   \fbox{$\CC$}
%   \fbox{$\CC[n]$}
% \end{center}
% 
% \subsection{Paréntesis y corchetes}
% Me resulta agradable normalizar el tamaño de los paréntesis y otros
% tipos de llaves. En general prefiero que en las expresiones
% matemáticas de tipo \emph{ecuación} o ``\emph{displaymath}'' los
% paréntesis sean ligeramente mayores que aquello que encierran. Pero
% prefiero paréntesis pequeños en las expresiones entre líneas dentro de
% los párrafos.  \medskip
% 
% \DescribeMacro{\parentesis}
% \DescribeMacro{\parentesis*}
% El comando \cs{parentesis} tiene 1
% argumento,\;\cs{parentesis}*\marg{contenido}. Escribe el
% \marg{contenido} entre los paréntesis |\big(| y |\big)| si se usa la
% versión con estrella (paréntesis medianos). Si no se incluye la
% estrella escribe el \marg{contenido} entre |(| y |)| (paréntesis
% pequeños)
% \begin{center}
%   |\parentesis{A} \parentesis*{A}| \hspace{1.2cm}
%   \fbox{$\parentesis{A}$} \fbox{$\parentesis*{A}$}
% \end{center}
% 
% \DescribeMacro{\Parentesis}
% \DescribeMacro{\Parentesis*}
% El comando \cs{Parentesis} tiene 1
% argumento,\;\cs{Parentesis}*\marg{contenido}. Escribe el
% \marg{contenido} entre los paréntesis |\left(| y |\right)| si se usa
% la versión con estrella (paréntesis ajustados al tamaño del
% contenido). Si no se incluye la estrella escribe el \marg{contenido}
% entre |\Big(| y |\Big)| (paréntesis grandes)
% \begin{center}
%   |\Parentesis{A} \Parentesis*{\int\limits_a^b h(x) dx}| \hspace{1.2cm}
%   \fbox{$\Parentesis{A}$} \fbox{$\Parentesis*{\int\limits_a^b h(x) dx}$}
% \end{center}
% 
% \DescribeMacro{\corchetes}
% \DescribeMacro{\corchetes*}
% El comando \cs{corchetes} tiene 1 argumento, y es similar a
% \cs{parentesis}, pero escribe el \marg{contenido} entre corchetes
% \begin{center}
%   |\corchetes{A} \corchetes*{A}| \hspace{1.2cm} \fbox{$\corchetes{A}$}
%   \fbox{$\corchetes*{A}$}
% \end{center}
% 
% \DescribeMacro{\Corchetes}
% \DescribeMacro{\Corchetes*}
% El comando \cs{Corchetes} tiene 1 argumento, y es similar a
% \cs{Parentesis}, pero escribe el \marg{contenido} entre corchetes
% \begin{center}
%   |\Corchetes{A} \Corchetes*{A}|
%   \hspace{1.2cm} \fbox{$\Corchetes{A}$}
%   \fbox{$\Corchetes*{A}$}
% \end{center}
% 
% \DescribeMacro{\angulos}
% \DescribeMacro{\angulos*}
% El comando \cs{angulos} tiene 1 argumento, y es similar a
% \cs{parentesis}, pero escribe el \marg{contenido} entre angulos
% \begin{center}
%   |\angulos{A} \angulos*{A}| \hspace{1.2cm} \fbox{$\angulos{A}$}
%   \fbox{$\angulos*{A}$}
% \end{center}
% 
% \DescribeMacro{\Angulos}
% \DescribeMacro{\Angulos*}
% El comando \cs{Angulos} tiene 1 argumento, y es similar a
% \cs{parentesis}, pero escribe el \marg{contenido} entre angulos
% \begin{center}
%   |\Angulos{A} \Angulos*{A}| \hspace{1.2cm} \fbox{$\angulos{A}$}
%   \fbox{$\Angulos*{A}$}
% \end{center}
% \bigskip
% 
% \subsubsection{Regla mnemotécnica para comandos que escriben expresiones con paréntesis}
% 
% \noindent
% \emph{Seguiré la siguiente regla con la nomenclatura de algunos
%   comandos y sus versiones con estrella} (\texttt{*}).
% \begin{itemize}
% \item Si terminan en ``|p|'' minúscula
%   \begin{itemize}
%   \item sin estrella se pondrá un paréntesis \emph{mediano} alrededor del
%     objeto sobre el que se esta realizando una operación
%   \item con estrella el paréntesis será \emph{pequeño}
%   \end{itemize}
% \item Si terminan en ``|P|'' mayúscula
%   \begin{itemize}
%   \item sin estrella se pondrá un paréntesis \emph{medianamente grande}
%     alrededor del objeto sobre el que se esta realizando una operación
%   \item con estrella el paréntesis tendrá un \emph{tamaño ajustado al
%       objeto}
%   \end{itemize}
% \item Si terminan en ``|pE|'' se pondrá un paréntesis alrededor de
%   toda la operación.
%   \begin{itemize}
%   \item sin estrella el paréntesis será \emph{mediano}
%   \item con estrella el paréntesis será \emph{pequeño}
%   \end{itemize}
% \item Si terminan en ``|PE|'' se pondrá un paréntesis alrededor de
%   toda la operación
%   \begin{itemize}
%   \item sin estrella se pondrá un paréntesis \emph{medianamente grande}
%   \item con estrella el tamaño del paréntesis quedará \emph{ajustado
%       al tamaño del objeto}
%   \end{itemize}
% \end{itemize}
% \subsection{Subíndices}
% \subsubsection{Subíndices y exponente}
% 
% \DescribeMacro{\LRidxE}
% \DescribeMacro{\LRidxEp}
% \DescribeMacro{\LRidxEp*}
% \DescribeMacro{\LRidxEP}
% \DescribeMacro{\LRidxEP*}
% \DescribeMacro{\LRidxEpE}
% \DescribeMacro{\LRidxEpE*}
% \DescribeMacro{\LRidxEPE}
% \DescribeMacro{\LRidxEPE*}
% El comando \cs{LRidxE} tiene 4
% argumentos,\;\cs{LRidxE}\marg{objeto}\marg{indIzda}\marg{indDcha}\marg{exponente},\; y
% pone un subíndice a cada lado del objeto (con exponente)
% \begin{center}
%   |\LRidxE{\Mat{A}}{1}{7}{'}|
%   \hspace{1.2cm}
%   \fbox{$\LRidxE{\Mat{A}}{1}{7}{'}$}
% \end{center}
% \begin{center}
%   |\LRidxEp{\Mat{A}}{1}{7}{'} \LRidxEp*{\Mat{A}}{1}{7}{'}|
%   \hspace{1.2cm}
%   \fbox{$\LRidxEp {\Mat{A}}{1}{7}{'}$}
%   \fbox{$\LRidxEp*{\Mat{A}}{1}{7}{'}$}
% \end{center}
% \begin{center}
%   |\LRidxEP{\Mat{A}}{1}{7}{'} \LRidxEP*{\Mat{A}}{1}{7}{'}|
%   \hspace{1.2cm}
%   \fbox{$\LRidxEP {\Mat{A}}{1}{7}{'}$}
%   \fbox{$\LRidxEP*{\Mat{A}}{1}{7}{'}$}
% \end{center}
% \begin{center}
%   |\LRidxEpE{\Mat{A}}{1}{7}{'} \LRidxEpE*{\Mat{A}}{1}{7}{'}|
%   \hspace{1.2cm}
%   \fbox{$\LRidxEpE {\Mat{A}}{1}{7}{'}$}
%   \fbox{$\LRidxEpE*{\Mat{A}}{1}{7}{'}$}
% \end{center}
% \begin{center}
%   |\LRidxEPE{\Mat{A}}{1}{7}{'} \LRidxEPE*{\Mat{A}}{1}{7}{'}|
%   \hspace{1.2cm}
%   \fbox{$\LRidxEPE {\Mat{A}}{1}{7}{'}$}
%   \fbox{$\LRidxEPE*{\Mat{A}}{1}{7}{'}$}
% \end{center}
% 
% \DescribeMacro{\LidxE}
% \DescribeMacro{\LidxEp}
% \DescribeMacro{\LidxEp*}
% \DescribeMacro{\LidxEP}
% \DescribeMacro{\LidxEP*}
% \DescribeMacro{\LidxEpE}
% \DescribeMacro{\LidxEpE*}
% \DescribeMacro{\LidxEPE}
% \DescribeMacro{\LidxEPE*}
% El comando \cs{LidxE} tiene 3
% argumentos,\;\cs{LidxE}\marg{objeto}\marg{indIzda}\marg{exponente},\;
% y pone un subíndice a la izquierda del objeto (con exponente)
% \begin{center}
%   |\LidxE{\Mat{A}}{1}{'}|
%   \hspace{1.2cm}
%   \fbox{$\LidxE{\Mat{A}}{1}{'}$}
% \end{center}
% \begin{center}
%   |\LidxEp{\Mat{A}}{1}{'} \LidxEp*{\Mat{A}}{1}{'}|
%   \hspace{1.2cm}
%   \fbox{$\LidxEp {\Mat{A}}{1}{'}$}
%   \fbox{$\LidxEp*{\Mat{A}}{1}{'}$}
% \end{center}
% \begin{center}
%   |\LidxEP{\Mat{A}}{1}{'} \LidxEP*{\Mat{A}}{1}{'}|
%   \hspace{1.2cm}
%   \fbox{$\LidxEP {\Mat{A}}{1}{'}$}
%   \fbox{$\LidxEP*{\Mat{A}}{1}{'}$}
% \end{center}
% \begin{center}
%   |\LidxEpE{\Mat{A}}{1}{'} \LidxEpE*{\Mat{A}}{1}{'}|
%   \hspace{1.2cm}
%   \fbox{$\LidxEpE {\Mat{A}}{1}{'}$}
%   \fbox{$\LidxEpE*{\Mat{A}}{1}{'}$}
% \end{center}
% \begin{center}
%   |\LidxEPE{\Mat{A}}{1}{'} \LidxEPE*{\Mat{A}}{1}{'}|
%   \hspace{1.2cm}
%   \fbox{$\LidxEPE {\Mat{A}}{1}{'}$}
%   \fbox{$\LidxEPE*{\Mat{A}}{1}{'}$}
% \end{center}
% 
% 
% \DescribeMacro{\RidxE} El comando \cs{RidxE} tiene 3
% argumentos,\;\cs{RidxE}\marg{objeto}\marg{indDcha}\marg{exponente},\; y
% pone un a la derecha del objeto (con exponente)
% \begin{center}
%   |\RidxE{\Mat{A}}{7}{'}|
%   \hspace{1.2cm}
%   \fbox{$\RidxE{\Mat{A}}{7}{'}$}
% \end{center}
% \begin{center}
%   |\RidxEp{\Mat{A}}{7}{'} \RidxEp*{\Mat{A}}{7}{'}|
%   \hspace{1.2cm}
%   \fbox{$\RidxEp {\Mat{A}}{7}{'}$}
%   \fbox{$\RidxEp*{\Mat{A}}{7}{'}$}
% \end{center}
% \begin{center}
%   |\RidxEP{\Mat{A}}{7}{'} \RidxEP*{\Mat{A}}{7}{'}|
%   \hspace{1.2cm}
%   \fbox{$\RidxEP {\Mat{A}}{7}{'}$}
%   \fbox{$\RidxEP*{\Mat{A}}{7}{'}$}
% \end{center}
% \begin{center}
%   |\RidxEpE{\Mat{A}}{7}{'} \RidxEpE*{\Mat{A}}{7}{'}|
%   \hspace{1.2cm}
%   \fbox{$\RidxEpE {\Mat{A}}{7}{'}$}
%   \fbox{$\RidxEpE*{\Mat{A}}{7}{'}$}
% \end{center}
% \begin{center}
%   |\RidxEPE{\Mat{A}}{7}{'} \RidxEPE*{\Mat{A}}{7}{'}|
%   \hspace{1.2cm}
%   \fbox{$\RidxEPE {\Mat{A}}{7}{'}$}
%   \fbox{$\RidxEPE*{\Mat{A}}{7}{'}$}
% \end{center}
% 
% \subsubsection{Solo subíndices}
% %%%%%%%%%%%%%%%%%%%%%%%%%%%%%%%%%%%%%%%%%%%%%%%%%%%%%%%%%%%%%%%%%%%%%%%%%%
% \DescribeMacro{\LRidx}
% \DescribeMacro{\LRidxp}
% \DescribeMacro{\LRidxp*}
% \DescribeMacro{\LRidxP}
% \DescribeMacro{\LRidxP*}
% \DescribeMacro{\LRidxpE}
% \DescribeMacro{\LRidxpE*}
% \DescribeMacro{\LRidxPE}
% \DescribeMacro{\LRidxPE*}
% Las versiones con y sin estrella tienen 3
% argumentos,\;|\LRidx<XX*>|\marg{objeto}\marg{indIzda}\marg{indDcha},\;
% y ponen un subíndice a cada lado del objeto
% \begin{center}
%   |\LRidx{\Mat{A}}{1}{7}|
%   \hspace{1.2cm}
%   \fbox{$\LRidx{\Mat{A}}{1}{7}$}
% \end{center}
% \begin{center}
%   |\LRidxp{\Mat{A}}{1}{7} \LRidxp*{\Mat{A}}{1}{7}|
%   \hspace{1.2cm}
%   \fbox{$\LRidxp {\Mat{A}}{1}{7}$}
%   \fbox{$\LRidxp*{\Mat{A}}{1}{7}$}
% \end{center}
% \begin{center}
%   |\LRidxP{\Mat{A}}{1}{7} \LRidxP*{\Mat{A}}{1}{7}|
%   \hspace{1.2cm}
%   \fbox{$\LRidxP {\Mat{A}}{1}{7}$}
%   \fbox{$\LRidxP*{\Mat{A}}{1}{7}$}
% \end{center}
% \begin{center}
%   |\LRidxpE{\Mat{A}}{1}{7} \LRidxpE*{\Mat{A}}{1}{7}|
%   \hspace{1.2cm}
%   \fbox{$\LRidxpE {\Mat{A}}{1}{7}$}
%   \fbox{$\LRidxpE*{\Mat{A}}{1}{7}$}
% \end{center}
% \begin{center}
%   |\LRidxPE{\Mat{A}}{1}{7} \LRidxPE*{\Mat{A}}{1}{7}|
%   \hspace{1.2cm}
%   \fbox{$\LRidxPE {\Mat{A}}{1}{7}$}
%   \fbox{$\LRidxPE*{\Mat{A}}{1}{7}$}
% \end{center}
% 
% %%%%%%%%%%%%%%%%%%%%%%%%%%%%%%%%%%%%%%%%%%%%%%%%%%%%%%%%%%%%%%%%%%%%%%%%%%
% \DescribeMacro{\Lidx}
% \DescribeMacro{\Lidxp}
% \DescribeMacro{\Lidxp*}
% \DescribeMacro{\LidxP}
% \DescribeMacro{\LidxP*}
% \DescribeMacro{\LidxpE}
% \DescribeMacro{\LidxpE*}
% \DescribeMacro{\LidxPE}
% \DescribeMacro{\LidxPE*}
% Las versiones con y sin estrella tienen 2
% argumentos,\;|\Lidx<XX*>|\marg{objeto}\marg{indIzda},\; y ponen un
% subíndice a la izquierda del objeto
% \begin{center}
%   |\Lidx{\Mat{A}}{1}|
%   \hspace{1.2cm}
%   \fbox{$\Lidx{\Mat{A}}{1}$}
% \end{center}
% \begin{center}
%   |\Lidxp{\widehat{\Mat{A}}}{1} \Lidxp*{\widehat{\Mat{A}}}{1}|
%   \hspace{1.2cm}
%   \fbox{$\Lidxp {\widehat{\Mat{A}}}{1}$}
%   \fbox{$\Lidxp*{\widehat{\Mat{A}}}{1}$}
% \end{center}
% \begin{center}
%   |\LidxP{\widehat{\Mat{A}}}{1} \LidxP*{\widehat{\Mat{A}}}{1}|
%   \hspace{1.2cm}
%   \fbox{$\LidxP {\widehat{\Mat{A}}}{1}$}
%   \fbox{$\LidxP*{\widehat{\Mat{A}}}{1}$}
% \end{center}
% \begin{center}
%   |\LidxpE{\widehat{\Mat{A}}}{1} \LidxpE*{\widehat{\Mat{A}}}{1}|
%   \hspace{1.2cm}
%   \fbox{$\LidxpE {\widehat{\Mat{A}}}{1}$}
%   \fbox{$\LidxpE*{\widehat{\Mat{A}}}{1}$}
% \end{center}
% \begin{center}
%   |\LidxPE{\widehat{\Mat{A}}}{1} \LidxPE*{\widehat{\Mat{A}}}{1}|
%   \hspace{1.2cm}
%   \fbox{$\LidxPE {\widehat{\Mat{A}}}{1}$}
%   \fbox{$\LidxPE*{\widehat{\Mat{A}}}{1}$}
% \end{center}
% 
% %%%%%%%%%%%%%%%%%%%%%%%%%%%%%%%%%%%%%%%%%%%%%%%%%%%%%%%%%%%%%%%%%%%%%%%%%%
% \DescribeMacro{\Ridx}
% \DescribeMacro{\Ridxp}
% \DescribeMacro{\Ridxp*}
% \DescribeMacro{\RidxP}
% \DescribeMacro{\RidxP*}
% \DescribeMacro{\RidxpE}
% \DescribeMacro{\RidxpE*}
% \DescribeMacro{\RidxPE}
% \DescribeMacro{\RidxPE*}
% Las versiones con y sin estrella tienen 2
% argumentos,\;|\Ridx<XX*>|\marg{objeto}\marg{indDcha},\; y ponen un
% subíndice a la derecha del objeto
% \begin{center}
%   |\Ridx{\Mat{A}}{7}|
%   \hspace{1.2cm}
%   \fbox{$\Ridx{\Mat{A}}{7}$}
% \end{center}
% \begin{center}
%   |\Ridxp{\widehat{\Mat{A}}}{1} \Ridxp*{\widehat{\Mat{A}}}{1}|
%   \hspace{1.2cm}
%   \fbox{$\Ridxp {\widehat{\Mat{A}}}{1}$}
%   \fbox{$\Ridxp*{\widehat{\Mat{A}}}{1}$}
% \end{center}
% \begin{center}
%   |\RidxP{\widehat{\Mat{A}}}{1} \RidxP*{\widehat{\Mat{A}}}{1}|
%   \hspace{1.2cm}
%   \fbox{$\RidxP {\widehat{\Mat{A}}}{1}$}
%   \fbox{$\RidxP*{\widehat{\Mat{A}}}{1}$}
% \end{center}
% \begin{center}
%   |\RidxpE{\widehat{\Mat{A}}}{1} \RidxpE*{\widehat{\Mat{A}}}{1}|
%   \hspace{1.2cm}
%   \fbox{$\RidxpE {\widehat{\Mat{A}}}{1}$}
%   \fbox{$\RidxpE*{\widehat{\Mat{A}}}{1}$}
% \end{center}
% \begin{center}
%   |\RidxPE{\widehat{A}}{1} \RidxPE*{\widehat{A}}{1}|
%   \hspace{1.2cm}
%   \fbox{$\RidxPE {\widehat{A}}{1}$}
%   \fbox{$\RidxPE*{\widehat{A}}{1}$}
% \end{center}
% 
% \subsection{Operadores}
% \subsubsection{Conjugación y concatenación}
% 
% Definimos un operador con una barra ancha.\\
% \DescribeMacro{\widebar}
% El comando \cs{widebar} tiene 1 argumento,\;\cs{widebar}\marg{objeto},\;
% y pone una barra ancha sobre el \marg{objeto}.
% \begin{center}
%   |\widebar{x}|
%   \hspace{1.2cm}
%   \fbox{$\widebar{x}$}
% \end{center}
% 
% Con dicha barra ancha denotaremos el operador conjugación:\\
% \DescribeMacro{\conj}
% El comando \cs{conj} tiene 1 argumento,\;\cs{conj}\marg{objeto},\;
% y pone una barra ancha sobre el \marg{objeto}.
% \begin{center}
%   |\conj{5+2i}|
%   \hspace{1.2cm}
%   \fbox{$\conj{5+2i}$}
% \end{center}
% 
% Con el comando \cs{concat} denotaremos la concatenación de dos sistemas\\
% \DescribeMacro{\concat}
% El comando \cs{concat} no tiene argumentos,\;\cs{concat}.
% \begin{center}
%   |\concat|
%   \hspace{1.2cm}
%   \fbox{$\concat$}
% \end{center}
% 
% Con el comando \cs{bigtimes} denotaremos el rpoductorio cartesiano
% \DescribeMacro{\bigtimes}
% El comando \cs{bigtimes} no tiene argumentos,\;\cs{bigtimes}.
% \begin{center}
%   |\bigtimes \bigtimes_{i=1}^n a_i \bigtimes\limits_{i=1}^n a_i|
%   
%   \fbox{$\bigtimes$}
%   \fbox{$\bigtimes_{i=J} a_i$}
%   \fbox{$\bigtimes\limits_{i=J} a_i$}
% \end{center}
% 
% \subsubsection{Norma y valor absoluto}
% \DescribeMacro{\norma}
% \DescribeMacro{\norma*}
% El comando \cs{norma} tiene 2 argumentos,\;\cs{norma}\oarg{tipo}\marg{objeto},\; y
% denota la norma del \marg{objeto}. En la versión con estrella las
% dobles barras verticales se ajustan al tamaño del \marg{objeto}.
% \begin{center}
%   |\norma{f} \norma*{\int\limits_a^b h(x) dx}|
%   \hspace{1.2cm}
%   \fbox{$\norma {f}$}
%   \fbox{$\norma*{\int\limits_a^b h(x) dx}$}
% \end{center}
% \begin{center}
%   |\norma[L_2]{f}^2 \norma*[L_1]{\int\limits_a^b h(x) dx}|
%   \hspace{1.2cm}
%   \fbox{$\norma[L_2] {f}^2$}
%   \fbox{$\norma*[L_1]{\int\limits_a^b h(x) dx}$}
% \end{center}
% 
% \DescribeMacro{\modulus}
% \DescribeMacro{\modulus*}
% El comando \cs{modulus} tiene 1
% argumento,\;\cs{modulus}\marg{objeto},\; y denota el valor absoluto
% del \marg{objeto}. En la versión con estrella las barras verticales se
% ajustan al tamaño del \marg{objeto}.
% \begin{center}
%   |\modulus{A} \modulus{ \int\limits_a^b h(x) dx }|
%   \hspace{1.2cm}
%   \fbox{$\modulus {f}$}
%   \fbox{$\modulus*{\int\limits_a^b h(x) dx}$}
% \end{center}
% 
% \DescribeMacro{\abs}
% \DescribeMacro{\abs*}
% El comando \cs{abs} tiene 1
% argumento,\;\cs{abs}\marg{objeto},\; y denota el valor absoluto
% del \marg{objeto}. En la versión con estrella las barras verticales se
% ajustan al tamaño del \marg{objeto}.
% \begin{center}
%   |\abs{A} \abs{ \int\limits_a^b h(x) dx }|
%   \hspace{1.2cm}
%   \fbox{$\abs {f}$}
%   \fbox{$\abs*{\int\limits_a^b h(x) dx}$}
% \end{center}
% 
% \subsubsection{Transposición}
% 
% \DescribeMacro{\T}
% El comando \cs{T} no tiene argumentos y denota el símbolo de la transposición.
% \begin{center}
%   |\T|
%   \hspace{1.2cm}
%   \fbox{$\T$}
% \end{center}
% 
% \DescribeMacro{\Trans}
% \DescribeMacro{\Transp}
% \DescribeMacro{\Transp*}
% \DescribeMacro{\TransP}
% \DescribeMacro{\TransP*}
% \DescribeMacro{\TranspE}
% \DescribeMacro{\TranspE*}
% \DescribeMacro{\TransPE}
% \DescribeMacro{\TransPE*}
% El comando |Trans<XX*>| tiene 1 argumento,\;|Trans<XX*>|\marg{objeto},\; y
% denota la transposición del \marg{objeto}
% \begin{center}
%   |\Trans{\Mat{A}} \Trans{\Mat{M}}|
%   \hspace{1.2cm}
%   \fbox{$\Trans{\Mat{A}}$}
%   \fbox{$\Trans{\Mat{M}}$}
% \end{center}
% \begin{center}
%   |\Transp{\widehat{\Mat{A}}} \Transp*{\widehat{\Mat{A}}}|
%   \hspace{1.2cm}
%   \fbox{$\Transp {\widehat{\Mat{A}}}$}
%   \fbox{$\Transp*{\widehat{\Mat{A}}}$}
% \end{center}
% \begin{center}
%   |\TransP{\Mat{A}} \TransP*{\Mat{A}}|
%   \hspace{1.2cm}
%   \fbox{$\TransP {\Mat{A}}$}
%   \fbox{$\TransP*{\Mat{A}}$}
% \end{center}
% \begin{center}
%   |\TranspE{\Mat{A} \TranspE*{\Mat{A}}|
%   \hspace{1.2cm}
%   \fbox{$\TranspE {\Mat{A}}$}
%   \fbox{$\TranspE*{\Mat{A}}$}
% \end{center}
% \begin{center}
%   |\TransPE{\Mat{A}} \TransPE*{\Mat{A}}|
%   \hspace{1.2cm}
%   \fbox{$\TransPE {\Mat{A}}$}
%   \fbox{$\TransPE*{\Mat{A}}$}
% \end{center}
% 
% \subsubsection{Inversa}
% 
% Me gusta que el signo negativo que indica la inversa sea ligeramente
% más corto que el habitual. Así logramos que las expresiones sean un
% poco más compactas.
% 
% \DescribeMacro{\minus}
% El comando \cs{minus} no tiene argumentos
% \begin{center}
%   |\minus|
%   \hspace{1.2cm}
%   \fbox{$\minus$}
% \end{center}
% 
% \DescribeMacro{\Inv}
% \DescribeMacro{\Invp}
% \DescribeMacro{\Invp*}
% \DescribeMacro{\InvP}
% \DescribeMacro{\InvP*}
% \DescribeMacro{\InvpE}
% \DescribeMacro{\InvpE*}
% \DescribeMacro{\InvPE}
% \DescribeMacro{\InvPE*}
% Tiene 1 argumento,\;\cs{Inv}\marg{objeto},\; y denota el inverso del
% \marg{objeto}.
% \begin{center}
%   |\Inv{x}|
%   \hspace{1.2cm}
%   \fbox{$\Inv{x}$}
% \end{center}
% \begin{center}
%   |\Invp{x} \Invp*{x}|
%   \hspace{1.2cm}
%   \fbox{$\Invp {x}$}
%   \fbox{$\Invp*{x}$}
% \end{center}
% \begin{center}
%   |\InvP{x} \InvP*{\int\limits_a^b h(x)dx}|
%   \hspace{1.2cm}
%   \fbox{$\InvP {x}$}
%   \fbox{$\InvP*{\int\limits_a^b h(x) dx}$}
% \end{center}
% \begin{center}
%   |\InvpE{x} \InvpE*{x}|
%   \hspace{1.2cm}
%   \fbox{$\InvpE {x}$}
%   \fbox{$\InvpE*{x}$}
% \end{center}
% \begin{center}
%   |\InvPE{x} \InvPE*{x}|
%   \hspace{1.2cm}
%   \fbox{$\InvPE {x}$}
%   \fbox{$\InvPE*{x}$}
% \end{center}
% 
% \subsubsection{Operador selector}
% 
% Denotaremos el operador selector con una barra vertical.
% 
% \DescribeMacro{\getItem}
% El comando \cs{getItem} no tiene argumentos
% \begin{center}
%   |\getItem|
%   \hspace{1.2cm}
%   \fbox{$\getItem$}
% \end{center}
% 
% \DescribeMacro{\getitemL}
% El comando \cs{getitemL} tiene 1 argumento,\;\cs{getitemL}\marg{objeto}.
% \begin{center}
%   |\getitemL{i}|
%   \hspace{1.2cm}
%   \fbox{$\getitemL{i}$}
% \end{center}
% 
% \DescribeMacro{\getitemR}
% El comando \cs{getitemR} tiene 1 argumento,\;\cs{getitemR}\marg{objeto}.
% \begin{center}
%   |\getitemR{j}|
%   \hspace{1.2cm}
%   \fbox{$\getitemR{j}$}
% \end{center}
% 
% \paragraph{por la izquierda de un objeto}
% 
% \DescribeMacro{\elemL}
% \DescribeMacro{\elemLp}
% \DescribeMacro{\elemLp*}
% \DescribeMacro{\elemLP}
% \DescribeMacro{\elemLP*}
% \DescribeMacro{\elemLpE}
% \DescribeMacro{\elemLpE*}
% \DescribeMacro{\elemLPE}
% \DescribeMacro{\elemLPE*}
% El comando |\elemL<XX*>| tiene 2 argumentos,
% \begin{center}
% |\elemL<XX*>|\marg{objeto}\marg{índice(s)},
% \end{center}
% y denota la selección de elementos por la izquierda.
% \begin{center}
%   |\elemL{\Mat{A}}{i}|
%   \hspace{1.2cm}
%   \fbox{$\elemL{\Mat{A}}{i}$}
% \end{center}
% \begin{center}
%   |\elemLp{\Mat{A}}{i} \elemLp*{\Mat{A}}{i}|
%   \hspace{1.2cm}
%   \fbox{$\elemLp {\Mat{A}}{i}$}
%   \fbox{$\elemLp*{\Mat{A}}{i}$}
% \end{center}
% \begin{center}
%   |\elemLP{\Mat{A}}{i} \elemLP*{\Mat{A}}{i}|
%   \hspace{1.2cm}
%   \fbox{$\elemLP {\Mat{A}}{i}$}
%   \fbox{$\elemLP*{\Mat{A}}{i}$}
% \end{center}
% \begin{center}
%   |\elemLpE{\Mat{A}}{i} \elemLpE*{\Mat{A}}{i}|
%   \hspace{1.2cm}
%   \fbox{$\elemLpE {\Mat{A}}{i}$}
%   \fbox{$\elemLpE*{\Mat{A}}{i}$}
% \end{center}
% \begin{center}
%   |\elemLPE{\Mat{A}}{i} \elemLPE{\Mat{A}}{i}|
%   \hspace{1.2cm}
%   \fbox{$\elemLPE {\Mat{A}}{i}$}
%   \fbox{$\elemLPE*{\Mat{A}}{i}$}
% \end{center}
% 
% \paragraph{por la derecha de un objeto}
% 
% \DescribeMacro{\elemR}
% \DescribeMacro{\elemRp}
% \DescribeMacro{\elemRp*}
% \DescribeMacro{\elemRP}
% \DescribeMacro{\elemRP*}
% \DescribeMacro{\elemRpE}
% \DescribeMacro{\elemRpE*}
% \DescribeMacro{\elemRPE}
% \DescribeMacro{\elemRPE*}
% El comando |\elemR<XX*>| tiene 2 argumentos,
% \begin{center}
% |\elemR<XX*>|\marg{objeto}\marg{índice(s)},
% \end{center}
% y denota la selección de elementos por la derecha.
% \begin{center}
%   |\elemR{\Mat{A}}{i}|
%   \hspace{1.2cm}
%   \fbox{$\elemR{\Mat{A}}{i}$}
% \end{center}
% \begin{center}
%   |\elemRp{\Mat{A}}{i} \elemRp*{\Mat{A}}{i}|
%   \hspace{1.2cm}
%   \fbox{$\elemRp {\Mat{A}}{i}$}
%   \fbox{$\elemRp*{\Mat{A}}{i}$}
% \end{center}
% \begin{center}
%   |\elemRP{\Mat{A}}{i} \elemRP*{\Mat{A}}{i}|
%   \hspace{1.2cm}
%   \fbox{$\elemRP {\Mat{A}}{i}$}
%   \fbox{$\elemRP*{\Mat{A}}{i}$}
% \end{center}
% \begin{center}
%   |\elemRpE{\Mat{A}}{i} \elemRpE*{\Mat{A}}{i}|
%   \hspace{1.2cm}
%   \fbox{$\elemRpE {\Mat{A}}{i}$}
%   \fbox{$\elemRpE*{\Mat{A}}{i}$}
% \end{center}
% \begin{center}
%   |\elemRPE{\Mat{A}}{i} \elemRPE{\Mat{A}}{i}|
%   \hspace{1.2cm}
%   \fbox{$\elemRPE {\Mat{A}}{i}$}
%   \fbox{$\elemRPE*{\Mat{A}}{i}$}
% \end{center}
% 
% \paragraph{por ambos lados de un objeto}
% 
% \DescribeMacro{\elemLR}
% \DescribeMacro{\elemLRp}
% \DescribeMacro{\elemLRp*}
% \DescribeMacro{\elemLRP}
% \DescribeMacro{\elemLRP*}
% \DescribeMacro{\elemLRpE}
% \DescribeMacro{\elemLRpE*}
% \DescribeMacro{\elemLRPE}
% \DescribeMacro{\elemLRPE*}
% El comando |\elemLR<XX*>| tiene 3
% argumentos,
% \begin{center}
% \;|\elemLR<XX*>|\marg{objeto}\marg{indice(s)Izda}\marg{indice(s)Dcha},\;
% \end{center}
% y denota la selección de elementos por ambos lados.
% \begin{center}
%   |\elemLR{\Mat{A}}{i}{j}|
%   \hspace{1.2cm}
%   \fbox{$\elemLR{\Mat{A}}{i}{j}$}
% \end{center}
% \begin{center}
%   |\elemLRp{\Mat{A}}{i}{j} \elemLRp*{\Mat{A}}{i}{j}|
%   \hspace{1.2cm}
%   \fbox{$\elemLRp {\Mat{A}}{i}{j}$}
%   \fbox{$\elemLRp*{\Mat{A}}{i}{j}$}
% \end{center}
% \begin{center}
%   |\elemLRP{\Mat{A}}{i}{j} \elemLRP*{\Mat{A}}{i}{j}|
%   \hspace{1.2cm}
%   \fbox{$\elemLRP {\Mat{A}}{i}{j}$}
%   \fbox{$\elemLRP*{\Mat{A}}{i}{j}$}
% \end{center}
% \begin{center}
%   |\elemLRpE{\Mat{A}}{i}{j} \elemLRpE*{\Mat{A}}{i}{j}|
%   \hspace{1.2cm}
%   \fbox{$\elemLRpE {\Mat{A}}{i}{j}$}
%   \fbox{$\elemLRpE*{\Mat{A}}{i}{j}$}
% \end{center}
% \begin{center}
%   |\elemLRPE{\Mat{A}}{i}{j} \elemLRPE*{\Mat{A}}{i}{j}|
%   \hspace{1.2cm}
%   \fbox{$\elemLRPE {\Mat{A}}{i}{j}$}
%   \fbox{$\elemLRPE*{\Mat{A}}{i}{j}$}
% \end{center}
% 
% %%%%%%%%%%%%%%%%%%%%%%%%%%%%%%%%%%%%%%
% \paragraph{por la izquierda de un vector}
% 
% \DescribeMacro{\eleVL}
% \DescribeMacro{\eleVLp}
% \DescribeMacro{\eleVLp*}
% \DescribeMacro{\eleVLP}
% \DescribeMacro{\eleVLP*}
% \DescribeMacro{\eleVLpE}
% \DescribeMacro{\eleVLpE*}
% \DescribeMacro{\eleVLPE}
% \DescribeMacro{\eleVLPE*}
% El comando |\eleVL<XX*>| tiene 3 argumentos,
% \begin{center}
%   |\eleVL<XX*>|\oarg{subíndice}\marg{nombre}\marg{índice(s)},
% \end{center}
% y denota la
% selección de elementos por la izquierda de un vector.
% \begin{center}
%   |\eleVL{a}{i} \eleVL[h]{a}{i}|
%   \hspace{1.2cm}
%   \fbox{$\eleVL   {a}{i}$}
%   \fbox{$\eleVL[h]{a}{i}$}
% \end{center}
% 
% \begin{center}
%   |\eleVLp{a}{i} \eleVLp[h]{a}{i}|
%   \hspace{1.2cm}
%   \fbox{$\eleVLp    {a}{i}$}
%   \fbox{$\eleVLp[h] {a}{i}$}
% \end{center}
% \begin{center}
%   |\eleVLp*{a}{i} \eleVLp*[h]{a}{i}|
%   \hspace{1.2cm}
%   \fbox{$\eleVLp*   {a}{i}$}
%   \fbox{$\eleVLp*[h]{a}{i}$}
% \end{center}
% 
% \begin{center}
%   |\eleVLP{a}{i} \eleVLP[h]{a}{i}|
%   \hspace{1.2cm}
%   \fbox{$\eleVLP    {a}{i}$}
%   \fbox{$\eleVLP[h] {a}{i}$}
% \end{center}
% \begin{center}
%   |\eleVLP*{a}{i} \eleVLP*[h]{a}{i}|
%   \hspace{1.2cm}
%   \fbox{$\eleVLP*   {a}{i}$}
%   \fbox{$\eleVLP*[h]{a}{i}$}
% \end{center}
% 
% \begin{center}
%   |\eleVLpE{a}{i} \eleVLpE[h]{a}{i}|
%   \hspace{1.2cm}
%   \fbox{$\eleVLpE    {a}{i}$}
%   \fbox{$\eleVLpE[h] {a}{i}$}
% \end{center}
% \begin{center}
%   |\eleVLpE*{a}{i} \eleVLpE*[h]{a}{i}|
%   \hspace{1.2cm}
%   \fbox{$\eleVLpE*   {a}{i}$}
%   \fbox{$\eleVLpE*[h]{a}{i}$}
% \end{center}
% 
% \begin{center}
%   |\eleVLPE{a}{i} \eleVLPE[h]{a}{i}|
%   \hspace{1.2cm}
%   \fbox{$\eleVLPE    {a}{i}$}
%   \fbox{$\eleVLPE[h] {a}{i}$}
% \end{center}
% \begin{center}
%   |\eleVLPE*{a}{i} \eleVLPE*[h]{a}{i}|
%   \hspace{1.2cm}
%   \fbox{$\eleVLPE*   {a}{i}$}
%   \fbox{$\eleVLPE*[h]{a}{i}$}
% \end{center}
% 
% %%%%%%%%%%%%%%%%%%%%%%%%%%%%%%%%%%%%%%
% \paragraph{por la derecha de un vector}
% 
% \DescribeMacro{\eleVR}
% \DescribeMacro{\eleVRp}
% \DescribeMacro{\eleVRp*}
% \DescribeMacro{\eleVRP}
% \DescribeMacro{\eleVRP*}
% \DescribeMacro{\eleVRpE}
% \DescribeMacro{\eleVRpE*}
% \DescribeMacro{\eleVRPE}
% \DescribeMacro{\eleVRPE*}
% El comando |\eleVR<XX*>| tiene 3 argumentos,
% \begin{center}
%   |\eleVR<XX*>|\oarg{subíndice}\marg{nombre}\marg{índice(s)},
% \end{center}
% y denota la selección de elementos por la derecha de un vector.
% \begin{center}
%   |\eleVR{a}{i} \eleVR[h]{a}{i}|
%   \hspace{1.2cm}
%   \fbox{$\eleVR   {a}{i}$}
%   \fbox{$\eleVR[h]{a}{i}$}
% \end{center}
% 
% \begin{center}
%   |\eleVRp{a}{i} \eleVRp[h]{a}{i}|
%   \hspace{1.2cm}
%   \fbox{$\eleVRp    {a}{i}$}
%   \fbox{$\eleVRp[h] {a}{i}$}
% \end{center}
% \begin{center}
%   |\eleVRp*{a}{i} \eleVRp*[h]{a}{i}|
%   \hspace{1.2cm}
%   \fbox{$\eleVRp*   {a}{i}$}
%   \fbox{$\eleVRp*[h]{a}{i}$}
% \end{center}
% 
% \begin{center}
%   |\eleVRP{a}{i} \eleVRP[h]{a}{i}|
%   \hspace{1.2cm}
%   \fbox{$\eleVRP    {a}{i}$}
%   \fbox{$\eleVRP[h] {a}{i}$}
% \end{center}
% \begin{center}
%   |\eleVRP*{a}{i} \eleVRP*[h]{a}{i}|
%   \hspace{1.2cm}
%   \fbox{$\eleVRP*   {a}{i}$}
%   \fbox{$\eleVRP*[h]{a}{i}$}
% \end{center}
% 
% \begin{center}
%   |\eleVRpE{a}{i} \eleVRpE[h]{a}{i}|
%   \hspace{1.2cm}
%   \fbox{$\eleVRpE    {a}{i}$}
%   \fbox{$\eleVRpE[h] {a}{i}$}
% \end{center}
% \begin{center}
%   |\eleVRpE*{a}{i} \eleVRpE*[h]{a}{i}|
%   \hspace{1.2cm}
%   \fbox{$\eleVRpE*   {a}{i}$}
%   \fbox{$\eleVRpE*[h]{a}{i}$}
% \end{center}
% 
% \begin{center}
%   |\eleVRPE{a}{i} \eleVRPE[h]{a}{i}|
%   \hspace{1.2cm}
%   \fbox{$\eleVRPE    {a}{i}$}
%   \fbox{$\eleVRPE[h] {a}{i}$}
% \end{center}
% \begin{center}
%   |\eleVRPE*{a}{i} \eleVRPE*[h]{a}{i}|
%   \hspace{1.2cm}
%   \fbox{$\eleVRPE*   {a}{i}$}
%   \fbox{$\eleVRPE*[h]{a}{i}$}
% \end{center}
% 
% %%%%%%%%%%%%%%%%%%%%%%%%%%%%%%%%%%%%%%
% \paragraph{por la izquierda de una matriz (filas)}
% 
% \DescribeMacro{\VectF}
% \DescribeMacro{\VectFp}
% \DescribeMacro{\VectFp*}
% \DescribeMacro{\VectFP}
% \DescribeMacro{\VectFP*}
% \DescribeMacro{\VectFpE}
% \DescribeMacro{\VectFpE*}
% \DescribeMacro{\VectFPE}
% \DescribeMacro{\VectFPE*}
% El comando |\VectF<XX*>| tiene 3 argumentos,
% \begin{center}
%   |\VectF<XX*>|\oarg{subíndice}\marg{nombre}\marg{índice(s)},
% \end{center}
% y denota la selección de filas de una matriz (nótese que
% automáticamente se añade un paréntesis cuando la matriz lleva un
% subíndice y la expresión lo requiere)
% \begin{center}
%   |\VectF{A}{i} \VectF[h]{A}{i}|
%   \hspace{1.2cm}
%   \fbox{$\VectF   {A}{i}$}
%   \fbox{$\VectF[h]{A}{i}$}
% \end{center}
% \begin{center}
%   |\VectFp{A}{i} \VectFp[h]{A}{i}|
%   \hspace{1.2cm}
%   \fbox{$\VectFp   {A}{i}$}
%   \fbox{$\VectFp[h]{A}{i}$}
% \end{center}
% \begin{center}
%   |\VectFp*{A}{i} \VectFp*[h]{A}{i}|
%   \hspace{1.2cm}
%   \fbox{$\VectFp*   {A}{i}$}
%   \fbox{$\VectFp*[h]{A}{i}$}
% \end{center}
% \begin{center}
%   |\VectFP{A}{i} \VectFP[h]{A}{i}|
%   \hspace{1.2cm}
%   \fbox{$\VectFP   {A}{i}$}
%   \fbox{$\VectFP[h]{A}{i}$}
% \end{center}
% \begin{center}
%   |\VectFP*{A}{i} \VectFP*[h]{A}{i}|
%   \hspace{1.2cm}
%   \fbox{$\VectFP*   {A}{i}$}
%   \fbox{$\VectFP*[h]{A}{i}$}
% \end{center}
% \begin{center}
%   |\VectFpE{A}{i} \VectFpE[h]{A}{i}|
%   \hspace{1.2cm}
%   \fbox{$\VectFpE   {A}{i}$}
%   \fbox{$\VectFpE[h]{A}{i}$}
% \end{center}
% \begin{center}
%   |\VectFpE*{A}{i} \VectFpE*[h]{A}{i}|
%   \hspace{1.2cm}
%   \fbox{$\VectFpE*   {A}{i}$}
%   \fbox{$\VectFpE*[h]{A}{i}$}
% \end{center}
% \begin{center}
%   |\VectFPE{A}{i} \VectFPE[h]{A}{i}|
%   \hspace{1.2cm}
%   \fbox{$\VectFPE   {A}{i}$}
%   \fbox{$\VectFPE[h]{A}{i}$}
% \end{center}
% \begin{center}
%   |\VectFPE*{A}{i} \VectFPE*[h]{A}{i}|
%   \hspace{1.2cm}
%   \fbox{$\VectFPE*   {A}{i}$}
%   \fbox{$\VectFPE*[h]{A}{i}$}
% \end{center}
% 
% 
% \DescribeMacro{\VectTF}
% \DescribeMacro{\VectTFp}
% \DescribeMacro{\VectTFp*}
% \DescribeMacro{\VectTFP}
% \DescribeMacro{\VectTFP*}
% \DescribeMacro{\VectTFpE}
% \DescribeMacro{\VectTFpE*}
% \DescribeMacro{\VectTFPE}
% \DescribeMacro{\VectTFPE*}
% El comando |\VectTF<XX*>| tiene 3 argumentos,
% \begin{center}
%   |\VectTF<XX*>|\oarg{subíndice}\marg{nombre}\marg{índice(s)},
% \end{center}
% y denota la selección de filas de una matriz (nótese que
% automáticamente se añade un paréntesis cuando la matriz lleva un
% subíndice y la expresión lo requiere)
% \begin{center}
%   |\VectTF{A}{i} \VectTF[h]{A}{i}|
%   \hspace{1.2cm}
%   \fbox{$\VectTF   {A}{i}$}
%   \fbox{$\VectTF[h]{A}{i}$}
% \end{center}
% \begin{center}
%   |\VectTFp{A}{i} \VectTFp[h]{A}{i}|
%   \hspace{1.2cm}
%   \fbox{$\VectTFp   {A}{i}$}
%   \fbox{$\VectTFp[h]{A}{i}$}
% \end{center}
% \begin{center}
%   |\VectTFp*{A}{i} \VectTFp*[h]{A}{i}|
%   \hspace{1.2cm}
%   \fbox{$\VectTFp*   {A}{i}$}
%   \fbox{$\VectTFp*[h]{A}{i}$}
% \end{center}
% \begin{center}
%   |\VectTFP{A}{i} \VectTFP[h]{A}{i}|
%   \hspace{1.2cm}
%   \fbox{$\VectTFP   {A}{i}$}
%   \fbox{$\VectTFP[h]{A}{i}$}
% \end{center}
% \begin{center}
%   |\VectTFP*{A}{i} \VectTFP*[h]{A}{i}|
%   \hspace{1.2cm}
%   \fbox{$\VectTFP*   {A}{i}$}
%   \fbox{$\VectTFP*[h]{A}{i}$}
% \end{center}
% \begin{center}
%   |\VectTFpE{A}{i} \VectTFpE[h]{A}{i}|
%   \hspace{1.2cm}
%   \fbox{$\VectTFpE   {A}{i}$}
%   \fbox{$\VectTFpE[h]{A}{i}$}
% \end{center}
% \begin{center}
%   |\VectTFpE*{A}{i} \VectTFpE*[h]{A}{i}|
%   \hspace{1.2cm}
%   \fbox{$\VectTFpE*   {A}{i}$}
%   \fbox{$\VectTFpE*[h]{A}{i}$}
% \end{center}
% \begin{center}
%   |\VectTFPE{A}{i} \VectTFPE[h]{A}{i}|
%   \hspace{1.2cm}
%   \fbox{$\VectTFPE   {A}{i}$}
%   \fbox{$\VectTFPE[h]{A}{i}$}
% \end{center}
% \begin{center}
%   |\VectTFPE*{A}{i} \VectTFPE*[h]{A}{i}|
%   \hspace{1.2cm}
%   \fbox{$\VectTFPE*   {A}{i}$}
%   \fbox{$\VectTFPE*[h]{A}{i}$}
% \end{center}
% 
% %%%%%%%%%%%%%%%%%%%%%%%%%%%%%%%%%%%%%%
% \paragraph{por la derecha de una matriz (columnas)}
% 
% \DescribeMacro{\VectC}
% \DescribeMacro{\VectCp}
% \DescribeMacro{\VectCp*}
% \DescribeMacro{\VectCP}
% \DescribeMacro{\VectCP*}
% \DescribeMacro{\VectCpE}
% \DescribeMacro{\VectCpE*}
% \DescribeMacro{\VectCPE}
% \DescribeMacro{\VectCPE*}
% El comando |\VectC<XX*>| tiene 3 argumentos,
% \begin{center}
%   |\VectC<XX*>|\oarg{subíndice}\marg{nombre}\marg{índice(s)},
% \end{center}
% y denota la selección de columnas de una matriz (nótese que
% automáticamente se añade un paréntesis cuando la matriz lleva un
% subíndice y la expresión lo requiere)
% \begin{center}
%   |\VectC{A}{i} \VectC[h]{A}{i}|
%   \hspace{1.2cm}
%   \fbox{$\VectC   {A}{i}$}
%   \fbox{$\VectC[h]{A}{i}$}
% \end{center}
% \begin{center}
%   |\VectCp{A}{i} \VectCp[h]{A}{i}|
%   \hspace{1.2cm}
%   \fbox{$\VectCp   {A}{i}$}
%   \fbox{$\VectCp[h]{A}{i}$}
% \end{center}
% \begin{center}
%   |\VectCp*{A}{i} \VectCp*[h]{A}{i}|
%   \hspace{1.2cm}
%   \fbox{$\VectCp*   {A}{i}$}
%   \fbox{$\VectCp*[h]{A}{i}$}
% \end{center}
% \begin{center}
%   |\VectCP{A}{i} \VectCP[h]{A}{i}|
%   \hspace{1.2cm}
%   \fbox{$\VectCP   {A}{i}$}
%   \fbox{$\VectCP[h]{A}{i}$}
% \end{center}
% \begin{center}
%   |\VectCP*{A}{i} \VectCP*[h]{A}{i}|
%   \hspace{1.2cm}
%   \fbox{$\VectCP*   {A}{i}$}
%   \fbox{$\VectCP*[h]{A}{i}$}
% \end{center}
% \begin{center}
%   |\VectCpE{A}{i} \VectCpE[h]{A}{i}|
%   \hspace{1.2cm}
%   \fbox{$\VectCpE   {A}{i}$}
%   \fbox{$\VectCpE[h]{A}{i}$}
% \end{center}
% \begin{center}
%   |\VectCpE*{A}{i} \VectCpE*[h]{A}{i}|
%   \hspace{1.2cm}
%   \fbox{$\VectCpE*   {A}{i}$}
%   \fbox{$\VectCpE*[h]{A}{i}$}
% \end{center}
% \begin{center}
%   |\VectCPE{A}{i} \VectCPE[h]{A}{i}|
%   \hspace{1.2cm}
%   \fbox{$\VectCPE   {A}{i}$}
%   \fbox{$\VectCPE[h]{A}{i}$}
% \end{center}
% \begin{center}
%   |\VectCPE*{A}{i} \VectCPE*[h]{A}{i}|
%   \hspace{1.2cm}
%   \fbox{$\VectCPE*   {A}{i}$}
%   \fbox{$\VectCPE*[h]{A}{i}$}
% \end{center}
% 
% 
% \DescribeMacro{\VectTC}
% \DescribeMacro{\VectTCp}
% \DescribeMacro{\VectTCp*}
% \DescribeMacro{\VectTCP}
% \DescribeMacro{\VectTCP*}
% \DescribeMacro{\VectTCpE}
% \DescribeMacro{\VectTCpE*}
% \DescribeMacro{\VectTCPE}
% \DescribeMacro{\VectTCPE*}
% El comando |\VectTC<XX*>| tiene 3 argumentos,
% \begin{center}
%   |\VectTC<XX*>|\oarg{subíndice}\marg{nombre}\marg{índice(s)},
% \end{center}
% y denota la selección de filas de una matriz (nótese que
% automáticamente se añade un paréntesis cuando la matriz lleva un
% subíndice y la expresión lo requiere)
% \begin{center}
%   |\VectTC{A}{i} \VectTC[h]{A}{i}|
%   \hspace{1.2cm}
%   \fbox{$\VectTC   {A}{i}$}
%   \fbox{$\VectTC[h]{A}{i}$}
% \end{center}
% \begin{center}
%   |\VectTCp{A}{i} \VectTCp[h]{A}{i}|
%   \hspace{1.2cm}
%   \fbox{$\VectTCp   {A}{i}$}
%   \fbox{$\VectTCp[h]{A}{i}$}
% \end{center}
% \begin{center}
%   |\VectTCp*{A}{i} \VectTCp*[h]{A}{i}|
%   \hspace{1.2cm}
%   \fbox{$\VectTCp*   {A}{i}$}
%   \fbox{$\VectTCp*[h]{A}{i}$}
% \end{center}
% \begin{center}
%   |\VectTCP{A}{i} \VectTCP[h]{A}{i}|
%   \hspace{1.2cm}
%   \fbox{$\VectTCP   {A}{i}$}
%   \fbox{$\VectTCP[h]{A}{i}$}
% \end{center}
% \begin{center}
%   |\VectTCP*{A}{i} \VectTCP*[h]{A}{i}|
%   \hspace{1.2cm}
%   \fbox{$\VectTCP*   {A}{i}$}
%   \fbox{$\VectTCP*[h]{A}{i}$}
% \end{center}
% \begin{center}
%   |\VectTCpE{A}{i} \VectTCpE[h]{A}{i}|
%   \hspace{1.2cm}
%   \fbox{$\VectTCpE   {A}{i}$}
%   \fbox{$\VectTCpE[h]{A}{i}$}
% \end{center}
% \begin{center}
%   |\VectTCpE*{A}{i} \VectTCpE*[h]{A}{i}|
%   \hspace{1.2cm}
%   \fbox{$\VectTCpE*   {A}{i}$}
%   \fbox{$\VectTCpE*[h]{A}{i}$}
% \end{center}
% \begin{center}
%   |\VectTCPE{A}{i} \VectTCPE[h]{A}{i}|
%   \hspace{1.2cm}
%   \fbox{$\VectTCPE   {A}{i}$}
%   \fbox{$\VectTCPE[h]{A}{i}$}
% \end{center}
% \begin{center}
%   |\VectTCPE*{A}{i} \VectTCPE*[h]{A}{i}|
%   \hspace{1.2cm}
%   \fbox{$\VectTCPE*   {A}{i}$}
%   \fbox{$\VectTCPE*[h]{A}{i}$}
% \end{center}
% 
% %%%%%%%%%%%%%%%%%%%%%%%%%%%%%%%%%%%%%%
% \paragraph{de elementos de una matriz}
% 
% \DescribeMacro{\eleM}
% \DescribeMacro{\eleMp}
% \DescribeMacro{\eleMp*}
% \DescribeMacro{\eleMP}
% \DescribeMacro{\eleMP*}
% \DescribeMacro{\eleMpE}
% \DescribeMacro{\eleMpE*}
% \DescribeMacro{\eleMPE}
% \DescribeMacro{\eleMPE*}
% El comando |\eleM<XX*>| tiene 4 argumentos,
% \begin{center}
%   |\eleM<XX*>|\oarg{subíndice}\marg{nombre}\marg{índice(s)Fil}\marg{índice(s)Col},
% \end{center}
% y denota la selección de filas y columnas de una matriz (nótese que
% automáticamente se añade un paréntesis cuando la matriz lleva un
% subíndice y la expresión lo requiere)
% \begin{center}
%   |\eleM{A}{i}{j} \eleM[h]{A}{i}{j}|
%   \hspace{1.2cm}
%   \fbox{$\eleM   {A}{i}{j}$}
%   \fbox{$\eleM[h]{A}{i}{j}$}
% \end{center}
% 
% \begin{center}
%   |\eleMp{A}{i}{j} \eleMp[h]{A}{i}{j}|
%   \hspace{1.2cm}
%   \fbox{$\eleMp   {A}{i}{j}$}
%   \fbox{$\eleMp[h]{A}{i}{j}$}
% \end{center}
% \begin{center}
%   |\eleMp*{A}{i}{j} \eleMp*[h]{A}{i}{j}|
%   \hspace{1.2cm}
%   \fbox{$\eleMp*   {A}{i}{j}$}
%   \fbox{$\eleMp*[h]{A}{i}{j}$}
% \end{center}
% 
% \begin{center}
%   |\eleMP{A}{i}{j} \eleMP[h]{A}{i}{j}|
%   \hspace{1.2cm}
%   \fbox{$\eleMP   {A}{i}{j}$}
%   \fbox{$\eleMP[h]{A}{i}{j}$}
% \end{center}
% \begin{center}
%   |\eleMP*{A}{i}{j} \eleMP*[h]{A}{i}{j}|
%   \hspace{1.2cm}
%   \fbox{$\eleMP*   {A}{i}{j}$}
%   \fbox{$\eleMP*[h]{A}{i}{j}$}
% \end{center}
% 
% \begin{center}
%   |\eleMpE{A}{i}{j} \eleMpE[h]{A}{i}{j}|
%   \hspace{1.2cm}
%   \fbox{$\eleMpE   {A}{i}{j}$}
%   \fbox{$\eleMpE[h]{A}{i}{j}$}
% \end{center}
% \begin{center}
%   |\eleMpE*{A}{i}{j} \eleMpE*[h]{A}{i}{j}|
%   \hspace{1.2cm}
%   \fbox{$\eleMpE*   {A}{i}{j}$}
%   \fbox{$\eleMpE*[h]{A}{i}{j}$}
% \end{center}
% 
% \begin{center}
%   |\eleMPE{A}{i}{j} \eleMPE[h]{A}{i}{j}|
%   \hspace{1.2cm}
%   \fbox{$\eleMPE   {A}{i}{j}$}
%   \fbox{$\eleMPE[h]{A}{i}{j}$}
% \end{center}
% \begin{center}
%   |\eleMPE*{A}{i}{j} \eleMPE*[h]{A}{i}{j}|
%   \hspace{1.2cm}
%   \fbox{$\eleMPE*   {A}{i}{j}$}
%   \fbox{$\eleMPE*[h]{A}{i}{j}$}
% \end{center}
% 
% 
% %%%%%%%%%%%%%%%%%%%%%%%%%%%%%%%%%%%%%%
% \paragraph{de elementos de una matriz transpuesta}
% 
% \DescribeMacro{\eleMT}
% \DescribeMacro{\eleMTp}
% \DescribeMacro{\eleMTp*}
% \DescribeMacro{\eleMTP}
% \DescribeMacro{\eleMTP*}
% \DescribeMacro{\eleMTpE}
% \DescribeMacro{\eleMTpE*}
% \DescribeMacro{\eleMTPE}
% \DescribeMacro{\eleMTPE*}
% El comando |\eleMT<XX*>| tiene 4 argumentos,
% \begin{center}
%   |\eleMT<XX*>|\oarg{subíndice}\marg{nombre}\marg{índice(s)Fil}\marg{índice(s)Col},
% \end{center}
% y denota la selección de filas y columnas de una matriz (nótese que
% automáticamente se añade un paréntesis cuando la matriz lleva un
% subíndice y la expresión lo requiere)
% \begin{center}
%   |\eleMT{A}{i}{j} \eleMT[h]{A}{i}{j}|
%   \hspace{1.2cm}
%   \fbox{$\eleMT   {A}{i}{j}$}
%   \fbox{$\eleMT[h]{A}{i}{j}$}
% \end{center}
% 
% \begin{center}
%   |\eleMTp{A}{i}{j} \eleMTp[h]{A}{i}{j}|
%   \hspace{1.2cm}
%   \fbox{$\eleMTp   {A}{i}{j}$}
%   \fbox{$\eleMTp[h]{A}{i}{j}$}
% \end{center}
% \begin{center}
%   |\eleMTp*{A}{i}{j} \eleMTp*[h]{A}{i}{j}|
%   \hspace{1.2cm}
%   \fbox{$\eleMTp*   {A}{i}{j}$}
%   \fbox{$\eleMTp*[h]{A}{i}{j}$}
% \end{center}
% 
% \begin{center}
%   |\eleMTP{A}{i}{j} \eleMTP[h]{A}{i}{j}|
%   \hspace{1.2cm}
%   \fbox{$\eleMTP   {A}{i}{j}$}
%   \fbox{$\eleMTP[h]{A}{i}{j}$}
% \end{center}
% \begin{center}
%   |\eleMTP*{A}{i}{j} \eleMTP*[h]{A}{i}{j}|
%   \hspace{1.2cm}
%   \fbox{$\eleMTP*   {A}{i}{j}$}
%   \fbox{$\eleMTP*[h]{A}{i}{j}$}
% \end{center}
% 
% \begin{center}
%   |\eleMTpE{A}{i}{j} \eleMTpE[h]{A}{i}{j}|
%   \hspace{1.2cm}
%   \fbox{$\eleMTpE   {A}{i}{j}$}
%   \fbox{$\eleMTpE[h]{A}{i}{j}$}
% \end{center}
% \begin{center}
%   |\eleMTpE*{A}{i}{j} \eleMTpE*[h]{A}{i}{j}|
%   \hspace{1.2cm}
%   \fbox{$\eleMTpE*   {A}{i}{j}$}
%   \fbox{$\eleMTpE*[h]{A}{i}{j}$}
% \end{center}
% 
% \begin{center}
%   |\eleMTPE{A}{i}{j} \eleMTPE[h]{A}{i}{j}|
%   \hspace{1.2cm}
%   \fbox{$\eleMTPE   {A}{i}{j}$}
%   \fbox{$\eleMTPE[h]{A}{i}{j}$}
% \end{center}
% \begin{center}
%   |\eleMTPE*{A}{i}{j} \eleMTPE*[h]{A}{i}{j}|
%   \hspace{1.2cm}
%   \fbox{$\eleMTPE*   {A}{i}{j}$}
%   \fbox{$\eleMTPE*[h]{A}{i}{j}$}
% \end{center}
% 
% 
% \subsubsection{Operaciones elementales}
% \label{sec:TransfElem}
% 
%  Primero fijamos la notación de las operaciones elementales tipo I y II, los intercambios y las reordenaciones (o permutaciones).
% 
% \DescribeMacro{\su}
% El comando \cs{su} tiene 3
% argumentos,\;\cs{pe}\marg{escalar}\marg{índice}\marg{índice},\; e indica una
% transformación Tipo I.
% \begin{center}
%   |\su{a}{j}{k}|
%   \hspace{1.2cm}
%   \fbox{$\su{a}{j}{k}$}
% \end{center}
% 
% \DescribeMacro{\pr}
% El comando \cs{pr} tiene 2
% argumento,\;\cs{pr}\marg{escalar}\marg{índice},\; e indica una
% transformación Tipo II.
% \begin{center}
%   |\pr{a}{k}|
%   \hspace{1.2cm}
%   \fbox{$\pr{a}{k}$}
% \end{center}
% 
% \DescribeMacro{\pe}
% El comando \cs{pr} tiene 2
% argumento,\;\cs{pr}\marg{índice}\marg{índice},\; e indica un intercambio.
% \begin{center}
%   |\pe{i}{k}|
%   \hspace{1.2cm}
%   \fbox{$\pe{i}{k}$}
% \end{center}
% 
% \DescribeMacro{\perm}
% El comando \cs{perm} no tiene
% argumentos e indica un reordenamiento o permutación.
% \begin{center}
%   |\perm|
%   \hspace{1.2cm}
%   \fbox{$\perm$}
% \end{center}
% \bigskip
% 
% \emph{Usaremos letra griega tau para denotar una operación elemental (o una
% secuencia de ellas)}.
% \bigskip
% 
% \DescribeMacro{\TrEl}
% El comando \cs{TrEl} no tiene argumentos
% \begin{center}
%   |\TrEl|
%   \hspace{1.2cm}
%   \fbox{$\TrEl$}
% \end{center}
% 
% %%%%%%%%%%%%%%%%%%%%%%%%%%%%%
% \DescribeMacro{\OpE}
% El comando \cs{OpE} tiene 1
% argumento,\;\cs{OpE}\marg{detalles},\; e indica una operación elemental.
% \begin{center}
%   |\OpE{xyz}|
%   \hspace{1.2cm}
%   \fbox{$\OpE{xyz}$}
% \end{center}
% 
% \DescribeMacro{\OEsu}
% El comando \cs{OEsu} tiene 3
% argumentos,\;\cs{OEsu}\marg{num}\marg{índice}\marg{índice},\; e indica una operación elemental de Tipo I
% \begin{center}
%   |\OEsu{a}{j}{k}|
%   \hspace{1.2cm}
%   \fbox{$\OEsu{a}{j}{k}$}
% \end{center}
% 
% \DescribeMacro{\OEpr}
% El comando \cs{OEpr} tiene 2
% argumentos,\;\cs{OEpr}\marg{num}\marg{índice},\; e indica una operación elemental de Tipo II
% \begin{center}
%   |\OEpr{a}{j}|
%   \hspace{1.2cm}
%   \fbox{$\OEpr{a}{j}$}
% \end{center}
% 
% \DescribeMacro{\OEin}
% El comando \cs{OEin} tiene 2
% argumentos,\;\cs{OEin}\marg{índice}\marg{índice},\; e indica un intercambio de posición entre componentes
% \begin{center}
%   |\OEin{k}{j}|
%   \hspace{1.2cm}
%   \fbox{$\OEin{k}{j}$}
% \end{center}
% 
% \DescribeMacro{\OEper}
% El comando \cs{OEper} no tiene
% argumentos e indica un reordenamiento o permutación entre componentes
% \begin{center}
%   |\OEper|
%   \hspace{1.2cm}
%   \fbox{$\OEper$}
% \end{center}
% 
% \DescribeMacro{\EOEsu}
% El comando \cs{EOEsu} tiene 3
% argumentos,\;\cs{EOEsu}\marg{num}\marg{índice}\marg{índice},\; e indica la operación espejo de una elemental de Tipo I
% \begin{center}
%   |\EOEsu{a}{j}{k}|
%   \hspace{1.2cm}
%   \fbox{$\EOEsu{a}{j}{k}$}
% \end{center}
% 
% \DescribeMacro{\EOEpr}
% El comando \cs{EOEpr} tiene 2
% argumentos,\;\cs{EOEpr}\marg{num}\marg{índice},\; e indica la operación espejo de una elemental de Tipo II
% \begin{center}
%   |\EOEpr{a}{j}|
%   \hspace{1.2cm}
%   \fbox{$\EOEpr{a}{j}$}
% \end{center}
% 
% \paragraph{Operaciones elementales genéricas.} Los siguientes comandos
% \emph{tienen argumentos opcionales, que no funcionan al escribir
%   preguntas para Moodle}.
% \bigskip
% 
% \DescribeMacro{\OEg}
% El comando \cs{OEg} tiene 2
% argumentos opcionales,\;\cs{OEg}\oarg{índice}\oarg{exponente},\; e indica una
% operación elemental genérica
% \begin{center}
%   |\OEg \OEg[k] \OEg[][*] \OEg[k][*]|
%   \hspace{1.2cm}
%   \fbox{$\OEg$} \fbox{$\OEg[k]$} \fbox{$\OEg[][*]$} \fbox{$\OEg[k][*]$}
% \end{center}
% 
% \DescribeMacro{\EOEg}
% El comando \cs{EOEg} tiene 2
% argumentos opcionales,\;\cs{EOEg}\oarg{índice}\oarg{exponente},\; e indica la
% operación espejo de una elemental genérica
% \begin{center}
%   |\EOEg \EOEg[k] \EOEg[][*] \EOEg[k][*]|
%   \hspace{1.2cm}
%   \fbox{$\EOEg$} \fbox{$\EOEg[k]$} \fbox{$\EOEg[][*]$} \fbox{$\EOEg[k][*]$}
% \end{center}
% 
% \DescribeMacro{\InvOEg}
% El comando \cs{InvEOEg} tiene 1
% argumento opcional,\;\cs{InvOEg}\oarg{índice},\; e indica la
% operación inversa de una elemental genérica
% \begin{center}
%   |\InvOEg \InvOEg[k]|
%   \hspace{1.2cm}
%   \fbox{$\InvOEg$} \fbox{$\InvOEg[k]$}
% \end{center}
% 
% \DescribeMacro{\EInvOEg}
% El comando \cs{EInvOEg} tiene 1
% argumento opcional,\;\cs{EInvOEg}\oarg{índice},\; e indica la
% operación espejo de la inversa de una elemental genérica
% \begin{center}
%   |\EInvOEg \EInvOEg[k]|
%   \hspace{1.2cm}
%   \fbox{$\EInvOEg$} \fbox{$\EInvOEg[k]$}
% \end{center}
% 
% \DescribeMacro{\SEg}
% El comando \cs{SOEg} tiene 3 argumentos
% opcionales,\;\cs{SOEg}\oarg{indiceInic}\oarg{indiceFin}\oarg{exponente},\;
% e indica una sucesión de operaciones elementales genéricas
% \begin{center}
%   |\SOEg|
%   \hspace{1.2cm}
%   \fbox{$\SOEg$}
% \end{center}
% \begin{center}
%   |\SOEg[8] \SOEg[8][p] \SOEg[8][p][*]|
%   \hspace{1.2cm}
%   \fbox{$\SOEg[8]$} \fbox{$\SOEg[8][p]$} \fbox{$\SOEg[8][p][*]$}
% \end{center}
% 
% \subparagraph{Comandos duplicados para las operaciones elementales
%   generales.}  Desgraciadamente para el propósito de este paquete, las
% macros que he definido al escribir el
% \href{https://github.com/mbujosab/CursoDeAlgebraLineal}{libro} usan
% mayoritariamente argumentos opcionales, que en Moodle no se pueden
% usar. Cambiar las macros originales supondría modificar los archivos
% del libro, las transparencias de clase, los problemas propuestos, los
% exámenes pasados\dots{} demasiado trabajo. La alternativa que me queda
% tampoco me gusta, pero al menos no supone tanto trabajo. Dicha
% alternativa consiste en duplicar comandos, es decir, que por cada
% comando original (con argumentos opcionales) crearé otro comando que
% pinte los mismos símbolos pero sin argumentos opcionales (esta
% solución ya la he tomado con los comandos de notación de los conjuntos
% de números, de manera que para escribir \R[n] ahora tenemos |\R[n]|
% (el argumento opcional es el superíndice) o bien |\Rr^n| (que no tiene
% argumentos opcionales y que es lo que tendremos que usar si queremos
% escribir dicha expresión en en las preguntas para Moodle).
% 
% El criterio de nomenclatura que he adoptado ha sido repetir la letra
% del comando pero en minúscula (salvo en el caso de los complejos); es
% decir, los comandos definidos para el libro son: |\N|, |\Z|, |\R| y
% |\CC| (debido a que |\C| ya es un comando del paquete
% \textsf{hyperref}). Así, que los nuevos comandos que he creado para
% duplicar los anteriores pero sin argumentos opcionales son |\Nn|,
% |\Zz|, |\Rr| y |\Cc|.
% 
% Ahora tengo que pensar en un criterio análogo para que sea fácil pasar
% del comando original a duplicado sin argumentos opcionales. No lo
% tengo claro así que voy a probar con mantener los mismo nombres pero
% con una |d| delante para indicar que es el comando duplicado (no sé
% que tal resultará esta solución).
% 
% 
% \DescribeMacro{\dOEgE}
% El comando \cs{dOEgE} tiene 2
% argumentos,\;\cs{dOEgE}\marg{índice}\marg{exponente},\; e indica una
% operación elemental genérica con un exponente (y replica el
% comando |\OEg| que tiene argumentos opcionales)
% \begin{center}
%   |\dOEgE{}{} \dOEgE{k}{} \dOEgE{k}{*}|
%   \hspace{1.2cm}
%   \fbox{$\dOEgE{}{}$} \fbox{$\dOEgE{k}{}$} \fbox{$\dOEgE{k}{*}$}
% \end{center}
% 
% \DescribeMacro{\dOEg}
% El comando \cs{dOEg} tiene 1
% argumento,\;\cs{dOEg}\marg{índice},\; e indica una
% operación elemental genérica (y replica el
% comando |\OEg| que tiene argumentos opcionales)
% \begin{center}
%   |\dOEg{} \dOEg{k}|
%   \hspace{1.2cm}
%   \fbox{$\dOEg{}$} \fbox{$\dOEg{k}$}
% \end{center}
% 
% También fijamos la notación para operación inversa, la operación espejo y el espejo de la inversa de una operación elemental
% 
% \DescribeMacro{\dEOEgE}
% El comando \cs{dEOEgE} tiene 2
% argumentos,\;\cs{dEOEgE}\marg{índice}\marg{exponente},\; e indica la
% operación espejo de una elemental genérica con un exponente (y replica
% el comando |\EOEg| que tiene argumentos opcionales)
% \begin{center}
%   |\dEOEgE{}{} \dEOEgE{k}{*}|
%   \hspace{1.2cm}
%   \fbox{$\dEOEgE{}{}$} \fbox{$\dEOEgE{k}{*}$}
% \end{center}
% 
% \DescribeMacro{\dEOEg}
% El comando \cs{dEOEg} tiene 1
% argumento,\;\cs{dEOEgE}\marg{índice},\; e indica la
% operación espejo de una elemental genérica (y replica
% el comando |\EOEg| que tiene argumentos opcionales)
% \begin{center}
%   |\dEOEg{} \dEOEg{k}|
%   \hspace{1.2cm}
%   \fbox{$\dEOEg{}$} \fbox{$\dEOEg{k}$}
% \end{center}
% 
% \DescribeMacro{\dInvOEg}
% El comando \cs{dInvOEg} tiene 1
% argumento,\;\cs{dInvOEgE}\marg{índice},\; e indica la inversa de una
% elemental genérica (y replica el comando |\InvOEg| que tiene
% argumentos opcionales)
% \begin{center}
%   |\dInvOEg{} \dInvOEg{k}|
%   \hspace{1.2cm}
%   \fbox{$\dInvOEg{}$} \fbox{$\dInvOEg{k}$}
% \end{center}
% 
% \DescribeMacro{\dEInvOEg}
% El comando \cs{dEInvOEg} tiene 1
% argumento,\;\cs{dEInvOEgE}\marg{índice},\; e indica la
% operación espejo de la inversa de una elemental genérica (y replica
% el comando |\EInvOEg| que tiene argumentos opcionales)
% \begin{center}
%   |\dEInvOEg{} \dEInvOEg{k}|
%   \hspace{1.2cm}
%   \fbox{$\dEInvOEg{}$} \fbox{$\dEInvOEg{k}$}
% \end{center}
% 
% \DescribeMacro{\dSOEgE}
% El comando \cs{dSOEgE} tiene 3
% argumento3,\;\cs{dSOEgE}\marg{indiceInic}\marg{indiceFin}\marg{exponente},\; e indica una sucesión de operaciones elementales genéricas con exponente
% \begin{center}
%   |\dSOEgE{j}{k}{*}|
%   \hspace{1.2cm}
%   \fbox{$\dSOEgE{j}{k}{*}$}
% \end{center}
% 
% \DescribeMacro{\dSOEg}
% El comando \cs{dSOEg} tiene 2
% argumento3,\;\cs{dSOEg}\marg{indiceInic}\marg{indiceFin},\; e indica una sucesión de operaciones elementales genéricas
% \begin{center}
%   |\dSOEg{j}{k|
%   \hspace{1.2cm}
%   \fbox{$\dSOEg{j}{k}$}
% \end{center}
% 
% \subsubsection{Transformaciones elementales particulares}
% 
% \paragraph{Transf. elemental aplicada la izquierda o derecha de un objeto}
% 
% \DescribeMacro{\TESF}
% \DescribeMacro{\TESFp}
% \DescribeMacro{\TESFp*}
% \DescribeMacro{\TESFP}
% \DescribeMacro{\TESFP*}
% \DescribeMacro{\TESFpE}
% \DescribeMacro{\TESFpE*}
% \DescribeMacro{\TESFPE}
% \DescribeMacro{\TESFPE*}
% El comando \cs{TESF} tiene 4
% argumentos,\;\cs{TESF}\marg{escalar}\marg{índice}\marg{índice}\marg{objeto},\;
% e indica una transformación elemental de Tipo  I por la izquierda del objeto.
% \begin{center}
%   |\TESF{a}{i}{j}{\Mat{A}}|
%   \hspace{1.2cm}
%   \fbox{$\TESF{a}{i}{j}{\Mat{A}}$}
% \end{center}
% \begin{center}
%   |\TESFp{a}{i}{j}{\Mat{A}} \TESFp*{a}{i}{j}{\Mat{A}}|
%   \hspace{1.2cm}
%   \fbox{$\TESFp {a}{i}{j}{\Mat{A}}$}
%   \fbox{$\TESFp*{a}{i}{j}{\Mat{A}}$}
% \end{center}
% \begin{center}
%   |\TESFP{a}{i}{j}{\Mat{A}} \TESFP*{a}{i}{j}{\Mat{A}}|
%   \hspace{1.2cm}
%   \fbox{$\TESFP {a}{i}{j}{\Mat{A}}$}
%   \fbox{$\TESFP*{a}{i}{j}{\Mat{A}}$}
% \end{center}
% \begin{center}
%   |\TESFpE{a}{i}{j}{\Mat{A}} \TESFpE*{a}{i}{j}{\Mat{A}}|
%   \hspace{1.2cm}
%   \fbox{$\TESFpE {a}{i}{j}{\Mat{A}}$}
%   \fbox{$\TESFpE*{a}{i}{j}{\Mat{A}}$}
% \end{center}
% \begin{center}
%   |\TESFPE{a}{i}{j}{\Mat{A}} \TESFPE*{a}{i}{j}{\Mat{A}}|
%   \hspace{1.2cm}
%   \fbox{$\TESFPE {a}{i}{j}{\Mat{A}}$}
%   \fbox{$\TESFPE*{a}{i}{j}{\Mat{A}}$}
% \end{center}
% 
% \DescribeMacro{\TESC}
% \DescribeMacro{\TESCp}
% \DescribeMacro{\TESCp*}
% \DescribeMacro{\TESCP}
% \DescribeMacro{\TESCP*}
% \DescribeMacro{\TESCpE}
% \DescribeMacro{\TESCpE*}
% \DescribeMacro{\TESCPE}
% \DescribeMacro{\TESCPE*}
% El comando \cs{TESC} tiene 4
% argumentos,\;\cs{TESC}\marg{escalar}\marg{índice}\marg{índice}\marg{objeto},\;
% e indica una transformación elemental de Tipo I por la derecha del objeto.
% \begin{center}
%   |\TESC{a}{i}{j}{\Mat{A}}|
%   \hspace{1.2cm}
%   \fbox{$\TESC{a}{i}{j}{\Mat{A}}$}
% \end{center}
% \begin{center}
%   |\TESCp{a}{i}{j}{\Mat{A}} \TESCp*{a}{i}{j}{\Mat{A}}|
%   \hspace{1.2cm}
%   \fbox{$\TESCp {a}{i}{j}{\Mat{A}}$}
%   \fbox{$\TESCp*{a}{i}{j}{\Mat{A}}$}
% \end{center}
% \begin{center}
%   |\TESCP{a}{i}{j}{\Mat{A}} \TESCP*{a}{i}{j}{\Mat{A}}|
%   \hspace{1.2cm}
%   \fbox{$\TESCP {a}{i}{j}{\Mat{A}}$}
%   \fbox{$\TESCP*{a}{i}{j}{\Mat{A}}$}
% \end{center}
% \begin{center}
%   |\TESCpE{a}{i}{j}{\Mat{A}} \TESCpE*{a}{i}{j}{\Mat{A}}|
%   \hspace{1.2cm}
%   \fbox{$\TESCpE {a}{i}{j}{\Mat{A}}$}
%   \fbox{$\TESCpE*{a}{i}{j}{\Mat{A}}$}
% \end{center}
% \begin{center}
%   |\TESCPE{a}{i}{j}{\Mat{A}} \TESCPE*{a}{i}{j}{\Mat{A}}|
%   \hspace{1.2cm}
%   \fbox{$\TESCPE {a}{i}{j}{\Mat{A}}$}
%   \fbox{$\TESCPE*{a}{i}{j}{\Mat{A}}$}
% \end{center}
% 
% %%%%%%%%%%%%%%%%%%%%%%%%%%%%%%%%%%%%%%%%%%%%%%%%%%%%%%%%
% 
% \DescribeMacro{\TEPF}
% \DescribeMacro{\TEPFp}
% \DescribeMacro{\TEPFp*}
% \DescribeMacro{\TEPFP}
% \DescribeMacro{\TEPFP*}
% \DescribeMacro{\TEPFpE}
% \DescribeMacro{\TEPFpE*}
% \DescribeMacro{\TEPFPE}
% \DescribeMacro{\TEPFPE*}
% El comando \cs{TEPF} tiene 3
% argumentos,\;\cs{TEPF}\marg{escalar}\marg{índice}\marg{objeto},\;
% e indica una transformación elemental de Tipo II por la izquierda del objeto.
% \begin{center}
%   |\TEPF{a}{i}{\Mat{A}}|
%   \hspace{1.2cm}
%   \fbox{$\TEPF{a}{i}{\Mat{A}}$}
% \end{center}
% \begin{center}
%   |\TEPFp{a}{i}{\Mat{A}} \TEPFp*{a}{i}{\Mat{A}}|
%   \hspace{1.2cm}
%   \fbox{$\TEPFp {a}{i}{\Mat{A}}$}
%   \fbox{$\TEPFp*{a}{i}{\Mat{A}}$}
% \end{center}
% \begin{center}
%   |\TEPFP{a}{i}{\Mat{A}} \TEPFP*{a}{i}{\Mat{A}}|
%   \hspace{1.2cm}
%   \fbox{$\TEPFP {a}{i}{\Mat{A}}$}
%   \fbox{$\TEPFP*{a}{i}{\Mat{A}}$}
% \end{center}
% \begin{center}
%   |\TEPFpE{a}{i}{\Mat{A}} \TEPFpE*{a}{i}{\Mat{A}}|
%   \hspace{1.2cm}
%   \fbox{$\TEPFpE {a}{i}{\Mat{A}}$}
%   \fbox{$\TEPFpE*{a}{i}{\Mat{A}}$}
% \end{center}
% \begin{center}
%   |\TEPFPE{a}{i}{\Mat{A}} \TEPFPE*{a}{i}{\Mat{A}}|
%   \hspace{1.2cm}
%   \fbox{$\TEPFPE {a}{i}{\Mat{A}}$}
%   \fbox{$\TEPFPE*{a}{i}{\Mat{A}}$}
% \end{center}
% 
% \DescribeMacro{\TEPC}
% \DescribeMacro{\TEPCp}
% \DescribeMacro{\TEPCp*}
% \DescribeMacro{\TEPCP}
% \DescribeMacro{\TEPCP*}
% \DescribeMacro{\TEPCpE}
% \DescribeMacro{\TEPCpE*}
% \DescribeMacro{\TEPCPE}
% \DescribeMacro{\TEPCPE*}
% El comando \cs{TEPC} tiene 3
% argumentos,\;\cs{TEPC}\marg{escalar}\marg{índice}\marg{objeto},\;
% e indica una transformación elemental de Tipo II por la derecha del objeto.
% \begin{center}
%   |\TEPC{a}{j}{\Mat{A}}|
%   \hspace{1.2cm}
%   \fbox{$\TEPC{a}{j}{\Mat{A}}$}
% \end{center}
% \begin{center}
%   |\TEPCp{a}{j}{\Mat{A}} \TEPCp*{a}{j}{\Mat{A}}|
%   \hspace{1.2cm}
%   \fbox{$\TEPCp {a}{j}{\Mat{A}}$}
%   \fbox{$\TEPCp*{a}{j}{\Mat{A}}$}
% \end{center}
% \begin{center}
%   |\TEPCP{a}{j}{\Mat{A}} \TEPCP*{a}{j}{\Mat{A}}|
%   \hspace{1.2cm}
%   \fbox{$\TEPCP {a}{j}{\Mat{A}}$}
%   \fbox{$\TEPCP*{a}{j}{\Mat{A}}$}
% \end{center}
% \begin{center}
%   |\TEPCpE{a}{j}{\Mat{A}} \TEPCpE*{a}{j}{\Mat{A}}|
%   \hspace{1.2cm}
%   \fbox{$\TEPCpE {a}{j}{\Mat{A}}$}
%   \fbox{$\TEPCpE*{a}{j}{\Mat{A}}$}
% \end{center}
% \begin{center}
%   |\TEPCPE{a}{j}{\Mat{A}} \TEPCPE*{a}{j}{\Mat{A}}|
%   \hspace{1.2cm}
%   \fbox{$\TEPCPE {a}{j}{\Mat{A}}$}
%   \fbox{$\TEPCPE*{a}{j}{\Mat{A}}$}
% \end{center}
% 
% %%%%%%%%%%%%%%%%%%%%%%%%%%%%%%%%%%%%%%%%%%%%%%%%%%%%%%%%
% 
% \DescribeMacro{\TEIF}
% \DescribeMacro{\TEIFp}
% \DescribeMacro{\TEIFp*}
% \DescribeMacro{\TEIFP}
% \DescribeMacro{\TEIFP*}
% \DescribeMacro{\TEIFpE}
% \DescribeMacro{\TEIFpE*}
% \DescribeMacro{\TEIFPE}
% \DescribeMacro{\TEIFPE*}
% El comando \cs{TEIF} tiene 3
% argumentos,\;\cs{TEIF}\marg{índice}\marg{índice}\marg{objeto},\;
% e indica un intercambio por la izquierda del objeto.
% \begin{center}
%   |\TEIF{k}{j}{\Mat{A}}|
%   \hspace{1.2cm}
%   \fbox{$\TEIF{k}{j}{\Mat{A}}$}
% \end{center}
% \begin{center}
%   |\TEIFp{k}{j}{\Mat{A}} \TEIFp*{k}{j}{\Mat{A}}|
%   \hspace{1.2cm}
%   \fbox{$\TEIFp {k}{j}{\Mat{A}}$}
%   \fbox{$\TEIFp*{k}{j}{\Mat{A}}$}
% \end{center}
% \begin{center}
%   |\TEIFP{k}{j}{\Mat{A}} \TEIFP*{k}{j}{\Mat{A}}|
%   \hspace{1.2cm}
%   \fbox{$\TEIFP {k}{j}{\Mat{A}}$}
%   \fbox{$\TEIFP*{k}{j}{\Mat{A}}$}
% \end{center}
% \begin{center}
%   |\TEIFpE{k}{j}{\Mat{A}} \TEIFpE*{k}{j}{\Mat{A}}|
%   \hspace{1.2cm}
%   \fbox{$\TEIFpE {k}{j}{\Mat{A}}$}
%   \fbox{$\TEIFpE*{k}{j}{\Mat{A}}$}
% \end{center}
% \begin{center}
%   |\TEIFPE{k}{j}{\Mat{A}} \TEIFPE*{k}{j}{\Mat{A}}|
%   \hspace{1.2cm}
%   \fbox{$\TEIFPE {k}{j}{\Mat{A}}$}
%   \fbox{$\TEIFPE*{k}{j}{\Mat{A}}$}
% \end{center}
% 
% \DescribeMacro{\TEIC}
% \DescribeMacro{\TEICp}
% \DescribeMacro{\TEICp*}
% \DescribeMacro{\TEICP}
% \DescribeMacro{\TEICP*}
% \DescribeMacro{\TEICpE}
% \DescribeMacro{\TEICpE*}
% \DescribeMacro{\TEICPE}
% \DescribeMacro{\TEICPE*}
% El comando \cs{TEIC} tiene 3
% argumentos,\;\cs{TEIC}\marg{índice}\marg{índice}\marg{objeto},\;
% e indica un intercambio por la derecha del objeto.
% \begin{center}
%   |\TEIC{k}{j}{\Mat{A}}|
%   \hspace{1.2cm}
%   \fbox{$\TEIC{k}{j}{\Mat{A}}$}
% \end{center}
% \begin{center}
%   |\TEICp{k}{j}{\Mat{A}} \TEICp*{k}{j}{\Mat{A}}|
%   \hspace{1.2cm}
%   \fbox{$\TEICp {k}{j}{\Mat{A}}$}
%   \fbox{$\TEICp*{k}{j}{\Mat{A}}$}
% \end{center}
% \begin{center}
%   |\TEICP{k}{j}{\Mat{A}} \TEICP*{k}{j}{\Mat{A}}|
%   \hspace{1.2cm}
%   \fbox{$\TEICP {k}{j}{\Mat{A}}$}
%   \fbox{$\TEICP*{k}{j}{\Mat{A}}$}
% \end{center}
% \begin{center}
%   |\TEICpE{k}{j}{\Mat{A}} \TEICpE*{k}{j}{\Mat{A}}|
%   \hspace{1.2cm}
%   \fbox{$\TEICpE {k}{j}{\Mat{A}}$}
%   \fbox{$\TEICpE*{k}{j}{\Mat{A}}$}
% \end{center}
% \begin{center}
%   |\TEICPE{k}{j}{\Mat{A}} \TEICPE*{k}{j}{\Mat{A}}|
%   \hspace{1.2cm}
%   \fbox{$\TEICPE {k}{j}{\Mat{A}}$}
%   \fbox{$\TEICPE*{k}{j}{\Mat{A}}$}
% \end{center}
% 
% %%%%%%%%%%%%%%%%%%%%%%%%%%%%%%%%%%%%%%%%%%%%%%%%%%%%%%%%
% 
% \DescribeMacro{\Mint}
% El comando \cs{Mint} tiene 2
% argumentos,\;\cs{Mint}\marg{índice}\marg{índice},\;
% e indica una matriz intercambio.
% \begin{center}
%   |\Mint{i}{j}|
%   \hspace{1.2cm}
%   \fbox{$\Mint{i}{j}$}
% \end{center}
% 
% \DescribeMacro{\MintT}
% El comando \cs{MintT} tiene 2
% argumentos,\;\cs{MintT}\marg{índice}\marg{índice},\;
% e indica una matriz intercambio (filas).
% \begin{center}
%   |\MintT{i}{j}|
%   \hspace{1.2cm}
%   \fbox{$\MintT{i}{j}$}
% \end{center}
% 
% \DescribeMacro{\PF}
% El comando \cs{PF} tiene 1
% argumento,\;\cs{PF}\marg{objeto},\;
% e indica una permutación de los elementos de un objeto por la izquierda.
% \begin{center}
%   |\PF{\Mat{A}}|
%   \hspace{1.2cm}
%   \fbox{$\PF{\Mat{A}}$}
% \end{center}
% 
% \DescribeMacro{\PC}
% El comando \cs{PC} tiene 1
% argumento,\;\cs{PC}\marg{objeto},\;
% e indica una permutación de los elementos de un objeto por la derecha.
% \begin{center}
%   |\PC{\Mat{A}}|
%   \hspace{1.2cm}
%   \fbox{$\PC{\Mat{A}}$}
% \end{center}
% 
% \DescribeMacro{\MP}
% El comando \cs{MP} no tiene argumentos e indica
% una matriz permutación.
% \begin{center}
%   |\MP|
%   \hspace{1.2cm}
%   \fbox{$\MP$}
% \end{center}
% 
% \DescribeMacro{\MPT}
% El comando \cs{MPT} no tiene argumentos e indica
% una matriz permutación.
% \begin{center}
%   |\MPT|
%   \hspace{1.2cm}
%   \fbox{$\MPT$}
% \end{center}
% 
% %%%%%%%%%%%%%%%%%%%%%%%%%%%%%%%%%%%%%%%%%%%%%%%%%%%%%%%%
% \paragraph{Sucesiones indiciadas de Transf. elementales} por la
% izquierda, la derecha, o por ambos lados.
% \bigskip
% 
% \DescribeMacro{\SITEF}
% \DescribeMacro{\SITEFp}
% \DescribeMacro{\SITEFp*}
% \DescribeMacro{\SITEFP}
% \DescribeMacro{\SITEFP*}
% \DescribeMacro{\SITEFpE}
% \DescribeMacro{\SITEFpE*}
% \DescribeMacro{\SITEFPE}
% \DescribeMacro{\SITEFPE*}
% El comando \cs{SITEF} tiene 3
% argumentos,\;\cs{SITEF}\marg{indInic}\marg{indFinal}\marg{objeto},\; e indica una
% sucesión de transformaciones elementales genéricas por la izquierda del \marg{objeto}.
% \begin{center}
%   |\SITEF{j}{k}{\Mat{A}}|
%   \hspace{1.2cm}
%   \fbox{$\SITEF{j}{k}{\Mat{A}}$}
% \end{center}
% 
% \begin{center}
%   |\SITEFp{j}{k}{\Mat{A}}|
%   \hspace{1.2cm}
%   \fbox{$\SITEFp {j}{k}{\Mat{A}}$}
% \end{center}
% \begin{center}
%   |\SITEFp*{j}{k}{\Mat{A}}|
%   \hspace{1.2cm}
%   \fbox{$\SITEFp*{j}{k}{\Mat{A}}$}
% \end{center}
% 
% \begin{center}
%   |\SITEFP{j}{k}{\Mat{A}}|
%   \hspace{1.2cm}
%   \fbox{$\SITEFP {j}{k}{\Mat{A}}$}
% \end{center}
% \begin{center}
%   |\SITEFP*{j}{k}{\Mat{A}}|
%   \hspace{1.2cm}
%   \fbox{$\SITEFP*{j}{k}{\Mat{A}}$}
% \end{center}
% 
% \begin{center}
%   |\SITEFpE{j}{k}{\Mat{A}}|
%   \hspace{1.2cm}
%   \fbox{$\SITEFpE {j}{k}{\Mat{A}}$}
% \end{center}
% \begin{center}
%   |\SITEFpE*{j}{k}{\Mat{A}}|
%   \hspace{1.2cm}
%   \fbox{$\SITEFpE*{j}{k}{\Mat{A}}$}
% \end{center}
% 
% \begin{center}
%   |\SITEFPE{j}{k}{\Mat{A}}|
%   \hspace{1.2cm}
%   \fbox{$\SITEFPE {j}{k}{\Mat{A}}$}
% \end{center}
% \begin{center}
%   |\SITEFPE*{j}{k}{\Mat{A}}|
%   \hspace{1.2cm}
%   \fbox{$\SITEFPE*{j}{k}{\Mat{A}}$}
% \end{center}
% 
% 
% \DescribeMacro{\SITEC}
% \DescribeMacro{\SITECp}
% \DescribeMacro{\SITECp*}
% \DescribeMacro{\SITECP}
% \DescribeMacro{\SITECP*}
% \DescribeMacro{\SITECpE}
% \DescribeMacro{\SITECpE*}
% \DescribeMacro{\SITECPE}
% \DescribeMacro{\SITECPE*}
% El comando \cs{SITEC} tiene 3
% argumentos,\;\cs{SITEC}\marg{indInic}\marg{indFinal}\marg{objeto},\; e indica una
% sucesión de transformaciones elementales genéricas por la derecha del \marg{objeto}.
% \begin{center}
%   |\SITEC{j}{k}{\Mat{A}}|
%   \hspace{1.2cm}
%   \fbox{$\SITEC{j}{k}{\Mat{A}}$}
% \end{center}
% 
% \begin{center}
%   |\SITECp{j}{k}{\Mat{A}}|
%   \hspace{1.2cm}
%   \fbox{$\SITECp {j}{k}{\Mat{A}}$}
% \end{center}
% \begin{center}
%   |\SITECp*{j}{k}{\Mat{A}}|
%   \hspace{1.2cm}
%   \fbox{$\SITECp*{j}{k}{\Mat{A}}$}
% \end{center}
% 
% \begin{center}
%   |\SITECP{j}{k}{\Mat{A}}|
%   \hspace{1.2cm}
%   \fbox{$\SITECP {j}{k}{\Mat{A}}$}
% \end{center}
% \begin{center}
%   |\SITECP*{j}{k}{\Mat{A}}|
%   \hspace{1.2cm}
%   \fbox{$\SITECP*{j}{k}{\Mat{A}}$}
% \end{center}
% 
% \begin{center}
%   |\SITECpE{j}{k}{\Mat{A}}|
%   \hspace{1.2cm}
%   \fbox{$\SITECpE {j}{k}{\Mat{A}}$}
% \end{center}
% \begin{center}
%   |\SITECpE*{j}{k}{\Mat{A}}|
%   \hspace{1.2cm}
%   \fbox{$\SITECpE*{j}{k}{\Mat{A}}$}
% \end{center}
% 
% \begin{center}
%   |\SITECPE{j}{k}{\Mat{A}}|
%   \hspace{1.2cm}
%   \fbox{$\SITECPE {j}{k}{\Mat{A}}$}
% \end{center}
% \begin{center}
%   |\SITECPE*{j}{k}{\Mat{A}}|
%   \hspace{1.2cm}
%   \fbox{$\SITECPE*{j}{k}{\Mat{A}}$}
% \end{center}
% 
% \DescribeMacro{\SITEFC}
% \DescribeMacro{\SITEFCp}
% \DescribeMacro{\SITEFCp*}
% \DescribeMacro{\SITEFCP}
% \DescribeMacro{\SITEFCP*}
% \DescribeMacro{\SITEFCpE}
% \DescribeMacro{\SITEFCpE*}
% \DescribeMacro{\SITEFCPE}
% \DescribeMacro{\SITEFCPE*}
% El comando \cs{SITEFC} tiene 3
% argumentos,\;\cs{SITEFC}\marg{indInic}\marg{indFinal}\marg{objeto},\; e
% indica una sucesión de transformaciones elementales genéricas por la derecha y
% la izquierda del \marg{objeto} (fíjese en el orden de los índices a cada lado).
% \begin{center}
%   |\SITEFC{j}{k}{\Mat{A}}|
%   \hspace{1.2cm}
%   \fbox{$\SITEFC{j}{k}{\Mat{A}}$}
% \end{center}
% 
% \begin{center}
%   |\SITEFCp{j}{k}{\Mat{A}}|
%   \hspace{1.2cm}
%   \fbox{$\SITEFCp {j}{k}{\Mat{A}}$}
% \end{center}
% \begin{center}
%   |\SITEFCp*{j}{k}{\Mat{A}}|
%   \hspace{1.2cm}
%   \fbox{$\SITEFCp*{j}{k}{\Mat{A}}$}
% \end{center}
% 
% \begin{center}
%   |\SITEFCP{j}{k}{\Mat{A}}|
%   \hspace{1.2cm}
%   \fbox{$\SITEFCP {j}{k}{\Mat{A}}$}
% \end{center}
% \begin{center}
%   |\SITEFCP*{j}{k}{\Mat{A}}|
%   \hspace{1.2cm}
%   \fbox{$\SITEFCP*{j}{k}{\Mat{A}}$}
% \end{center}
% 
% \begin{center}
%   |\SITEFCpE{j}{k}{\Mat{A}}|
%   \hspace{1.2cm}
%   \fbox{$\SITEFCpE {j}{k}{\Mat{A}}$}
% \end{center}
% \begin{center}
%   |\SITEFCpE*{j}{k}{\Mat{A}}|
%   \hspace{1.2cm}
%   \fbox{$\SITEFCpE*{j}{k}{\Mat{A}}$}
% \end{center}
% 
% \begin{center}
%   |\SITEFCPE{j}{k}{\Mat{A}}|
%   \hspace{1.2cm}
%   \fbox{$\SITEFCPE {j}{k}{\Mat{A}}$}
% \end{center}
% \begin{center}
%   |\SITEFCPE*{j}{k}{\Mat{A}}|
%   \hspace{1.2cm}
%   \fbox{$\SITEFCPE*{j}{k}{\Mat{A}}$}
% \end{center}
% 
% 
% \DescribeMacro{\SITEFCR}
% \DescribeMacro{\SITEFCRp}
% \DescribeMacro{\SITEFCRp*}
% \DescribeMacro{\SITEFCRP}
% \DescribeMacro{\SITEFCRP*}
% \DescribeMacro{\SITEFCRpE}
% \DescribeMacro{\SITEFCRpE*}
% \DescribeMacro{\SITEFCRPE}
% \DescribeMacro{\SITEFCRPE*}
% El comando \cs{SITEFCR} tiene 3
% argumentos,\;\cs{SITEFCR}\marg{indInic}\marg{indFinal}\marg{objeto},\; e
% indica una sucesión de transformaciones elementales genéricas por la derecha y
% la izquierda del \marg{objeto} (fíjese en el orden de los índices a cada lado).
% \begin{center}
%   |\SITEFCR{j}{k}{\Mat{A}}|
%   \hspace{1.2cm}
%   \fbox{$\SITEFCR{j}{k}{\Mat{A}}$}
% \end{center}
% 
% \begin{center}
%   |\SITEFCRp{j}{k}{\Mat{A}}|
%   \hspace{1.2cm}
%   \fbox{$\SITEFCRp {j}{k}{\Mat{A}}$}
% \end{center}
% \begin{center}
%   |\SITEFCRp*{j}{k}{\Mat{A}}|
%   \hspace{1.2cm}
%   \fbox{$\SITEFCRp*{j}{k}{\Mat{A}}$}
% \end{center}
% 
% \begin{center}
%   |\SITEFCRP{j}{k}{\Mat{A}}|
%   \hspace{1.2cm}
%   \fbox{$\SITEFCRP {j}{k}{\Mat{A}}$}
% \end{center}
% \begin{center}
%   |\SITEFCRP*{j}{k}{\Mat{A}}|
%   \hspace{1.2cm}
%   \fbox{$\SITEFCRP*{j}{k}{\Mat{A}}$}
% \end{center}
% 
% \begin{center}
%   |\SITEFCRpE{j}{k}{\Mat{A}}|
%   \hspace{1.2cm}
%   \fbox{$\SITEFCRpE {j}{k}{\Mat{A}}$}
% \end{center}
% \begin{center}
%   |\SITEFCRpE*{j}{k}{\Mat{A}}|
%   \hspace{1.2cm}
%   \fbox{$\SITEFCRpE*{j}{k}{\Mat{A}}$}
% \end{center}
% 
% \begin{center}
%   |\SITEFCRPE{j}{k}{\Mat{A}}|
%   \hspace{1.2cm}
%   \fbox{$\SITEFCRPE {j}{k}{\Mat{A}}$}
% \end{center}
% \begin{center}
%   |\SITEFCRPE*{j}{k}{\Mat{A}}|
%   \hspace{1.2cm}
%   \fbox{$\SITEFCRPE*{j}{k}{\Mat{A}}$}
% \end{center}
% 
% %%%%%%%%%%%%%%%%%%%%%%%%%%%%%%%%%%%%%%
% 
% \DescribeMacro{\TrF}
% \DescribeMacro{\TrFp}
% \DescribeMacro{\TrFp*}
% \DescribeMacro{\TrFP}
% \DescribeMacro{\TrFP*}
% \DescribeMacro{\TrFpE}
% \DescribeMacro{\TrFpE*}
% \DescribeMacro{\TrFPE}
% \DescribeMacro{\TrFPE*}
% El comando \cs{TrF} tiene 2
% argumentos,\;\cs{TrF}\oarg{trasformaciones}\marg{objeto},\; e indica
% la aplicación de transformaciones elementales por la izquierda del
% \marg{objeto}.
% \begin{center}
%   |\TrF{\Mat{A}} \TrF[\OEpr{-7}{j}]{\Mat{A}}|
%   \hspace{1.2cm}
%   \fbox{$\TrF         {\Mat{A}}$}
%   \fbox{$\TrF[\OEpr{-7}{j}]{\Mat{A}}$}
% \end{center}
% 
% \begin{center}
%   |\TrFp{\Mat{A}} \TrFp[\OEpr{-7}{j}]{\Mat{A}}|
%   \hspace{1.2cm}
%   \fbox{$\TrFp         {\Mat{A}}$}
%   \fbox{$\TrFp[\OEpr{-7}{j}]{\Mat{A}}$}
% \end{center}
% \begin{center}
%   |\TrFp*{\Mat{A}} \TrFp*[\OEpr{-7}{j}]{\Mat{A}}|
%   \hspace{1.2cm}
%   \fbox{$\TrFp*         {\Mat{A}}$}
%   \fbox{$\TrFp*[\OEpr{-7}{j}]{\Mat{A}}$}
% \end{center}
% \begin{center}
%   |\TrFP{\Mat{A}} \TrFP[\OEpr{-7}{j}]{\Mat{A}}|
%   \hspace{1.2cm}
%   \fbox{$\TrFP         {\Mat{A}}$}
%   \fbox{$\TrFP[\OEpr{-7}{j}]{\Mat{A}}$}
% \end{center}
% \begin{center}
%   |\TrFP*{\Mat{A}} \TrFP*[\OEpr{-7}{j}]{\Mat{A}}|
%   \hspace{1.2cm}
%   \fbox{$\TrFP*         {\Mat{A}}$}
%   \fbox{$\TrFP*[\OEpr{-7}{j}]{\Mat{A}}$}
% \end{center}
% 
% \begin{center}
%   |\TrFpE{\Mat{A}} \TrFp[\OEpr{-7}{j}]{\Mat{A}}|
%   \hspace{1.2cm}
%   \fbox{$\TrFpE         {\Mat{A}}$}
%   \fbox{$\TrFpE[\OEpr{-7}{j}]{\Mat{A}}$}
% \end{center}
% \begin{center}
%   |\TrFpE*{\Mat{A}} \TrFpE*[\OEpr{-7}{j}]{\Mat{A}}|
%   \hspace{1.2cm}
%   \fbox{$\TrFpE*         {\Mat{A}}$}
%   \fbox{$\TrFpE*[\OEpr{-7}{j}]{\Mat{A}}$}
% \end{center}
% \begin{center}
%   |\TrFPE{\Mat{A}} \TrFPE[\OEpr{-7}{j}]{\Mat{A}}|
%   \hspace{1.2cm}
%   \fbox{$\TrFPE         {\Mat{A}}$}
%   \fbox{$\TrFPE[\OEpr{-7}{j}]{\Mat{A}}$}
% \end{center}
% \begin{center}
%   |\TrFPE*{\Mat{A}} \TrFPE*[\OEpr{-7}{j}]{\Mat{A}}|
%   \hspace{1.2cm}
%   \fbox{$\TrFPE*         {\Mat{A}}$}
%   \fbox{$\TrFPE*[\OEpr{-7}{j}]{\Mat{A}}$}
% \end{center}
% 
% \DescribeMacro{\TrC}
% \DescribeMacro{\TrCp}
% \DescribeMacro{\TrCp*}
% \DescribeMacro{\TrCP}
% \DescribeMacro{\TrCP*}
% \DescribeMacro{\TrCpE}
% \DescribeMacro{\TrCpE*}
% \DescribeMacro{\TrCPE}
% \DescribeMacro{\TrCPE*}
% El comando \cs{TrC} tiene 2
% argumentos,\;\cs{TrC}\oarg{trasformaciones}\marg{objeto},\; e indica
% la aplicación de transformaciones elementales por la derecha del
% \marg{objeto}.
% \begin{center}
%   |\TrC{\SV{A}} \TrC[\OEpr{-7}{j}]{\SV{A}}|
%   \hspace{1.2cm}
%   \fbox{$\TrC         {\SV{A}}$}
%   \fbox{$\TrC[\OEpr{-7}{j}]{\SV{A}}$}
% \end{center}
% 
% \begin{center}
%   |\TrCp{\SV{A}} \TrCp[\OEpr{-7}{j}]{\SV{A}}|
%   \hspace{1.2cm}
%   \fbox{$\TrCp         {\SV{A}}$}
%   \fbox{$\TrCp[\OEpr{-7}{j}]{\SV{A}}$}
% \end{center}
% \begin{center}
%   |\TrCp*{\SV{A}} \TrCp*[\OEpr{-7}{j}]{\SV{A}}|
%   \hspace{1.2cm}
%   \fbox{$\TrCp*         {\SV{A}}$}
%   \fbox{$\TrCp*[\OEpr{-7}{j}]{\SV{A}}$}
% \end{center}
% \begin{center}
%   |\TrCP{\SV{A}} \TrCP[\OEpr{-7}{j}]{\SV{A}}|
%   \hspace{1.2cm}
%   \fbox{$\TrCP         {\SV{A}}$}
%   \fbox{$\TrCP[\OEpr{-7}{j}]{\SV{A}}$}
% \end{center}
% \begin{center}
%   |\TrCP*{\SV{A}} \TrCP*[\OEpr{-7}{j}]{\SV{A}}|
%   \hspace{1.2cm}
%   \fbox{$\TrCP*         {\SV{A}}$}
%   \fbox{$\TrCP*[\OEpr{-7}{j}]{\SV{A}}$}
% \end{center}
% 
% \begin{center}
%   |\TrCpE{\SV{A}} \TrCp[\OEpr{-7}{j}]{\SV{A}}|
%   \hspace{1.2cm}
%   \fbox{$\TrCpE         {\SV{A}}$}
%   \fbox{$\TrCpE[\OEpr{-7}{j}]{\SV{A}}$}
% \end{center}
% \begin{center}
%   |\TrCpE*{\SV{A}} \TrCpE*[\OEpr{-7}{j}]{\SV{A}}|
%   \hspace{1.2cm}
%   \fbox{$\TrCpE*         {\SV{A}}$}
%   \fbox{$\TrCpE*[\OEpr{-7}{j}]{\SV{A}}$}
% \end{center}
% \begin{center}
%   |\TrCPE{\SV{A}} \TrCPE[\OEpr{-7}{j}]{\SV{A}}|
%   \hspace{1.2cm}
%   \fbox{$\TrCPE         {\SV{A}}$}
%   \fbox{$\TrCPE[\OEpr{-7}{j}]{\SV{A}}$}
% \end{center}
% \begin{center}
%   |\TrCPE*{\SV{A}} \TrCPE*[\OEpr{-7}{j}]{\SV{A}}|
%   \hspace{1.2cm}
%   \fbox{$\TrCPE*         {\SV{A}}$}
%   \fbox{$\TrCPE*[\OEpr{-7}{j}]{\SV{A}}$}
% \end{center}
% 
% 
% \DescribeMacro{\TrFC}
% \DescribeMacro{\TrFCp}
% \DescribeMacro{\TrFCp*}
% \DescribeMacro{\TrFCP}
% \DescribeMacro{\TrFCP*}
% \DescribeMacro{\TrFCpE}
% \DescribeMacro{\TrFCpE*}
% \DescribeMacro{\TrFCPE}
% \DescribeMacro{\TrFCPE*}
% El comando \cs{TrFCC} tiene 3
% argumentos,\;\cs{TrFCC}\oarg{trasformacionesIzda}\oarg{trasformacionesDcha}\marg{objeto},\; e indica
% la aplicación de transformaciones elementales por la izquierda y la derecha del
% \marg{objeto}.
% \begin{center}
%   |\TrFC{\Mat{A}} \TrFC[\OEg[3]][\OEper]{\Mat{A}}|
%   \hspace{1.2cm}
%   \fbox{$\TrFC                 {\Mat{A}}$}
%   \fbox{$\TrFC[\OEg[3]][\OEper]{\Mat{A}}$}
% \end{center}
% 
% \begin{center}
%   |\TrFCp{\Mat{A}} \TrFCp[\OEg[3]][\OEper]{\Mat{A}}|
%   \hspace{1.2cm}
%   \fbox{$\TrFCp                 {\Mat{A}}$}
%   \fbox{$\TrFCp[\OEg[3]][\OEper]{\Mat{A}}$}
% \end{center}
% 
% \begin{center}
%   |\TrFCp*{\Mat{A}} \TrFCp*[\OEg[3]][\OEper]{\Mat{A}}|
%   \hspace{1.2cm}
%   \fbox{$\TrFCp*                 {\Mat{A}}$}
%   \fbox{$\TrFCp*[\OEg[3]][\OEper]{\Mat{A}}$}
% \end{center}
% 
% \begin{center}
%   |\TrFCP{\Mat{A}} \TrFCP[\OEg[3]][\OEper]{\Mat{A}}|
%   \hspace{1.2cm}
%   \fbox{$\TrFCP                 {\Mat{A}}$}
%   \fbox{$\TrFCP[\OEg[3]][\OEper]{\Mat{A}}$}
% \end{center}
% 
% \begin{center}
%   |\TrFCP*{\Mat{A}} \TrFCP*[\OEg[3]][\OEper]{\Mat{A}}|
%   \hspace{1.2cm}
%   \fbox{$\TrFCP*                 {\Mat{A}}$}
%   \fbox{$\TrFCP*[\OEg[3]][\OEper]{\Mat{A}}$}
% \end{center}
% 
% \begin{center}
%   |\TrFCpE{\SV{A}} \TrFCpE[\OEg[3]][\OEper]{\SV{A}}|
%   \hspace{1.2cm}
%   \fbox{$\TrFCpE                 {\SV{A}}$}
%   \fbox{$\TrFCpE[\OEg[3]][\OEper]{\SV{A}}$}
% \end{center}
% 
% \begin{center}
%   |\TrFCpE*{\SV{A}} \TrFCpE*[\OEg[3]][\OEper]{\SV{A}}|
%   \hspace{1.2cm}
%   \fbox{$\TrFCpE*                 {\SV{A}}$}
%   \fbox{$\TrFCpE*[\OEg[3]][\OEper]{\SV{A}}$}
% \end{center}
% 
% \begin{center}
%   |\TrFCPE{\SV{A}} \TrFCPE[\OEg[3]][\OEper]{\SV{A}}|
%   \hspace{1.2cm}
%   \fbox{$\TrFCPE                 {\SV{A}}$}
%   \fbox{$\TrFCPE[\OEg[3]][\OEper]{\SV{A}}$}
% \end{center}
% 
% \begin{center}
%   |\TrFCPE*{\SV{A}} \TrFCPE*[\OEg][\OEper]{\SV{A}}|
%   \hspace{1.2cm}
%   \fbox{$\TrFCPE*              {\SV{A}}$}
%   \fbox{$\TrFCPE*[\OEg][\OEper]{\SV{A}}$}
% \end{center}
% 
% %%%%%%%%%%%%%%%%%%%%%%%%%%%%%%%%%%%%%
% 
% \paragraph{Transf. elemental genérica aplicada a la izquierda de un objeto.} Cuando la aplicamos a la izquierda de una matriz corresponde a una transformación de sus filas
% 
% \DescribeMacro{\TEF}
% \DescribeMacro{\TEFp}
% \DescribeMacro{\TEFp*}
% \DescribeMacro{\TEFP}
% \DescribeMacro{\TEFP*}
% \DescribeMacro{\TEFpE}
% \DescribeMacro{\TEFpE*}
% \DescribeMacro{\TEFPE}
% \DescribeMacro{\TEFPE*}
% El comando |TEF<XX*>| tiene 3
% argumentos,\;|TEF<XX*>|\oarg{índice}\oarg{exponente}\marg{objeto},\; e
% indica una transformación elemental genérica por la izquierda del
% objeto.
% \begin{center}
%   |\TEF{\Mat{A}} \TEF[k]{\Mat{A}} \TEF[k][']{\Mat{A}}|
%   \hspace{1.2cm}
%   \fbox{$\TEF{\Mat{A}}$} \fbox{$\TEF[k]{\Mat{A}}$} \fbox{$\TEF[k][']{\Mat{A}}$}
% \end{center}
% 
% \begin{center}
%   |\TEFp{\Mat{A}} \TEFp[k]{\Mat{A}} \TEFp[k][']{\Mat{A}}|
%   \hspace{1.2cm}
%   \fbox{$\TEFp{\Mat{A}}$} \fbox{$\TEFp[k]{\Mat{A}}$} \fbox{$\TEFp[k][']{\Mat{A}}$}
% \end{center}
% \begin{center}
%   |\TEFp*{\Mat{A}} \TEFp*[k]{\Mat{A}} \TEFp*[k][']{\Mat{A}}|
%   \hspace{1.2cm}
%   \fbox{$\TEFp*{\Mat{A}}$} \fbox{$\TEFp*[k]{\Mat{A}}$} \fbox{$\TEFp*[k][']{\Mat{A}}$}
% \end{center}
% 
% \begin{center}
%   |\TEFP{\Mat{A}} \TEFP[k]{\Mat{A}} \TEFP[k][']{\Mat{A}}|
%   \hspace{1.2cm}
%   \fbox{$\TEFP{\Mat{A}}$} \fbox{$\TEFP[k]{\Mat{A}}$} \fbox{$\TEFP[k][']{\Mat{A}}$}
% \end{center}
% \begin{center}
%   |\TEFP*{\Mat{A}} \TEFP*[k]{\Mat{A}} \TEFP*[k][']{\Mat{A}}|
%   \hspace{1.2cm}
%   \fbox{$\TEFP*{\Mat{A}}$} \fbox{$\TEFP*[k]{\Mat{A}}$} \fbox{$\TEFP*[k][']{\Mat{A}}$}
% \end{center}
% 
% \begin{center}
%   |\TEFpE{\Mat{A}} \TEFpE[k][']{\Mat{A}}|
%   \hspace{1.2cm}
%   \fbox{$\TEFpE{\Mat{A}}$} \fbox{$\TEFpE[k][']{\Mat{A}}$}
% \end{center}
% \begin{center}
%   |\TEFpE*{\Mat{A}} \TEFpE*[k][']{\Mat{A}}|
%   \hspace{1.2cm}
%   \fbox{$\TEFpE*{\Mat{A}}$} \fbox{$\TEFpE*[k][']{\Mat{A}}$}
% \end{center}
% 
% \begin{center}
%   |\TEFPE{\Mat{A}} \TEFPE[k][']{\Mat{A}}|
%   \hspace{1.2cm}
%   \fbox{$\TEFPE{\Mat{A}}$} \fbox{$\TEFPE[k][']{\Mat{A}}$}
% \end{center}
% \begin{center}
%   |\TEFPE*{\Mat{A}}  \TEFPE*[k][']{\Mat{A}}|
%   \hspace{1.2cm}
%   \fbox{$\TEFPE*{\Mat{A}}$} \fbox{$\TEFPE*[k][']{\Mat{A}}$}
% \end{center}
% 
% \paragraph{Transf. elemental genérica aplicada a la derecha de un objeto.} Cuando la aplicamos a la derechade una matriz corresponde a una transformación de sus columnas
% 
% \DescribeMacro{\TEC}
% \DescribeMacro{\TECp}
% \DescribeMacro{\TECp*}
% \DescribeMacro{\TECP}
% \DescribeMacro{\TECP*}
% \DescribeMacro{\TECpE}
% \DescribeMacro{\TECpE*}
% \DescribeMacro{\TECPE}
% \DescribeMacro{\TECPE*}
% El comando |TEC<XX*>| tiene 3
% argumentos,\;|TEC<XX*>|\oarg{índice}\oarg{exponente}\marg{objeto},\; e
% indica una transformación elemental genérica por la izquierda del
% objeto.
% \begin{center}
%   |\TEC{\Mat{A}} \TEC[k]{\Mat{A}} \TEC[k][']{\Mat{A}}|
%   \hspace{1.2cm}
%   \fbox{$\TEC{\Mat{A}}$} \fbox{$\TEC[k]{\Mat{A}}$} \fbox{$\TEC[k][']{\Mat{A}}$}
% \end{center}
% 
% \begin{center}
%   |\TECp{\Mat{A}} \TECp[k]{\Mat{A}} \TECp[k][']{\Mat{A}}|
%   \hspace{1.2cm}
%   \fbox{$\TECp{\Mat{A}}$} \fbox{$\TECp[k]{\Mat{A}}$} \fbox{$\TECp[k][']{\Mat{A}}$}
% \end{center}
% \begin{center}
%   |\TECp*{\Mat{A}} \TECp*[k]{\Mat{A}} \TECp*[k][']{\Mat{A}}|
%   \hspace{1.2cm}
%   \fbox{$\TECp*{\Mat{A}}$} \fbox{$\TECp*[k]{\Mat{A}}$} \fbox{$\TECp*[k][']{\Mat{A}}$}
% \end{center}
% 
% \begin{center}
%   |\TECP{\Mat{A}} \TECP[k]{\Mat{A}} \TECP[k][']{\Mat{A}}|
%   \hspace{1.2cm}
%   \fbox{$\TECP{\Mat{A}}$} \fbox{$\TECP[k]{\Mat{A}}$} \fbox{$\TECP[k][']{\Mat{A}}$}
% \end{center}
% \begin{center}
%   |\TECP*{\Mat{A}} \TECP*[k]{\Mat{A}} \TECP*[k][']{\Mat{A}}|
%   \hspace{1.2cm}
%   \fbox{$\TECP*{\Mat{A}}$} \fbox{$\TECP*[k]{\Mat{A}}$} \fbox{$\TECP*[k][']{\Mat{A}}$}
% \end{center}
% 
% \begin{center}
%   |\TECpE{\Mat{A}} \TECpE[k][']{\Mat{A}}|
%   \hspace{1.2cm}
%   \fbox{$\TECpE{\Mat{A}}$} \fbox{$\TECpE[k][']{\Mat{A}}$}
% \end{center}
% \begin{center}
%   |\TECpE*{\Mat{A}} \TECpE*[k][']{\Mat{A}}|
%   \hspace{1.2cm}
%   \fbox{$\TECpE*{\Mat{A}}$} \fbox{$\TECpE*[k][']{\Mat{A}}$}
% \end{center}
% 
% \begin{center}
%   |\TECPE{\Mat{A}} \TECPE[k][']{\Mat{A}}|
%   \hspace{1.2cm}
%   \fbox{$\TECPE{\Mat{A}}$} \fbox{$\TECPE[k][']{\Mat{A}}$}
% \end{center}
% \begin{center}
%   |\TECPE*{\Mat{A}}  \TECPE*[k][']{\Mat{A}}|
%   \hspace{1.2cm}
%   \fbox{$\TECPE*{\Mat{A}}$} \fbox{$\TECPE*[k][']{\Mat{A}}$}
% \end{center}
% 
% 
% \paragraph{Espejo de una Transf. elemental genérica aplicada a la izquierda de un objeto.}{\mbox{\ }}
% 
% \DescribeMacro{\ETEF}
% \DescribeMacro{\ETEFp}
% \DescribeMacro{\ETEFp*}
% \DescribeMacro{\ETEFP}
% \DescribeMacro{\ETEFP*}
% \DescribeMacro{\ETEFpE}
% \DescribeMacro{\ETEFpE*}
% \DescribeMacro{\ETEFPE}
% \DescribeMacro{\ETEFPE*}
% El comando |ETEF<XX*>| tiene 3
% argumentos,\;|ETEF<XX*>|\oarg{índice}\oarg{exponente}\marg{objeto},\; e
% indica una transformación elemental genérica \emph{espejo} por la izquierda del
% objeto.
% \begin{center}
%   |\ETEF{\Mat{A}} \ETEF[k]{\Mat{A}} \ETEF[k][']{\Mat{A}}|
%   %\hspace{0.5cm}
%   \fbox{$\ETEF{\Mat{A}}$} \fbox{$\ETEF[k]{\Mat{A}}$} \fbox{$\ETEF[k][']{\Mat{A}}$}
% \end{center}
% 
% \begin{center}
%   |\ETEFp{\Mat{A}} \ETEFp[k][']{\Mat{A}}|
%   %\hspace{1.2cm}
%   \fbox{$\ETEFp{\Mat{A}}$} \fbox{$\ETEFp[k][']{\Mat{A}}$}
% \end{center}
% \begin{center}
%   |\ETEFp*{\Mat{A}} \ETEFp*[k][']{\Mat{A}}|
%   \hspace{1.2cm}
%   \fbox{$\ETEFp*{\Mat{A}}$} \fbox{$\ETEFp*[k][']{\Mat{A}}$}
% \end{center}
% 
% \begin{center}
%   |\ETEFP{\Mat{A}} \ETEFP[k][']{\Mat{A}}|
%   \hspace{1.2cm}
%   \fbox{$\ETEFP{\Mat{A}}$} \fbox{$\ETEFP[k][']{\Mat{A}}$}
% \end{center}
% \begin{center}
%   |\ETEFP*{\Mat{A}} \ETEFP*[k][']{\Mat{A}}|
%   \hspace{1.2cm}
%   \fbox{$\ETEFP*{\Mat{A}}$} \fbox{$\ETEFP*[k][']{\Mat{A}}$}
% \end{center}
% 
% \begin{center}
%   |\ETEFpE{\Mat{A}} \ETEFpE[k][']{\Mat{A}}|
%   \hspace{1.2cm}
%   \fbox{$\ETEFpE{\Mat{A}}$} \fbox{$\ETEFpE[k][']{\Mat{A}}$}
% \end{center}
% \begin{center}
%   |\ETEFpE*{\Mat{A}} \ETEFpE*[k][']{\Mat{A}}|
%   \hspace{1.2cm}
%   \fbox{$\ETEFpE*{\Mat{A}}$} \fbox{$\ETEFpE*[k][']{\Mat{A}}$}
% \end{center}
% 
% \begin{center}
%   |\ETEFPE{\Mat{A}} \ETEFPE[k][']{\Mat{A}}|
%   \hspace{1.2cm}
%   \fbox{$\ETEFPE{\Mat{A}}$} \fbox{$\ETEFPE[k][']{\Mat{A}}$}
% \end{center}
% \begin{center}
%   |\ETEFPE*{\Mat{A}}  \ETEFPE*[k][']{\Mat{A}}|
%   \hspace{1.2cm}
%   \fbox{$\ETEFPE*{\Mat{A}}$} \fbox{$\ETEFPE*[k][']{\Mat{A}}$}
% \end{center}
% 
% \paragraph{Espejo de una Transf. elemental genérica aplicada a la derecha de un objeto.}{\mbox{\ }}
% 
% \DescribeMacro{\ETEC}
% \DescribeMacro{\ETECp}
% \DescribeMacro{\ETECp*}
% \DescribeMacro{\ETECP}
% \DescribeMacro{\ETECP*}
% \DescribeMacro{\ETECpE}
% \DescribeMacro{\ETECpE*}
% \DescribeMacro{\ETECPE}
% \DescribeMacro{\ETECPE*}
% El comando |ETEC<XX*>| tiene 3
% argumentos,\;|ETEC<XX*>|\oarg{índice}\oarg{exponente}\marg{objeto},\; e
% indica una transformación elemental genérica \emph{espejo} por la izquierda del
% objeto.
% \begin{center}
%   |\ETEC{\Mat{A}} \ETEC[k]{\Mat{A}} \ETEC[k][']{\Mat{A}}|
%   %\hspace{0.5cm}
%   \fbox{$\ETEC{\Mat{A}}$} \fbox{$\ETEC[k]{\Mat{A}}$} \fbox{$\ETEC[k][']{\Mat{A}}$}
% \end{center}
% 
% \begin{center}
%   |\ETECp{\Mat{A}} \ETECp[k][']{\Mat{A}}|
%   %\hspace{1.2cm}
%   \fbox{$\ETECp{\Mat{A}}$} \fbox{$\ETECp[k][']{\Mat{A}}$}
% \end{center}
% \begin{center}
%   |\ETECp*{\Mat{A}} \ETECp*[k][']{\Mat{A}}|
%   \hspace{1.2cm}
%   \fbox{$\ETECp*{\Mat{A}}$} \fbox{$\ETECp*[k][']{\Mat{A}}$}
% \end{center}
% 
% \begin{center}
%   |\ETECP{\Mat{A}} \ETECP[k][']{\Mat{A}}|
%   \hspace{1.2cm}
%   \fbox{$\ETECP{\Mat{A}}$} \fbox{$\ETECP[k][']{\Mat{A}}$}
% \end{center}
% \begin{center}
%   |\ETECP*{\Mat{A}} \ETECP*[k][']{\Mat{A}}|
%   \hspace{1.2cm}
%   \fbox{$\ETECP*{\Mat{A}}$} \fbox{$\ETECP*[k][']{\Mat{A}}$}
% \end{center}
% 
% \begin{center}
%   |\ETECpE{\Mat{A}} \ETECpE[k][']{\Mat{A}}|
%   \hspace{1.2cm}
%   \fbox{$\ETECpE{\Mat{A}}$} \fbox{$\ETECpE[k][']{\Mat{A}}$}
% \end{center}
% \begin{center}
%   |\ETECpE*{\Mat{A}} \ETECpE*[k][']{\Mat{A}}|
%   \hspace{1.2cm}
%   \fbox{$\ETECpE*{\Mat{A}}$} \fbox{$\ETECpE*[k][']{\Mat{A}}$}
% \end{center}
% 
% \begin{center}
%   |\ETECPE{\Mat{A}} \ETECPE[k][']{\Mat{A}}|
%   \hspace{1.2cm}
%   \fbox{$\ETECPE{\Mat{A}}$} \fbox{$\ETECPE[k][']{\Mat{A}}$}
% \end{center}
% \begin{center}
%   |\ETECPE*{\Mat{A}}  \ETECPE*[k][']{\Mat{A}}|
%   \hspace{1.2cm}
%   \fbox{$\ETECPE*{\Mat{A}}$} \fbox{$\ETECPE*[k][']{\Mat{A}}$}
% \end{center}
% 
% %%%%%%%%%%%%%%%%%%%%%%%%%%%%%%%%%%%%%%%%%%%%%%%%%%%%%%%%%%%%%%%%%%%%%%%%%%%
% 
% \paragraph{Inversa de una Transf. elemental genérica aplicada a la izquierda de un objeto.}{\mbox{\ }}
% 
% \DescribeMacro{\InvTEF}
% \DescribeMacro{\InvTEFp}
% \DescribeMacro{\InvTEFp*}
% \DescribeMacro{\InvTEFP}
% \DescribeMacro{\InvTEFP*}
% \DescribeMacro{\InvTEFpE}
% \DescribeMacro{\InvTEFpE*}
% \DescribeMacro{\InvTEFPE}
% \DescribeMacro{\InvTEFPE*}
% El comando |InvTEF<XX*>| tiene 2
% argumentos,\;|InvTEF<XX*>|\oarg{índice}\marg{objeto},\; e
% indica la inversa de una transformación elemental genérica por la izquierda del
% objeto.
% \begin{center}
%   |\InvTEF{\Mat{A}} \InvTEF[k]{\Mat{A}}|
%   %\hspace{0.5cm}
%   \fbox{$\InvTEF{\Mat{A}}$} \fbox{$\InvTEF[k]{\Mat{A}}$}
% \end{center}
% 
% \begin{center}
%   |\InvTEFp{\Mat{A}} \InvTEFp[k]{\Mat{A}}|
%   %\hspace{1.2cm}
%   \fbox{$\InvTEFp{\Mat{A}}$} \fbox{$\InvTEFp[k]{\Mat{A}}$}
% \end{center}
% \begin{center}
%   |\InvTEFp*{\Mat{A}} \InvTEFp*[k]{\Mat{A}}|
%   \hspace{1.2cm}
%   \fbox{$\InvTEFp*{\Mat{A}}$} \fbox{$\InvTEFp*[k]{\Mat{A}}$}
% \end{center}
% 
% \begin{center}
%   |\InvTEFP{\Mat{A}} \InvTEFP[k]{\Mat{A}}|
%   \hspace{1.2cm}
%   \fbox{$\InvTEFP{\Mat{A}}$} \fbox{$\InvTEFP[k]{\Mat{A}}$}
% \end{center}
% \begin{center}
%   |\InvTEFP*{\Mat{A}} \InvTEFP*[k]{\Mat{A}}|
%   \hspace{1.2cm}
%   \fbox{$\InvTEFP*{\Mat{A}}$} \fbox{$\InvTEFP*[k]{\Mat{A}}$}
% \end{center}
% 
% \begin{center}
%   |\InvTEFpE{\Mat{A}} \InvTEFpE[k]{\Mat{A}}|
%   \hspace{1.2cm}
%   \fbox{$\InvTEFpE{\Mat{A}}$} \fbox{$\InvTEFpE[k]{\Mat{A}}$}
% \end{center}
% \begin{center}
%   |\InvTEFpE*{\Mat{A}} \InvTEFpE*[k]{\Mat{A}}|
%   \hspace{1.2cm}
%   \fbox{$\InvTEFpE*{\Mat{A}}$} \fbox{$\InvTEFpE*[k]{\Mat{A}}$}
% \end{center}
% 
% \begin{center}
%   |\InvTEFPE{\Mat{A}} \InvTEFPE[k]{\Mat{A}}|
%   \hspace{1.2cm}
%   \fbox{$\InvTEFPE{\Mat{A}}$} \fbox{$\InvTEFPE[k]{\Mat{A}}$}
% \end{center}
% \begin{center}
%   |\InvTEFPE*{\Mat{A}}  \InvTEFPE*[k]{\Mat{A}}|
%   \hspace{1.2cm}
%   \fbox{$\InvTEFPE*{\Mat{A}}$} \fbox{$\InvTEFPE*[k]{\Mat{A}}$}
% \end{center}
% 
% \paragraph{Inversa de una Transf. elemental genérica aplicada a la derecha de un objeto.}{\mbox{\ }}
% 
% \DescribeMacro{\InvTEC}
% \DescribeMacro{\InvTECp}
% \DescribeMacro{\InvTECp*}
% \DescribeMacro{\InvTECP}
% \DescribeMacro{\InvTECP*}
% \DescribeMacro{\InvTECpE}
% \DescribeMacro{\InvTECpE*}
% \DescribeMacro{\InvTECPE}
% \DescribeMacro{\InvTECPE*}
% El comando |InvTEC<XX*>| tiene 2
% argumentos,\;|InvTEC<XX*>|\oarg{índice}\marg{objeto},\; e
% indica la inversa de una transformación elemental genérica por la izquierda del
% objeto.
% \begin{center}
%   |\InvTEC{\Mat{A}} \InvTEC[k]{\Mat{A}}|
%   %\hspace{0.5cm}
%   \fbox{$\InvTEC{\Mat{A}}$} \fbox{$\InvTEC[k]{\Mat{A}}$}
% \end{center}
% 
% \begin{center}
%   |\InvTECp{\Mat{A}} \InvTECp[k]{\Mat{A}}|
%   %\hspace{1.2cm}
%   \fbox{$\InvTECp{\Mat{A}}$} \fbox{$\InvTECp[k]{\Mat{A}}$}
% \end{center}
% \begin{center}
%   |\InvTECp*{\Mat{A}} \InvTECp*[k]{\Mat{A}}|
%   \hspace{1.2cm}
%   \fbox{$\InvTECp*{\Mat{A}}$} \fbox{$\InvTECp*[k]{\Mat{A}}$}
% \end{center}
% 
% \begin{center}
%   |\InvTECP{\Mat{A}} \InvTECP[k]{\Mat{A}}|
%   \hspace{1.2cm}
%   \fbox{$\InvTECP{\Mat{A}}$} \fbox{$\InvTECP[k]{\Mat{A}}$}
% \end{center}
% \begin{center}
%   |\InvTECP*{\Mat{A}} \InvTECP*[k]{\Mat{A}}|
%   \hspace{1.2cm}
%   \fbox{$\InvTECP*{\Mat{A}}$} \fbox{$\InvTECP*[k]{\Mat{A}}$}
% \end{center}
% 
% \begin{center}
%   |\InvTECpE{\Mat{A}} \InvTECpE[k]{\Mat{A}}|
%   \hspace{1.2cm}
%   \fbox{$\InvTECpE{\Mat{A}}$} \fbox{$\InvTECpE[k]{\Mat{A}}$}
% \end{center}
% \begin{center}
%   |\InvTECpE*{\Mat{A}} \InvTECpE*[k]{\Mat{A}}|
%   \hspace{1.2cm}
%   \fbox{$\InvTECpE*{\Mat{A}}$} \fbox{$\InvTECpE*[k]{\Mat{A}}$}
% \end{center}
% 
% \begin{center}
%   |\InvTECPE{\Mat{A}} \InvTECPE[k]{\Mat{A}}|
%   \hspace{1.2cm}
%   \fbox{$\InvTECPE{\Mat{A}}$} \fbox{$\InvTECPE[k]{\Mat{A}}$}
% \end{center}
% \begin{center}
%   |\InvTECPE*{\Mat{A}}  \InvTECPE*[k]{\Mat{A}}|
%   \hspace{1.2cm}
%   \fbox{$\InvTECPE*{\Mat{A}}$} \fbox{$\InvTECPE*[k]{\Mat{A}}$}
% \end{center}
% 
% %%%%%%%%%%%%%%%%%%%%%%%%%%%%%%%%%%%%%%%%%%%%%%%%%%%%%%%%%%%%%%%%%%%%%%%%
% 
% %%%%%%%%%%%%%%%%%%%%%%%%%%%%%%%%%%%%%%%%%%%%%%%%%%%%%%%%%%%%%%%%%%%%%%%%%%%
% 
% \paragraph{Espejo de la inversa de una Transf. elemental genérica aplicada a la izquierda de un objeto.}{\mbox{\ }}
% 
% \DescribeMacro{\EInvTEF}
% \DescribeMacro{\EInvTEFp}
% \DescribeMacro{\EInvTEFp*}
% \DescribeMacro{\EInvTEFP}
% \DescribeMacro{\EInvTEFP*}
% \DescribeMacro{\EInvTEFpE}
% \DescribeMacro{\EInvTEFpE*}
% \DescribeMacro{\EInvTEFPE}
% \DescribeMacro{\EInvTEFPE*}
% El comando |EInvTEF<XX*>| tiene 2
% argumentos,\;|EInvTEF<XX*>|\oarg{índice}\marg{objeto},\; e
% indica la inversa de una transformación elemental genérica por la izquierda del
% objeto.
% \begin{center}
%   |\EInvTEF{\Mat{A}} \EInvTEF[k]{\Mat{A}}|
%   %\hspace{0.5cm}
%   \fbox{$\EInvTEF{\Mat{A}}$} \fbox{$\EInvTEF[k]{\Mat{A}}$}
% \end{center}
% 
% \begin{center}
%   |\EInvTEFp{\Mat{A}} \EInvTEFp[k]{\Mat{A}}|
%   %\hspace{1.2cm}
%   \fbox{$\EInvTEFp{\Mat{A}}$} \fbox{$\EInvTEFp[k]{\Mat{A}}$}
% \end{center}
% \begin{center}
%   |\EInvTEFp*{\Mat{A}} \EInvTEFp*[k]{\Mat{A}}|
%   \hspace{1.2cm}
%   \fbox{$\EInvTEFp*{\Mat{A}}$} \fbox{$\EInvTEFp*[k]{\Mat{A}}$}
% \end{center}
% 
% \begin{center}
%   |\EInvTEFP{\Mat{A}} \EInvTEFP[k]{\Mat{A}}|
%   \hspace{1.2cm}
%   \fbox{$\EInvTEFP{\Mat{A}}$} \fbox{$\EInvTEFP[k]{\Mat{A}}$}
% \end{center}
% \begin{center}
%   |\EInvTEFP*{\Mat{A}} \EInvTEFP*[k]{\Mat{A}}|
%   \hspace{1.2cm}
%   \fbox{$\EInvTEFP*{\Mat{A}}$} \fbox{$\EInvTEFP*[k]{\Mat{A}}$}
% \end{center}
% 
% \begin{center}
%   |\EInvTEFpE{\Mat{A}} \EInvTEFpE[k]{\Mat{A}}|
%   \hspace{1.2cm}
%   \fbox{$\EInvTEFpE{\Mat{A}}$} \fbox{$\EInvTEFpE[k]{\Mat{A}}$}
% \end{center}
% \begin{center}
%   |\EInvTEFpE*{\Mat{A}} \EInvTEFpE*[k]{\Mat{A}}|
%   \hspace{1.2cm}
%   \fbox{$\EInvTEFpE*{\Mat{A}}$} \fbox{$\EInvTEFpE*[k]{\Mat{A}}$}
% \end{center}
% 
% \begin{center}
%   |\EInvTEFPE{\Mat{A}} \EInvTEFPE[k]{\Mat{A}}|
%   \hspace{1.2cm}
%   \fbox{$\EInvTEFPE{\Mat{A}}$} \fbox{$\EInvTEFPE[k]{\Mat{A}}$}
% \end{center}
% \begin{center}
%   |\EInvTEFPE*{\Mat{A}}  \EInvTEFPE*[k]{\Mat{A}}|
%   \hspace{1.2cm}
%   \fbox{$\EInvTEFPE*{\Mat{A}}$} \fbox{$\EInvTEFPE*[k]{\Mat{A}}$}
% \end{center}
% 
% \paragraph{Espejo de la inversa de una Transf. elemental genérica aplicada a la derecha de un objeto.}{\mbox{\ }}
% 
% \DescribeMacro{\EInvTEC}
% \DescribeMacro{\EInvTECp}
% \DescribeMacro{\EInvTECp*}
% \DescribeMacro{\EInvTECP}
% \DescribeMacro{\EInvTECP*}
% \DescribeMacro{\EInvTECpE}
% \DescribeMacro{\EInvTECpE*}
% \DescribeMacro{\EInvTECPE}
% \DescribeMacro{\EInvTECPE*}
% El comando |EInvTEC<XX*>| tiene 2
% argumentos,\;|EInvTEC<XX*>|\oarg{índice}\marg{objeto},\; e
% indica la inversa de una transformación elemental genérica por la izquierda del
% objeto.
% \begin{center}
%   |\EInvTEC{\Mat{A}} \EInvTEC[k]{\Mat{A}}|
%   %\hspace{0.5cm}
%   \fbox{$\EInvTEC{\Mat{A}}$} \fbox{$\EInvTEC[k]{\Mat{A}}$}
% \end{center}
% 
% \begin{center}
%   |\EInvTECp{\Mat{A}} \EInvTECp[k]{\Mat{A}}|
%   %\hspace{1.2cm}
%   \fbox{$\EInvTECp{\Mat{A}}$} \fbox{$\EInvTECp[k]{\Mat{A}}$}
% \end{center}
% \begin{center}
%   |\EInvTECp*{\Mat{A}} \EInvTECp*[k]{\Mat{A}}|
%   \hspace{1.2cm}
%   \fbox{$\EInvTECp*{\Mat{A}}$} \fbox{$\EInvTECp*[k]{\Mat{A}}$}
% \end{center}
% 
% \begin{center}
%   |\EInvTECP{\Mat{A}} \EInvTECP[k]{\Mat{A}}|
%   \hspace{1.2cm}
%   \fbox{$\EInvTECP{\Mat{A}}$} \fbox{$\EInvTECP[k]{\Mat{A}}$}
% \end{center}
% \begin{center}
%   |\EInvTECP*{\Mat{A}} \EInvTECP*[k]{\Mat{A}}|
%   \hspace{1.2cm}
%   \fbox{$\EInvTECP*{\Mat{A}}$} \fbox{$\EInvTECP*[k]{\Mat{A}}$}
% \end{center}
% 
% \begin{center}
%   |\EInvTECpE{\Mat{A}} \EInvTECpE[k]{\Mat{A}}|
%   \hspace{1.2cm}
%   \fbox{$\EInvTECpE{\Mat{A}}$} \fbox{$\EInvTECpE[k]{\Mat{A}}$}
% \end{center}
% \begin{center}
%   |\EInvTECpE*{\Mat{A}} \EInvTECpE*[k]{\Mat{A}}|
%   \hspace{1.2cm}
%   \fbox{$\EInvTECpE*{\Mat{A}}$} \fbox{$\EInvTECpE*[k]{\Mat{A}}$}
% \end{center}
% 
% \begin{center}
%   |\EInvTECPE{\Mat{A}} \EInvTECPE[k]{\Mat{A}}|
%   \hspace{1.2cm}
%   \fbox{$\EInvTECPE{\Mat{A}}$} \fbox{$\EInvTECPE[k]{\Mat{A}}$}
% \end{center}
% \begin{center}
%   |\EInvTECPE*{\Mat{A}}  \EInvTECPE*[k]{\Mat{A}}|
%   \hspace{1.2cm}
%   \fbox{$\EInvTECPE*{\Mat{A}}$} \fbox{$\EInvTECPE*[k]{\Mat{A}}$}
% \end{center}
% 
% 
% 
% %%%%%%%%%%%%%%%%%%%%%%%%%%%%%%%%%%%%%%
% \paragraph{Transf. elemental genérica aplicada a la izquierda de un objeto (funciones duplicadas sin argumentos opcionales).} Cuando la aplicamos a la izquierda de una matriz corresponde a una transformación de sus filas
% 
% \DescribeMacro{\dTEEF}
% \DescribeMacro{\dTEEFp}
% \DescribeMacro{\dTEEFP}
% \DescribeMacro{\dTEEFpE}
% \DescribeMacro{\dTEEFPE}
% El comando \cs{dTEEF} tiene 3
% argumentos,\;\cs{dTEEF}\marg{índice}\marg{exponente}\marg{objeto},\; e
% indica una transformación elemental genérica (con exponente) por la
% izquierda del objeto.
% \begin{center}
%   |\dTEEF{}{}{\SV{A}} \dTEEF{2}{}{\SV{A}} \dTEEF{2}{*}{\SV{A}}|
%   \hspace{1.2cm}
%   \fbox{$\dTEEF{}{}{\SV{A}}$} \fbox{$\dTEEF{2}{}{\SV{A}}$} \fbox{$\dTEEF{2}{*}{\SV{A}}$}
% \end{center}
% 
% \begin{center}
%   |\dTEEFp{}{}{A} \dTEEFp{2}{}{A} \dTEEFp{2}{*}{A}|
%   \hspace{1.2cm}
%   \fbox{$\dTEEFp{}{}{A}$} \fbox{$\dTEEFp{2}{}{A}$} \fbox{$\dTEEFp{2}{*}{A}$}
% \end{center}
% 
% \begin{center}
%   |\dTEEFP{}{}{A} \dTEEFP{2}{}{A} \dTEEFP{2}{*}{A}|
%   \hspace{1.2cm}
%   \fbox{$\dTEEFP{}{}{A}$} \fbox{$\dTEEFP{2}{}{A}$} \fbox{$\dTEEFP{2}{*}{A}$}
% \end{center}
% 
% \begin{center}
%   |\dTEEFpE{}{}{A} \dTEEFpE{2}{}{A} \dTEEFpE{2}{*}{A}|
%   \hspace{1.2cm}
%   \fbox{$\dTEEFpE{}{}{A}$} \fbox{$\dTEEFpE{2}{}{A}$} \fbox{$\dTEEFpE{2}{*}{A}$}
% \end{center}
% 
% \begin{center}
%   |\dTEEFPE{}{}{A} \dTEEFPE{2}{}{A} \dTEEFPE{2}{*}{A}|
%   \hspace{1.2cm}
%   \fbox{$\dTEEFPE{}{}{A}$} \fbox{$\dTEEFPE{2}{}{A}$} \fbox{$\dTEEFPE{2}{*}{A}$}
% \end{center}
% 
% \DescribeMacro{\dTEF}
% \DescribeMacro{\dTEFp}
% \DescribeMacro{\dTEFP}
% \DescribeMacro{\dTEFpE}
% \DescribeMacro{\dTEFPE}
% El comando \cs{dTEF} tiene 2
% argumentos,\;\cs{dTEF}\marg{índice}\marg{objeto},\;
% e indica una transformación elemental genérica por la izquierda del objeto.
% \begin{center}
%   |\dTEF{}{\Mat{A}} \dTEF{2}{\Mat{A}}|
%   \hspace{1.2cm}
%   \fbox{$\dTEF{}{\Mat{A}}$} \fbox{$\dTEF{2}{\Mat{A}}$}
% \end{center}
% 
% \begin{center}
%   |\dTEFpE{}{\Mat{A}} \dTEFpE{2}{\Mat{A}}|
%   \hspace{1.2cm}
%   \fbox{$\dTEFpE{}{\Mat{A}}$} \fbox{$\dTEFpE{2}{\Mat{A}}$}
% \end{center}
% 
% \begin{center}
%   |\dTEFPE{}{\Mat{A}} \dTEFPE{2}{\Mat{A}}|
%   \hspace{1.2cm}
%   \fbox{$\dTEFPE{}{\Mat{A}}$} \fbox{$\dTEFPE{2}{\Mat{A}}$}
% \end{center}
% 
% \DescribeMacro{\dETEF}
% \DescribeMacro{\dETEFp}
% \DescribeMacro{\dETEFP}
% \DescribeMacro{\dETEFpE}
% \DescribeMacro{\dETEFPE}
% El comando \cs{dETEF} tiene 2
% argumentos,\;\cs{dETEF}\marg{índice}\marg{objeto},\;
% e indica una transformación elemental espejo genérica por la izquierda del objeto.
% \begin{center}
%   |\dETEF{}{\Mat{A}} \dETEF{2}{\Mat{A}}|
%   \hspace{1.2cm}
%   \fbox{$\dETEF{}{\Mat{A}}$} \fbox{$\dETEF{2}{\Mat{A}}$}
% \end{center}
% 
% \begin{center}
%   |\dETEFp{}{\Mat{A}} \dETEFp{2}{\Mat{A}}|
%   \hspace{1.2cm}
%   \fbox{$\dETEFp{}{\Mat{A}}$} \fbox{$\dETEFp{2}{\Mat{A}}$}
% \end{center}
% 
% \begin{center}
%   |\dETEFP{}{\Mat{A}} \dETEFP{2}{\Mat{A}}|
%   \hspace{1.2cm}
%   \fbox{$\dETEFP{}{\Mat{A}}$} \fbox{$\dETEFP{2}{\Mat{A}}$}
% \end{center}
% 
% \begin{center}
%   |\dETEFpE{}{\Mat{A}} \dETEFpE{2}{\Mat{A}}|
%   \hspace{1.2cm}
%   \fbox{$\dETEFpE{}{\Mat{A}}$} \fbox{$\dETEFpE{2}{\Mat{A}}$}
% \end{center}
% 
% \begin{center}
%   |\dETEFPE{}{\Mat{A}} \dETEFPE{2}{\Mat{A}}|
%   \hspace{1.2cm}
%   \fbox{$\dETEFPE{}{\Mat{A}}$} \fbox{$\dETEFPE{2}{\Mat{A}}$}
% \end{center}
% 
% \DescribeMacro{\dInvTEF}
% \DescribeMacro{\dInvTEFp}
% \DescribeMacro{\dInvTEFP}
% \DescribeMacro{\dInvTEFpE}
% \DescribeMacro{\dInvTEFPE}
% El comando \cs{dInvTEF} tiene 2
% argumentos,\;\cs{dInvTEF}\marg{índice}\marg{objeto},\;
% e indica una transformación elemental espejo inversa genérica por la izquierda del objeto.
% \begin{center}
%   |\dInvTEF{}{\Mat{A}} \dInvTEF{2}{\Mat{A}}|
%   \hspace{1.2cm}
%   \fbox{$\dInvTEF{}{\Mat{A}}$} \fbox{$\dInvTEF{2}{\Mat{A}}$}
% \end{center}
% 
% \begin{center}
%   |\dInvTEFp{}{\Mat{A}} \dInvTEFp{2}{\Mat{A}}|
%   \hspace{1.2cm}
%   \fbox{$\dInvTEFp{}{\Mat{A}}$} \fbox{$\dInvTEFp{2}{\Mat{A}}$}
% \end{center}
% 
% \begin{center}
%   |\dInvTEFP{}{\Mat{A}} \dInvTEFP{2}{\Mat{A}}|
%   \hspace{1.2cm}
%   \fbox{$\dInvTEFP{}{\Mat{A}}$} \fbox{$\dInvTEFP{2}{\Mat{A}}$}
% \end{center}
% 
% \begin{center}
%   |\dInvTEFpE{}{\Mat{A}} \dInvTEFpE{2}{\Mat{A}}|
%   \hspace{1.2cm}
%   \fbox{$\dInvTEFpE{}{\Mat{A}}$} \fbox{$\dInvTEFpE{2}{\Mat{A}}$}
% \end{center}
% 
% \begin{center}
%   |\dInvTEFPE{}{\Mat{A}} \dInvTEFPE{2}{\Mat{A}}|
%   \hspace{1.2cm}
%   \fbox{$\dInvTEFPE{}{\Mat{A}}$} \fbox{$\dInvTEFPE{2}{\Mat{A}}$}
% \end{center}
% 
% \DescribeMacro{\dEInvTEF}
% \DescribeMacro{\dEInvTEFp}
% \DescribeMacro{\dEInvTEFP}
% \DescribeMacro{\dEInvTEFpE}
% \DescribeMacro{\dEInvTEFPE}
% El comando \cs{dEInvTEF} tiene 2
% argumentos,\;\cs{dEInvTEF}\marg{índice}\marg{objeto},\;
% e indica una transformación elemental espejo inversa genérica por la izquierda del objeto.
% \begin{center}
%   |\dEInvTEF{}{\Mat{A}} \dEInvTEF{2}{\Mat{A}}|
%   \hspace{1.2cm}
%   \fbox{$\dEInvTEF{}{\Mat{A}}$} \fbox{$\dEInvTEF{2}{\Mat{A}}$}
% \end{center}
% 
% \begin{center}
%   |\dEInvTEFp{}{\Mat{A}} \dEInvTEFp{2}{\Mat{A}}|
%   \hspace{1.2cm}
%   \fbox{$\dEInvTEFp{}{\Mat{A}}$} \fbox{$\dEInvTEFp{2}{\Mat{A}}$}
% \end{center}
% 
% \begin{center}
%   |\dEInvTEFP{}{\Mat{A}} \dEInvTEFP{2}{\Mat{A}}|
%   \hspace{1.2cm}
%   \fbox{$\dEInvTEFP{}{\Mat{A}}$} \fbox{$\dEInvTEFP{2}{\Mat{A}}$}
% \end{center}
% 
% \begin{center}
%   |\dEInvTEFpE{}{\Mat{A}} \dEInvTEFpE{2}{\Mat{A}}|
%   \hspace{1.2cm}
%   \fbox{$\dEInvTEFpE{}{\Mat{A}}$} \fbox{$\dEInvTEFpE{2}{\Mat{A}}$}
% \end{center}
% 
% \begin{center}
%   |\dEInvTEFPE{}{\Mat{A}} \dEInvTEFPE{2}{\Mat{A}}|
%   \hspace{1.2cm}
%   \fbox{$\dEInvTEFPE{}{\Mat{A}}$} \fbox{$\dEInvTEFPE{2}{\Mat{A}}$}
% \end{center}
% 
% %%%%%%%%%%%%%%%%%%%%%%%%%%%%%%%%%%
% 
% \paragraph{Transf. elemental aplicada la derecha de un objeto (funciones duplicadas sin argumentos opcionales).} Cuando la aplicamos a la derecha de una matriz corresponde a una transformación de sus columnas
% 
% \DescribeMacro{\dTEEC}
% \DescribeMacro{\dTEECp}
% \DescribeMacro{\dTEECP}
% \DescribeMacro{\dTEECpE}
% \DescribeMacro{\dTEECPE}
% El comando \cs{dTEEC} tiene 3
% argumentos,\;\cs{dTEEC}\marg{índice}\marg{exponente}\marg{objeto},\; e
% indica una transformación elemental genérica (con exponente) por la
% derecha del objeto.
% \begin{center}
%   |\dTEEC{}{}{\SV{A}} \dTEEC{2}{}{\SV{A}} \dTEEC{2}{*}{\SV{A}}|
%   \hspace{1.2cm}
%   \fbox{$\dTEEC{}{}{\SV{A}}$} \fbox{$\dTEEC{2}{}{\SV{A}}$} \fbox{$\dTEEC{2}{*}{\SV{A}}$}
% \end{center}
% 
% \begin{center}
%   |\dTEECp{}{}{A} \dTEECp{2}{}{A} \dTEECp{2}{*}{A}|
%   \hspace{1.2cm}
%   \fbox{$\dTEECp{}{}{A}$} \fbox{$\dTEECp{2}{}{A}$} \fbox{$\dTEECp{2}{*}{A}$}
% \end{center}
% 
% \begin{center}
%   |\dTEECP{}{}{A} \dTEECP{2}{}{A} \dTEECP{2}{*}{A}|
%   \hspace{1.2cm}
%   \fbox{$\dTEECP{}{}{A}$} \fbox{$\dTEECP{2}{}{A}$} \fbox{$\dTEECP{2}{*}{A}$}
% \end{center}
% 
% \begin{center}
%   |\dTEECpE{}{}{A} \dTEECpE{2}{}{A} \dTEECpE{2}{*}{A}|
%   \hspace{1.2cm}
%   \fbox{$\dTEECpE{}{}{A}$} \fbox{$\dTEECpE{2}{}{A}$} \fbox{$\dTEECpE{2}{*}{A}$}
% \end{center}
% 
% \begin{center}
%   |\dTEECPE{}{}{A} \dTEECPE{2}{}{A} \dTEECPE{2}{*}{A}|
%   \hspace{1.2cm}
%   \fbox{$\dTEECPE{}{}{A}$} \fbox{$\dTEECPE{2}{}{A}$} \fbox{$\dTEECPE{2}{*}{A}$}
% \end{center}
% 
% \DescribeMacro{\dTEC}
% \DescribeMacro{\dTECp}
% \DescribeMacro{\dTECP}
% \DescribeMacro{\dTECpE}
% \DescribeMacro{\dTECPE}
% El comando \cs{dTEC} tiene 2
% argumentos,\;\cs{dTEC}\marg{índice}\marg{objeto},\;
% e indica una transformación elemental genérica por la derecha del objeto.
% \begin{center}
%   |\dTEC{}{\Mat{A}} \dTEC{2}{\Mat{A}}|
%   \hspace{1.2cm}
%   \fbox{$\dTEC{}{\Mat{A}}$} \fbox{$\dTEC{2}{\Mat{A}}$}
% \end{center}
% 
% \begin{center}
%   |\dTECpE{}{\Mat{A}} \dTECpE{2}{\Mat{A}}|
%   \hspace{1.2cm}
%   \fbox{$\dTECpE{}{\Mat{A}}$} \fbox{$\dTECpE{2}{\Mat{A}}$}
% \end{center}
% 
% \begin{center}
%   |\dTECPE{}{\Mat{A}} \dTECPE{2}{\Mat{A}}|
%   \hspace{1.2cm}
%   \fbox{$\dTECPE{}{\Mat{A}}$} \fbox{$\dTECPE{2}{\Mat{A}}$}
% \end{center}
% 
% \DescribeMacro{\dETEC}
% \DescribeMacro{\dETECp}
% \DescribeMacro{\dETECP}
% \DescribeMacro{\dETECpE}
% \DescribeMacro{\dETECPE}
% El comando \cs{dETEC} tiene 2
% argumentos,\;\cs{dETEC}\marg{índice}\marg{objeto},\;
% e indica una transformación elemental espejo genérica por la derecha del objeto.
% \begin{center}
%   |\dETEC{}{\Mat{A}} \dETEC{2}{\Mat{A}}|
%   \hspace{1.2cm}
%   \fbox{$\dETEC{}{\Mat{A}}$} \fbox{$\dETEC{2}{\Mat{A}}$}
% \end{center}
% 
% \begin{center}
%   |\dETECp{}{\Mat{A}} \dETECp{2}{\Mat{A}}|
%   \hspace{1.2cm}
%   \fbox{$\dETECp{}{\Mat{A}}$} \fbox{$\dETECp{2}{\Mat{A}}$}
% \end{center}
% 
% \begin{center}
%   |\dETECP{}{\Mat{A}} \dETECP{2}{\Mat{A}}|
%   \hspace{1.2cm}
%   \fbox{$\dETECP{}{\Mat{A}}$} \fbox{$\dETECP{2}{\Mat{A}}$}
% \end{center}
% 
% \begin{center}
%   |\dETECpE{}{\Mat{A}} \dETECpE{2}{\Mat{A}}|
%   \hspace{1.2cm}
%   \fbox{$\dETECpE{}{\Mat{A}}$} \fbox{$\dETECpE{2}{\Mat{A}}$}
% \end{center}
% 
% \begin{center}
%   |\dETECPE{}{\Mat{A}} \dETECPE{2}{\Mat{A}}|
%   \hspace{1.2cm}
%   \fbox{$\dETECPE{}{\Mat{A}}$} \fbox{$\dETECPE{2}{\Mat{A}}$}
% \end{center}
% 
% \DescribeMacro{\dInvTEC}
% \DescribeMacro{\dInvTECp}
% \DescribeMacro{\dInvTECP}
% \DescribeMacro{\dInvTECpE}
% \DescribeMacro{\dInvTECPE}
% El comando \cs{dInvTEC} tiene 2
% argumentos,\;\cs{dInvTEC}\marg{índice}\marg{objeto},\;
% e indica una transformación elemental espejo inversa genérica por la derecha del objeto.
% \begin{center}
%   |\dInvTEC{}{\Mat{A}} \dInvTEC{2}{\Mat{A}}|
%   \hspace{1.2cm}
%   \fbox{$\dInvTEC{}{\Mat{A}}$} \fbox{$\dInvTEC{2}{\Mat{A}}$}
% \end{center}
% 
% \begin{center}
%   |\dInvTECp{}{\Mat{A}} \dInvTECp{2}{\Mat{A}}|
%   \hspace{1.2cm}
%   \fbox{$\dInvTECp{}{\Mat{A}}$} \fbox{$\dInvTECp{2}{\Mat{A}}$}
% \end{center}
% 
% \begin{center}
%   |\dInvTECP{}{\Mat{A}} \dInvTECP{2}{\Mat{A}}|
%   \hspace{1.2cm}
%   \fbox{$\dInvTECP{}{\Mat{A}}$} \fbox{$\dInvTECP{2}{\Mat{A}}$}
% \end{center}
% 
% \begin{center}
%   |\dInvTECpE{}{\Mat{A}} \dInvTECpE{2}{\Mat{A}}|
%   \hspace{1.2cm}
%   \fbox{$\dInvTECpE{}{\Mat{A}}$} \fbox{$\dInvTECpE{2}{\Mat{A}}$}
% \end{center}
% 
% \begin{center}
%   |\dInvTECPE{}{\Mat{A}} \dInvTECPE{2}{\Mat{A}}|
%   \hspace{1.2cm}
%   \fbox{$\dInvTECPE{}{\Mat{A}}$} \fbox{$\dInvTECPE{2}{\Mat{A}}$}
% \end{center}
% 
% \DescribeMacro{\dEInvTEC}
% \DescribeMacro{\dEInvTECp}
% \DescribeMacro{\dEInvTECP}
% \DescribeMacro{\dEInvTECpE}
% \DescribeMacro{\dEInvTECPE}
% El comando \cs{dEInvTEC} tiene 2
% argumentos,\;\cs{dEInvTEC}\marg{índice}\marg{objeto},\;
% e indica una transformación elemental espejo inversa genérica por la derecha del objeto.
% \begin{center}
%   |\dEInvTEC{}{\Mat{A}} \dEInvTEC{2}{\Mat{A}}|
%   \hspace{1.2cm}
%   \fbox{$\dEInvTEC{}{\Mat{A}}$} \fbox{$\dEInvTEC{2}{\Mat{A}}$}
% \end{center}
% 
% \begin{center}
%   |\dEInvTECp{}{\Mat{A}} \dEInvTECp{2}{\Mat{A}}|
%   \hspace{1.2cm}
%   \fbox{$\dEInvTECp{}{\Mat{A}}$} \fbox{$\dEInvTECp{2}{\Mat{A}}$}
% \end{center}
% 
% \begin{center}
%   |\dEInvTECP{}{\Mat{A}} \dEInvTECP{2}{\Mat{A}}|
%   \hspace{1.2cm}
%   \fbox{$\dEInvTECP{}{\Mat{A}}$} \fbox{$\dEInvTECP{2}{\Mat{A}}$}
% \end{center}
% 
% \begin{center}
%   |\dEInvTECpE{}{\Mat{A}} \dEInvTECpE{2}{\Mat{A}}|
%   \hspace{1.2cm}
%   \fbox{$\dEInvTECpE{}{\Mat{A}}$} \fbox{$\dEInvTECpE{2}{\Mat{A}}$}
% \end{center}
% 
% \begin{center}
%   |\dEInvTECPE{}{\Mat{A}} \dEInvTECPE{2}{\Mat{A}}|
%   \hspace{1.2cm}
%   \fbox{$\dEInvTECPE{}{\Mat{A}}$} \fbox{$\dEInvTECPE{2}{\Mat{A}}$}
% \end{center}
% 
% %%%%%%%%%%%%%%%%%%%%%%%%%%%%%%%%%%%%%%
% \paragraph{Transformaciones elementales particulares} Aquí describimos la notación de transformaciones específicas.
% 
% \DescribeMacro{\dTrF}
% \DescribeMacro{\dTrFp}
% \DescribeMacro{\dTrFP}
% \DescribeMacro{\dTrFpE}
% \DescribeMacro{\dTrFPE}
% El comando \cs{dTrF} tiene 2
% argumentos,\;\cs{dTrF}\marg{operación(es)}\marg{objeto},\; e indica una
% transformación (o transformaciones) elemental(es) por la izquierda del objeto.
% \begin{center}
%   |\dTrF{ \dOEgE{1}{'}\cdots\dOEgE{p}{'} }{\Mat{I}}|
%   \hspace{1.2cm}
%   \fbox{$\dTrF{\dOEgE{1}{'}\cdots\dOEgE{p}{'}}{\Mat{I}}$}
% \end{center}
% \begin{center}
%   |\dTrF{ \OpE{\su{5}{i}{j}}\OpE{\pr{-7}{j}} }{\Mat{A}}|
%   \hspace{1.2cm}
%   \fbox{$\dTrF{\OpE{\su{5}{i}{j}}\OpE{\pr{-7}{j}}}{\Mat{A}}$}
% \end{center}
% 
% \begin{center}
%   |\dTrFp{ \dOEgE{1}{'}\cdots\dOEgE{p}{'} }{\Mat{I}}|
%   \hspace{1.2cm}
%   \fbox{$\dTrFp{\dOEgE{1}{'}\cdots\dOEgE{p}{'}}{\Mat{I}}$}
% \end{center}
% \begin{center}
%   |\dTrFp{ \OpE{\su{5}{i}{j}}\OpE{\pr{-7}{j}} }{\Mat{A}}|
%   \hspace{1.2cm}
%   \fbox{$\dTrFp{\OpE{\su{5}{i}{j}}\OpE{\pr{-7}{j}}}{\Mat{A}}$}
% \end{center}
% 
% \begin{center}
%   |\dTrFP{ \dOEgE{1}{'}\cdots\dOEgE{p}{'} }{\Mat{I}}|
%   \hspace{1.2cm}
%   \fbox{$\dTrFP{\dOEgE{1}{'}\cdots\dOEgE{p}{'}}{\Mat{I}}$}
% \end{center}
% \begin{center}
%   |\dTrFP{ \OpE{\su{5}{i}{j}}\OpE{\pr{-7}{j}} }{\Mat{A}}|
%   \hspace{1.2cm}
%   \fbox{$\dTrFP{\OpE{\su{5}{i}{j}}\OpE{\pr{-7}{j}}}{\Mat{A}}$}
% \end{center}
% 
% \begin{center}
%   |\dTrFpE{ \dOEgE{1}{'}\cdots\dOEgE{p}{'} }{\Mat{I}}|
%   \hspace{1.2cm}
%   \fbox{$\dTrFpE{\dOEgE{1}{'}\cdots\dOEgE{p}{'}}{\Mat{I}}$}
% \end{center}
% \begin{center}
%   |\dTrFpE{ \OpE{\su{5}{i}{j}}\OpE{\pr{-7}{j}} }{\Mat{A}}|
%   \hspace{1.2cm}
%   \fbox{$\dTrFpE{\OpE{\su{5}{i}{j}}\OpE{\pr{-7}{j}}}{\Mat{A}}$}
% \end{center}
% 
% \begin{center}
%   |\dTrFPE{ \dOEgE{1}{'}\cdots\dOEgE{p}{'} }{\Mat{I}}|
%   \hspace{1.2cm}
%   \fbox{$\dTrFPE{\dOEgE{1}{'}\cdots\dOEgE{p}{'}}{\Mat{I}}$}
% \end{center}
% \begin{center}
%   |\dTrFPE{ \OpE{\su{5}{i}{j}}\OpE{\pr{-7}{j}} }{\Mat{A}}|
%   \hspace{1.2cm}
%   \fbox{$\dTrFPE{\OpE{\su{5}{i}{j}}\OpE{\pr{-7}{j}}}{\Mat{A}}$}
% \end{center}
% 
% \DescribeMacro{\dTrC}
% \DescribeMacro{\dTrCp}
% \DescribeMacro{\dTrCP}
% \DescribeMacro{\dTrCpE}
% \DescribeMacro{\dTrCPE}
% El comando \cs{dTrC} tiene 2
% argumentos,\;\cs{dTrC}\marg{operación(es)}\marg{objeto},\; e indica una
% transformación (o transformaciones) elemental(es) por la derecha del objeto.
% \begin{center}
%   |\dTrC{ \dOEgE{1}{'}\cdots\dOEgE{p}{'} }{\Mat{I}}|
%   \hspace{1.2cm}
%   \fbox{$\dTrC{\dOEgE{1}{'}\cdots\dOEgE{p}{'}}{\Mat{I}}$}
% \end{center}
% \begin{center}
%   |\dTrC{ \OpE{\su{5}{i}{j}}\OpE{\pr{-7}{j}} }{\Mat{A}}|
%   \hspace{1.2cm}
%   \fbox{$\dTrC{\OpE{\su{5}{i}{j}}\OpE{\pr{-7}{j}}}{\Mat{A}}$}
% \end{center}
% 
% \begin{center}
%   |\dTrCp{ \dOEgE{1}{'}\cdots\dOEgE{p}{'} }{\Mat{I}}|
%   \hspace{1.2cm}
%   \fbox{$\dTrCp{\dOEgE{1}{'}\cdots\dOEgE{p}{'}}{\Mat{I}}$}
% \end{center}
% \begin{center}
%   |\dTrCp{ \OpE{\su{5}{i}{j}}\OpE{\pr{-7}{j}} }{\Mat{A}}|
%   \hspace{1.2cm}
%   \fbox{$\dTrCp{\OpE{\su{5}{i}{j}}\OpE{\pr{-7}{j}}}{\Mat{A}}$}
% \end{center}
% 
% \begin{center}
%   |\dTrCP{ \dOEgE{1}{'}\cdots\dOEgE{p}{'} }{\Mat{I}}|
%   \hspace{1.2cm}
%   \fbox{$\dTrCP{\dOEgE{1}{'}\cdots\dOEgE{p}{'}}{\Mat{I}}$}
% \end{center}
% \begin{center}
%   |\dTrCP{ \OpE{\su{5}{i}{j}}\OpE{\pr{-7}{j}} }{\Mat{A}}|
%   \hspace{1.2cm}
%   \fbox{$\dTrCP{\OpE{\su{5}{i}{j}}\OpE{\pr{-7}{j}}}{\Mat{A}}$}
% \end{center}
% 
% \begin{center}
%   |\dTrCpE{ \dOEgE{1}{'}\cdots\dOEgE{p}{'} }{\Mat{I}}|
%   \hspace{1.2cm}
%   \fbox{$\dTrCpE{\dOEgE{1}{'}\cdots\dOEgE{p}{'}}{\Mat{I}}$}
% \end{center}
% \begin{center}
%   |\dTrCpE{ \OpE{\su{5}{i}{j}}\OpE{\pr{-7}{j}} }{\Mat{A}}|
%   \hspace{1.2cm}
%   \fbox{$\dTrCpE{\OpE{\su{5}{i}{j}}\OpE{\pr{-7}{j}}}{\Mat{A}}$}
% \end{center}
% 
% \begin{center}
%   |\dTrCPE{ \dOEgE{1}{'}\cdots\dOEgE{p}{'} }{\Mat{I}}|
%   \hspace{1.2cm}
%   \fbox{$\dTrCPE{\dOEgE{1}{'}\cdots\dOEgE{p}{'}}{\Mat{I}}$}
% \end{center}
% \begin{center}
%   |\dTrCPE{ \OpE{\su{5}{i}{j}}\OpE{\pr{-7}{j}} }{\Mat{A}}|
%   \hspace{1.2cm}
%   \fbox{$\dTrCPE{\OpE{\su{5}{i}{j}}\OpE{\pr{-7}{j}}}{\Mat{A}}$}
% \end{center}
% 
% \DescribeMacro{\dTrFC}
% \DescribeMacro{\dTrFCp}
% \DescribeMacro{\dTrFCP}
% \DescribeMacro{\dTrFCpE}
% \DescribeMacro{\dTrFCPE}
% El comando \cs{dTrFC} tiene 3
% argumentos,\;\cs{dTrFC}\marg{operacionesIzda}\marg{operacionesDcha}\marg{objeto},\;
% e indica una transformación (o transformaciones) elemental(es) por
% cada lado del objeto.
% \begin{center}
%   |\dTrFC{\OpE{\su{-5}{i}{j}}}{\OpE{\pr{-7}{j}}}{\Mat{A}}|
%   \hspace{1.2cm}
%   \fbox{$\dTrFC{\OpE{\su{-5}{i}{j}}}{\OpE{\pr{-7}{j}}}{\Mat{A}}$}
% \end{center}
% 
% \begin{center}
%   |\dTrFCp{\OpE{\su{-5}{i}{j}}}{\OpE{\pr{-7}{j}}}{\Mat{A}}|
%   \hspace{1.2cm}
%   \fbox{$\dTrFCp{\OpE{\su{-5}{i}{j}}}{\OpE{\pr{-7}{j}}}{\Mat{A}}$}
% \end{center}
% 
% \begin{center}
%   |\dTrFCP{\OpE{\su{-5}{i}{j}}}{\OpE{\pr{-7}{j}}}{\Mat{A}}|
%   \hspace{1.2cm}
%   \fbox{$\dTrFCP{\OpE{\su{-5}{i}{j}}}{\OpE{\pr{-7}{j}}}{\Mat{A}}$}
% \end{center}
% 
% \begin{center}
%   |\dTrFCpE{\OpE{\su{-5}{i}{j}}}{\OpE{\pr{-7}{j}}}{\Mat{A}}|
%   \hspace{1.2cm}
%   \fbox{$\dTrFCpE{\OpE{\su{-5}{i}{j}}}{\OpE{\pr{-7}{j}}}{\Mat{A}}$}
% \end{center}
% 
% \begin{center}
%   |\dTrFCPE{\OpE{\su{-5}{i}{j}}}{\OpE{\pr{-7}{j}}}{\Mat{A}}|
%   \hspace{1.2cm}
%   \fbox{$\dTrFCPE{\OpE{\su{-5}{i}{j}}}{\OpE{\pr{-7}{j}}}{\Mat{A}}$}
% \end{center}
% 
% \subsubsection{Operador que quita un elemento}
% 
% \DescribeMacro{\fueraitemL}
% El comando \cs{fueraitemL} tiene 1
% argumento,\;\cs{fueraitemL}\marg{indice},\; y denota la eliminación por
% la izquierda del elemento correspondiente al \marg{indice}
% \begin{center}
%   |\fueraitemL{i}|
%   \hspace{1.2cm}
%   \fbox{$\fueraitemL{i}$}
% \end{center}
% 
% \DescribeMacro{\fueraitemR}
% El comando \cs{fueraitemR} tiene 1
% argumento,\;\cs{fueraitemR}\marg{indice},\; y denota la eliminación por
% la derecha del elemento correspondiente al \marg{indice}
% \begin{center}
%   |\fueraitemR{j}|
%   \hspace{1.2cm}
%   \fbox{$\fueraitemR{j}$}
% \end{center}
% 
% \DescribeMacro{\quitaLR}
% El comando \cs{quitaLR} tiene 3
% argumentos,\;\cs{quitaLR}\marg{objeto}\marg{indIzda}\marg{indDcha},\;
% y denota el resultante de quitar un elemento por la izquierda y otro
% por la derecha
% \begin{center}
%   |\quitaLR{\Mat{A}}{i}{j}|
%   \hspace{1.2cm}
%   \fbox{$\quitaLR{\Mat{A}}{i}{j}$}
% \end{center}
% 
% \DescribeMacro{\quitaL}
% El comando \cs{quitaL} tiene 2
% argumentos,\;\cs{quitaL}\marg{objeto}\marg{indIzda},\;
% y denota el resultante de quitar un elemento por la izquierda
% \begin{center}
%   |\quitaL{\Mat{A}}{i}|
%   \hspace{1.2cm}
%   \fbox{$\quitaL{\Mat{A}}{i}$}
% \end{center}
% 
% \DescribeMacro{\quitaR}
% El comando \cs{quitaR} tiene 2
% argumentos,\;\cs{quitaR}\marg{objeto}\marg{indDcha},\;
% y denota el resultante de quitar un elemento por la derecha
% \begin{center}
%   |\quitaR{\Mat{A}}{j}|
%   \hspace{1.2cm}
%   \fbox{$\quitaR{\Mat{A}}{j}$}
% \end{center}
% 
% \subsubsection{Selección de elementos sin emplear el operador selector}
% 
% \DescribeMacro{\elemUUU}
% El comando \cs{elemUUU} tiene 2
% argumentos,\;\cs{elemUUU}\marg{sistema}\marg{indice},\; y denota la
% selección del elemento correspondiente al \marg{indice}
% \begin{center}
%   |\elemUUU{\SV{Z}}{i}|
%   \hspace{1.2cm}
%   \fbox{$\elemUUU{\SV{Z}}{i}$}
% \end{center}
% 
% \DescribeMacro{\VectFFF}
% \DescribeMacro{\VectFFFT}
% El comando \cs{VectFFF} tiene 2
% argumentos,\;\cs{VectFFF}\marg{nombre}\marg{indice},\; y denota la
% selección de la fila correspondiente al \marg{indice}
% \begin{center}
%   |\VectFFF{A}{i} \VectFFFT{A}{i}|
%   \hspace{1.2cm}
%   \fbox{$\VectFFF {A}{i}$}
%   \fbox{$\VectFFFT{A}{i}$}
% \end{center}
% 
% \DescribeMacro{\VectCCC}
% \DescribeMacro{\VectCCCT}
% El comando \cs{VectCCC} tiene 2
% argumentos,\;\cs{VectCCC}\marg{nombre}\marg{indice},\; y denota la
% selección de la columna correspondiente al \marg{indice}
% \begin{center}
%   |\VectCCC{A}{i} \VectCCCT{A}{i}|
%   \hspace{1.2cm}
%   \fbox{$\VectCCC {A}{i}$}
%   \fbox{$\VectCCCT{A}{i}$}
% \end{center}
% 
% \DescribeMacro{\eleVVV}
% \DescribeMacro{\eleVV}
%  tiene 2
% argumentos,\;\marg{nombre}\marg{indice},\; y denota la
% selección del elemento de un vector correspondiente al índice indicado
% \begin{center}
%   |\eleVVV{A}{i} \eleVV{A}{i}|
%   \hspace{1.2cm}
%   \fbox{$\eleVVV {A}{i}$}
%   \fbox{$\eleVV  {A}{i}$}
% \end{center}
% 
% \DescribeMacro{\eleMMM}
% \DescribeMacro{\eleMMMT}
% \DescribeMacro{\eleMM}
%  tiene 3
% argumentos,\;\marg{nombre}\marg{indiceFil}\marg{indiceCol},\; y denota la
% selección del elemento de una matriz correspondiente a los índices indicados
% \begin{center}
%   |\eleMMM{A}{i}{j} \eleMMMT{A}{i}{j} \eleMM{A}{i}{j}|
%   \hspace{1.2cm}
%   \fbox{$\eleMMM {A}{i}{j}$}
%   \fbox{$\eleMMMT{A}{i}{j}$}
%   \fbox{$\eleMM  {A}{i}{j}$}
% \end{center}
% 
% %%%%%%%%%%%%%%%%%%%%%%%%%%%%%%%%%%%%%%
% \subsection{Sistemas genéricos}
% 
% \DescribeMacro{\SV}
% El comando \cs{SV} tiene 2 argumentos,\;\cs{SV}\oarg{subíndice}\marg{nombre}
% \begin{center}
%   |\SV{A} \SV[h]{A}|
%   \hspace{1.2cm}
%   \fbox{$\SV   {A}$}
%   \fbox{$\SV[h]{A}$}
% \end{center}
% 
% \DescribeMacro{\concatSV}
% El comando \cs{concatSV} tiene 2
% argumentos,\;\cs{concatSV}\marg{sistemaA}\marg{sistemaB},\; y denota
% la concatenación del \marg{sistemaA} con el \marg{sistemaB}.
% \begin{center}
%   |\concatSV{\Mat{A}}{\Mat{B}}|
%   \hspace{1.2cm}
%   \fbox{$\concatSV{\Mat{A}}{\Mat{B}}$}
% \end{center}
% 
% %%%%%%%%%%%%%%%%%%%%%%%%%%%%%%%%%%%%%%
% 
% \subsection{Vectores y matrices}
% 
% \subsubsection{Vectores genéricos}
% 
% \DescribeMacro{\vect}
% \DescribeMacro{\vectp}
% \DescribeMacro{\vectp*}
% \DescribeMacro{\vectP}
% \DescribeMacro{\vectP*}
% tiene 2 argumentos,\;|vect<X*>|\oarg{subíndice}\marg{nombre},\; y denota un vector
% genérico.
% 
% \begin{center}
%   |\vect{a} \vect[h]{a}|
%   \hspace{1.2cm}
%   \fbox{$\vect   {a}$}
%   \fbox{$\vect[h]{a}$}
% \end{center}
% 
% \begin{center}
%   |\vectp{a} \vectp*{a} \vectP{a} \vectP*{a}|
%   \hspace{1.2cm}
%   \fbox{$\vectp {a}$}
%   \fbox{$\vectp*{a}$}
%   \fbox{$\vectP {a}$}
%   \fbox{$\vectP*{a}$}
% \end{center}
% 
% \begin{center}
%   |\vectp[h]{a} \vectp*[h]{a} \vectP[h]{a} \vectP*[h]{a}|
%   \hspace{1.2cm}
%   \fbox{$\vectp [h]{a}$}
%   \fbox{$\vectp*[h]{a}$}
%   \fbox{$\vectP [h]{a}$}
%   \fbox{$\vectP*[h]{a}$}
% \end{center}
% 
% %%%%%%%%%%%%%%%%%%%%%%%%%%%%%%%%%%%%%%
% \subsubsection{Vectores de $\R[n]$}
% 
% \DescribeMacro{\Vect}
% \DescribeMacro{\Vectp}
% \DescribeMacro{\Vectp*}
% \DescribeMacro{\VectP}
% \DescribeMacro{\VectP*}
% tiene 3 argumentos,\;|Vect<X*>|\oarg{subíndice}\oarg{superíndice}\marg{nombre},\; y denota un vector de
% $\R[n]$
% \begin{center}
%   |\Vect{a} \Vect[h]{a}  \Vect[h][2]{a}|
%   \hspace{1.2cm}
%   \fbox{$\Vect      {a}$}
%   \fbox{$\Vect[h]   {a}$}
%   \fbox{$\Vect[h][2]{a}$}
% \end{center}
% 
% \begin{center}
%   |\Vectp{a} \Vectp*[][2]{a} \VectP{a} \VectP*{a}|
%   \hspace{1.2cm}
%   \fbox{$\Vectp {a}$}
%   \fbox{$\Vectp*[][2]{a}$}
%   \fbox{$\VectP {a}$}
%   \fbox{$\VectP*{a}$}
% \end{center}
% 
% \begin{center}
%   |\Vectp[h]{a} \Vectp*[h][2]{a} \VectP[h]{a} \VectP*[h]{a}|
%   
%   \fbox{$\Vectp [h]{a}$}
%   \fbox{$\Vectp*[h][2]{a}$}
%   \fbox{$\VectP [h]{a}$}
%   \fbox{$\VectP*[h]{a}$}
% \end{center}
% 
% \DescribeMacro{\irvec}
% tiene 3 argumentos,\;|irvec|\oarg{subíndiceInic}\oarg{subíndiceFin}\marg{nombre},\; y escribe  una sucesión de vectores de $\R[n]$
% \begin{center}
%   |\irvec{a} \irvec[p]{a} \irvec[p][q]{a}|
%   \hspace{0.2cm}
%   \fbox{$\irvec      {a}$}
%   \fbox{$\irvec[p]   {a}$}
%   \fbox{$\irvec[p][q]{a}$}
% \end{center}
% 
% \DescribeMacro{\irvec}C
% tiene 3 argumentos,\;|irvec|\oarg{subíndiceInic}\oarg{subíndiceFin}\marg{nombre},\; y escribe  una sucesión de columnas de una matriz
% \begin{center}
%   |\irvecC{a} \irvecC[p]{a} \irvecC[p][q]{a}|
%   \hspace{0.2cm}
%   \fbox{$\irvecC      {a}$}
%   \fbox{$\irvecC[p]   {a}$}
%   \fbox{$\irvecC[p][q]{a}$}
% \end{center}
% 
% \subsubsection{Matrices}
% 
% \DescribeMacro{\Mat}
% tiene 2 argumentos,\;|Mat<X*>|\oarg{subíndice}\marg{nombre},\; y denota una matriz
% \begin{center}
%   |\Mat{A} \Mat[h]{A} \Mat[h]{A}^2|
%   \hspace{1.2cm}
%   \fbox{$\Mat   {A}$}
%   \fbox{$\Mat[h]{A}$}
%   \fbox{$\Mat[h]{A}^2$}
% \end{center}
% \begin{center}
%   |\Matp{A} \Matp*{A} \MatP{A} \MatP*{A}|
%   \hspace{1.2cm}
%   \fbox{$\Matp {A}$}
%   \fbox{$\Matp*{A}$}
%   \fbox{$\MatP {A}$}
%   \fbox{$\MatP*{A}$}
% \end{center}
% \begin{center}
%   |\Matp{A}[h] \Matp*{A}[h] \MatP{A}[h] \MatP*{A}[h]|
%   \hspace{1.2cm}
%   \fbox{$\Matp [h]{A}$}
%   \fbox{$\Matp*[h]{A}$}
%   \fbox{$\MatP [h]{A}$}
%   \fbox{$\MatP*[h]{A}$}
% \end{center}
% 
% %%%%%%%%%%%%%%%%%%%%%%%%%%%%%%%%%%%%%%
% \paragraph{Matrices transpuestas.}{\mbox{ }}
% 
% \DescribeMacro{\MatT}
% \DescribeMacro{\MatTp}
% \DescribeMacro{\MatTp*}
% \DescribeMacro{\MatTP}
% \DescribeMacro{\MatTP*}
% \DescribeMacro{\MatTpE}
% \DescribeMacro{\MatTpE*}
% \DescribeMacro{\MatTPE}
% \DescribeMacro{\MatTPE*}
% El comando |MatT<XX*>| tiene 2 argumentos,\;|MatT<XX*>|\oarg{subíndice}\marg{nombre}
% \begin{center}
%   |\MatT{A} \MatT[h]{A}|
%   \hspace{1.2cm}
%   \fbox{$\MatT   {A}$}
%   \fbox{$\MatT[h]{A}$}
% \end{center}
% \begin{center}
%   |\MatTp{A} \MatTp*{A} \MatTp[h]{A} \MatTp*[h]{A}|
%   \hspace{1.2cm}
%   \fbox{$\MatTp    {A}$}
%   \fbox{$\MatTp*   {A}$}
%   \fbox{$\MatTp [h]{A}$}
%   \fbox{$\MatTp*[h]{A}$}
% \end{center}
% \begin{center}
%   |\MatTP{A} \MatTP*{A} \MatTP[h]{A} \MatTP*[h]{A}|
%   \hspace{1.2cm}
%   \fbox{$\MatTP    {A}$}
%   \fbox{$\MatTP*   {A}$}
%   \fbox{$\MatTP [h]{A}$}
%   \fbox{$\MatTP*[h]{A}$}
% \end{center}
% \begin{center}
%   |\MatTpE{A} \MatTpE*{A} \MatTpE[h]{A} \MatTpE*[h]{A}|
%   \hspace{1.2cm}
%   \fbox{$\MatTpE    {A}$}
%   \fbox{$\MatTpE*   {A}$}
%   \fbox{$\MatTpE [h]{A}$}
%   \fbox{$\MatTpE*[h]{A}$}
% \end{center}
% \begin{center}
%   |\MatTPE{A} \MatTPE*{A} \MatTPE[h]{A} \MatTPE*[h]{A}|
%   \hspace{1.2cm}
%   \fbox{$\MatTPE    {A}$}
%   \fbox{$\MatTPE*   {A}$}
%   \fbox{$\MatTPE [h]{A}$}
%   \fbox{$\MatTPE*[h]{A}$}
% \end{center}
% 
% %%%%%%%%%%%%%%%%%%%%%%%%%%%%%%%%%%%%%%
% \subparagraph{Matriz transpuesta de la transpuesta.}{\mbox{ }}
% 
% \DescribeMacro{\MatTT}
% \DescribeMacro{\MatTT*}
% \DescribeMacro{\MatTTPE}
% \DescribeMacro{\MatTTPE*}
% El comando \cs{MatTT} tiene 2 argumentos,\;|MatTT<X*>|\oarg{subíndice}\marg{nombre}
% \begin{center}
%   |\MatTT{A} \MatTT*{A} \MatTT[h]{A} \MatTT*[h]{A}|
%   \hspace{0.7cm}
%   \fbox{$\MatTT    {A}$}
%   \fbox{$\MatTT*   {A}$}
%   \fbox{$\MatTT [h]{A}$}
%   \fbox{$\MatTT*[h]{A}$}
% \end{center}
% 
% \begin{center}
%   |\MatTTPE{A} \MatTTPE*{A}|
%   \hspace{0.7cm}
%   \fbox{$\MatTTPE    {A}$}
%   \fbox{$\MatTTPE*   {A}$}
% \end{center}
% 
% \begin{center}
%   |\MatTTPE[h]{A} \MatTTPE*[h]{A}|
%   \hspace{0.7cm}
%   \fbox{$\MatTTPE [h]{A}$}
%   \fbox{$\MatTTPE*[h]{A}$}
% \end{center}
% 
% %%%%%%%%%%%%%%%%%%%%%%%%%%%%%%%%%%%%%%
% \paragraph{Matrices columna}{\mbox{ }}
% 
% \DescribeMacro{\MVect}
% \DescribeMacro{\MVect*}
% El comando \cs{MVect} tiene 2 argumentos,\;\cs{MVect}\oarg{subíndice}\marg{nombre},\;
% y denota una matriz columna creada a partir de un vector
% \begin{center}
%   |\MVect{a} \MVect*{a}|
%   \hspace{1.2cm}
%   \fbox{$\MVect {a}$}
%   \fbox{$\MVect*{a}$}
% \end{center}
% \begin{center}
%   |\MVectF[h]{a} \MVectF*[h]{a}|
%   \hspace{1.2cm}
%   \fbox{$\MVect [h]{a}$}
%   \fbox{$\MVect*[h]{a}$}
% \end{center}
% 
% \DescribeMacro{\MVectF}
% \DescribeMacro{\MVectF*}
% El comando \cs{MVectF} tiene 3 argumentos,\;\cs{MVectF}\oarg{subíndice}\marg{nombre}\marg{índice},\;
% y denota una matriz columna creada a partir de una \emph{fila} de una matriz
% \begin{center}
%   |\MVectF{A}{j} \MVectF*{A}{j}|
%   \hspace{1.2cm}
%   \fbox{$\MVectF    {A}{j}$}
%   \fbox{$\MVectF*   {A}{j}$}
% \end{center}
% \begin{center}
%   |\MVectF[h]{A}{j} \MVectF*[h]{A}{j}|
%   \hspace{1.2cm}
%   \fbox{$\MVectF [h]{A}{j}$}
%   \fbox{$\MVectF*[h]{A}{j}$}
% \end{center}
% 
% \DescribeMacro{\MVectC}
% \DescribeMacro{\MVectC*}
% El comando \cs{MVectC} tiene 3 argumentos,\;\cs{MVectC}\oarg{subíndice}\marg{nombre}\marg{índice},\;
% y denota una matriz columna creada a partir de una \emph{columna} de una matriz
% \begin{center}
%   |\MVectC{A}{j} \MVectC*{A}{j}|
%   \hspace{1.2cm}
%   \fbox{$\MVectC    {A}{j}$}
%   \fbox{$\MVectC*   {A}{j}$}
% \end{center}
% \begin{center}
%   |\MVectC[h]{A}{j} \MVectC*[h]{A}{j}|
%   \hspace{1.2cm}
%   \fbox{$\MVectC [h]{A}{j}$}
%   \fbox{$\MVectC*[h]{A}{j}$}
% \end{center}
% 
% %%%%%%%%%%%%%%%%%%%%%%%%%%%%%%%%%%%%%%
% \paragraph{Matrices fila}{\mbox{ }}
% 
% \DescribeMacro{\MVectT}
% \DescribeMacro{\MVectT*}
% El comando \cs{MVectT} tiene 2 argumentos,\;\cs{MVectT}\oarg{subíndice}\marg{nombre},\;
% y denota una matriz fila creada a partir de un vector
% \begin{center}
%   |\MVectT{a} \MVectT*{a}|
%   \hspace{1.2cm}
%   \fbox{$\MVectT {a}$}
%   \fbox{$\MVectT*{a}$}
% \end{center}
% \begin{center}
%   |\MVectT[h]{a} \MVectT*[h]{a}|
%   \hspace{1.2cm}
%   \fbox{$\MVectT [h]{a}$}
%   \fbox{$\MVectT*[h]{a}$}
% \end{center}
% 
% 
% \DescribeMacro{\MVectFT}
% \DescribeMacro{\MVectFT*}
% El comando \cs{MVectFT} tiene 3 argumentos,\;\cs{MVectFT}\oarg{subíndice}\marg{nombre}\marg{índice},\;
% y denota una matriz fila creada a partir de una \emph{fila} de una matriz
% \begin{center}
%   |\MVectFT{A}{j} \MVectFT*{A}{j}|
%   \hspace{1.2cm}
%   \fbox{$\MVectFT    {A}{j}$}
%   \fbox{$\MVectFT*   {A}{j}$}
% \end{center}
% \begin{center}
%   |\MVectFT[h]{A}{j} \MVectFT*[h]{A}{j}|
%   \hspace{1.2cm}
%   \fbox{$\MVectFT [h]{A}{j}$}
%   \fbox{$\MVectFT*[h]{A}{j}$}
% \end{center}
% 
% \DescribeMacro{\MVectCT}
% \DescribeMacro{\MVectCT*}
% El comando \cs{MVectCT} tiene 3 argumentos,\;\cs{MVectCT}\oarg{subíndice}\marg{nombre}\marg{índice},\;
% y denota una matriz fila creada a partir de una \emph{columna} de una matriz
% \begin{center}
%   |\MVectCT{A}{j} \MVectCT*{A}{j}|
%   \hspace{1.2cm}
%   \fbox{$\MVectCT    {A}{j}$}
%   \fbox{$\MVectCT*   {A}{j}$}
% \end{center}
% \begin{center}
%   |\MVectCT[h]{A}{j} \MVectCT*[h]{A}{j}|
%   \hspace{1.2cm}
%   \fbox{$\MVectCT [h]{A}{j}$}
%   \fbox{$\MVectCT*[h]{A}{j}$}
% \end{center}
% %%%%%%%%%%%%%%%%%%%%%%%%%%%%%%%%%%%%%%
% \paragraph{Matriz inversa} Notación para las matrices inversas
% 
% \DescribeMacro{\InvMat}
% \DescribeMacro{\InvMatp}
% \DescribeMacro{\InvMatp*}
% \DescribeMacro{\InvMatP}
% \DescribeMacro{\InvMatP*}
% \DescribeMacro{\InvMatpE}
% \DescribeMacro{\InvMatpE*}
% \DescribeMacro{\InvMatPE}
% \DescribeMacro{\InvMatPE*}
% El comando \cs{InvMat} tiene 2
% argumentos,\;|InvMat<XX*>|\oarg{índice}\marg{nombre},\; y denota la
% inversa de una matriz
% \begin{center}
%   |\InvMat{A} \InvMat[h]{A}|
%   \hspace{1.2cm}
%   \fbox{$\InvMat{A}$}
%   \fbox{$\InvMat[h]{A}$}
% \end{center}
% 
% \begin{center}
%   |\InvMatp{A} \InvMatp*{A}|
%   \hspace{1.2cm}
%   \fbox{$\InvMatp    {A}$}
%   \fbox{$\InvMatp*   {A}$}
% \end{center}
% 
% \begin{center}
%   |\InvMatp[h]{A} \InvMatp*[h]{A}|
%   \hspace{1.2cm}
%   \fbox{$\InvMatp [h]{A}$}
%   \fbox{$\InvMatp*[h]{A}$}
% \end{center}
% 
% \begin{center}
%   |\InvMatP{A} \InvMatP*{A}|
%   \hspace{1.2cm}
%   \fbox{$\InvMatP    {A}$}
%   \fbox{$\InvMatP*   {A}$}
% \end{center}
% 
% \begin{center}
%   |\InvMatP[h]{A} \InvMatP*[h]{A}|
%   \hspace{1.2cm}
%   \fbox{$\InvMatP [h]{A}$}
%   \fbox{$\InvMatP*[h]{A}$}
% \end{center}
% 
% \begin{center}
%   |\InvMatpE{A} \InvMatpE*{A}|
%   \hspace{1.2cm}
%   \fbox{$\InvMatpE    {A}$}
%   \fbox{$\InvMatpE*   {A}$}
% \end{center}
% 
% \begin{center}
%   |\InvMatpE[h]{A} \InvMatpE*[h]{A}|
%   \hspace{1.2cm}
%   \fbox{$\InvMatpE [h]{A}$}
%   \fbox{$\InvMatpE*[h]{A}$}
% \end{center}
% 
% \begin{center}
%   |\InvMatPE{A} \InvMatPE*{A}|
%   \hspace{1.2cm}
%   \fbox{$\InvMatPE    {A}$}
%   \fbox{$\InvMatPE*   {A}$}
% \end{center}
% 
% \begin{center}
%   |\InvMatPE[h]{A} \InvMatPE*[h]{A}|
%   \hspace{1.2cm}
%   \fbox{$\InvMatPE [h]{A}$}
%   \fbox{$\InvMatPE*[h]{A}$}
% \end{center}
% 
% \DescribeMacro{\InvMatT}
% \DescribeMacro{\InvMatTpE}
% \DescribeMacro{\InvMatTpE*}
% \DescribeMacro{\InvMatTPE}
% \DescribeMacro{\InvMatTPE*}
% El comando \cs{InvMatT} tiene 2
% argumentos,\;|InvMatT<XX*>|\oarg{índice}\marg{nombre},\; y denota la
% inversa de una matriz transpuesta
% \begin{center}
%   |\InvMatT{A} \InvMatT*{A}|
%   \hspace{1.2cm}
%   \fbox{$\InvMatT    {A}$}
%   \fbox{$\InvMatT*   {A}$}
% \end{center}
% 
% \begin{center}
%   |\InvMatT[h]{A} \InvMatT*[h]{A}|
%   \hspace{1.2cm}
%   \fbox{$\InvMatT [h]{A}$}
%   \fbox{$\InvMatT*[h]{A}$}
% \end{center}
% 
% \begin{center}
%   |\InvMatTpE{A} \InvMatTpE*{A}|
%   \hspace{1.2cm}
%   \fbox{$\InvMatTpE    {A}$}
%   \fbox{$\InvMatTpE*   {A}$}
% \end{center}
% 
% \begin{center}
%   |\InvMatTpE[h]{A} \InvMatTpE*[h]{A}|
%   \hspace{1.2cm}
%   \fbox{$\InvMatTpE [h]{A}$}
%   \fbox{$\InvMatTpE*[h]{A}$}
% \end{center}
% 
% \begin{center}
%   |\InvMatTPE{A} \InvMatTPE*{A}|
%   \hspace{1.2cm}
%   \fbox{$\InvMatTPE    {A}$}
%   \fbox{$\InvMatTPE*   {A}$}
% \end{center}
% 
% \begin{center}
%   |\InvMatTPE[h]{A} \InvMatTPE*[h]{A}|
%   \hspace{1.2cm}
%   \fbox{$\InvMatTPE [h]{A}$}
%   \fbox{$\InvMatTPE*[h]{A}$}
% \end{center}
% 
% \DescribeMacro{\TInvMat}
% \DescribeMacro{\TInvMatpE}
% \DescribeMacro{\TInvMatpE*}
% \DescribeMacro{\TInvMatPE}
% \DescribeMacro{\TInvMatPE*}
% El comando \cs{TInvMat} tiene 2 argumentos,\;|TInvMat<XX*>|\oarg{índice}\marg{nombre},\;
% y denota la transpuesta de la inversa de una matriz
% \begin{center}
%   |\TInvMat{A} \TInvMat*{A}|
%   \hspace{1.2cm}
%   \fbox{$\TInvMat    {A}$}
%   \fbox{$\TInvMat*   {A}$}
% \end{center}
% 
% \begin{center}
%   |\TInvMat[h]{A} \TInvMat*[h]{A}|
%   \hspace{1.2cm}
%   \fbox{$\TInvMat [h]{A}$}
%   \fbox{$\TInvMat*[h]{A}$}
% \end{center}
% 
% \begin{center}
%   |\TInvMatpE{A} \TInvMatpE*{A}|
%   \hspace{1.2cm}
%   \fbox{$\TInvMatpE    {A}$}
%   \fbox{$\TInvMatpE*   {A}$}
% \end{center}
% 
% \begin{center}
%   |\TInvMatpE[h]{A} \TInvMatpE*[h]{A}|
%   \hspace{1.2cm}
%   \fbox{$\TInvMatpE [h]{A}$}
%   \fbox{$\TInvMatpE*[h]{A}$}
% \end{center}
% 
% \begin{center}
%   |\TInvMatPE{A} \TInvMatPE*{A}|
%   \hspace{1.2cm}
%   \fbox{$\TInvMatPE    {A}$}
%   \fbox{$\TInvMatPE*   {A}$}
% \end{center}
% 
% \begin{center}
%   |\TInvMatPE[h]{A} \TInvMatPE*[h]{A}|
%   \hspace{1.2cm}
%   \fbox{$\TInvMatPE [h]{A}$}
%   \fbox{$\TInvMatPE*[h]{A}$}
% \end{center}
% 
% %%%%%%%%%%%%%%%%%%%%%%%%%%%%%%%%%%%%%%
% \subsubsection{Miscelánea matrices}
% 
% \DescribeMacro{\Traza}
% El comando \cs{Traza} no  tiene
% argumentos
% \begin{center}
%   |\Traza|
%   \hspace{1.2cm}
%   \fbox{$\Traza$}
% \end{center}
% 
% \DescribeMacro{\rg}
% El comando \cs{rg} no  tiene
% argumentos
% \begin{center}
%   |\rg|
%   \hspace{1.2cm}
%   \fbox{$\rg$}
% \end{center}
% 
% \DescribeMacro{\traza}
% \DescribeMacro{\traza*}
% El comando \cs{traza} tiene 1
% argumento,\;\cs{traza}\marg{objeto}
% \begin{center}
%   |\traza{\Mat{A}} \traza*{\Mat{A}}|
%   \hspace{1.2cm}
%   \fbox{$\traza {\Mat{A}}$}
%   \fbox{$\traza*{\Mat{A}}$}
% \end{center}
% 
% \DescribeMacro{\rango}
% \DescribeMacro{\rango*}
% El comando \cs{rango} tiene 1
% argumento,\;\cs{rango}\marg{objeto}
% \begin{center}
%   |\rango{\Mat{A}} \rango*{\Mat{A}}|
%   \hspace{1.2cm}
%   \fbox{$\rango {\Mat{A}}$}
%   \fbox{$\rango*{\Mat{A}}$}
% \end{center}
% 
% %%%%%%%%%%%%%%%%%%%%%%%%%%%%%%%%%%%%%%
% \paragraph{Determinante de una matriz}{\mbox{ } }
% 
% \DescribeMacro{\cof}
% El comando \cs{cof} no  tiene
% argumentos
% \begin{center}
%   |\cof|
%   \hspace{1.2cm}
%   \fbox{$\cof$}
% \end{center}
% 
% \DescribeMacro{\adj}
% El comando \cs{adj} no  tiene
% argumentos
% \begin{center}
%   |\adj|
%   \hspace{1.2cm}
%   \fbox{$\adj$}
% \end{center}
% 
% \DescribeMacro{\determinante}
% \DescribeMacro{\determinante*}
% El comando \cs{determinante} tiene 1
% argumento,\;\cs{determinante}\marg{objeto},\; y denota el determinante
% del \marg{objeto} usando las barras verticales
% \begin{center}
%   |\determinante{\Mat{A}} \determinante*{\Mat{A}}|
%   \hspace{1.2cm}
%   \fbox{$\determinante {\Mat{A}}$}
%   \fbox{$\determinante*{\Mat{A}}$}
% \end{center}
% 
% \DescribeMacro{\detp}
% \DescribeMacro{\detp*}
% El comando \cs{detp} tiene 1
% argumento,\;\cs{detp}\marg{objeto},\; y denota el determinante
% del \marg{objeto} usando la abreviatura det y paréntesis
% \begin{center}
%   |\detp{\Mat{A}} \detp*{\Mat{A}}|
%   \hspace{1.2cm}
%   \fbox{$\detp {\Mat{A}}$}
%   \fbox{$\detp*{\Mat{A}}$}
% \end{center}
% 
% \DescribeMacro{\detP}
% \DescribeMacro{\detP*}
% El comando \cs{detP} tiene 1
% argumento,\;\cs{detP}\marg{objeto},\; y denota el determinante
% del \marg{objeto} usando la abreviatura det y Paréntesis
% \begin{center}
%   |\detP{\Mat{A}} \detP*{\Mat{A}}|
%   \hspace{1.2cm}
%   \fbox{$\detP {\Mat{A}}$}
%   \fbox{$\detP*{\Mat{A}}$}
% \end{center}
% 
% \DescribeMacro{\subMat}
% El comando \cs{subMat} tiene 3
% argumentos,\;\cs{subMat}\marg{nombre}\marg{indIzda}\marg{indDcha},\; y denota la submatriz resultante de quitar una o más filas y columnas de la matriz \marg{nombre}
% \begin{center}
%   |\subMat{A}{i}{j}|
%   \hspace{1.2cm}
%   \fbox{$\subMat{A}{i}{j}$}
% \end{center}
% 
% \DescribeMacro{\Menor}
% \DescribeMacro{\MenorR}
% tiene 3 argumentos,\;\marg{nombre}\marg{indFila}\marg{indCol},\; y denota el menor de la matriz correspondiente a la fila y columna indicadas
% \begin{center}
%   |\Menor{A}{i}{j} \MenoR{A}{i}{j}|
%   \hspace{1.2cm}
%   \fbox{$\Menor{A}{i}{j}$}
%   \fbox{$\MenoR{A}{i}{j}$}
% \end{center}
% 
% \DescribeMacro{\Cof}
% \DescribeMacro{\Cof*}
% El comando \cs{Cof} tiene 3
% argumentos,\;\cs{Cof}\marg{nombre}\marg{indFila}\marg{indCol},\; y denota el cofactor de la fila y columna indicadas
% \begin{center}
%   |\Cof{A}{i}{j} \Cof*{A}{i}{j}|
%   \hspace{1.2cm}
%   \fbox{$\Cof {A}{i}{j}$}
%   \fbox{$\Cof*{A}{i}{j}$}
% \end{center}
% 
% %%%%%%%%%%%%%%%%%%%%%%%%%%%%%%%%%%%%%%
% \paragraph{Orden de las matrices}
% 
% \DescribeMacro{\Dim}
% \DescribeMacro{\Dimp}
% \DescribeMacro{\Dimp*}
% \DescribeMacro{\DimP}
% \DescribeMacro{\DimP*}
% \DescribeMacro{\DimpE}
% \DescribeMacro{\DimpE*}
% \DescribeMacro{\DimPE}
% \DescribeMacro{\DimPE*}
% El comando \cs{Dim} tiene 3
% argumentos,\;\cs{Dim}\marg{objeto}\marg{filas}\marg{columnas}
% \begin{center}
%   |\Dim{xxx}{n}{m}|
%   \hspace{1.2cm}
%   \fbox{$\Dim{xxx}{n}{m}$}
% \end{center}
% 
% \begin{center}
%   |\Dimp{x}{n}{m} \Dimp*{x}{n}{m}|
%   \hspace{1.2cm}
%   \fbox{$\Dimp {x}{n}{m}$}
%   \fbox{$\Dimp*{x}{n}{m}$}
% \end{center}
% 
% \begin{center}
%   |\DimP{x}{n}{m} \DimP*{x}{n}{m}|
%   \hspace{1.2cm}
%   \fbox{$\DimP {x}{n}{m}$}
%   \fbox{$\DimP*{x}{n}{m}$}
% \end{center}
% 
% \begin{center}
%   |\DimpE{x}{n}{m} \DimpE*{x}{n}{m}|
%   \hspace{1.2cm}
%   \fbox{$\DimpE {x}{n}{m}$}
%   \fbox{$\DimpE*{x}{n}{m}$}
% \end{center}
% 
% \begin{center}
%   |\DimPE{x}{n}{m} \DimPE*{x}{n}{m}|
%   \hspace{1.2cm}
%   \fbox{$\DimPE {x}{n}{m}$}
%   \fbox{$\DimPE*{x}{n}{m}$}
% \end{center}
% 
% %%%%%%%%%%%%%%%%%%%%%%%%%%%%%%%%%%%%%%
% 
% \DescribeMacro{\Matdim}
% \DescribeMacro{\Matdim}
% \DescribeMacro{\Matdimp}
% \DescribeMacro{\Matdimp*}
% \DescribeMacro{\MatdimP}
% \DescribeMacro{\MatdimP*}
% \DescribeMacro{\MatdimpE}
% \DescribeMacro{\MatdimpE*}
% \DescribeMacro{\MatdimPE}
% \DescribeMacro{\MatdimPE*}
% El comando \cs{Matdim} tiene 3
% argumentos,\;|Matdim<XX*>|\marg{nombre}\marg{filas}\marg{columnas}
% \begin{center}
%   |\Matdim{xxx}{n}{m}|
%   \hspace{1.2cm}
%   \fbox{$\Matdim{xxx}{n}{m}$}
% \end{center}
% 
% \begin{center}
%   |\Matdimp{x}{n}{m} \Matdimp*{x}{n}{m}|
%   \hspace{1.2cm}
%   \fbox{$\Matdimp {x}{n}{m}$}
%   \fbox{$\Matdimp*{x}{n}{m}$}
% \end{center}
% 
% \begin{center}
%   |\MatdimP{x}{n}{m} \MatdimP*{x}{n}{m}|
%   \hspace{1.2cm}
%   \fbox{$\MatdimP {x}{n}{m}$}
%   \fbox{$\MatdimP*{x}{n}{m}$}
% \end{center}
% 
% \begin{center}
%   |\MatdimpE{x}{n}{m} \MatdimpE*{x}{n}{m}|
%   \hspace{1.2cm}
%   \fbox{$\MatdimpE {x}{n}{m}$}
%   \fbox{$\MatdimpE*{x}{n}{m}$}
% \end{center}
% 
% \begin{center}
%   |\MatdimPE{x}{n}{m} \MatdimPE*{x}{n}{m}|
%   \hspace{1.2cm}
%   \fbox{$\MatdimPE {x}{n}{m}$}
%   \fbox{$\MatdimPE*{x}{n}{m}$}
% \end{center}
% 
% \DescribeMacro{\MatTdim}
% \DescribeMacro{\MatTdim}
% \DescribeMacro{\MatTdimp}
% \DescribeMacro{\MatTdimp*}
% \DescribeMacro{\MatTdimP}
% \DescribeMacro{\MatTdimP*}
% \DescribeMacro{\MatTdimpE}
% \DescribeMacro{\MatTdimpE*}
% \DescribeMacro{\MatTdimPE}
% \DescribeMacro{\MatTdimPE*}
% El comando \cs{Matdim} tiene 3
% argumentos,\;|Matdim<XX*>|\marg{nombre}\marg{filas}\marg{columnas}
% \begin{center}
%   |\MatTdim{X}{n}{m}|
%   \hspace{1.2cm}
%   \fbox{$\MatTdim{xxx}{n}{m}$}
% \end{center}
% 
% \begin{center}
%   |\MatTdimp{x}{n}{m} \MatTdimp*{x}{n}{m}|
%   \hspace{1.2cm}
%   \fbox{$\MatTdimp {x}{n}{m}$}
%   \fbox{$\MatTdimp*{x}{n}{m}$}
% \end{center}
% 
% \begin{center}
%   |\MatTdimP{x}{n}{m} \MatTdimP*{x}{n}{m}|
%   \hspace{1.2cm}
%   \fbox{$\MatTdimP {x}{n}{m}$}
%   \fbox{$\MatTdimP*{x}{n}{m}$}
% \end{center}
% 
% \begin{center}
%   |\MatTdimpE{x}{n}{m} \MatTdimpE*{x}{n}{m}|
%   \hspace{1.2cm}
%   \fbox{$\MatTdimpE {x}{n}{m}$}
%   \fbox{$\MatTdimpE*{x}{n}{m}$}
% \end{center}
% 
% \begin{center}
%   |\MatTdimPE{x}{n}{m} \MatTdimPE*{x}{n}{m}|
%   \hspace{1.2cm}
%   \fbox{$\MatTdimPE {x}{n}{m}$}
%   \fbox{$\MatTdimPE*{x}{n}{m}$}
% \end{center}
% 
% %%%%%%%%%%%%%%%%%%%%%%%%%%%%%%%%%%%%%%
% \paragraph{Nombre de la matriz de autovalores}{\mbox{ } }
% 
% \DescribeMacro{\MDaV}
% \cs{MDaV} no  tiene
% argumentos e indica la letra usada par las matrices de autovalores
% \begin{center}
%   |\MDaV|
%   \hspace{1.2cm}
%   \fbox{$\MDaV$}
% \end{center}
% 
% \paragraph{Matriz triangular superior unitaria}  (según denominación de G. H. Golub y C. F. Van Loan)
% 
% \DescribeMacro{\UMat}
% \DescribeMacro{\InvUMat}
% \cs{Umat} y \cs{InvUmat} tienen 1 argumento opcional
% \begin{center}
%   |\UMat{A} \UMat[k]{A}|
%   \hspace{1.2cm}
%   \fbox{$\UMat{A}$}
%   \fbox{$\UMat[k]{A}$}
% \end{center}
% \begin{center}
%   |\InvUMat{A} \InvUMat[k]{A}|
%   \hspace{1.2cm}
%   \fbox{$\InvUMat{A}$}
%   \fbox{$\InvUMat[k]{A}$}
% \end{center}
% 
% \paragraph{Matriz triangular inferior unitaria}  (según denominación de G. H. Golub y C. F. Van Loan)
% 
% \DescribeMacro{\UMatT}
% \cs{UMatT} tiene 1 argumento opcional
% \begin{center}
%   |\UMatT{A} \UMatT[k]{A}|
%   \hspace{1.2cm}
%   \fbox{$\UMatT{A}$}
%   \fbox{$\UMatT[k]{A}$}
% \end{center}
% 
% \paragraph{Matriz de eliminación gaussiana (por columnas)}
% \DescribeMacro{\MatGC}
% \DescribeMacro{\InvMatGC}
% \cs{MatGC} e \cs{InvMatGC} tienen 1 argumento 
% \begin{center}
%   |\MatGC{h} \InvMatGC{h}|
%   \hspace{1.2cm}
%   \fbox{$\MatGC{h}$}
%   \fbox{$\InvMatGC{h}$}
% \end{center}
% 
% %%%%%%%%%%%%%%%%%%%%%%%%%%%%%%%%%%%%%%
% \subsection{Productos entre vectores}
% 
% \subsubsection{Producto escalar}
% 
% \DescribeMacro{\eSc}
% \DescribeMacro{\eSc*}
% tiene 3 argumentos,\;\cs{eSc}\oarg{espacio}\marg{objeto}\marg{objeto},\; y denota el
% producto escalar entre dos objetos (con asterisco los ángulos se
% ajustan al contenido)
% \begin{center}
%   |\eSc{f(x)}{g(x)} \eSc*{f(x)}{g(x)}|
%   \hspace{1.2cm}
%   \fbox{$\eSc {f(x)}{g(x)}$}
%   \fbox{$\eSc*{f(x)}{g(x)}$}
% \end{center}
% \begin{center}
%   |\eSc[L_2]{f(x)}{g(x)} \eSc[L_2]*{f(x)}{g(x)}|
%   \hspace{1.2cm}
%   \fbox{$\eSc [L_2]{f(x)}{g(x)}$}
%   \fbox{$\eSc*[L_2]{f(x)}{g(x)}$}
% \end{center}
% 
% \DescribeMacro{\esc}
% \DescribeMacro{\esc*}
% tiene 3 argumentos,\;\cs{esc}\oarg{espacio}\oarg{espacio}\marg{nombre}\marg{nombre},\; y denota el
% producto escalar entre dos vectores genéricos (con asterisco los
% ángulos se ajustan al contenido)
% \begin{center}
%   |\esc{a}{b} \esc*{a}{b}|
%   \hspace{1.2cm}
%   \fbox{$\esc {a}{b}$}
%   \fbox{$\esc*{a}{b}$}
% \end{center}
% \begin{center}
%   |\esc[L_2]{f(x)}{g(x)} \esc[L_2]*{f(x)}{g(x)}|
%   \hspace{1.2cm}
%   \fbox{$\esc [L_2]{f(x)}{g(x)}$}
%   \fbox{$\esc*[L_2]{f(x)}{g(x)}$}
% \end{center}
% 
% \subsubsection{Producto punto}
% \emph{¡Ojo! en las versiones con paréntesis no he sido consistente con
%   el convenio seguido anteriormente y, en lugar de terminar en
%   \texttt{pE} o \texttt{PE}, sencillamente terminan en \texttt{p} o
%   \texttt{P}}.
% 
% \DescribeMacro{\dotProd}
% \DescribeMacro{\dotProdp}
% \DescribeMacro{\dotProdp*}
% \DescribeMacro{\dotProdP}
% \DescribeMacro{\dotProdP*}
% tiene 2 argumentos,\;\cs{dotProd}\marg{objeto}\marg{objeto},\; y
% denota el producto punto entre dos objetos
% \begin{center}
%   |\dotProd{(\Vect{a}+\Vect{b})}{\Vect{c}}|
%   \hspace{1.2cm}
%   \fbox{$\dotProd{(\Vect{a}+\Vect{b})}{\Vect{c}}$}
% \end{center}
% \begin{center}
%   |\dotProdp{\Vect{a}}{\Vect{b}} \dotProdp*{\Vect{a}}{\Vect{b}}|
%   \hspace{1.2cm}
%   \fbox{$\dotProdp {\Vect{a}}{\Vect{b}}$}
%   \fbox{$\dotProdp*{\Vect{a}}{\Vect{b}}$}
% \end{center}
% \begin{center}
%   |\dotProdP{\Vect{a}}{\Vect{b}} \dotProdP*{\Vect{a}}{\Vect{b}}|
%   \hspace{1.2cm}
%   \fbox{$\dotProdP {\Vect{a}}{\Vect{b}}$}
%   \fbox{$\dotProdP*{\Vect{a}}{\Vect{b}}$}
% \end{center}
% 
% \DescribeMacro{\dotprod}
% \DescribeMacro{\dotprodp}
% \DescribeMacro{\dotprodp*}
% \DescribeMacro{\dotprodP}
% \DescribeMacro{\dotprodP*}
% tiene 2 argumentos,\;\cs{dotprod}\oarg{subíndice1}\marg{nombre}\oarg{subíndice2}\marg{nombre},\; y
% denota el producto punto entre dos vectores de $\R[n]$
% \begin{center}
%   |\dotprod[k]{a}[h]{b}|
%   \hspace{1.2cm}
%   \fbox{$\dotprod[k]{a}[h]{b}$}
% \end{center}
% 
% \begin{center}
%   |\dotprodp{a}{b} \dotprodp*{a}{b}|
%   \hspace{1.2cm}
%   \fbox{$\dotprodp {a}{b}$}
%   \fbox{$\dotprodp*{a}{b}$}
% \end{center}
% 
% \begin{center}
%   |\dotprodP{a}{b} \dotprodP*[h]{a}[k]{b}|
%   \hspace{1.2cm}
%   \fbox{$\dotprodP {a}{b}$}
%   \fbox{$\dotprodP*[h]{a}[k]{b}$}
% \end{center}
% 
% \subsubsection{Producto punto a punto o \emph{Hadamard}}
% 
% \DescribeMacro{\prodH}
% \DescribeMacro{\prodHp}
% \DescribeMacro{\prodHp*}
% \DescribeMacro{\prodHP}
% \DescribeMacro{\prodHP*}
% \DescribeMacro{\prodH}
% tiene 2 argumentos,\;\cs{prodH}\marg{objeto}\marg{objeto},\; y
% denota el producto punto a punto entre dos objetos
% \begin{center}
%   |\prodH{(\Vect{a}+\Vect{b})}{\Vect{c}}|
%   \hspace{1.2cm}
%   \fbox{$\prodH{(\Vect{a}+\Vect{b})}{\Vect{c}}$}
% \end{center}
% \begin{center}
%   |\prodHp{\Vect{a}}{\Vect{b}} \prodHp*{\Vect{a}}{\Vect{b}}|
%   \hspace{1.2cm}
%   \fbox{$\prodHp {\Vect{a}}{\Vect{b}}$}
%   \fbox{$\prodHp*{\Vect{a}}{\Vect{b}}$}
% \end{center}
% \begin{center}
%   |\prodHP{\Vect{a}}{\Vect{b}} \prodHP*{\Vect{a}}{\Vect{b}}|
%   \hspace{1.2cm}
%   \fbox{$\prodHP {\Vect{a}}{\Vect{b}}$}
%   \fbox{$\prodHP*{\Vect{a}}{\Vect{b}}$}
% \end{center}
% 
% \DescribeMacro{\prodh}
% \DescribeMacro{\prodhp}
% \DescribeMacro{\prodhp*}
% \DescribeMacro{\prodhP}
% \DescribeMacro{\prodhP*}
% tiene 4 argumentos,\;\cs{prodh}\oarg{índice}\marg{nombre}\oarg{índice}\marg{nombre},\; y
% denota el producto punto a punto entre dos vectores de $\R[n]$
% \begin{center}
%   |\prodh{a}{b} \prodh[j]{a}[k]{b}|
%   \hspace{1.2cm}
%   \fbox{$\prodh{a}{b}$}
%   \fbox{$\prodh[j]{a}[k]{b}$}
% \end{center}
% 
% \begin{center}
%   |\prodhp{a}{b} \prodhp*{a}{b}|
%   \hspace{1.2cm}
%   \fbox{$\prodhp {a}{b}$}
%   \fbox{$\prodhp*{a}{b}$}
% \end{center}
% 
% \begin{center}
%   |\prodhP{a}{b} \prodhP*[j]{a}[k]{b}|
%   \hspace{1.2cm}
%   \fbox{$\prodhP {a}{b}$}
%   \fbox{$\prodhP*[j]{a}[k]{b}$}
% \end{center}
% 
% \subsection{Matriz por vector y vector por matriz}
% 
% \DescribeMacro{\MV}
% \DescribeMacro{\MVpE}
% \DescribeMacro{\MVpE*}
% \DescribeMacro{\MVPE}
% \DescribeMacro{\MVPE*}
% tiene 4 argumentos,\;
% \cs{MV}\oarg{indMatriz}\marg{nombre}\oarg{indVector}\marg{nombre},\; y
% denota el producto de una matriz por un vector de $\R[n]$
% \begin{center}
%   |\MV{A}{b} \MV[j]{A}[k]{b}|
%   \hspace{1.2cm}
%   \fbox{$\MV{A}{b}$}
%   \fbox{$\MV[j]{A}[k]{b}$}
% \end{center}
% \begin{center}
%   |\MVpE{A}{b} \MVpE*{A}{b} \MVpE*[j]{A}[k]{b}|
%   \hspace{1.2cm}
%   \fbox{$\MVpE {A}{b}$}
%   \fbox{$\MVpE*{A}{b}$}
%   \fbox{$\MVpE*[j]{A}[k]{b}$}
% \end{center}
% \begin{center}
%   |\MVPE{A}{b} \MVPE*{A}{b} \MVPE*[j]{A}[k]{b}|
%   \hspace{1.2cm}
%   \fbox{$\MVPE {A}{b}$}
%   \fbox{$\MVPE*{A}{b}$}
%   \fbox{$\MVPE*[j]{A}[k]{b}$}
% \end{center}
% 
% \DescribeMacro{\VM}
% \DescribeMacro{\VMpE}
% \DescribeMacro{\VMpE*}
% \DescribeMacro{\VMPE}
% \DescribeMacro{\VMPE*}
% tiene 4 argumentos,\;
% \cs{VM}\oarg{indVector}\marg{nombre}\oarg{indMatriz}\marg{nombre},\; y
% denota el producto de un vector de $\R[n]$ por una matriz
% \begin{center}
%   |\VM{a}{B} \VM[k]{a}[j]{B}|
%   \hspace{1.2cm}
%   \fbox{$\VM{a}{B}$}
%   \fbox{$\VM[k]{a}[j]{B}$}
% \end{center}
% \begin{center}
%   |\VMpE{a}{B} \VMpE*{a}{B} \VMpE*[j]{A}[k]{b}|
%   \hspace{1.2cm}
%   \fbox{$\VMpE {a}{B}$}
%   \fbox{$\VMpE*{a}{B}$}
%   \fbox{$\VMpE*[k]{A}[j]{b}$}
% \end{center}
% \begin{center}
%   |\VMPE{a}{B} \VMPE*{a}{B} \VMpE*[j]{A}[k]{b}|
%   \hspace{1.2cm}
%   \fbox{$\VMPE {a}{B}$}
%   \fbox{$\VMPE*{a}{B}$}
%   \fbox{$\VMPE*[k]{A}[j]{b}$}
% \end{center}
% 
% \DescribeMacro{\MTV}
% \DescribeMacro{\MTVp}
% \DescribeMacro{\MTVp*}
% \DescribeMacro{\MTVP}
% \DescribeMacro{\MTVP*}
% tiene 4 argumentos,\;
% \cs{MTV}\oarg{indMatriz}\marg{nombre}\oarg{indVector}\marg{nombre},\; y
% denota el producto de una matriz transpuesta por un vector de $\R[n]$
% \begin{center}
%   |\MTV{A}{b} \MTV[j]{A}[k]{b}|
%   \hspace{1.2cm}
%   \fbox{$\MTV{A}{b}$}
%   \fbox{$\MTV[j]{A}[k]{b}$}
% \end{center}
% \begin{center}
%   |\MTVp{A}{b} \MTVp*{A}{b} \MTVp*[j]{A}[k]{b}|
%   \hspace{1.2cm}
%   \fbox{$\MTVp {A}{b}$}
%   \fbox{$\MTVp*{A}{b}$}
%   \fbox{$\MTVp*[j]{A}[k]{b}$}
% \end{center}
% \begin{center}
%   |\MTVP{A}{b} \MTVP*{A}{b} \MTVP*[j]{A}[k]{b}|
%   \hspace{1.2cm}
%   \fbox{$\MTVP {A}{b}$}
%   \fbox{$\MTVP*{A}{b}$}
%   \fbox{$\MTVP*[j]{A}[k]{b}$}
% \end{center}
% 
% \DescribeMacro{\VMT}
% \DescribeMacro{\VMTp}
% \DescribeMacro{\VMTp*}
% \DescribeMacro{\VMTP}
% \DescribeMacro{\VMTP*}
% tiene 4 argumentos,\;
% \cs{VMT}\oarg{indVector}\marg{nombre}\oarg{indMatriz}\marg{nombre},\; y
% denota el producto de un vector de $\R[n]$ por una matriz transpuesta
% \begin{center}
%   |\VMT{a}{B} \VMT[k]{a}[j]{B}|
%   \hspace{1.2cm}
%   \fbox{$\VMT{a}{B}$}
%   \fbox{$\VMT[k]{a}[j]{B}$}
% \end{center}
% \begin{center}
%   |\VMTp{a}{B} \VMTp*{a}{B} \VMTp*[j]{A}[k]{b}|
%   \hspace{1.2cm}
%   \fbox{$\VMTp {a}{B}$}
%   \fbox{$\VMTp*{a}{B}$}
%   \fbox{$\VMTp*[k]{A}[j]{b}$}
% \end{center}
% \begin{center}
%   |\VMTP{a}{B} \VMTP*{a}{B} \VMTP*[j]{A}[k]{b}|
%   \hspace{1.2cm}
%   \fbox{$\VMTP {a}{B}$}
%   \fbox{$\VMTP*{a}{B}$}
%   \fbox{$\VMTP*[k]{A}[j]{b}$}
% \end{center}
% 
% \subsection{Matriz por matriz}
% 
% \DescribeMacro{\MN}
% tiene 4
% argumentos,\;\cs{MN}\oarg{subíndice1}\marg{nombre1}\oarg{subíndice2}\marg{nombre1},\;
% y denota el producto matriz por matriz
% \begin{center}
%   |\MN{A}{B}|
%   \hspace{1.2cm}
%   \fbox{$\MN{A}{B}$}
% \end{center}
% \begin{center}
%   |\MN[h]{A}{B} \MN{A}[k]{B} \MN[h]{A}[k]{B}|
%   \hspace{1.2cm}
%   \fbox{$\MN[h]{A}{B}$}
%   \fbox{$\MN{A}[k]{B}$}
%   \fbox{$\MN[h]{A}[k]{B}$}
% \end{center}
% 
% \DescribeMacro{\MTN}
% \DescribeMacro{\MTNp}
% \DescribeMacro{\MTNp*}
% \DescribeMacro{\MTNP}
% \DescribeMacro{\MTNP*}
% tiene 4
% argumentos,\;\cs{MTN}\oarg{subíndice1}\marg{nombre1}\oarg{subíndice2}\marg{nombre1},\;
% y denota el producto matriz transpuesta por matriz
% \begin{center}
%   |\MTN{A}{B}|
%   \hspace{1.2cm}
%   \fbox{$\MTN{A}{B}$}
% \end{center}
% \begin{center}
%   |\MTN[h]{A}{B} \MTN{A}[k]{B} \MTN[h]{A}[k]{B}|
%   \hspace{1.2cm}
%   \fbox{$\MTN[h]{A}{B}$}
%   \fbox{$\MTN{A}[k]{B}$}
%   \fbox{$\MTN[h]{A}[k]{B}$}
% \end{center}
% \begin{center}
%   |\MTNp{A}{B} \MTNp*{A}{B}|
%   \hspace{1.2cm}
%   \fbox{$\MTNp {A}{B}$}
%   \fbox{$\MTNp*{A}{B}$}
% \end{center}
% \begin{center}
%   |\MTNp[h]{A}[k]{B} \MTNp*[h]{A}[k]{B}|
%   \hspace{1.2cm}
%   \fbox{$\MTNp [h]{A}[k]{B}$}
%   \fbox{$\MTNp*[h]{A}[k]{B}$}
% \end{center}
% \begin{center}
%   |\MTNP{A}{B} \MTNP*{A}{B}|
%   \hspace{1.2cm}
%   \fbox{$\MTNP {A}{B}$}
%   \fbox{$\MTNP*{A}{B}$}
% \end{center}
% \begin{center}
%   |\MTNP[h]{A}[k]{B} \MTNP*[h]{A}[k]{B}|
%   \hspace{1.2cm}
%   \fbox{$\MTNP [h]{A}[k]{B}$}
%   \fbox{$\MTNP*[h]{A}[k]{B}$}
% \end{center}
% 
% \DescribeMacro{\MNT}
% \DescribeMacro{\MNTp}
% \DescribeMacro{\MNTp*}
% \DescribeMacro{\MNTP}
% \DescribeMacro{\MNTP*}
% tiene 4
% argumentos,\;\cs{MNT}\oarg{subíndice1}\marg{nombre1}\oarg{subíndice2}\marg{nombre1},\;
% y denota el producto matriz por matriz transpuesta
% \begin{center}
%   |\MNT{A}{B}|
%   \hspace{1.2cm}
%   \fbox{$\MNT{A}{B}$}
% \end{center}
% \begin{center}
%   |\MNT[h]{A}{B} \MNT{A}[k]{B} \MNT[h]{A}[k]{B}|
%   \hspace{1.2cm}
%   \fbox{$\MNT[h]{A}{B}$}
%   \fbox{$\MNT{A}[k]{B}$}
%   \fbox{$\MNT[h]{A}[k]{B}$}
% \end{center}
% \begin{center}
%   |\MNTp{A}{B} \MNTp*{A}{B}|
%   \hspace{1.2cm}
%   \fbox{$\MNTp {A}{B}$}
%   \fbox{$\MNTp*{A}{B}$}
% \end{center}
% \begin{center}
%   |\MNTp[h]{A}[k]{B} \MNTp*[h]{A}[k]{B}|
%   \hspace{1.2cm}
%   \fbox{$\MNTp [h]{A}[k]{B}$}
%   \fbox{$\MNTp*[h]{A}[k]{B}$}
% \end{center}
% \begin{center}
%   |\MNTP{A}{B} \MNTP*{A}{B}|
%   \hspace{1.2cm}
%   \fbox{$\MNTP {A}{B}$}
%   \fbox{$\MNTP*{A}{B}$}
% \end{center}
% \begin{center}
%   |\MNTP[h]{A}[k]{B} \MNTP*[h]{A}[k]{B}|
%   \hspace{1.2cm}
%   \fbox{$\MNTP [h]{A}[k]{B}$}
%   \fbox{$\MNTP*[h]{A}[k]{B}$}
% \end{center}
% 
% \DescribeMacro{\MTM}
% \DescribeMacro{\MTMp}
% \DescribeMacro{\MTMp*}
% \DescribeMacro{\MTMP}
% \DescribeMacro{\MTMP*}
% tiene 2
% argumentos,\;\cs{MTM}\oarg{subíndice}\marg{nombre},\;
% y denota el producto matriz transpuesta por matriz
% \begin{center}
%   |\MTM{A} \MTM[h]{A}|
%   \hspace{1.2cm}
%   \fbox{$\MTM   {A}$}
%   \fbox{$\MTM[h]{A}$}
% \end{center}
% \begin{center}
%   |\MTMp{A} \MTMp*{A} \MTMp[h]{A} \MTMp*[h]{A}|
%   \hspace{1.2cm}
%   \fbox{$\MTMp {A}$}
%   \fbox{$\MTMp*{A}$}
%   \fbox{$\MTMp [h]{A}$}
%   \fbox{$\MTMp*[h]{A}$}
% \end{center}
% \begin{center}
%   |\MTMP{A} \MTMP*{A} \MTMP[h]{A} \MTMP*[h]{A}|
%   \hspace{1.2cm}
%   \fbox{$\MTMP {A}$}
%   \fbox{$\MTMP*{A}$}
%   \fbox{$\MTMP [h]{A}$}
%   \fbox{$\MTMP*[h]{A}$}
% \end{center}
% 
% \DescribeMacro{\MMT}
% \DescribeMacro{\MMTp}
% \DescribeMacro{\MMTp*}
% \DescribeMacro{\MMTP}
% \DescribeMacro{\MMTP*}
% tiene 2 argumentos,\;\cs{MMT}\oarg{subíndice}\marg{nombre},\; y
% denota el producto matriz por su transpuesta
% \begin{center}
%   |\MMT{A}|
%   \hspace{1.2cm}
%   \fbox{$\MMT{A}$}
% \end{center}
% \begin{center}
%   |\MMTp{A} \MMTp*{A} \MMTp[h]{A} \MMTp*[h]{A}|
%   \hspace{1.2cm}
%   \fbox{$\MMTp {A}$}
%   \fbox{$\MMTp*{A}$}
%   \fbox{$\MMTp [h]{A}$}
%   \fbox{$\MMTp*[h]{A}$}
% \end{center}
% \begin{center}
%   |\MMTP{A} \MMTP*{A} \MMTP[h]{A} \MMTP*[h]{A}|
%   \hspace{1.2cm}
%   \fbox{$\MMTP {A}$}
%   \fbox{$\MMTP*{A}$}
%   \fbox{$\MMTP [h]{A}$}
%   \fbox{$\MMTP*[h]{A}$}
% \end{center}
% 
% \DescribeMacro{\MNMT}
% \DescribeMacro{\MNMTp}
% \DescribeMacro{\MNMTp*}
% \DescribeMacro{\MNMTP}
% \DescribeMacro{\MNMTP*}
% tiene 4
% argumentos,\;\cs{MNMT}\oarg{subíndice1}\marg{nombre1}\oarg{subíndice2}\marg{nombre1},\;
% y denota el producto matriz por matriz por matriz transpuesta
% \begin{center}
%   |\MNMT{A}{D} \MNMT[h]{A}[k]{D}|
%   \hspace{1.2cm}
%   \fbox{$\MNMT   {A}   {B}$}
%   \fbox{$\MNMT[h]{A}[k]{B}$}
% \end{center}
% \begin{center}
%   |\MNMTp{A}{D} \MNMTp*{A}{D}|
%   \hspace{1.2cm}
%   \fbox{$\MNMTp {A}{B}$}
%   \fbox{$\MNMTp*{A}{B}$}
% \end{center}
% \begin{center}
%   |\MNMTp[h]{A}[k]{D} \MNMTp*[h]{A}[k]{D}|
%   \hspace{1.2cm}
%   \fbox{$\MNMTp [h]{A}[k]{B}$}
%   \fbox{$\MNMTp*[h]{A}[k]{B}$}
% \end{center}
% \begin{center}
%   |\MNMTP{A}{D} \MNMTP*{A}{D}|
%   \hspace{1.2cm}
%   \fbox{$\MNMTP {A}{B}$}
%   \fbox{$\MNMTP*{A}{B}$}
% \end{center}
% \begin{center}
%   |\MNMTP[h]{A}[k]{D} \MNMTP*[h]{A}[k]{D}|
%   \hspace{1.2cm}
%   \fbox{$\MNMTP [h]{A}[k]{B}$}
%   \fbox{$\MNMTP*[h]{A}[k]{B}$}
% \end{center}
% 
% \DescribeMacro{\MTNM}
% \DescribeMacro{\MTNMp}
% \DescribeMacro{\MTNMp*}
% \DescribeMacro{\MTNMP}
% \DescribeMacro{\MTNMP*}
% tiene 4 argumentos,
% \;\cs{MNMT}\oarg{subíndice1}\marg{nombre1}\oarg{subíndice2}\marg{nombre1},\;
% y denota el producto matriz transpuesta por matriz por matriz
% transpuesta
% \begin{center}
%   |\MTNM{A}{D} \MTNM[h]{A}[k]{D}|
%   \hspace{1.2cm}
%   \fbox{$\MTNM   {A}   {B}$}
%   \fbox{$\MTNM[h]{A}[k]{B}$}
% \end{center}
% \begin{center}
%   |\MTNMp{A}{D} \MTNMp*{A}{D}|
%   \hspace{1.2cm}
%   \fbox{$\MTNMp {A}{B}$}
%   \fbox{$\MTNMp*{A}{B}$}
% \end{center}
% \begin{center}
%   |\MTNMp[h]{A}[k]{D} \MTNMp*[h]{A}[k]{D}|
%   \hspace{1.2cm}
%   \fbox{$\MTNMp [h]{A}[k]{B}$}
%   \fbox{$\MTNMp*[h]{A}[k]{B}$}
% \end{center}
% \begin{center}
%   |\MTNMP{A}{D} \MTNMP*{A}{D}|
%   \hspace{1.2cm}
%   \fbox{$\MTNMP {A}{B}$}
%   \fbox{$\MTNMP*{A}{B}$}
% \end{center}
% \begin{center}
%   |\MTNMP[h]{A}[k]{D} \MTNMP*[h]{A}[k]{D}|
%   \hspace{1.2cm}
%   \fbox{$\MTNMP [h]{A}[k]{B}$}
%   \fbox{$\MTNMP*[h]{A}[k]{B}$}
% \end{center}
% 
% \subsection{Otros productos entre matrices y vectores}
% 
% \DescribeMacro{\MTMV}
% \DescribeMacro{\MTMVp}
% \DescribeMacro{\MTMVp*}
% \DescribeMacro{\MTMVP}
% \DescribeMacro{\MTMVP*}
% tiene 2 argumentos,\;\cs{MTMV}\marg{nombre}\marg{nombre},\; y
% denota el producto matriz transpuesta por matriz por vector
% \begin{center}
%   |\MTMV{A}{b}|
%   \hspace{1.2cm}
%   \fbox{$\MTMV{A}{b}$}
% \end{center}
% \begin{center}
%   |\MTMVp{A}{b} \MTMVp*{A}{b}|
%   \hspace{1.2cm}
%   \fbox{$\MTMVp {A}{b}$}
%   \fbox{$\MTMVp*{A}{b}$}
% \end{center}
% \begin{center}
%   |\MTMVP{A}{b} \MTMVP*{A}{b}|
%   \hspace{1.2cm}
%   \fbox{$\MTMVP {A}{b}$}
%   \fbox{$\MTMVP*{A}{b}$}
% \end{center}
% 
% \DescribeMacro{\VMW}
% tiene 3 argumentos,\;\cs{VMW}\marg{nombre}\marg{nombre}\marg{nombre},\; y
% denota el producto vector por matriz por vector
% \begin{center}
%   |\VMW{a}{B}{c}|
%   \hspace{1.2cm}
%   \fbox{$\VMW{a}{B}{c}$}
% \end{center}
% 
% \DescribeMacro{\VMV}
% tiene 2 argumentos,\;\cs{VMV}\marg{nombre}\marg{nombre},\; y
% denota el producto vector por matriz por vector
% \begin{center}
%   |\VMV{a}{B}|
%   \hspace{1.2cm}
%   \fbox{$\VMV{a}{B}$}
% \end{center}
% 
% \DescribeMacro{\VMTW}
% \DescribeMacro{\VMTWp}
% \DescribeMacro{\VMTWp*}
% \DescribeMacro{\VMTWP}
% \DescribeMacro{\VMTWP*}
% tiene 3 argumentos,\;\cs{VMTW}\marg{nombre}\marg{nombre}\marg{nombre},\; y
% denota el producto vector por matriz transpuesta por vector
% \begin{center}
%   |\VMTW{a}{B}{c}|
%   \hspace{1.2cm}
%   \fbox{$\VMTW{a}{B}{c}$}
% \end{center}
% \begin{center}
%   |\VMTWp{a}{B}{c} \VMTWp*{a}{B}{c}|
%   \hspace{1.2cm}
%   \fbox{$\VMTWp {a}{B}{c}$}
%   \fbox{$\VMTWp*{a}{B}{c}$}
% \end{center}
% \begin{center}
%   |\VMTWP{a}{B}{c} \VMTWP*{a}{B}{c}|
%   \hspace{1.2cm}
%   \fbox{$\VMTWP {a}{B}{c}$}
%   \fbox{$\VMTWP*{a}{B}{c}$}
% \end{center}
% 
% 
% \DescribeMacro{\VMTV}
% \DescribeMacro{\VMTVp}
% \DescribeMacro{\VMTVp*}
% \DescribeMacro{\VMTVP}
% \DescribeMacro{\VMTVP*}
% tiene 2 argumentos,\;\cs{VMTV}\marg{nombre}\marg{nombre},\; y
% denota el producto vector por matriz por vector
% \begin{center}
%   |\VMTV{a}{B}|
%   \hspace{1.2cm}
%   \fbox{$\VMTV{a}{B}$}
% \end{center}
% \begin{center}
%   |\VMTVp{a}{B} \VMTVp*{a}{B}|
%   \hspace{1.2cm}
%   \fbox{$\VMTVp {a}{B}$}
%   \fbox{$\VMTVp*{a}{B}$}
% \end{center}
% \begin{center}
%   |\VMTVP{a}{B} \VMTVP*{a}{B}|
%   \hspace{1.2cm}
%   \fbox{$\VMTVP {a}{B}$}
%   \fbox{$\VMTVP*{a}{B}$}
% \end{center}
% 
% \DescribeMacro{\InvMTM}
% \DescribeMacro{\InvMTM*}
% tiene 1 argumento,\;\cs{InvMTM}\marg{nombre},\; y
% denota la inversa del producto de una matriz transpuesta por ella misma
% \begin{center}
%   |\InvMTM{A} \InvMTM*{A}|
%   \hspace{1.2cm}
%   \fbox{$\InvMTM {A}$}
%   \fbox{$\InvMTM*{A}$}
% \end{center}
% \begin{center}
%   |\InvMTM[h]{A} \InvMTM*[h]{A}|
%   \hspace{1.2cm}
%   \fbox{$\InvMTM [h]{A}$}
%   \fbox{$\InvMTM*[h]{A}$}
% \end{center}
% 
% \DescribeMacro{\InvXTX}
% no tiene argumentos y
% denota la inversa del producto de la matriz X transpuesta por ella misma
% \begin{center}
%   |\InvXTX|
%   \hspace{1.2cm}
%   \fbox{$\InvXTX$}
% \end{center}
% 
% \DescribeMacro{\MInvMTMMT}
% \DescribeMacro{\MInvMTMMT*}
% tiene 2 argumentos,\;\cs{MInvMTMMT}\oarg{subíndice}\marg{nombre},\; y
% denota la matriz proyección sobre el espacio columna de la matriz de rango completo por columnas indicada por su \marg{nombre}
% \begin{center}
%   |\MInvMTMMT{A} \MInvMTMMT*{a}|
%   \hspace{1.2cm}
%   \fbox{$\MInvMTMMT {A}$}
%   \fbox{$\MInvMTMMT*{a}$}
% \end{center}
% \begin{center}
%   |\MInvMTMMT[h]{A} \MInvMTMMT*[h]{a}|
%   \hspace{1.2cm}
%   \fbox{$\MInvMTMMT [h]{A}$}
%   \fbox{$\MInvMTMMT*[h]{a}$}
% \end{center}
% 
% \DescribeMacro{\VTW}
% tiene 4 argumentos,\;\cs{VTW}\oarg{subíndice1}\marg{nombre1}\oarg{subíndice2}\marg{nombre2},\; y
% denota el producto de una matriz fila por una matriz columna
% \begin{center}
%   |\VTW{a}{b} \VTW[h]{a}[k]{b}|
%   \hspace{1.2cm}
%   \fbox{$\VTW   {a}   {b}$}
%   \fbox{$\VTW[h]{a}[k]{b}$}
% \end{center}
% 
% \DescribeMacro{\VTV}
% tiene 2 argumentos,\;\cs{VTV}\oarg{subíndice}\marg{nombre},\; y
% denota el producto de una matriz fila por su transpuesta
% \begin{center}
%   |\VTV{a} \VTV[h]{a}|
%   \hspace{1.2cm}
%   \fbox{$\VTV   {a}$}
%   \fbox{$\VTV[h]{a}$}
% \end{center}
% 
% \DescribeMacro{\VWT}
% tiene 2 argumentos,\;\cs{VWT}\oarg{subíndice1}\marg{nombre}\oarg{subíndice2}\marg{nombre},\; y
% denota el producto de una matriz columna por una matriz fila
% \begin{center}
%   |\VWT[h]{a}[k]{b}|
%   \hspace{1.2cm}
%   \fbox{$\VWT[h]{a}[k]{b}$}
% \end{center}
% 
% \DescribeMacro{\VVT}
% tiene 2 argumentos,\;\cs{VVT}\oarg{subíndice}\marg{nombre},\; y
% denota el producto de una matriz columna por su transpuesta
% \begin{center}
%   |\VVT{a} \VVT[h]{a}|
%   \hspace{1.2cm}
%   \fbox{$\VVT   {a}$}
%   \fbox{$\VVT[h]{a}$}
% \end{center}
% 
% \subsection{Sistemas de ecuaciones}
% 
% \DescribeMacro{\SEL}
% tiene 3 argumentos,\;\cs{SEL}\marg{nombre}\marg{nombre}\marg{nombre},\; y
% denota un sistema de ecuaciones lineales (con notación matricial)
% \begin{center}
%   |\SEL{A}{x}{b}|
%   \hspace{1.2cm}
%   \fbox{$\SEL{A}{x}{b}$}
% \end{center}
% 
% \DescribeMacro{\SELT}
% tiene 3
% argumentos,\;\cs{SELT}\marg{nombre}\marg{nombre}\marg{nombre},\; y
% denota un sistema de ecuaciones lineales (con notación matricial y
% matriz de coeficientes transpuesta)
% \begin{center}
%   |\SELT{A}{x}{b}|
%   \hspace{1.2cm}
%   \fbox{$\SELT{A}{x}{b}$}
% \end{center}
% 
% \DescribeMacro{\SELTP}
% tiene 3
% argumentos,\;\cs{SELTP}\marg{nombre}\marg{nombre}\marg{nombre},\; y
% denota un sistema de ecuaciones lineales (con notación matricial y
% matriz de coeficientes transpuesta entre paréntesis)
% \begin{center}
%   |\SELTP{A}{x}{b}|
%   \hspace{1.2cm}
%   \fbox{$\SELTP{A}{x}{b}$}
% \end{center}
% 
% \DescribeMacro{\SELF}
% tiene 3
% argumentos,\;\cs{SELF}\marg{nombre}\marg{nombre}\marg{nombre},\; y
% denota un sistema de ecuaciones lineales en forma de combinaciones de
% lineales de las filas de la matriz de coeficientes (con notación
% matricial)
% \begin{center}
%   |\SELF{y}{A}{b}|
%   \hspace{1.2cm}
%   \fbox{$\SELF{y}{A}{b}$}
% \end{center}
% 
% \subsection{Espacios vectoriales}
% 
% \DescribeMacro{\EV}
% tiene 3
% argumentos,\;\cs{EV}\oarg{subíndice}\oarg{superíndice}\marg{nombre},\; y
% denota un espacio vectorial
% \begin{center}
%   |\EV{A} \EV{V} \EV[\R]{E}  \EV[\R][*]{E}|
%   \hspace{1.2cm}
%   \fbox{$\EV{A} \EV{V} \EV[\R]{E} \EV[\R][*]{E}$}
% \end{center}
% 
% \DescribeMacro{\EspacioNul}
% no tiene argumentos y denota al espacio nulo (o núcleo)
% \begin{center}
%   |\EspacioNul|
%   \hspace{1.2cm}
%   \fbox{$\EspacioNul$}
% \end{center}
% 
% \DescribeMacro{\EspacioCol}
% no tiene argumentos y denota al espacio columna
% \begin{center}
%   |\EspacioCol|
%   \hspace{1.2cm}
%   \fbox{$\EspacioCol$}
% \end{center}
% 
% \DescribeMacro{\Nulls}
% \DescribeMacro{\Nulls*}
% tiene 1 argumento,\;\cs{Nulls}\marg{objeto},\; y denota el espacio
% nulo (o núcleo) del objeto
% \begin{center}
%   |\Nulls{f} \Nulls*{f}|
%   \hspace{1.2cm}
%   \fbox{$\Nulls {f}$}
%   \fbox{$\Nulls*{f}$}
% \end{center}
% 
% \DescribeMacro{\nulls}
% \DescribeMacro{\nulls*}
% tiene 1 argumento,\;\cs{nulls}\marg{nombre},\; y denota el espacio
% nulo (o núcleo) de una matriz
% \begin{center}
%   |\nulls{A} \nulls*{A}|
%   \hspace{1.2cm}
%   \fbox{$\nulls {A}$}
%   \fbox{$\nulls*{A}$}
% \end{center}
% 
% \DescribeMacro{\Cols}
% \DescribeMacro{\Cols*}
% tiene 1 argumento,\;\cs{Cols}\marg{objeto},\; y denota el espacio
% columna del objeto
% \begin{center}
%   |\Cols{f} \Cols*{f}|
%   \hspace{1.2cm}
%   \fbox{$\Cols {f}$}
%   \fbox{$\Cols*{f}$}
% \end{center}
% 
% \DescribeMacro{\cols}
% \DescribeMacro{\cols*}
% tiene 1 argumento,\;\cs{cols}\marg{nombre},\; y denota el espacio
% columna de una matriz
% \begin{center}
%   |\cols{A} \cols*{A}|
%   \hspace{1.2cm}
%   \fbox{$\cols {A}$}
%   \fbox{$\cols*{A}$}
% \end{center}
% 
% \DescribeMacro{\Span}
% \DescribeMacro{\Span*}
% tiene 1 argumento,\;\cs{Span}\marg{sistema},\; y denota el espacio
% vectorial generado con los elementos del \marg{sistema} o conjunto
% \begin{center}
%   |\Span{\SV{Z}} \Span*{\SV{Z}}|
%   \hspace{1.2cm}
%   \fbox{$\Span {\SV{Z}}$}
%   \fbox{$\Span*{\SV{Z}}$}
% \end{center}
% 
% \DescribeMacro{\coord}
% \DescribeMacro{\coordP}
% \DescribeMacro{\coordP*}
% \DescribeMacro{\coordPE}
% \DescribeMacro{\coordPE*}
% tiene 1 argumento,\;\cs{coord}\marg{vector}\marg{base},\; y denota
% las coordenadas de un vector respecto de una base
% \begin{center}
%   |\coord{\vect{x}}{\SV{Z}}|
%   \hspace{1.2cm}
%   \fbox{$\coord{\vect{x}}{\SV{Z}}$}
% \end{center}
% \begin{center}
%   |\coordP{\vect{x}+\vect{y}}{\SV{Z}} \coordP*{\vect{x}+\vect{y}}{\SV{Z}}|
%   \hspace{1.2cm}
%   \fbox{$\coordP {\vect{x}+\vect{y}}{\SV{Z}}$}
%   \fbox{$\coordP*{\vect{x}+\vect{y}}{\SV{Z}}$}
% \end{center}
% \begin{center}
%   |\coordPE{\Vect{x}}{\Mat{B}} \coordPE*{\Vect{x}}{\Mat{B}}|
%   \hspace{1.2cm}
%   \fbox{$\coordPE {\Vect{x}}{\Mat{B}}$}
%   \fbox{$\coordPE*{\Vect{x}}{\Mat{B}}$}
% \end{center}
% 
% \subsection{Notación funcional}
% 
% \DescribeMacro{\dom}
% El comando \cs{dom} no  tiene
% argumentos y denota el \emph{dominio} de una función
% \begin{center}
%   |\dom(f)|
%   \hspace{1.2cm}
%   \fbox{$\dom(f)$}
% \end{center}
% 
% \DescribeMacro{\imagen}
% El comando \cs{imagen} no  tiene
% argumentos y denota la \emph{imagen} de una función
% \begin{center}
%   |\imagen(f)|
%   \hspace{1.2cm}
%   \fbox{$\imagen(f)$}
% \end{center}
% 
% \DescribeMacro{\imrec}
% El comando \cs{imrec} tiene 2 argumentos
% ,\;\cs{imrec}\marg{funcion}\marg{valor},\;
% y denota la \emph{imagen inversa}
% \begin{center}
%   |\imrec{f}{\lambda}=\{x\mid f(x)=\lambda\}  \imrec{\Vect{a}}{\lambda}=\{i\mid \eleVR{a}{i}=\lambda\}|
%   \hspace{1.2cm}
%   \fbox{$\imrec{f}{\lambda}=\{x\mid f(x)=\lambda\}$}
%   \fbox{$\imrec{\Vect{a}}{\lambda}=\{i\mid \eleVR{a}{i}=\lambda\}$}
% \end{center}
% 
% \DescribeMacro{\mifun}
% tiene 3
% argumentos,\;\cs{mifun}\marg{nombre}\marg{dominio}\marg{conjLlegada},\;
% y denota una función que asigna a los elementos de su dominio
% elementos del \emph{conjunto de llegada}
% \begin{center}
%   |\mifun{f}{X}{Y} \mifun*{f}{X}{Y}|
%   \hspace{1.2cm}
%   \fbox{$\mifun {f}{X}{Y}$}
%   \fbox{$\mifun*{f}{X}{Y}$}
% \end{center}
% 
% \DescribeMacro{\deffun}
% tiene 3 argumentos,
% \;\cs{deffun}\marg{nombre}\marg{dominio}\marg{conjLlegada}\marg{variable}\marg{imagen},\;
% y denota una función que asigna a los elementos de su dominio
% elementos del \emph{conjunto de llegada}
% \begin{center}
%   |\deffun{f}{\Z}{\N}{x}{x^2}|
%   \hspace{1.2cm}
%   \fbox{$\deffun{f}{\Z}{\N}{x}{x^2}$}
% \end{center}
% 
% \DescribeMacro{\sproy}
% El comando \cs{sproy} no  tiene
% argumentos y denota el \emph{operador proyección ortogonal}
% \begin{center}
%   |\sproy|
%   \hspace{1.2cm}
%   \fbox{$\sproy$}
% \end{center}
% 
% 
% \DescribeMacro{\proy}
% \DescribeMacro{\proy*}
% El comando \cs{proy}\oarg{subespacio}\marg{vector} tiene 2
% argumentos y denota la proyección ortogonal de un \marg{vector} sobre un \oarg{subespacio}
% \begin{center}
%   |\proy{\vect{x}} \proy*[\cols*{X}]{\Vect{y}} \proy[\indUno]{\ind{A}}|
%   
%   \fbox{$\proy{\vect{x}}$}
%   \fbox{$\proy*[\cols*{X}]{\Vect{y}}$}
%   \fbox{$\proy[\indUno]{\ind{A}}$}
% \end{center}
% 
% %%%%%%%%%%%%%%%%%%%%%%%%%%%%%%%%%%%%%%%%%%%%%
% 
% \subsection{Probabilidad}
% 
% \DescribeMacro{\ind}
% El comando \cs{ind}\marg{conjunto} tiene 1
% argumento y denota la función indicatriz del \marg{conjunto}
% \begin{center}
%   |\ind{\Omega}|
%   \hspace{1.2cm}
%   \fbox{$\ind{\Omega}$}
% \end{center}
% 
% \DescribeMacro{\indCero}
% El comando \cs{indCero} no tiene 
% argumentos denota la función indicatriz nula
% \begin{center}
%   |\indCero|
%   \hspace{1.2cm}
%   \fbox{$\indCero$}
% \end{center}
% 
% \DescribeMacro{\indUno}
% El comando \cs{indUno} no tiene 
% argumentos denota la función indicatriz constante uno
% \begin{center}
%   |\indUno|
%   \hspace{1.2cm}
%   \fbox{$\indUno$}
% \end{center}
% 
% \DescribeMacro{\Ind}
% El comando \cs{Ind} no tiene
% argumentos y denota la función indicatriz constante uno
% \begin{center}
%   |\Ind|
%   \hspace{1.2cm}
%   \fbox{$\Ind$}
% \end{center}
% 
% 
% \DescribeMacro{\sspi}
% El comando \cs{sspi}\oarg{espacio}\oarg{exponente} tiene 2 argumentos
% y especifica el símbolo para el semi-producto interior definido en un
% \oarg{espacio} concreto
% \begin{center}
%   |\sspi \sspi[\EV{E}] \sspi[\EV{E}][*]|
%   \hspace{1.2cm}
%   \fbox{$\sspi$}
%   \fbox{$\sspi[\EV{E}]$}
%   \fbox{$\sspi[\EV{E}][*]$}
% \end{center}
% 
% 
% \DescribeMacro{\SPI}
% \DescribeMacro{\SPI*}
% El comando \cs{SPI}\oarg{semi-producto int.}\oarg{exponente semi-producto int.}\marg{objeto}\marg{objeto} tiene 4
% argumentos y denota el semi-producto interios entre los dos objetos
% \begin{center}
%   |\SPI{X}{Y} \SPI*{X}{Y} \SPI[\EV{E}]{X}{Y} \SPI*[\EV{E}][*]{\sum_{n=1}^k \esuc*{f}}{Y}|
% 
%   \fbox{$\SPI{X}{Y}$}
%   \fbox{$\SPI*{X}{Y}$}
%   \fbox{$\SPI[\EV{E}]{X}{Y}$}
%   \fbox{$\SPI*[\EV{E}][*]{\sum_{n=1}^k \esuc*{f}}{Y}$}
% \end{center}
% 
% \DescribeMacro{\sesp}
% El comando \cs{sesp}\oarg{semi-producto int.} tiene 1
% argumento y especifica el símbolo para la esperanza (la integral de Lebesgue)
% \begin{center}
%   |\sesp \sesp[\sspi]|
%   \hspace{1.2cm}
%   \fbox{$\sesp$}
%   \fbox{$\sesp[\sspi]$}
% \end{center}
% 
% 
% \DescribeMacro{\ESP}
% \DescribeMacro{\ESP*}
% El comando \cs{ESP}\oarg{semi-producto int.}\marg{objeto} tiene 2
% argumentos y denota la esperanza (la integral de Lebesgue) de un
% \marg{objeto}
% \begin{center}
%   |\ESP{X} \ESP[\sspi]{X} \ESP*[\sspi]{\sum\limits_{i=1}^n \esuc{f}}|
% 
%   \fbox{$\ESP{X}$}
%   \fbox{$\ESP[\sspi]{X}$}
%   \fbox{$\ESP*[\sspi]{\sum\limits_{i=1}^n \esuc{f}}$}
% \end{center}
% 
% 
% \DescribeMacro{\domesp}
% El comando \cs{domesp}\marg{espacio} tiene 1
% argumento y denota es dominio de la función esperanza (integral de Legesgue)
% en un \marg{espacio} concreto
% \begin{center}
%   |\domesp{\EV{E}}|
%   \hspace{1.2cm}
%   \fbox{$\domesp{\EV{E}}$}
% \end{center}
% 
% 
% \DescribeMacro{\spro}
% El comando \cs{spro}\oarg{semi-producto int.} tiene 1
% argumento y especifica el símbolo para la probabilidad
% \begin{center}
%   |\spro \spro[\sspi]|
%   \hspace{1.2cm}
%   \fbox{$\spro$}
%   \fbox{$\spro[\sspi]$}
% \end{center}
% 
% 
% \DescribeMacro{\PRO}
% \DescribeMacro{\PRO*}
% El comando \cs{PRO}\oarg{semi-producto int.}\marg{suceso} tiene 2
% argumentos y denota la probabilidad de un \marg{suceso}
% \begin{center}
%   |\PRO{A} \PRO*[\sspi]{A}|
%   \hspace{1.2cm}
%   \fbox{$\PRO{A}$}
%   \fbox{$\PRO*[\sspi]{A}$}
% \end{center}
% 
% \DescribeMacro{\PRObh}
% \DescribeMacro{\PRObh*}
% El comando \cs{PRObh}\marg{suceso}\marg{hipótesis} tiene 2
% argumentos y denota la probabilidad de un \marg{suceso} bajo cierta hipótesis
% \begin{center}
%   |\PRObh{A}{\Hnula} \PRObh*{A}{\Hnula}|
%   \hspace{1.2cm}
%   \fbox{$\PRObh{A}{\Hnula}$}
%   \fbox{$\PRObh*{A}{\Hnula}$}
% \end{center}
% 
% Con el comando \cs{pindep} denotaremos la independencia probabilística\\
% \DescribeMacro{\pindep}
% El comando \cs{pindep} no tiene argumentos,\;\cs{pindep}.
% \begin{center}
%   |A \pindep B|
%   \hspace{1.2cm}
%   \fbox{$A \pindep B$}
% \end{center}
% 
% El comando \cs{dperp} es otra alternativa para denotar la independencia probabilística\\
% \DescribeMacro{\dperp}
% El comando \cs{dperp} no tiene argumentos,\;\cs{dperp}.
% \begin{center}
%   |A \dperp B|
%   \hspace{1.2cm}
%   \fbox{$A \dperp B$}
% \end{center}
% 
% El comando \cs{ndperp} niega la independencia probabilística\\
% \DescribeMacro{\ndperp}
% El comando \cs{ndperp} no tiene argumentos,\;\cs{ndperp}.
% \begin{center}
%   |A \ndperp B|
%   \hspace{1.2cm}
%   \fbox{$A \ndperp B$}
% \end{center}
% 
% \DescribeMacro{\PSpan}
% \DescribeMacro{\PSpan*}
% tiene 1 argumento,\;\cs{PSpan}\marg{sistema},\; y denota el espacio semi-euclídeo de probabilidad generado con los elementos del \marg{sistema} o conjunto
% \begin{center}
%   |\PSpan{\SV{Z}} \PSpan*{\SV{Z}}|
%   \hspace{1.2cm}
%   \fbox{$\PSpan {\SV{Z}}$}
%   \fbox{$\PSpan*{\SV{Z}}$}
% \end{center}
% 
% \DescribeMacro{\Clase}
% tiene 1 argumento,\;\cs{Clase}\marg{representante},\; y denota la
% clase de equivalencia  del \marg{representante}
% \begin{center}
%   |\Clase{\VA{Z}} \Clase{\cteVA{1}}|
%   \hspace{1.2cm}
%   \fbox{$\Clase{\VA{Z}}$}
%   \fbox{$\Clase{\cteVA{1}}$}
% \end{center}
% 
% \DescribeMacro{\Media}
% \DescribeMacro{\Mediap}
% \DescribeMacro{\Mediap*}
% \DescribeMacro{\MediaP}
% \DescribeMacro{\MediaP*}
% El comando \cs{Media}\marg{objeto} tiene 1
% argumento y pinta una barra horizontal que denota la media (proyección ortogonal sobre los vectores contantes) del \marg{objeto}
% \begin{center}
%   |\Media{\Vect{x}} \Mediap{\Vect{x}+\Vect{y}} \MediaP*{\Vect{x}+\Vect{y}}|
% 
%   \fbox{$\Media{\Vect{x}}$}
%   \fbox{$\Mediap{\Vect{x}+\Vect{y}}$}
%   \fbox{$\MediaP*{\Vect{x}+\Vect{y}}$}
% \end{center}
% 
% 
% \DescribeMacro{\Smedia}
% El comando \cs{Smedia} no tiene
% argumentos y pinta el símbolo del valor medio
% \begin{center}
%   |\Smedia|
%   \hspace{1.2cm}
%   \fbox{$\Smedia$}
% \end{center}
% 
% \DescribeMacro{\SmediaM}
% El comando \cs{SmediaM} no tiene
% argumentos y pinta el símbolo de la media muestral
% \begin{center}
%   |\SmediaM|
%   \hspace{1.2cm}
%   \fbox{$\SmediaM$}
% \end{center}
% 
% \DescribeMacro{\Svar}
% El comando \cs{Svar} no tiene
% argumentos y pinta el símbolo de la varianza
% \begin{center}
%   |\Svar|
%   \hspace{1.2cm}
%   \fbox{$\Svar$}
% \end{center}
% 
% \DescribeMacro{\SvarM}
% El comando \cs{SvarM} no tiene
% argumentos y pinta el símbolo de la varianza muestral
% \begin{center}
%   |\SvarM|
%   \hspace{1.2cm}
%   \fbox{$\SvarM$}
% \end{center}
% 
% \DescribeMacro{\Scov}
% El comando \cs{Scov} no tiene
% argumentos y pinta el símbolo de la covarianza
% \begin{center}
%   |\Scov|
%   \hspace{1.2cm}
%   \fbox{$\Scov$}
% \end{center}
% 
% \DescribeMacro{\ScovM}
% El comando \cs{ScovM} no tiene
% argumentos y pinta el símbolo de la covarianza muestral
% \begin{center}
%   |\ScovM|
%   \hspace{1.2cm}
%   \fbox{$\ScovM$}
% \end{center}
% 
% \DescribeMacro{\Scorr}
% El comando \cs{Scorr} no tiene
% argumentos y pinta el símbolo de la correlación
% \begin{center}
%   |\Scorr|
%   \hspace{1.2cm}
%   \fbox{$\Scorr$}
% \end{center}
% 
% \DescribeMacro{\ScorrM}
% El comando \cs{ScorrM} no tiene
% argumentos y pinta el símbolo de la correlación muestral
% \begin{center}
%   |\ScorrM|
%   \hspace{1.2cm}
%   \fbox{$\ScorrM$}
% \end{center}
% 
% \DescribeMacro{\media}
% \DescribeMacro{\mediap}
% \DescribeMacro{\mediap*}
% \DescribeMacro{\mediaP}
% \DescribeMacro{\mediaP*}
% El comando \cs{media} tiene 1 argumento, \cs{media}\marg{objeto}, y
% denota el valor medio del objeto.
% \begin{center}
%   |\media{\Vect{x}} \media{\Vect{x}}^2 \media{} |
%   \hspace{1.2cm}
%   \fbox{$\media{\Vect{x}}$}
%   \fbox{$\media{\Vect{x}}^2$}
%   \fbox{$\media{}$}
% \end{center}
% \begin{center}
%   |\mediap{\Vect{x}^2} \mediap*{\Vect{x}^2} \mediaP*{\Vect{x}^2}^2 |
%   \hspace{1.2cm}
%   \fbox{$\mediap{\Vect{x}^2}$}
%   \fbox{$\mediap*{\Vect{x}^2}$}
%   \fbox{$\mediaP*{\Vect{x}^2}^2$}
% \end{center}
% 
% \DescribeMacro{\mediaM}
% \DescribeMacro{\mediaMp}
% \DescribeMacro{\mediaMp*}
% \DescribeMacro{\mediaMP}
% \DescribeMacro{\mediaMP*}
% El comando \cs{mediaM} tiene 1 argumento, \cs{mediaM}\marg{muestra}, y
% denota la media muestral.
% \begin{center}
%   |\mediaM{\Vect{x}} \mediaM{\Vect{x}}^2 \mediaM{} |
%   \hspace{1.2cm}
%   \fbox{$\mediaM{\Vect{x}}$}
%   \fbox{$\mediaM{\Vect{x}}^2$}
%   \fbox{$\mediaM{}$}
% \end{center}
% \begin{center}
%   |\mediaMp{\Vect{x}^2} \mediaMp*{\Vect{x}^2} \mediaMP*{\Vect{x}^2}^2 |
%   \hspace{1.2cm}
%   \fbox{$\mediaMp{\Vect{x}^2}$}
%   \fbox{$\mediaMp*{\Vect{x}^2}$}
%   \fbox{$\mediaMP*{\Vect{x}^2}^2$}
% \end{center}
% 
% \DescribeMacro{\dt}
% \DescribeMacro{\dtp}
% \DescribeMacro{\dtp*}
% \DescribeMacro{\dtP}
% \DescribeMacro{\dtP*}
% El comando \cs{dt} tiene 1 argumento, \cs{dt}\marg{objeto}, y
% denota la desviación típica del objeto.
% \begin{center}
%   |\dt{\Vect{x}} \dt{} |
%   \hspace{1.2cm}
%   \fbox{$\dt{\Vect{x}}$}
%   \fbox{$\dt{}$}
% \end{center}
% \begin{center}
%   |\dtp{\Vect{x}^2} \dtp*{\Vect{x}^2} |
%   \hspace{1.2cm}
%   \fbox{$\dtp{\Vect{x}^2}$}
%   \fbox{$\dtp*{\Vect{x}^2}$}
% \end{center}
% 
% \DescribeMacro{\dtM}
% \DescribeMacro{\dtMp}
% \DescribeMacro{\dtMp*}
% \DescribeMacro{\dtMP}
% \DescribeMacro{\dtMP*}
% El comando \cs{dtM} tiene 1 argumento, \cs{dtM}\marg{muestra}, y
% denota la desviación típica muestral.
% \begin{center}
%   |\dtM{\Vect{x}} \dtM{} |
%   \hspace{1.2cm}
%   \fbox{$\dtM{\Vect{x}}$}
%   \fbox{$\dtM{}$}
% \end{center}
% \begin{center}
%   |\dtMp{\Vect{x}^2} \dtMp*{\Vect{x}^2} |
%   \hspace{1.2cm}
%   \fbox{$\dtMp{\Vect{x}^2}$}
%   \fbox{$\dtMp*{\Vect{x}^2}$}
% \end{center}
% 
% \DescribeMacro{\var}
% \DescribeMacro{\varp}
% \DescribeMacro{\varp*}
% \DescribeMacro{\varP}
% \DescribeMacro{\varP*}
% El comando \cs{var} tiene 1 argumento, \cs{var}\marg{objeto}, y
% denota la varianza del objeto.
% \begin{center}
%   |\var{\Vect{x}} \var{} |
%   \hspace{1.2cm}
%   \fbox{$\var{\Vect{x}}$}
%   \fbox{$\var{}$}
% \end{center}
% \begin{center}
%   |\varp{\Vect{x}^2} \varp*{\Vect{x}^2} |
%   \hspace{1.2cm}
%   \fbox{$\varp{\Vect{x}^2}$}
%   \fbox{$\varp*{\Vect{x}^2}$}
% \end{center}
% 
% \DescribeMacro{\varM}
% \DescribeMacro{\varMp}
% \DescribeMacro{\varMp*}
% \DescribeMacro{\varMP}
% \DescribeMacro{\varMP*}
% El comando \cs{varM} tiene 1 argumento, \cs{varM}\marg{muestra}, y
% denota la varianza muestral.
% \begin{center}
%   |\varM{\Vect{x}} \varM{} |
%   \hspace{1.2cm}
%   \fbox{$\varM{\Vect{x}}$}
%   \fbox{$\varM{}$}
% \end{center}
% \begin{center}
%   |\varMp{\Vect{x}^2} \varMp*{\Vect{x}^2} |
%   \hspace{1.2cm}
%   \fbox{$\varMp{\Vect{x}^2}$}
%   \fbox{$\varMp*{\Vect{x}^2}$}
% \end{center}
% 
% \DescribeMacro{\cvarM}
% \DescribeMacro{\cvarMp}
% \DescribeMacro{\cvarMp*}
% \DescribeMacro{\cvarMP}
% \DescribeMacro{\cvarMP*}
% El comando \cs{cvarM} tiene 1 argumento, \cs{cvarM}\marg{muestra}, y
% denota la cuasi-varianza muestral.
% \begin{center}
%   |\cvarM{\Vect{x}} \cvarM{} |
%   \hspace{1.2cm}
%   \fbox{$\cvarM{\Vect{x}}$}
%   \fbox{$\cvarM{}$}
% \end{center}
% \begin{center}
%   |\cvarMp{\Vect{x}^2} \cvarMp*{\Vect{x}^2} |
%   \hspace{1.2cm}
%   \fbox{$\cvarMp{\Vect{x}^2}$}
%   \fbox{$\cvarMp*{\Vect{x}^2}$}
% \end{center}
% 
% \DescribeMacro{\cov}
% \DescribeMacro{\covp}
% \DescribeMacro{\covp*}
% \DescribeMacro{\covP}
% \DescribeMacro{\covP*}
% El comando \cs{cov} tiene 2 argumentos, \cs{cov}\marg{objeto1}\marg{objeto2}, y
% denota la covarianza entre \marg{objeto1} y \marg{objeto2}.
% \begin{center}
%   |\cov{\Vect{x}}{\Vect{y}} \cov{}{} |
%   \hspace{1.2cm}
%   \fbox{$\cov{\Vect{x}}{\Vect{y}}$}
%   \fbox{$\cov{}{}$}
% \end{center}
% \begin{center}
%   |\covp{\Vect{x}^2}{\Vect{y}} \covp*{\Vect{x}}{\Vect{y}} |
%   \hspace{1.2cm}
%   \fbox{$\covp{\Vect{x}^2}{\Vect{y}}$}
%   \fbox{$\covp*{\Vect{x}}{\Vect{y}}$}
% \end{center}
% 
% \DescribeMacro{\covM}
% \DescribeMacro{\covMp}
% \DescribeMacro{\covMp*}
% \DescribeMacro{\covMP}
% \DescribeMacro{\covMP*}
% El comando \cs{covM} tiene 2 argumentos, \cs{covM}\marg{muestra1}\marg{muestra2}, y
% denota la covarianza muestral.
% \begin{center}
%   |\covM{\Vect{x}}{\Vect{y}} \covM{}{} |
%   \hspace{1.2cm}
%   \fbox{$\covM{\Vect{x}}{\Vect{y}}$}
%   \fbox{$\covM{}{}$}
% \end{center}
% \begin{center}
%   |\covMp{\Vect{x}^2}{\Vect{y}} \covMp*{\Vect{x}}{\Vect{y}} |
%   \hspace{1.2cm}
%   \fbox{$\covMp{\Vect{x}^2}{\Vect{y}}$}
%   \fbox{$\covMp*{\Vect{x}}{\Vect{y}}$}
% \end{center}
% 
% \DescribeMacro{\corr}
% \DescribeMacro{\corrp}
% \DescribeMacro{\corrp*}
% \DescribeMacro{\corrP}
% \DescribeMacro{\corrP*}
% El comando \cs{corr} tiene 2 argumentos, \cs{corr}\marg{objeto1}\marg{objeto2}, y
% denota la correlación entre \marg{objeto1} y \marg{objeto2}.
% \begin{center}
%   |\corr{\Vect{x}}{\Vect{y}} \corr{}{} |
%   \hspace{1.2cm}
%   \fbox{$\corr{\Vect{x}}{\Vect{y}}$}
%   \fbox{$\corr{}{}$}
% \end{center}
% \begin{center}
%   |\corrp{\Vect{x}^2}{\Vect{y}} \corrp*{\Vect{x}}{\Vect{y}} |
%   \hspace{1.2cm}
%   \fbox{$\corrp{\Vect{x}^2}{\Vect{y}}$}
%   \fbox{$\corrp*{\Vect{x}}{\Vect{y}}$}
% \end{center}
% 
% \DescribeMacro{\corrM}
% \DescribeMacro{\corrMp}
% \DescribeMacro{\corrMp*}
% \DescribeMacro{\corrMP}
% \DescribeMacro{\corrMP*}
% El comando \cs{corrM} tiene 2 argumentos, \cs{corrM}\marg{muestra1}\marg{muestra2}, y
% denota la correlación muestral.
% \begin{center}
%   |\corrM{\Vect{x}}{\Vect{y}} \corrM{}{} |
%   \hspace{1.2cm}
%   \fbox{$\corrM{\Vect{x}}{\Vect{y}}$}
%   \fbox{$\corrM{}{}$}
% \end{center}
% \begin{center}
%   |\corrMp{\Vect{x}^2}{\Vect{y}} \corrMp*{\Vect{x}}{\Vect{y}} |
%   \hspace{1.2cm}
%   \fbox{$\corrMp{\Vect{x}^2}{\Vect{y}}$}
%   \fbox{$\corrMp*{\Vect{x}}{\Vect{y}}$}
% \end{center}
% 
% %%%%%%%%%%%%%%%%%%%%%%%%%%%%%%%%%%%%%%%%%%%%%
% 
% \subsection{Econometría}
% 
% \DescribeMacro{TM}
% El comando \cs{TM} no tiene
% argumentos y denota el tamaño muestral
% \begin{center}
%   |\TM|
%   \hspace{1.2cm}
%   \fbox{$\TM$}
% \end{center}
% 
% \DescribeMacro{resi}
% El comando \cs{resi} tiene 1 
% argumento \cs{resi}\marg{índice} y pinta error de ajuste MCO correspondiente al índice 
% \begin{center}
%   |\resi{j}|
%   \hspace{1.2cm}
%   \fbox{$\resi{j}$}
% \end{center}
% 
% \DescribeMacro{res}
% El comando \cs{res} no tiene
% argumentos y pinta el vector de residuos de un ajuste MCO
% \begin{center}
%   |\res|
%   \hspace{1.2cm}
%   \fbox{$\res$}
% \end{center}
% 
% \DescribeMacro{SRC}
% El comando \cs{SRC} no tiene
% argumentos y denota la suma de residuos MCO al cuadrado
% \begin{center}
%   |\SRC|
%   \hspace{1.2cm}
%   \fbox{$\SRC$}
% \end{center}
% 
% \DescribeMacro{ColorA}
% El comando \cs{ColorA} tiene 1
% argumento, \cs{ColorA}\marg{objeto}, y denota con color que el \marg{objeto} es una variable aleatoria (vector de un espacio euclídeo probabilístico)
% \begin{center}
%   |\ColorA{X}|
%   \hspace{1.2cm}
%   \fbox{$\ColorA{X}$}
% \end{center}
% 
% \DescribeMacro{VColorA}
% El comando \cs{VColorA} tiene 1
% argumento, \cs{VColorA}\marg{objeto}, y denota un vector con color que indica que está formado por variables aleatorias
% \begin{center}
%   |\VColorA{y}|
%   \hspace{1.2cm}
%   \fbox{$\VColorA{y}$}
% \end{center}
% 
% \DescribeMacro{VAn}
% El comando \cs{VAn} tiene 2
% argumentos, \cs{VAn}\marg{nombre}\marg{índice}, y denota una variable aleatoria con subíndice
% \begin{center}
%   |\VAn{x}{k}|
%   \hspace{1.2cm}
%   \fbox{$\VAn{x}{k}$}
% \end{center}
% 
% \DescribeMacro{VAi}
% El comando \cs{VAi} tiene 2
% argumentos, \cs{VAi}\oarg{índice}\marg{nombre}, y denota una variable aleatoria
% \begin{center}
%   |\VAi{x} \VAi[k]{x}|
%   \hspace{1.2cm}
%   \fbox{$\VAi{x}$} \fbox{$\VAi[k]{x}$}
% \end{center}
% 
% \DescribeMacro{VA}
% El comando \cs{VA} tiene 2
% argumentos, \cs{VA}\oarg{índice}\marg{nombre}, y denota una variable aleatoria
% \begin{center}
%   |\VA{x} \VA[k]{x}|
%   \hspace{1.2cm}
%   \fbox{$\VA{x}$} \fbox{$\VA[k]{x}$}
% \end{center}
% 
% \DescribeMacro{VAind}
% El comando \cs{VAind} tiene 1
% argumento, \cs{VAind}\marg{suceso}, y denota una variable aleatoria indicatriz
% \begin{center}
%   |\VAind{\Omega}|
%   \hspace{1.2cm}
%   \fbox{$\VAind{\Omega}$} 
% \end{center}
% 
% \DescribeMacro{VAindCero}
% El comando \cs{VAindCero} no tiene
% argumentos, \cs{VAindCero}, y denota la variable aleatoria cero
% \begin{center}
%   |\VAindCero|
%   \hspace{1.2cm}
%   \fbox{$\VAindCero$} 
% \end{center}
% 
% \DescribeMacro{VAindUno}
% El comando \cs{VAindUno} no tiene
% argumentos, \cs{VAindUno}, y denota la variable aleatoria constante uno
% \begin{center}
%   |\VAindUno|
%   \hspace{1.2cm}
%   \fbox{$\VAindUno$} 
% \end{center}
% 
% \DescribeMacro{cteVA}
% El comando \cs{cteVA} tiene 1
% argumento, \cs{cteVA}\marg{número}, y denota la variable aleatoria constante casi seguro
% \begin{center}
%   |\cteVA{0} \cteVA{1} \cteVA{2}|
%   \hspace{1.2cm}
%   \fbox{$\cteVA{0}$} 
%   \fbox{$\cteVA{1}$} 
%   \fbox{$\cteVA{2}$} 
% \end{center}
% 
% \DescribeMacro{VVA}
% El comando \cs{VVA} tiene 2
% argumentos, \cs{VVA}\oarg{índice}\marg{nombre}, y denota un vector aleatorio
% \begin{center}
%   |\VVA{y} \VVA[k]{y}|
%   \hspace{1.2cm}
%   \fbox{$\VVA{y}$} \fbox{$\VVA[k]{y}$}
% \end{center}
% 
% \DescribeMacro{MVA}
% \DescribeMacro{MVAp}
% \DescribeMacro{MVAP}
% El comando \cs{MVA} tiene 2
% argumentos, \cs{MVA}\oarg{índice}\marg{nombre}, y denota una matriz aleatoria
% \begin{center}
%   |\MVA{X} \MVA[k]{X} \MVAp*[k]{X}|
%   \hspace{1.2cm}
%   \fbox{$\MVA{X}$} \fbox{$\MVA[k]{X}$} \fbox{$\MVAp*[k]{X}$}
% \end{center}
% 
% \DescribeMacro{MVAT}
% \DescribeMacro{MVATp}
% \DescribeMacro{MVATP}
% El comando \cs{MVAT} tiene 2
% argumentos, \cs{MVAT}\oarg{índice}\marg{nombre}, y denota una matriz aleatoria
% \begin{center}
%   |\MVAT{X} \MVAT[k]{X} \MVATp*[k]{X} \MVATpE[k]{X}|
%   \hspace{1.2cm}
%   \fbox{$\MVAT{X}$} \fbox{$\MVAT[k]{X}$} \fbox{$\MVATp*[k]{X}$} \fbox{$\MVATpE[k]{X}$}
% \end{center}
% 
% \DescribeMacro{SVA}
% El comando \cs{SVA} tiene 2
% argumento2, \cs{SVA}\oarg{índice}\marg{nombre}, y denota un sistema de variables aleatorias
% \begin{center}
%   |\SVA{X} \SVA[n]{X}|
%   \hspace{1.2cm}
%   \fbox{$\SVA{X}$}\fbox{$\SVA[n]{X}$}
% \end{center}
% 
% \DescribeMacro{SVAT}
% El comando \cs{SVAT} tiene 2
% argumentos, \cs{SVAT}\oarg{índice}\marg{nombre}, y denota un sistema de variables aleatorias
% transpuesto
% \begin{center}
%   |\SVAT{X} \SVAT[j]{X}|
%   \hspace{1.2cm}
%   \fbox{$\SVAT{X}$}
%   \fbox{$\SVAT[j]{X}$}
% \end{center}
% 
% \DescribeMacro{\per}
% El comando \cs{per} no tiene argumentos y
% denota el término de perturbación de un modelo
% \begin{center}
%   |\per|
%   \hspace{1.2cm}
%   \fbox{$\per$}
% \end{center}
% 
% \DescribeMacro{\peri}
% El comando \cs{peri} tiene 1 argumento, \cs{peri}\oarg{índice}, y
% denota el término de perturbación (con un subíndice) de un modelo
% \begin{center}
%   |\peri \peri[t]|
%   \hspace{1.2cm}
%   \fbox{$\peri$} \fbox{$\peri[t]$}
% \end{center}
% 
% \DescribeMacro{\Vper}
% El comando \cs{Vper} no tiene argumento y
% denota un vector de perturbaciones
% \begin{center}
%   |\Vper|
%   \hspace{1.2cm}
%   \fbox{$\Vper$}
% \end{center}
% 
% \DescribeMacro{\esperanza}
% El comando \cs{esperanza} no tiene 
% argumentos y especifica el símbolo para el operador esperanza
% \begin{center}
%   |\esperanza|
%   \hspace{1.2cm}
%   \fbox{$\esperanza$}
% \end{center}
% 
% 
% \DescribeMacro{\E}
% \DescribeMacro{\E*}
% El comando \cs{E}\marg{variable aleatoria} tiene 1
% argumento y denota la esperanza de una \marg{variable aleatoria}
% \begin{center}
%   |\E{\VA{X}} \E*{\sum\limits_{i=1}^n \VAn{X}{i}}|
% 
%   \fbox{$\E{\VA{X}}$}
%   \fbox{$\E*{\sum\limits_{i=1}^n \VAn{X}{i}}$}
% \end{center}
% 
% \DescribeMacro{\desviaciontipica}
% El comando \cs{desviaciontipica} no tiene 
% argumentos y especifica el símbolo para la desviación típica
% \begin{center}
%   |\desviaciontipica|
%   \hspace{1.2cm}
%   \fbox{$\desviaciontipica$}
% \end{center}
% 
% \DescribeMacro{\Dt}
% \DescribeMacro{\Dt*}
% El comando \cs{Dt}\marg{variable aleatoria} tiene 1
% argumento y denota la desviación típica de una \marg{variable aleatoria}
% \begin{center}
%   |\Dt{\VA{X}} \Dt*{\sum\limits_{i=1}^n \VAn{X}{i}}|
% 
%   \fbox{$\Dt{\VA{X}}$}
%   \fbox{$\Dt*{\sum\limits_{i=1}^n \VAn{X}{i}}$}
% \end{center}
% 
% \DescribeMacro{\varianza}
% El comando \cs{varianza} no tiene 
% argumentos y especifica el símbolo para la varianza
% \begin{center}
%   |\varianza|
%   \hspace{1.2cm}
%   \fbox{$\varianza$}
% \end{center}
% 
% \DescribeMacro{\Var}
% \DescribeMacro{\Var*}
% El comando \cs{Var}\marg{variable aleatoria} tiene 1
% argumento y denota la varianza de una \marg{variable aleatoria}
% \begin{center}
%   |\Var{\VA{X}} \Var*{\sum\limits_{i=1}^n \VAn{X}{i}}|
% 
%   \fbox{$\Var{\VA{X}}$}
%   \fbox{$\Var*{\sum\limits_{i=1}^n \VAn{X}{i}}$}
% \end{center}
% 
% \DescribeMacro{\covarianza}
% El comando \cs{covarianza} no tiene 
% argumentos y especifica el símbolo para la covarianza
% \begin{center}
%   |\covarianza|
%   \hspace{1.2cm}
%   \fbox{$\covarianza$}
% \end{center}
% 
% 
% \DescribeMacro{\Cov}
% \DescribeMacro{\Cov*}
% El comando \cs{Cov}\marg{variable aleatoria}\marg{variable aleatoria} tiene 2
% argumentos y denota la covarianza entre dos variables aleatorias
% \begin{center}
%   |\Cov{\VA{X}}{\VA{Y}} \Cov*{\VA{Y}}{\sum\limits_{i=1}^n \VAn{X}{i}}|
% 
%   \fbox{$\Cov{\VA{X}}{\VA{Y}}$}
%   \fbox{$\Cov*{\VA{Y}}{\sum\limits_{i=1}^n \VAn{X}{i}}$}
% \end{center}
% 
% \DescribeMacro{\correlacion}
% El comando \cs{correlacion} no tiene 
% argumentos y especifica el símbolo para la correlación
% \begin{center}
%   |\correlacion|
%   \hspace{1.2cm}
%   \fbox{$\correlacion$}
% \end{center}
% 
% 
% \DescribeMacro{\Corr}
% \DescribeMacro{\Corr*}
% El comando \cs{Corr}\marg{variable aleatoria}\marg{variable aleatoria} tiene 2
% argumentos y denota la correlación entre dos variables aleatorias
% \begin{center}
%   |\Corr{\VA{X}}{\VA{Y}} \Corr*{\VA{Y}}{\sum\limits_{i=1}^n \VAn{X}{i}}|
% 
%   \fbox{$\Corr{\VA{X}}{\VA{Y}}$}
%   \fbox{$\Corr*{\VA{Y}}{\sum\limits_{i=1}^n \VAn{X}{i}}$}
% \end{center}
% 
% \DescribeMacro{\ECond}
% \DescribeMacro{\ECond*}
% El comando \cs{ECond} tiene 2 argumentos, \cs{ECond}\marg{V. aleatoria}\marg{V. aleatoria o sistema} y denota la esperanza de \marg{V. aleatoria} condicionada a una VA o sistema de variables aleatorias
% \begin{center}
%   |\ECond{\VA{Y}}{\VA{X}} \ECond*{\VA{Y}}{\SVA{Z}}|
%   \hspace{1.2cm}
%   \fbox{$\ECond{\VA{Y}}{\VA{X}}$}
%   \fbox{$\ECond*{\VA{Y}}{\SVA{Z}}$}
% \end{center}
% 
% \DescribeMacro{\ECondYX}
% \DescribeMacro{\ECondYX*}
% El comando \cs{ECondYX} tiene 2 argumentos, \cs{ECondYX}\marg{V. aleatoria}\marg{Sist. VA} y denota la esperanza de \marg{V. aleatoria} condicionada a un sistema de variables aleatorias
% \begin{center}
%   |\ECondYX{\VA{Y}}{X} \ECondYX*{\VA{Y}}{Z}|
%   \hspace{1.2cm}
%   \fbox{$\ECondYX{\VA{Y}}{X}$}
%   \fbox{$\ECondYX*{\VA{Y}}{Z}$}
% \end{center}
% 
% \DescribeMacro{\VarCond}
% \DescribeMacro{\VarCond*}
% El comando \cs{VarCond} tiene 2 argumentos, \cs{VarCond}\marg{V. aleatoria}\marg{V. aleatoria o sistema} y denota la varianza de \marg{V. aleatoria} condicionada a una VA o sistema de variables aleatorias
% \begin{center}
%   |\VarCond{\VA{Y}}{\VA{X}} \VarCond*{\VA{Y}}{\SVA{Z}}|
%   \hspace{1.2cm}
%   \fbox{$\VarCond {\VA{Y}}{\VA{X}}$}
%   \fbox{$\VarCond*{\VA{Y}}{\SVA{Z}}$}
% \end{center}
% 
% \DescribeMacro{\DtCond}
% \DescribeMacro{\DtCond*}
% El comando \cs{DtCond} tiene 2 argumentos, \cs{DtCond}\marg{V. aleatoria}\marg{V. aleatoria o sistema} y denota la desviación típica de \marg{V. aleatoria} condicionada a una VA o sistema de variables aleatorias
% \begin{center}
%   |\DtCond{\VA{Y}}{\VA{X}} \DtCond*{\VA{Y}}{\SVA{Z}}|
%   \hspace{1.2cm}
%   \fbox{$\DtCond {\VA{Y}}{\VA{X}}$}
%   \fbox{$\DtCond*{\VA{Y}}{\SVA{Z}}$}
% \end{center}
% 
% \DescribeMacro{\VarCondYX}
% \DescribeMacro{\VarCondYX*}
% El comando \cs{VarCondYX} tiene 2 argumentos, \cs{VarCondYX}\marg{V. aleatoria}\marg{Sist. VA} y denota la varianza de \marg{V. aleatoria} condicionada a un sistema de variables aleatorias
% \begin{center}
%   |\VarCondYX{\VA{Y}}{X} \VarCondYX*{\VA{Y}}{Z}|
%   \hspace{1.2cm}
%   \fbox{$\VarCondYX{\VA{Y}}{X}$}
%   \fbox{$\VarCondYX*{\VA{Y}}{Z}$}
% \end{center}
% 
% \DescribeMacro{\CovCond}
% \DescribeMacro{\CovCond*}
% El comando \cs{CovCond} tiene 3 argumentos, \cs{CovCond}\marg{V. aleatoria1}\marg{V. aleatoria2}\marg{V. aleatoria o sistema} y denota la covarianza entre \marg{V. aleatoria1} y \marg{V. aleatoria2} condicionada a una VA o sistema de variables aleatorias
% \begin{center}
%   |\CovCond{\VA{X}}{\VA{Y}}{\SVA{Z}} \CovCond*{\VA{X}}{\VA{Y}}{\SVA{Z}}|
%   \hspace{1.2cm}
%   \fbox{$\CovCond {\VA{X}}{\VA{Y}}{\SVA{Z}}$}
%   \fbox{$\CovCond*{\VA{X}}{\VA{Y}}{\SVA{Z}}$}
% \end{center}
% 
% \DescribeMacro{\CovCondXYZ}
% \DescribeMacro{\CovCondXYZ*}
% El comando \cs{CovCondXYZ} tiene 3 argumentos, \cs{CovCondXYZ}\marg{V. aleatoria1}\marg{V. aleatoria2}\marg{Sist. VA} y denota la covarianza entre \marg{V. aleatoria1} y \marg{V. aleatoria2} condicionada a un sistema de variables aleatorias
% \begin{center}
%   |\CovCondXYZ{\VA{X}}{\VA{Y}}{Z} \CovCondXYZ*{\VA{X}}{\VA{Y}}{Z}|
%   \hspace{1.2cm}
%   \fbox{$\CovCondXYZ {\VA{X}}{\VA{Y}}{Z}$}
%   \fbox{$\CovCondXYZ*{\VA{X}}{\VA{Y}}{Z}$}
% \end{center}
% 
% \DescribeMacro{\Estmc}
% El comando \cs{Estmc}\marg{objeto} tiene 1
% argumento y denota el ajuste MCO del \marg{objeto}\begin{center}
%   |\Estmc{A}|
%   \hspace{1.2cm}
%   \fbox{$\Estmc{A}$}
% \end{center}
% 
% \DescribeMacro{\VEstmc}
% El comando \cs{VEstmc}\marg{objeto} tiene 1
% argumento y denota el ajuste MCO del \marg{vector} de \R[n]
% \begin{center}
%   |\VEstmc{\beta} \VEstmc[k]{\beta}|
%   \hspace{1.2cm}
%   \fbox{$\VEstmc{\beta}$}
%   \fbox{$\VEstmc[k]{\beta}$}
% \end{center}
% 
% \DescribeMacro{\Estmd}
% El comando \cs{Estmd}\marg{objeto} tiene 1
% argumento y denota el estimador por MCO del \marg{objeto}\begin{center}
%   |\Estmd{A}|
%   \hspace{1.2cm}
%   \fbox{$\Estmd{A}$}
% \end{center}
% 
% \DescribeMacro{\VEstmd}
% El comando \cs{VEstmd}\marg{vector} tiene 1
% argumento y denota el estimador por MCO del \marg{vector} de \R[n]
% \begin{center}
%   |\VEstmd{\beta} \VEstmd[k]{\beta}|
%   \hspace{1.2cm}
%   \fbox{$\VEstmd{\beta}$}
%   \fbox{$\VEstmd[k]{\beta}$}
% \end{center}
% 
% \DescribeMacro{\MLT}
% El comando \cs{MLT} no tieneargumentos  y denota el modelo cuyo único regresor es \VAindUno
% \begin{center}
%   |\MLT|
%   \hspace{1.2cm}
%   \fbox{$\MLT$}
% \end{center}
% 
% \DescribeMacro{\MLS}
% El comando \cs{MLS} no tieneargumentos  y denota el modelo lineal simple
% \begin{center}
%   |\MLS|
%   \hspace{1.2cm}
%   \fbox{$\MLS$}
% \end{center}
% 
% \DescribeMacro{\MLG}
% El comando \cs{MLG} no tiene argumentos y escribe el Modelo Lineal General
% \begin{center}
%   |\MLG|
%   \hspace{1.2cm}
%   \fbox{$\MLG$}
% \end{center}
% 
% \DescribeMacro{\masMLT}
% El comando \cs{masMLT} no tieneargumentos  y denota el modelo muestral cuyo único regresor es \VVA{1}
% \begin{center}
%   |\masMLT|
%   \hspace{1.2cm}
%   \fbox{$\masMLT$}
% \end{center}
% 
% \DescribeMacro{\masMLS}
% El comando \cs{masMLS} no tieneargumentos  y denota el modelo muestral lineal simple
% \begin{center}
%   |\masMLS|
%   \hspace{1.2cm}
%   \fbox{$\masMLS$}
% \end{center}
% 
% \DescribeMacro{\masMLG}
% El comando \cs{masMLG} no tiene argumentos y escribe el Modelo muestral Lineal General
% \begin{center}
%   |\masMLG|
%   \hspace{1.2cm}
%   \fbox{$\masMLG$}
% \end{center}
% 
% \DescribeMacro{\MCO}
% El comando \cs{MCO} tiene 2 argumentos \cs{MCO}\marg{regresando}\marg{regresor} y escribe el cálculo de los parámetros del ajuste MCO
% \begin{center}
%   |\MCO{Y}{X}|
%   \hspace{1.2cm}
%   \fbox{$\MCO{Y}{X}$}
% \end{center}
% 
% \DescribeMacro{\MCOc}
% El comando \cs{MCOc} no tiene y escribe el cálculo de los parámetros del ajuste MCO del vector \Vect{y} sobre \cols{X}
% \begin{center}
%   |\MCOc|
%   \hspace{1.2cm}
%   \fbox{$\MCOc$}
% \end{center}
% 
% \DescribeMacro{\MCOd}
% El comando \cs{MCOd} no tiene y escribe el estimador de los parámetros del juste MCO
% \begin{center}
%   |\MCOd|
%   \hspace{1.2cm}
%   \fbox{$\MCOd$}
% \end{center}
% 
% \DescribeMacro{\ajusteMLT}
% El comando \cs{ajusteMLT} no tieneargumentos  y denota el ajuste del modelo cuyo único regresor el vector constante
% \begin{center}
%   |\ajusteMLT|
%   \hspace{1.2cm}
%   \fbox{$\ajusteMLT$}
% \end{center}
% 
% \DescribeMacro{\ajusteMLS}
% El comando \cs{ajusteMLS} no tieneargumentos  y denota el ajuste del modelo lineal simple
% \begin{center}
%   |\ajusteMLS|
%   \hspace{1.2cm}
%   \fbox{$\ajusteMLS$}
% \end{center}
% 
% \DescribeMacro{\ajusteMLG}
% El comando \cs{ajusteMLG} no tiene argumentos y escribe el ajuste del Modelo Lineal General
% \begin{center}
%   |\ajusteMLG|
%   \hspace{1.2cm}
%   \fbox{$\ajusteMLG$}
% \end{center}
% 
% \DescribeMacro{\SupI}
% El comando \cs{SupI} no tiene argumentos y escribe el primer supuesto del Modelo Lineal General
% \begin{center}
%   |\SupI|
%   \hspace{1.2cm}
%   \fbox{$\SupI$}
% \end{center}
% 
% \DescribeMacro{\SupII}
% El comando \cs{SupII} no tiene argumentos y escribe el segundo supuesto del Modelo Lineal General
% \begin{center}
%   |\SupII|
%   \hspace{1.2cm}
%   \fbox{$\SupII$}
% \end{center}
% 
% \DescribeMacro{\SupIII}
% El comando \cs{SupIII} no tiene argumentos y escribe el tercer supuesto del Modelo Lineal General
% \begin{center}
%   |\SupIII|
%   \hspace{1.2cm}
%   \fbox{$\SupIII$}
% \end{center}
% 
% \DescribeMacro{\SupIV}
% El comando \cs{SupIV} no tiene argumentos y escribe el cuarto supuesto del Modelo Lineal General
% \begin{center}
%   |\SupIV|
%   \hspace{1.2cm}
%   \fbox{$\SupIV$}
% \end{center}
% 
% \DescribeMacro{\SupIImas}
% El comando \cs{SupIImas} no tiene argumentos y escribe el segundo supuesto muestral del Modelo Lineal General
% \begin{center}
%   |\SupIImas|
%   \hspace{1.2cm}
%   \fbox{$\SupIImas$}
% \end{center}
% 
% \DescribeMacro{\SupIIImas}
% El comando \cs{SupIIImas} no tiene argumentos y escribe el tercer supuesto muestral del Modelo Lineal General
% \begin{center}
%   |\SupIIImas|
%   \hspace{1.2cm}
%   \fbox{$\SupIIImas$}
% \end{center}
% 
% \DescribeMacro{\SupIVmas}
% El comando \cs{SupIVmas} no tiene argumentos y escribe el cuarto supuesto muestral del Modelo Lineal General
% \begin{center}
%   |\SupIVmas|
%   \hspace{1.2cm}
%   \fbox{$\SupIVmas$}
% \end{center}
% 
% \DescribeMacro{\SupVmas}
% El comando \cs{SupVmas} no tiene argumentos y escribe el quinto supuesto muestral del Modelo Lineal General
% \begin{center}
%   |\SupVmas|
%   \hspace{1.2cm}
%   \fbox{$\SupVmas$}
% \end{center}
% 
% \DescribeMacro{\MVAR}
% El comando \cs{MVAR} tiene 1 argumento \cs{MVAR}\marg{regresores} y denota la matriz de varianzas y covarianzas de los \marg{regresores}
% \begin{center}
%   |\MVAR{X}|
%   \hspace{1.2cm}
%   \fbox{$\MVAR{X}$}
% \end{center}
% 
% \DescribeMacro{\VCOV}
% El comando \cs{VCOV} tiene 2 argumentos \cs{VCOV}\marg{regresores}\marg{regresando} y denota el vector de covarianzas entre los \marg{regresores} y el \marg{regresando}
% \begin{center}
%   |\VCOV{X}{y}|
%   \hspace{1.2cm}
%   \fbox{$\VCOV{X}{y}$}
% \end{center}
% 
% \DescribeMacro{\MVARM}
% El comando \cs{MVARM} tiene 1 argumento \cs{MVARM}\marg{regresores} y denota la matriz de varianzas y covarianzas muestral
% \begin{center}
%   |\MVARM{X}|
%   \hspace{1.2cm}
%   \fbox{$\MVARM{X}$}
% \end{center}
% 
% \DescribeMacro{\VCOVM}
% El comando \cs{VCOVM} tiene 2 argumentos \cs{VCOVM}\marg{regresores}\marg{regresando} y denota el vector de covarianzas muestral
% \begin{center}
%   |\VCOVM{X}{y}|
%   \hspace{1.2cm}
%   \fbox{$\VCOVM{X}{y}$}
% \end{center}
% 
% \DescribeMacro{\Normal}
% El comando \cs{Normal} tiene 2 argumentos \cs{Normal}\marg{esperanza}\marg{varianza} y denota la distribución de probabilidad Normal
% \begin{center}
%   |\Normal{\mu}{\sigma^2}|
%   \hspace{1.2cm}
%   \fbox{$\Normal{\mu}{\sigma^2}$}
% \end{center}
% 
% \DescribeMacro{\TStudent}
% El comando \cs{TStudent} tiene 1 argumento \cs{TStudent}\marg{gl} y denota la distribución de probabilidad t de Student
% \begin{center}
%   |\TStudent{N-k}|
%   \hspace{1.2cm}
%   \fbox{$\TStudent{N-k}$}
% \end{center}
% 
% \DescribeMacro{\FSnedecor}
% El comando \cs{FSnedecor} tiene 2 argumentos \cs{FSnedecor}\marg{gl}\marg{gl} y denota la distribución de probabilidad F de Snedecor
% \begin{center}
%   |\FSnedecor{N-k}{r}|
%   \hspace{1.2cm}
%   \fbox{$\FSnedecor{r}{N-k}$}
% \end{center}
% 
% \DescribeMacro{\ChiCuadrado}
% El comando \cs{ChiCuadrado} tiene 1 argumento \cs{ChiCuadrado}\marg{gl} y denota la distribución de probabilidad Chi cuadrado
% \begin{center}
%   |\ChiCuadrado{k}|
%   \hspace{1.2cm}
%   \fbox{$\ChiCuadrado{k}$}
% \end{center}
% 
% \DescribeMacro{\ValorC}
% El comando \cs{ValorCritico} tiene 3 argumentos
% \cs{ValorCritico}\marg{dist}\marg{grados}\marg{prob} y
% denota el valor crítico para una \marg{prob} dada
% \begin{center}
%   |\ValorC{t}{N-k}{\alpha} \ValorC{F}{\! r,N-k}{1-\alpha}|
% 
%   \fbox{$\ValorC{t}{N-k}{\alpha}$}
%   \fbox{$\ValorC{F}{\!r,N-k}{1-\alpha}$}
% \end{center}
% 
% \DescribeMacro{\EstmcE}
% El comando \cs{EstmcE} tiene 1 argumento \cs{EstmcE}\marg{objeto} y denota la estimación de la  esperanza del \marg{objeto}
% \begin{center}
%   |\EstmcE{\VA{Y}} \EstmcE*{\VA{Y}}|
%   \hspace{1.2cm}
%   \fbox{$\EstmcE {\VA{Y}}$}
%   \fbox{$\EstmcE*{\VA{Y}}$}
% \end{center}
% 
% \DescribeMacro{\EstmdE}
% El comando \cs{EstmdE} tiene 1 argumento \cs{EstmdE}\marg{objeto} y denota un estimador de la  esperanza del \marg{objeto}
% \begin{center}
%   |\EstmdE{\VA{Y}}|
%   \hspace{1.2cm}
%   \fbox{$\EstmdE{\VA{Y}}$}
% \end{center}
% 
% \DescribeMacro{\EstmcECond}
% El comando \cs{EstmcECond} tiene 2 argumentos
% \cs{EstmcECond}\marg{objeto1}\marg{objeto1}\marg{objeto2} y denota la
% estimación de esperanza del \marg{objeto1} condicionada al
% \marg{objeto2}
% \begin{center}
%   |\EstmcECond{\VA{Y}}{\SVA{X}} \EstmcECond*{\VA{Y}}{\SVA{X}}|
%   \hspace{1.2cm}
%   \fbox{$\EstmcECond {\VA{Y}}{\SVA{X}}$}
%   \fbox{$\EstmcECond*{\VA{Y}}{\SVA{X}}$}
% \end{center}
% 
% \DescribeMacro{\EstmdECond}
% El comando \cs{EstmdECond} tiene 2 argumentos
% \cs{EstmcECond}\marg{objeto1}\marg{objeto1}\marg{objeto2} y denota un 
% estimador de la esperanza del \marg{objeto1} condicionada al
% \marg{objeto2}
% \begin{center}
%   |\EstmdECond{\VA{Y}}{\SVA{X}} \EstmdECond*{\VA{Y}}{\SVA{X}}|
%   \hspace{1.2cm}
%   \fbox{$\EstmdECond {\VA{Y}}{\SVA{X}}$}
%   \fbox{$\EstmdECond*{\VA{Y}}{\SVA{X}}$}
% \end{center}
% 
% \DescribeMacro{\EstmcDt}
% El comando \cs{EstmcDt} tiene 1 argumento \cs{EstmcDt}\marg{objeto} y denota la estimación de la  desviación típica del \marg{objeto}
% \begin{center}
%   |\EstmcDt{\VA{Y}}|
%   \hspace{1.2cm}
%   \fbox{$\EstmcDt{\VA{Y}}$}
% \end{center}
% 
% \DescribeMacro{\EstmdDt}
% El comando \cs{EstmdDt} tiene 1 argumento \cs{EstmdDt}\marg{objeto} y denota un estimador de la  desviación típica del \marg{objeto}
% \begin{center}
%   |\EstmdDt{\VA{Y}}|
%   \hspace{1.2cm}
%   \fbox{$\EstmdDt{\VA{Y}}$}
% \end{center}
% 
% \DescribeMacro{\EstmcDtCond}
% El comando \cs{EstmcDtCond} tiene 2 argumentos
% \cs{EstmcDtCond}\marg{objeto1}\marg{objeto1}\marg{objeto2} y denota la
% estimación de la desviación típica del \marg{objeto1} condicionada al
% \marg{objeto2}
% \begin{center}
%   |\EstmcDtCond{\VA{Y}}{\SVA{X}}|
%   \hspace{1.2cm}
%   \fbox{$\EstmcDtCond {\VA{Y}}{\SVA{X}}$}
% \end{center}
% 
% \DescribeMacro{\EstmdDtCond}
% El comando \cs{EstmdDtCond} tiene 2 argumentos
% \cs{EstmcDtCond}\marg{objeto1}\marg{objeto1}\marg{objeto2} y denota un 
% estimador de la desviación típica del \marg{objeto1} condicionada al
% \marg{objeto2}
% \begin{center}
%   |\EstmdDtCond{\VA{Y}}{\SVA{X}}|
%   \hspace{1.2cm}
%   \fbox{$\EstmdDtCond {\VA{Y}}{\SVA{X}}$}
% \end{center}
% 
% \DescribeMacro{\EstmcVar}
% El comando \cs{EstmcVar} tiene 1 argumento \cs{EstmcVar}\marg{objeto} y denota la estimación de la varianza del \marg{objeto}
% \begin{center}
%   |\EstmcVar{\VA{Y}}|
%   \hspace{1.2cm}
%   \fbox{$\EstmcVar{\VA{Y}}$}
% \end{center}
% 
% \DescribeMacro{\EstmdVar}
% El comando \cs{EstmdVar} tiene 1 argumento \cs{EstmdVar}\marg{objeto} y denota un estimador de la varianza del \marg{objeto}
% \begin{center}
%   |\EstmdVar{\VA{Y}}|
%   \hspace{1.2cm}
%   \fbox{$\EstmdVar{\VA{Y}}$}
% \end{center}
% 
% \DescribeMacro{\EstmcVarCond}
% El comando \cs{EstmcVarCond} tiene 2 argumentos \cs{EstmcVar}\marg{objeto1}\marg{objeto2} y denota la estimación de la varianza del \marg{objeto1} condicionada al \marg{objeto2}
% \begin{center}
%   |\EstmcVarCond{\VA{Y}}{\SVA{X}}|
%   \hspace{1.2cm}
%   \fbox{$\EstmcVarCond{\VA{Y}}{\SVA{X}}$}
% \end{center}
% 
% \DescribeMacro{\EstmcVarCond}
% El comando \cs{EstmdVarCond} tiene 2 argumentos \cs{EstmcVar}\marg{objeto1}\marg{objeto2} y denota un estimador de la varianza del \marg{objeto1} condicionada al \marg{objeto2}
% \begin{center}
%   |\EstmdVarCond{\VA{Y}}{\SVA{X}}|
%   \hspace{1.2cm}
%   \fbox{$\EstmdVarCond{\VA{Y}}{\SVA{X}}$}
% \end{center}
% 
% \DescribeMacro{\EstmcCov}
% El comando \cs{EstmcCov} tiene 2 argumentos \cs{EstmcCov}\marg{objeto1}\marg{objeto2} y denota la estimación de la covarianza entre ambos objetos
% \begin{center}
%   |\EstmcCov{\VA{X}}{\VA{Y}}|
%   \hspace{1.2cm}
%   \fbox{$\EstmcCov{\VA{X}}{\VA{Y}}$}
% \end{center}
% 
% \DescribeMacro{\EstmdCov}
% El comando \cs{EstmdCov} tiene 2 argumentos \cs{EstmdCov}\marg{objeto1}\marg{objeto2} y denota un estimador de la covarianza entre ambos objetos
% \begin{center}
%   |\EstmdCov{\VA{X}}{\VA{Y}}|
%   \hspace{1.2cm}
%   \fbox{$\EstmdCov{\VA{X}}{\VA{Y}}$}
% \end{center}
% 
% \DescribeMacro{\EstmcCovCond}
% El comando \cs{EstmcCovCond} tiene 2 argumentos \cs{EstmcCovCond}\marg{objeto1}\marg{objeto2}\marg{objeto3} y denota la estimación de la covarianza entre \marg{objeto1} y \marg{objeto2} condicionanda al \marg{objeto3}
% \begin{center}
%   |\EstmcCovCond{\VA{X}}{\VA{Y}}{\SVA{Z}}|
%   \hspace{1.2cm}
%   \fbox{$\EstmcCovCond{\VA{X}}{\VA{Y}}{\SVA{Z}}$}
% \end{center}
% 
% \DescribeMacro{\EstmdCovCond}
% El comando \cs{EstmdCovCond} tiene 2 argumentos \cs{EstmdCovCond}\marg{objeto1}\marg{objeto2}\marg{objeto3} y denota un estimador de la covarianza entre \marg{objeto1} y \marg{objeto2} condicionanda al \marg{objeto3}
% \begin{center}
%   |\EstmdCovCond{\VA{X}}{\VA{Y}}{\SVA{Z}}|
%   \hspace{1.2cm}
%   \fbox{$\EstmdCovCond{\VA{X}}{\VA{Y}}{\SVA{Z}}$}
% \end{center}
% 
% \DescribeMacro{\EstmcCorr}
% El comando \cs{EstmcCorr} tiene 2 argumentos \cs{EstmcCorr}\marg{objeto1}\marg{objeto2} y denota la estimación de la correlación entre ambos objetos
% \begin{center}
%   |\EstmcCorr{\VA{X}}{\VA{Y}}|
%   \hspace{1.2cm}
%   \fbox{$\EstmcCorr{\VA{X}}{\VA{Y}}$}
% \end{center}
% 
% \DescribeMacro{\EstmdCorr}
% El comando \cs{EstmdCorr} tiene 2 argumentos \cs{EstmdCorr}\marg{objeto1}\marg{objeto2} y denota un estimador de la correlación entre ambos objetos
% \begin{center}
%   |\EstmdCorr{\VA{X}}{\VA{Y}}|
%   \hspace{1.2cm}
%   \fbox{$\EstmdCorr{\VA{X}}{\VA{Y}}$}
% \end{center}
% 
% \DescribeMacro{\EstmcCorrCond}
% El comando \cs{EstmcCorrCond} tiene 2 argumentos \cs{EstmcCorrCond}\marg{objeto1}\marg{objeto2}\marg{objeto3} y denota la estimación de la correlación entre \marg{objeto1} y \marg{objeto2} condicionanda al \marg{objeto3}
% \begin{center}
%   |\EstmcCorrCond{\VA{X}}{\VA{Y}}{\SVA{Z}}|
%   \hspace{1.2cm}
%   \fbox{$\EstmcCorrCond{\VA{X}}{\VA{Y}}{\SVA{Z}}$}
% \end{center}
% 
% \DescribeMacro{\EstmdCorrCond}
% El comando \cs{EstmdCorrCond} tiene 2 argumentos \cs{EstmdCorrCond}\marg{objeto1}\marg{objeto2}\marg{objeto3} y denota un estimador de la correlación entre \marg{objeto1} y \marg{objeto2} condicionanda al \marg{objeto3}
% \begin{center}
%   |\EstmdCorrCond{\VA{X}}{\VA{Y}}{\SVA{Z}}|
%   \hspace{1.2cm}
%   \fbox{$\EstmdCorrCond{\VA{X}}{\VA{Y}}{\SVA{Z}}$}
% \end{center}
% 
% \DescribeMacro{\estimEcond}
% El comando \cs{estimEcond} tiene 2 argumentos \cs{estimEcond}\marg{regresando}\marg{regresores} y denota la estimación de la esperanza condicional
% \begin{center}
%   |\estimEcond{\VA{P}}{\text{superficie}}|
%   \hspace{1.2cm}
%   \fbox{$\estimEcond{\VA{P}}{\text{superficie}}$}
% \end{center}
% 
% \DescribeMacro{\Hnula}
% El comando \cs{Hnula} no tiene argumentos y denota una hipótesis nula
% \begin{center}
%   |\Hnula|
%   \hspace{1.2cm}
%   \fbox{$\Hnula$}
% \end{center}
% 
% \DescribeMacro{\Halt}
% El comando \cs{Halt} no tiene argumentos y denota la hipótesis alternativa
% \begin{center}
%   |\Halt|
%   \hspace{1.2cm}
%   \fbox{$\Halt$}
% \end{center}
% 
% \DescribeMacro{\Rcritica}
% El comando \cs{Rcritica} no tiene argumentos y denota la región crítica
% \begin{center}
%   |\Rcritica|
%   \hspace{1.2cm}
%   \fbox{$\Rcritica$}
% \end{center}
% 
% \DescribeMacro{\Racept}
% El comando \cs{Racept} no tiene argumentos y denota la región complementaria a la región crítica
% \begin{center}
%   |\Racept|
%   \hspace{1.2cm}
%   \fbox{$\Racept$}
% \end{center}
% 
% \DescribeMacro{\fdppar}
% El comando \cs{fdppar} tiene 2 argumentos \cs{fpdpar}\oarg{parámetros}\marg{variable} y denota la función de densidad de la \marg{variable}
% \begin{center}
%   |\fdppar{X} \fdppar[\beta]{X}|
%   \hspace{1.2cm}
%   \fbox{$\fdppar       {X}$}
%   \fbox{$\fdppar[\beta]{X}$}
% \end{center}
% 
% \DescribeMacro{\testadistico}
% El comando \cs{testadistico} no tiene argumentos y denota el valor tomado por el estadístico t de student
% \begin{center}
%   |\testadistico|
%   \hspace{1.2cm}
%   \fbox{$\testadistico$}
% \end{center}
% 
% \DescribeMacro{\Testadistico}
% El comando \cs{Testadistico} no tiene argumentos y denota el estadístico t de student
% \begin{center}
%   |\Testadistico|
%   \hspace{1.2cm}
%   \fbox{$\Testadistico$}
% \end{center}
% 
% \DescribeMacro{\festadistico}
% El comando \cs{festadistico} no tiene argumentos y denota el valor tomado por el estadístico F de Snedecor
% \begin{center}
%   |\festadistico|
%   \hspace{1.2cm}
%   \fbox{$\festadistico$}
% \end{center}
% 
% \DescribeMacro{\Festadistico}
% El comando \cs{Festadistico} no tiene argumentos y denota el estadístico F de Snedecor
% \begin{center}
%   |\Festadistico|
%   \hspace{1.2cm}
%   \fbox{$\Festadistico$}
% \end{center}
% 
% \DescribeMacro{\simBajoCond}
% El comando \cs{simBajoCond} tiene 1 argumento \cs{simBajoCond}\marg{condición} y denota "distribución bajo condición"
% \begin{center}
%   |\simBajoCond{x=1}|
%   \hspace{1.2cm}
%   \fbox{$\simBajoCond{x=1}$}
% \end{center}
% 
% \DescribeMacro{\simnula}
% El comando \cs{simnula} no tiene argumentos y denota "distribución bajo \Hnula"
% \begin{center}
%   |\simnula|
%   \hspace{1.2cm}
%   \fbox{$\simnula$}
% \end{center}
% 
% \DescribeMacro{\simNula}
% El comando \cs{simNula} tiene 1 argumento \cs{simNula}\marg{hipótesis} y denota "distribución bajo cierta hipótesis nula"
% \begin{center}
%   |\simNula{\sigma=1}|
%   \hspace{1.2cm}
%   \fbox{$\simNula{\sigma=1}$}
% \end{center}
% 
% \DescribeMacro{\IConfc}
% El comando \cs{IConfc} tiene 2 argumentos \cs{IConfc}\marg{confianza}\marg{objeto} y denota el intervalo de \marg{confianza} del \marg{objeto}
% \begin{center}
%   |\IConfc{1-\alpha}{\MV{R}{\Estmd{\beta}}} \IConfd{1-\alpha}{\MV{R}{\Estmd{\beta}}}|
%   \hspace{1.2cm}
%   \fbox{$\IConfc{1-\alpha}{\MV{R}{\Estmd{\beta}}}$}
%   \fbox{$\IConfd{1-\alpha}{\MV{R}{\Estmd{\beta}}}$}
% \end{center}
% 
% 
% %%%%%%%%%%%%%%%%%%%%%%%%%%%%%%%%%%%%%%%%%%%%%
% 
% \subsection{Sucesiones}
% 
% \DescribeMacro{suc}
% \DescribeMacro{suc*}
% El comando \cs{suc} tiene tres argumentos
% ,\;\cs{esuc}\oarg{ind}\oarg{conjunto}\marg{nombre},\; y denota una
% sucesión. La versión con estrella emplea la notación tradicional y la
% versión sin estrella uso con una notación compacta (que omite los
% detalles relativos a los índides)
% \begin{center}
%   |\suc*{f} \suc*[j]{f} \suc*[j][\Z]{f} \esuc{f} \esuc[j]{f} \esuc[j][\Z]{f}|
%   \hspace{1.2cm}
%   \fbox{$\suc*{f}$}
%   \fbox{$\suc*[j]{f}$}
%   \fbox{$\suc*[j][\Z]{f}$}
%   \fbox{$\suc{f}$}
%   \fbox{$\suc[j]{f}$}
%   \fbox{$\suc[j][\Z]{f}$}
% \end{center}
% 
% \DescribeMacro{esuc}
% \DescribeMacro{esuc*}
% El comando \cs{esuc} tiene dos argumentos
% ,\;\cs{esuc}\oarg{ind}\marg{nombre},\; y denota el elemento de una
% sucesión. La versión con estrella emplea la notación tradicional y la
% versión sin estrella uso con una notación compacta
% \begin{center}
%   |\esuc*{f} \esuc*[j]{f} \esuc{f} \esuc[j]{f}|
%   \hspace{1.2cm}
%   \fbox{$\esuc*{f}$}
%   \fbox{$\esuc*[j]{f}$}
%   \fbox{$\esuc{f}$}
%   \fbox{$\esuc[j]{f}$}
% \end{center}
% 
% \StopEventually{\PrintChanges\PrintIndex}
%
% \section{Implementación}
%
%
% \iffalse
%%%%%%%%%%%%%%%%%%%%%%%%%%%%%%%%%%%%
%% --- Conjuntos de números
%%%%%%%%%%%%%%%%%%%%%%%%%%%%%%%%%%%%
% \fi
%
% \subsection{Conjuntos de números}
%
% \begin{macro}{\Nn}
% \begin{macro}{\Zz}
% \begin{macro}{\Rr}
% \begin{macro}{\Cc}
% \begin{macro}{\Kk}
% Números naturales, enteros, reales y complejos
%    \begin{macrocode}
\NewDocumentCommand\Nn{     }{\ensuremath{ {\mathbb{N}} }\xspace}
\NewDocumentCommand\Zz{     }{\ensuremath{ {\mathbb{Z}} }\xspace}
\NewDocumentCommand\Rr{     }{\ensuremath{ {\mathbb{R}} }\xspace}
\NewDocumentCommand\Cc{     }{\ensuremath{ {\mathbb{C}} }\xspace}
\NewDocumentCommand\Kk{     }{\ensuremath{ {\mathbb{K}} }\xspace}
%    \end{macrocode}
% \end{macro}
% \end{macro}
% \end{macro}
% \end{macro}
% \end{macro}
%
%
% \begin{macro}{\N}
% \begin{macro}{\Z}
% \begin{macro}{\R}
% \begin{macro}{\CC}
% \begin{macro}{\K}
% Números naturales, enteros, reales y complejos con exponente opcional
%    \begin{macrocode}
\NewDocumentCommand\N { O{} }{\ensuremath{ {\Nn}^{#1} }\xspace}
\NewDocumentCommand\Z { O{} }{\ensuremath{ {\Zz}^{#1} }\xspace}
\NewDocumentCommand\R { O{} }{\ensuremath{ {\Rr}^{#1} }\xspace}
\NewDocumentCommand\CC{ O{} }{\ensuremath{ {\Cc}^{#1} }\xspace}
\NewDocumentCommand\K { O{} }{\ensuremath{ {\Kk}^{#1} }\xspace}
%    \end{macrocode}
% \end{macro}
% \end{macro}
% \end{macro}
% \end{macro}
% \end{macro}
%
%
% \subsection{Paréntesis y corchetes}
%
% \iffalse
%%%%%%%%%%%%%%%%%%%%%%%%%%%%%%%%%%%%
%% --- Paréntesis y corchetes
%%%%%%%%%%%%%%%%%%%%%%%%%%%%%%%%%%%%
% \fi
%
% \begin{macro}{\parentesis}
% \begin{macro}{\parentesis*}
% Paréntesis pequeños
%    \begin{macrocode}
\NewDocumentCommand\parentesis{sm}{\ensuremath{\IfBooleanTF#1
       {     ( #2  )    }
       { \big( #2  \big)}    }\xspace}
%    \end{macrocode}
% \end{macro}
% \end{macro}
%
% \begin{macro}{\Parentesis}
% \begin{macro}{\Parentesis*}
% Paréntesis de tamaño variable
%    \begin{macrocode}
\NewDocumentCommand\Parentesis{sm}{\ensuremath{\IfBooleanTF#1
       {\left( #2 \right)}
       { \Big( #2   \Big)}    }\xspace}
%    \end{macrocode}
% \end{macro}
% \end{macro}
%
%
% \begin{macro}{\corchetes}
% \begin{macro}{\corchetes*}
% Corchetes pequeños
%    \begin{macrocode}
\NewDocumentCommand\corchetes{sm}{\ensuremath{\IfBooleanTF#1
       {[#2]}
       { \big[#2  \big]}    }\xspace}
%    \end{macrocode}
% \end{macro}
% \end{macro}
%
%
% \begin{macro}{\Corchetes}
% \begin{macro}{\Corchetes*}
% Corchetes de tamaño variable
%    \begin{macrocode}
\NewDocumentCommand\Corchetes{sm}{\ensuremath{\IfBooleanTF#1
       {\left[#2\right]}
       { \Big[#2  \Big]}    }\xspace}
%    \end{macrocode}
% \end{macro}
% \end{macro}
%
%
% \begin{macro}{\angulos}
% \begin{macro}{\angulos*}
% Angulos de tamaño variable
%    \begin{macrocode}
\NewDocumentCommand\angulos{sm}{\ensuremath{\IfBooleanTF#1
        {    \langle #2     \rangle}
        {\big\langle #2 \big\rangle} }\xspace}
%    \end{macrocode}
% \end{macro}
% \end{macro}
%
%
% \begin{macro}{\Angulos}
% \begin{macro}{\Angulos*}
% Angulos de tamaño variable
%    \begin{macrocode}
\NewDocumentCommand\Angulos{sm}{\ensuremath{\IfBooleanTF#1
        {\left\langle #2 \right\rangle}
        { \Big\langle #2   \Big\rangle} }\xspace}
%    \end{macrocode}
% \end{macro}
% \end{macro}
%
%
% \subsection{Subíndices}
%
% \iffalse
%%%%%%%%%%%%%%%%%%%%%%%%%%%%%%%%%%%%
%% --- Algunos operadores (subíndices a derecha e izquierda)
%%%%%%%%%%%%%%%%%%%%%%%%%%%%%%%%%%%%
% \fi
%
% \begin{macro}{\LRidxE}
% \begin{macro}{\LRidxEp}
% \begin{macro}{\LRidxEp*}
% \begin{macro}{\LRidxEP}
% \begin{macro}{\LRidxEP*}
% \begin{macro}{\LRidxEpE}
% \begin{macro}{\LRidxEpE*}
% \begin{macro}{\LRidxEPE}
% \begin{macro}{\LRidxEPE*}
% Comandos para escribir índices a derecha e izquierda de un objeto (con exponente)
%    \begin{macrocode}
\NewDocumentCommand\LRidxE {mmmm}{\ensuremath{\leftidx{_{#2}^{}}{{#1}}{_{#3}^{#4}}}\xspace}

\NewDocumentCommand\LRidxEp {smmmm}{\ensuremath{\IfBooleanTF#1
                {\LRidxE{\parentesis*{#2}}{#3}{#4}{#5}}
                {\LRidxE{\parentesis {#2}}{#3}{#4}{#5}}     }\xspace}

\NewDocumentCommand\LRidxEP {smmmm}{\ensuremath{\IfBooleanTF#1
                {\LRidxE{\Parentesis*{#2}}{#3}{#4}{#5}}
                {\LRidxE{\Parentesis {#2}}{#3}{#4}{#5}}     }\xspace}

\NewDocumentCommand\LRidxEpE {smmmm}{\ensuremath{\IfBooleanTF#1
                {\parentesis*{\LRidxE{#2}{#3}{#4}{#5}}}
                {\parentesis {\LRidxE{#2}{#3}{#4}{#5}}}     }\xspace}

\NewDocumentCommand\LRidxEPE {smmmm}{\ensuremath{\IfBooleanTF#1
                {\Parentesis*{\LRidxE{#2}{#3}{#4}{#5}}}
                {\Parentesis {\LRidxE{#2}{#3}{#4}{#5}}}     }\xspace}
%    \end{macrocode}
% \end{macro}
% \end{macro}
% \end{macro}
% \end{macro}
% \end{macro}
% \end{macro}
% \end{macro}
% \end{macro}
% \end{macro}
%
%
% \begin{macro}{\LidxE}
% \begin{macro}{\LidxEp}
% \begin{macro}{\LidxEp*}
% \begin{macro}{\LidxEP}
% \begin{macro}{\LidxEP*}
% \begin{macro}{\LidxEpE}
% \begin{macro}{\LidxEpE*}
% \begin{macro}{\LidxEPE}
% \begin{macro}{\LidxEPE*}
% Comandos para escribir índices a la izquierda de un objeto (con exponente)
%    \begin{macrocode}
\NewDocumentCommand\LidxE  { mmm}{\ensuremath{{\leftidx{_{#2}^{}}{{#1}}{_{  }^{#3}}}}\xspace}

\NewDocumentCommand\LidxEp {smmm}{\ensuremath{\IfBooleanTF#1
                {\LidxE{\parentesis*{#2}}{#3}{#4}}
                {\LidxE{\parentesis {#2}}{#3}{#4}}     }\xspace}

\NewDocumentCommand\LidxEP {smmm}{\ensuremath{\IfBooleanTF#1
                {\LidxE{\Parentesis*{#2}}{#3}{#4}}
                {\LidxE{\Parentesis {#2}}{#3}{#4}}     }\xspace}

\NewDocumentCommand\LidxEpE {smmm}{\ensuremath{\IfBooleanTF#1
                {\parentesis*{\LidxE{#2}{#3}{#4}}}
                {\parentesis {\LidxE{#2}{#3}{#4}}}     }\xspace}

\NewDocumentCommand\LidxEPE {smmm}{\ensuremath{\IfBooleanTF#1
                {\Parentesis*{\LidxE{#2}{#3}{#4}}}
                {\Parentesis {\LidxE{#2}{#3}{#4}}}     }\xspace}
%    \end{macrocode}
% \end{macro}
% \end{macro}
% \end{macro}
% \end{macro}
% \end{macro}
% \end{macro}
% \end{macro}
% \end{macro}
% \end{macro}
%
%
% \begin{macro}{\RidxE}
% \begin{macro}{\RidxEp}
% \begin{macro}{\RidxEp*}
% \begin{macro}{\RidxEP}
% \begin{macro}{\RidxEP*}
% \begin{macro}{\RidxEpE}
% \begin{macro}{\RidxEpE*}
% \begin{macro}{\RidxEPE}
% \begin{macro}{\RidxEPE*}
% Comandos para escribir índices a la derecha de un objeto (con exponente)
%    \begin{macrocode}
\NewDocumentCommand\RidxE  { mmm}{\ensuremath{{\leftidx{        }{{#1}}{_{#2}^{#3}}}}\xspace}

\NewDocumentCommand\RidxEp {smmm}{\ensuremath{\IfBooleanTF#1
                {\RidxE{\parentesis*{#2}}{#3}{#4}}
                {\RidxE{\parentesis {#2}}{#3}{#4}}     }\xspace}

\NewDocumentCommand\RidxEP {smmm}{\ensuremath{\IfBooleanTF#1
                {\RidxE{\Parentesis*{#2}}{#3}{#4}}
                {\RidxE{\Parentesis {#2}}{#3}{#4}}     }\xspace}

\NewDocumentCommand\RidxEpE {smmm}{\ensuremath{\IfBooleanTF#1
                {\parentesis*{\RidxE{#2}{#3}{#4}}}
                {\parentesis {\RidxE{#2}{#3}{#4}}}     }\xspace}

\NewDocumentCommand\RidxEPE {smmm}{\ensuremath{\IfBooleanTF#1
                {\Parentesis*{\RidxE{#2}{#3}{#4}}}
                {\Parentesis {\RidxE{#2}{#3}{#4}}}     }\xspace}
%    \end{macrocode}
% \end{macro}
% \end{macro}
% \end{macro}
% \end{macro}
% \end{macro}
% \end{macro}
% \end{macro}
% \end{macro}
% \end{macro}
%
%%%%%%%%%%%%%%%%%%%%%%%%%%%%%%%%%%%%%%%%%%%%%%%%%%%%%%%%%%%%%%%%%%%%%%%%%%
%
% \begin{macro}{\LRidx}
% Comando para escribir un índice a la derecha y otro a la izquierda de un objeto
%    \begin{macrocode}
\NewDocumentCommand\LRidx  { mmm}{\ensuremath{{\LRidxE{#1}{#2}{#3}{}}}\xspace}
%    \end{macrocode}
% \end{macro}
%
%
% \begin{macro}{\LRidxp}
% \begin{macro}{\LRidxp*}
% \begin{macro}{\LRidxP}
% \begin{macro}{\LRidxP*}
% Comandos para escribir un índice a la derecha y otro a la izquierda de un objeto entre paréntesis
%    \begin{macrocode}
\NewDocumentCommand\LRidxp {smmm}{\ensuremath{\IfBooleanTF#1
                {\LRidx{\parentesis*{#2}}{#3}{#4}}
                {\LRidx{\parentesis {#2}}{#3}{#4}}     }\xspace}

\NewDocumentCommand\LRidxP {smmm}{\ensuremath{\IfBooleanTF#1
                {\LRidx{\Parentesis*{#2}}{#3}{#4}}
                {\LRidx{\Parentesis {#2}}{#3}{#4}}     }\xspace}
%    \end{macrocode}
% \end{macro}
% \end{macro}
% \end{macro}
% \end{macro}
%
%
% \begin{macro}{\LRidxpE}
% \begin{macro}{\LRidxpE*}
% \begin{macro}{\LRidxPE}
% \begin{macro}{\LRidxPE*}
% Comando para escribir, entre paréntesis, un índice a la derecha y otro a la izquierda de un objeto
%    \begin{macrocode}
\NewDocumentCommand\LRidxpE {smmm}{\ensuremath{\IfBooleanTF#1
                {\parentesis*{\LRidx{#2}{#3}{#4}}}
                {\parentesis {\LRidx{#2}{#3}{#4}}}     }\xspace}

\NewDocumentCommand\LRidxPE {smmm}{\ensuremath{\IfBooleanTF#1
                {\Parentesis*{\LRidx{#2}{#3}{#4}}}
                {\Parentesis {\LRidx{#2}{#3}{#4}}}     }\xspace}
%    \end{macrocode}
% \end{macro}
% \end{macro}
% \end{macro}
% \end{macro}
%
%%%%%%%%%%%%%%%%%%%%%%%%%%%%%%%%%%%%%%%%%%%%%%%%%%%%%%%%%%%%%%%%%%%%%%%%%%
%
% \begin{macro}{\Lidx}
% Comando para escribir un índice a la izquierda de un objeto
%    \begin{macrocode}
\NewDocumentCommand\Lidx    {  mm}{\ensuremath{\LidxE {#1}{#2}{}    }\xspace}
%    \end{macrocode}
% \end{macro}
%
%
% \begin{macro}{\Lidxp}
% \begin{macro}{\Lidxp*}
% \begin{macro}{\LidxP}
% \begin{macro}{\LidxP*}
% Comandos para escribir un índice a la izquierda de un objeto entre paréntesis
%    \begin{macrocode}
\NewDocumentCommand\Lidxp   { smm}{\ensuremath{\IfBooleanTF#1
                {\Lidx{\parentesis*{#2}}{#3}}
                {\Lidx{\parentesis {#2}}{#3}}     }\xspace}

\NewDocumentCommand\LidxP   { smm}{\ensuremath{\IfBooleanTF#1
                {\Lidx{\Parentesis*{#2}}{#3}}
                {\Lidx{\Parentesis {#2}}{#3}}     }\xspace}
%    \end{macrocode}
% \end{macro}
% \end{macro}
% \end{macro}
% \end{macro}
%
%
% \begin{macro}{\LidxpE}
% \begin{macro}{\LidxpE*}
% \begin{macro}{\LidxPE}
% \begin{macro}{\LidxPE*}
% Comando para escribir, entre paréntesis, un índice a la izquierda de un objeto
%    \begin{macrocode}
\NewDocumentCommand\LidxpE   { smm}{\ensuremath{\IfBooleanTF#1
                {\parentesis*{\Lidx{#2}{#3}}}
                {\parentesis {\Lidx{#2}{#3}}}     }\xspace}

\NewDocumentCommand\LidxPE   { smm}{\ensuremath{\IfBooleanTF#1
                {\Parentesis*{\Lidx{#2}{#3}}}
                {\Parentesis {\Lidx{#2}{#3}}}     }\xspace}
%    \end{macrocode}
% \end{macro}
% \end{macro}
% \end{macro}
% \end{macro}
%
%%%%%%%%%%%%%%%%%%%%%%%%%%%%%%%%%%%%%%%%%%%%%%%%%%%%%%%%%%%%%%%%%%%%%%%%%%
%
% \begin{macro}{\Ridx}
% Comando para escribir un índice a la derecha de un objeto
%    \begin{macrocode}
\NewDocumentCommand\Ridx     {  mm}{\ensuremath{\RidxE {#1}{#2}{}    }\xspace}
%    \end{macrocode}
% \end{macro}
%
%
% \begin{macro}{\Ridxp}
% \begin{macro}{\Ridxp*}
% \begin{macro}{\RidxP}
% \begin{macro}{\RidxP*}
% Comandos para escribir un índice a la derecha de un objeto entre paréntesis
%    \begin{macrocode}
\NewDocumentCommand\Ridxp    {smm}{\ensuremath{\IfBooleanTF#1
                {\Ridx{\parentesis*{#2}}{#3}}
                {\Ridx{\parentesis {#2}}{#3}}     }\xspace}

\NewDocumentCommand\RidxP    {smm}{\ensuremath{\IfBooleanTF#1
                {\Ridx{\Parentesis*{#2}}{#3}}
                {\Ridx{\Parentesis {#2}}{#3}}     }\xspace}
%    \end{macrocode}
% \end{macro}
% \end{macro}
% \end{macro}
% \end{macro}
%
%
% \begin{macro}{\RidxpE}
% \begin{macro}{\RidxpE*}
% \begin{macro}{\RidxPE}
% \begin{macro}{\RidxPE*}
% Comando para escribir, entre paréntesis, un índice a la derecha de un objeto
%    \begin{macrocode}
\NewDocumentCommand\RidxpE {smm}{\ensuremath{\IfBooleanTF#1
                {\parentesis*{\Ridx{#2}{#3}}}
                {\parentesis {\Ridx{#2}{#3}}}     }\xspace}

\NewDocumentCommand\RidxPE {smm}{\ensuremath{\IfBooleanTF#1
                {\Parentesis*{\Ridx{#2}{#3}}}
                {\Parentesis {\Ridx{#2}{#3}}}     }\xspace}
%    \end{macrocode}
% \end{macro}
% \end{macro}
% \end{macro}
% \end{macro}
%
%
% \subsection{Operadores}
%
% \iffalse
%%%%%%%%%%%%%%%%%%%%%%%%%%%%%%%%%%%%
%% --- ALGUNOS OPERADORES
%%%%%%%%%%%%%%%%%%%%%%%%%%%%%%%%%%%%
% \fi
%
%
% \subsubsection{Conjugación y concatenación}
%
% \iffalse
%%%%%%%%%%%%%%%%%%%%%%%%%%%%%%%%%%%%
%% --- Conjugación y concatenación
%%%%%%%%%%%%%%%%%%%%%%%%%%%%%%%%%%%%
% \fi
% \begin{macro}{\widebar}
% Barra ancha para indicar media o conjugación
%    \begin{macrocode}
\NewDocumentCommand\widebar{m}{\mathop{\overline{#1}}}
%    \end{macrocode}
% \end{macro}
%
%
% \begin{macro}{\conj}
% Signo de conjugación
%    \begin{macrocode}
\NewDocumentCommand\conj   {m}{\ensuremath{\widebar{#1}}\xspace}
%    \end{macrocode}
% \end{macro}
%
%
% \begin{macro}{\concat}
% Concatenación
%    \begin{macrocode}
\newcommand{\concat}{\mathbin{\mathpalette\conc@t\relax}}
\newcommand{\conc@t}[2]{%
  \vcenter{\hbox{%
    \sbox\z@{$\m@th#1-$}%
    \setlength{\unitlength}{\wd\z@}%
    \begin{picture}(1,1)
    \roundcap
    \put(0.1,0.5){\line(1,0){0.8}}
    \put(0.35,0.1){\line(0,1){0.8}}
    \put(0.65,0.1){\line(0,1){0.8}}
    \end{picture}%
  }} }
%    \end{macrocode}
% \end{macro}
%
%
% \subsubsection{Norma y valor absoluto}
%
% \iffalse
%%%%%%%%%%%%%%%%%%%%%%%%%%%%%%%%%%%%
%% --- Normas y valor absoluto
%%%%%%%%%%%%%%%%%%%%%%%%%%%%%%%%%%%%
% \fi
%
% \begin{macro}{\norma}
% \begin{macro}{\norma*}
% Norma de un objeto
%    \begin{macrocode}
\NewDocumentCommand\norma{sO{}m}{\ensuremath{\IfBooleanTF#1
            {{\left\lVert{#3}\right\rVert}_{\scriptstyle{#2}}}
            {{     \lVert{#3}      \rVert}_{\scriptstyle{#2}}}  }\xspace}
%    \end{macrocode}
% \end{macro}
% \end{macro}
%
%
% \begin{macro}{\modulus}
% \begin{macro}{\modulus*}
% Valor absoluto
%    \begin{macrocode}
\NewDocumentCommand\modulus{sm}{\ensuremath{\IfBooleanTF#1
            {\left|{#2}\right|}
            {     |{#2}      |}                }\xspace}
%    \end{macrocode}
% \end{macro}
% \end{macro}
%
%
% \begin{macro}{\abs}
% \begin{macro}{\abs*}
% Valor absoluto
%    \begin{macrocode}
\NewDocumentCommand\abs{sm}{\ensuremath{\IfBooleanTF#1
            {\left|{#2}\right|}
            {     |{#2}      |}                }\xspace}
%    \end{macrocode}
% \end{macro}
% \end{macro}
%
%
% \subsubsection{Transposición}
%
% \iffalse
%%%%%%%%%%%%%%%%%%%%%%%%%%%%%%%%%%%%
%% --- Transposición
%%%%%%%%%%%%%%%%%%%%%%%%%%%%%%%%%%%%
% \fi
%
%
%
% \begin{macro}{\T}
% Signo de transposición
%    \begin{macrocode}
\NewDocumentCommand\T{}{\intercal}
%    \end{macrocode}
% \end{macro}
%
%
% \begin{macro}{\Trans}
% \begin{macro}{\Transp}
% \begin{macro}{\Transp*}
% \begin{macro}{\TransP}
% \begin{macro}{\TransP*}
% \begin{macro}{\TranspE}
% \begin{macro}{\TranspE*}
% \begin{macro}{\TransPE}
% \begin{macro}{\TransPE*}
% Transposición
%    \begin{macrocode}
\NewDocumentCommand\Trans  {sm}{\ensuremath{\IfBooleanTF#1
        {\RidxE{#2\big.}{}{\T} }
        {\RidxE{#2}{}{\T}      }                  }\xspace}

\NewDocumentCommand\Transp {sm}{\ensuremath{\IfBooleanTF#1
       {\Trans{{\parentesis*{#2}}}}
       {\Trans{{\parentesis {#2}}}}        }\xspace}

\NewDocumentCommand\TransP {sm}{\ensuremath{\IfBooleanTF#1
       {\Trans{{\Parentesis*{#2}}}}
       {\Trans{{\Parentesis {#2}}}}        }\xspace}

\NewDocumentCommand\TranspE{sm}{\ensuremath{\IfBooleanTF#1
       {\parentesis*{\Trans{#2}}}
       {\parentesis {\Trans{#2}}}          }\xspace}

\NewDocumentCommand\TransPE{sm}{\ensuremath{\IfBooleanTF#1
       {\Parentesis*{\Trans{#2}}}
       {\Parentesis {\Trans{#2}}}          }\xspace}
%    \end{macrocode}
% \end{macro}
% \end{macro}
% \end{macro}
% \end{macro}
% \end{macro}
% \end{macro}
% \end{macro}
% \end{macro}
% \end{macro}
%
%
% \subsubsection{Inversa}
%
% \iffalse
%%%%%%%%%%%%%%%%%%%%%%%%%%%%%%%%%%%%
%% --- Algunos operadores (Inversa)
%%%%%%%%%%%%%%%%%%%%%%%%%%%%%%%%%%%%
% \fi
%
% \begin{macro}{\minus}
% Signo negativo para indicar la inversa
%    \begin{macrocode}
\NewDocumentCommand\minus  { }{\hbox{-}}
%    \end{macrocode}
% \end{macro}
%
% \begin{macro}{\Inv}
% \begin{macro}{\Invp}
% \begin{macro}{\Invp*}
% \begin{macro}{\InvP}
% \begin{macro}{\InvP*}
% \begin{macro}{\InvpE}
% \begin{macro}{\InvpE*}
% \begin{macro}{\InvPE}
% \begin{macro}{\InvPE*}
% Notación de la inversa
%    \begin{macrocode}
\NewDocumentCommand\Inv     {m    }{\ensuremath{ \RidxE{#1}{}{\minus1} }\xspace}

\NewDocumentCommand\Invp    {sm   }{\ensuremath{\IfBooleanTF#1
            {\Inv{{\parentesis*{#2}}}}
            {\Inv{{\parentesis {#2}}}}            }\xspace}

\NewDocumentCommand\InvP    {sm   }{\ensuremath{\IfBooleanTF#1
            {\Inv{{\Parentesis*{#2}}}}
            {\Inv{{\Parentesis {#2}}}}            }\xspace}

\NewDocumentCommand\InvpE   {sm   }{\ensuremath{\IfBooleanTF#1
            {\parentesis*{\Inv{#2}}}
            {\parentesis {\Inv{#2}}}            }\xspace}

\NewDocumentCommand\InvPE   {sm   }{\ensuremath{\IfBooleanTF#1
            {\Parentesis*{\Inv{#2}}}
            {\Parentesis {\Inv{#2}}}            }\xspace}
%    \end{macrocode}
% \end{macro}
% \end{macro}
% \end{macro}
% \end{macro}
% \end{macro}
% \end{macro}
% \end{macro}
% \end{macro}
% \end{macro}
%
%
% \subsubsection{Operador selector}
%
% \iffalse
%%%%%%%%%%%%%%%%%%%%%%%%%%%%%%%%%%%%
%% --- OPERADOR SELECTOR
%%%%%%%%%%%%%%%%%%%%%%%%%%%%%%%%%%%%
% \fi
%
% \begin{macro}{\getItem}
% Signo de operador selector
%    \begin{macrocode}
\NewDocumentCommand\getItem { }{\ensuremath{ \boldsymbol{\mid} }\xspace}
%    \end{macrocode}
% \end{macro}
%
%
% \begin{macro}{\getitemL}
% Operador selector por la izquierda y operador selector por la derecha
%    \begin{macrocode}
\NewDocumentCommand\getitemL{m}{\ensuremath{ {#1} \getItem }\xspace}
\NewDocumentCommand\getitemR{m}{\ensuremath{ \getItem {#1} }\xspace}
%    \end{macrocode}
% \end{macro}
%
%
%%%%%%%%%%%%%%%%%%%%%%%%%%%%%%%%%%%%%%%%%%%%%
% \textbf{selector por la izquierda de un objeto}
%
% \begin{macro}{\elemL}
% \begin{macro}{\elemLp}
% \begin{macro}{\elemLp*}
% \begin{macro}{\elemLP}
% \begin{macro}{\elemLP*}
% \begin{macro}{\elemLpE}
% \begin{macro}{\elemLpE*}
% \begin{macro}{\elemLPE}
% \begin{macro}{\elemLPE*}
% Selector por la izquierda
%    \begin{macrocode}
\NewDocumentCommand\elemL   {mm}{\ensuremath{ \Lidx{#1}{\getitemL{#2}} }\xspace}

\NewDocumentCommand\elemLp {smm}{\ensuremath{\IfBooleanTF#1
      {\elemL{\parentesis*{#2}}{#3}}
      {\elemL{\parentesis {#2}}{#3}} }\xspace}

\NewDocumentCommand\elemLP {smm}{\ensuremath{\IfBooleanTF#1
      {\elemL{\Parentesis*{#2}}{#3}}
      {\elemL{\Parentesis {#2}}{#3}} }\xspace}

\NewDocumentCommand\elemLpE{smm}{\ensuremath{\IfBooleanTF#1
      {\parentesis*{\elemL{#2}{#3}}}
      {\parentesis {\elemL{#2}{#3}}} }\xspace}

\NewDocumentCommand\elemLPE{smm}{\ensuremath{\IfBooleanTF#1
      {\Parentesis*{\elemL{#2}{#3}}}
      {\Parentesis {\elemL{#2}{#3}}} }\xspace}
%    \end{macrocode}
% \end{macro}
% \end{macro}
% \end{macro}
% \end{macro}
% \end{macro}
% \end{macro}
% \end{macro}
% \end{macro}
% \end{macro}
%
%%%%%%%%%%%%%%%%%%%%%%%%%%%%%%%%%%%%%%%%%%%%%
% \textbf{por la derecha de un objeto}
%
% \begin{macro}{\elemR}
% \begin{macro}{\elemRp}
% \begin{macro}{\elemRp*}
% \begin{macro}{\elemRP}
% \begin{macro}{\elemRP*}
% \begin{macro}{\elemRpE}
% \begin{macro}{\elemRpE*}
% \begin{macro}{\elemRPE}
% \begin{macro}{\elemRPE*}
% Selector por la izquierda
%    \begin{macrocode}
\NewDocumentCommand\elemR   {mm}{\ensuremath{ \Ridx{#1}{\getitemR{#2}} }\xspace}

\NewDocumentCommand\elemRp {smm}{\ensuremath{\IfBooleanTF#1
      {\elemR{\parentesis*{#2}}{#3}}
      {\elemR{\parentesis {#2}}{#3}} }\xspace}

\NewDocumentCommand\elemRP {smm}{\ensuremath{\IfBooleanTF#1
      {\elemR{\Parentesis*{#2}}{#3}}
      {\elemR{\Parentesis {#2}}{#3}} }\xspace}

\NewDocumentCommand\elemRpE{smm}{\ensuremath{\IfBooleanTF#1
      {\parentesis*{\elemR{#2}{#3}}}
      {\parentesis {\elemR{#2}{#3}}} }\xspace}

\NewDocumentCommand\elemRPE{smm}{\ensuremath{\IfBooleanTF#1
      {\Parentesis*{\elemR{#2}{#3}}}
      {\Parentesis {\elemR{#2}{#3}}} }\xspace}
%    \end{macrocode}
% \end{macro}
% \end{macro}
% \end{macro}
% \end{macro}
% \end{macro}
% \end{macro}
% \end{macro}
% \end{macro}
% \end{macro}
%
%%%%%%%%%%%%%%%%%%%%%%%%%%%%%%%%%%%%%%%%%%%%%
% \textbf{por ambos lados de un objeto}
%
% \begin{macro}{\elemLR}
% \begin{macro}{\elemLRp}
% \begin{macro}{\elemLRp*}
% \begin{macro}{\elemLRP}
% \begin{macro}{\elemLRP*}
% \begin{macro}{\elemLRpE}
% \begin{macro}{\elemLRpE*}
% \begin{macro}{\elemLRPE}
% \begin{macro}{\elemLRPE*}
% Selectores por ambos lados
%    \begin{macrocode}
\NewDocumentCommand\elemLR{mmm}{
    \ensuremath{ \LRidx{#1}{\getitemL{#2}}{\getitemR{#3}} }\xspace}

\NewDocumentCommand\elemLRp {smmm}{\ensuremath{\IfBooleanTF#1
      {\elemLR{\parentesis*{#2}}{#3}{#4}}
      {\elemLR{\parentesis {#2}}{#3}{#4}}   }\xspace}

\NewDocumentCommand\elemLRpE{smmm}{\ensuremath{\IfBooleanTF#1
      {\parentesis*{\elemLR{#2}{#3}{#4}}}
      {\parentesis {\elemLR{#2}{#3}{#4}}}   }\xspace}

\NewDocumentCommand\elemLRP {smmm}{\ensuremath{\IfBooleanTF#1
      {\elemLR{\Parentesis*{#2}}{#3}{#4}}
      {\elemLR{\Parentesis {#2}}{#3}{#4}}   }\xspace}

\NewDocumentCommand\elemLRPE{smmm}{\ensuremath{\IfBooleanTF#1
      {\Parentesis*{\elemLR{#2}{#3}{#4}}}
      {\Parentesis {\elemLR{#2}{#3}{#4}}}   }\xspace}
%    \end{macrocode}
% \end{macro}
% \end{macro}
% \end{macro}
% \end{macro}
% \end{macro}
% \end{macro}
% \end{macro}
% \end{macro}
% \end{macro}
%
% \iffalse
%%%%%%%%%%%%%%%%%%%%%%%%%%%%%%%%%%%%
%% --- Operador selector con vectores
%%%%%%%%%%%%%%%%%%%%%%%%%%%%%%%%%%%%
% \fi
%
%%%%%%%%%%%%%%%%%%%%%%%%%%%%%%%%%%%%%%
% \textbf{por la izquierda de un vector}
%
% \begin{macro}{\eleVL}
% \begin{macro}{\eleVLp}
% \begin{macro}{\eleVLp*}
% \begin{macro}{\eleVLP}
% \begin{macro}{\eleVLP*}
% \begin{macro}{\eleVLpE}
% \begin{macro}{\eleVLpE*}
% \begin{macro}{\eleVLPE}
% \begin{macro}{\eleVLPE*}
% Selector de elementos de un vector por la izquierda
%    \begin{macrocode}
\NewDocumentCommand\eleVL{omm}{\ensuremath{\IfNoValueTF{#1}%
        {\elemL  {\Vect    {#2}}{#3}}
        {\elemLP*{\Vect[#1]{#2}}{#3}}    }\xspace}

\NewDocumentCommand\eleVLp{somm}{\ensuremath{\IfBooleanTF{#1}%
        {\elemLp*{\Vect[#2]{#3}}{#4}}
        {\elemLp {\Vect[#2]{#3}}{#4}}    }\xspace}

\NewDocumentCommand\eleVLP{somm}{\ensuremath{\IfBooleanTF{#1}%
        {\elemLP*{\Vect[#2]{#3}}{#4}}
        {\elemLP {\Vect[#2]{#3}}{#4}}    }\xspace}

\NewDocumentCommand\eleVLpE{somm}{\ensuremath{\IfBooleanTF#1%
       {\elemLpE*{\IfNoValueTF{#2}
                  {\Vect      {#3}}
                  {\Vectp*[#2]{#3}}}{#4}}
       {\elemLpE {\IfNoValueTF{#2}
                  {\Vect      {#3}}
                  {\Vectp*[#2]{#3}}}{#4}} }\xspace}

\NewDocumentCommand\eleVLPE{somm}{\ensuremath{\IfBooleanTF#1%
       {\elemLPE*{\IfNoValueTF{#2}
                  {\Vect      {#3}}
                  {\VectP*[#2]{#3}}}{#4}}
       {\elemLPE {\IfNoValueTF{#2}
                  {\Vect      {#3}}
                  {\VectP*[#2]{#3}}}{#4}} }\xspace}
%    \end{macrocode}
% \end{macro}
% \end{macro}
% \end{macro}
% \end{macro}
% \end{macro}
% \end{macro}
% \end{macro}
% \end{macro}
% \end{macro}
%
%%%%%%%%%%%%%%%%%%%%%%%%%%%%%%%%%%%%%%%%%%%%%
% \textbf{por la derecha de un vector}
%
% \begin{macro}{\eleVR}
% \begin{macro}{\eleVRp}
% \begin{macro}{\eleVRp*}
% \begin{macro}{\eleVRP}
% \begin{macro}{\eleVRP*}
% \begin{macro}{\eleVRpE}
% \begin{macro}{\eleVRpE*}
% \begin{macro}{\eleVRPE}
% \begin{macro}{\eleVRPE*}
% Selector de elementos de un vector por la derecha
%    \begin{macrocode}
\NewDocumentCommand\eleVR{omm}{\ensuremath{\IfNoValueTF{#1}%
        {\elemR  {\Vect    {#2}}{#3}}
        {\elemRP*{\Vect[#1]{#2}}{#3}}    }\xspace}

\NewDocumentCommand\eleVRp{somm}{\ensuremath{\IfBooleanTF{#1}%
        {\elemRp*{\Vect[#2]{#3}}{#4}}
        {\elemRp {\Vect[#2]{#3}}{#4}}    }\xspace}

\NewDocumentCommand\eleVRP{somm}{\ensuremath{\IfBooleanTF{#1}%
        {\elemRP*{\Vect[#2]{#3}}{#4}}
        {\elemRP {\Vect[#2]{#3}}{#4}}    }\xspace}

\NewDocumentCommand\eleVRpE{somm}{\ensuremath{\IfBooleanTF#1%
       {\elemRpE*{\IfNoValueTF{#2}
                  {\Vect      {#3}}
                  {\Vectp*[#2]{#3}}}{#4}}
       {\elemRpE {\IfNoValueTF{#2}
                  {\Vect      {#3}}
                  {\Vectp*[#2]{#3}}}{#4}} }\xspace}

\NewDocumentCommand\eleVRPE{somm}{\ensuremath{\IfBooleanTF#1%
       {\elemRPE*{\IfNoValueTF{#2}
                  {\Vect      {#3}}
                  {\VectP*[#2]{#3}}}{#4}}
       {\elemRPE {\IfNoValueTF{#2}
                  {\Vect      {#3}}
                  {\VectP*[#2]{#3}}}{#4}} }\xspace}
%    \end{macrocode}
% \end{macro}
% \end{macro}
% \end{macro}
% \end{macro}
% \end{macro}
% \end{macro}
% \end{macro}
% \end{macro}
% \end{macro}
%
% \iffalse
%%%%%%%%%%%%%%%%%%%%%%%%%%%%%%%%%%%%
%% --- Operador selector con matrices
%%%%%%%%%%%%%%%%%%%%%%%%%%%%%%%%%%%%
% \fi
%
% \textbf{de filas de una matriz}

% \begin{macro}{\VectF}
% \begin{macro}{\VectFp}
% \begin{macro}{\VectFp*}
% \begin{macro}{\VectFP}
% \begin{macro}{\VectFP*}
% \begin{macro}{\VectFpE}
% \begin{macro}{\VectFpE*}
% \begin{macro}{\VectFPE}
% \begin{macro}{\VectFPE*}
% Selector de filas de una matriz
%    \begin{macrocode}
\NewDocumentCommand\VectF{omm}{\ensuremath{\IfNoValueTF{#1}%
        {\elemL  {\Mat    {#2}}{#3}}
        {\elemLp*{\Mat[#1]{#2}}{#3}}    }\xspace}

\NewDocumentCommand\VectFp{somm}{\ensuremath{\IfBooleanTF#1%%
        {\elemLp*{\Mat[#2]{#3}}{#4}}
        {\elemLp {\Mat[#2]{#3}}{#4}}  }\xspace}

\NewDocumentCommand\VectFP{somm}{\ensuremath{\IfBooleanTF#1%%
        {\elemLP*{\Mat[#2]{#3}}{#4}}
        {\elemLP {\Mat[#2]{#3}}{#4}}  }\xspace}

\NewDocumentCommand\VectFpE{somm}{\ensuremath{\IfBooleanTF#1%
       {\elemLpE*{\IfNoValueTF{#2}
                  {\Mat      {#3}}
                  {\MatP*[#2]{#3}}}{#4}}
       {\elemLpE {\IfNoValueTF{#2}
                  {\Mat      {#3}}
                  {\Matp*[#2]{#3}}}{#4}} }\xspace}

\NewDocumentCommand\VectFPE{somm}{\ensuremath{\IfBooleanTF#1%
       {\elemLPE*{\IfNoValueTF{#2}
                  {\Mat      {#3}}
                  {\MatP*[#2]{#3}}}{#4}}
       {\elemLPE {\IfNoValueTF{#2}
                  {\Mat      {#3}}
                  {\MatP*[#2]{#3}}}{#4}} }\xspace}
%    \end{macrocode}
% \end{macro}
% \end{macro}
% \end{macro}
% \end{macro}
% \end{macro}
% \end{macro}
% \end{macro}
% \end{macro}
% \end{macro}
%
%
% \begin{macro}{\VectTF}
% \begin{macro}{\VectTFp}
% \begin{macro}{\VectTFp*}
% \begin{macro}{\VectTFP}
% \begin{macro}{\VectTFP*}
% \begin{macro}{\VectTFpE}
% \begin{macro}{\VectTFpE*}
% \begin{macro}{\VectTFPE}
% \begin{macro}{\VectTFPE*}
% Selector de filas de una matriz
%    \begin{macrocode}
\NewDocumentCommand\VectTF{omm}{\ensuremath{\IfNoValueTF{#1}%
        {\elemL  {\MatTpE* {#2}}{#3}}
        {\elemLp*{\MatT[#1]{#2}}{#3}}    }\xspace}

\NewDocumentCommand\VectTFp{somm}{\ensuremath{\IfBooleanTF#1%%
        {\elemLp*{\MatT[#2]{#3}}{#4}}
        {\elemLp {\MatT[#2]{#3}}{#4}}  }\xspace}

\NewDocumentCommand\VectTFP{somm}{\ensuremath{\IfBooleanTF#1%%
        {\elemLP*{\MatT[#2]{#3}}{#4}}
        {\elemLP {\MatT[#2]{#3}}{#4}}  }\xspace}

\NewDocumentCommand\VectTFpE{somm}{\ensuremath{\IfBooleanTF#1%
       {\elemLpE*{\IfNoValueTF{#2}
                  {\MatTpE*    {#3}}
                  {\MatTPE*[#2]{#3}}}{#4}}
       {\elemLpE {\IfNoValueTF{#2}
                  {\MatTpE*    {#3}}
                  {\MatTpE*[#2]{#3}}}{#4}} }\xspace}

\NewDocumentCommand\VectTFPE{somm}{\ensuremath{\IfBooleanTF#1%
       {\elemLPE*{\IfNoValueTF{#2}
                  {\MatTpE*    {#3}}
                  {\MatTPE*[#2]{#3}}}{#4}}
       {\elemLPE {\IfNoValueTF{#2}
                  {\MatTpE*    {#3}}
                  {\MatTPE*[#2]{#3}}}{#4}} }\xspace}
%    \end{macrocode}
% \end{macro}
% \end{macro}
% \end{macro}
% \end{macro}
% \end{macro}
% \end{macro}
% \end{macro}
% \end{macro}
% \end{macro}
%
%%%%%%%%%%%%%%%%%%%%%%%%%%%%%%%%%%%%%%%%%%%%%
% \textbf{de columnas de una matriz}
%
% \begin{macro}{\VectC}
% \begin{macro}{\VectCp}
% \begin{macro}{\VectCp*}
% \begin{macro}{\VectCP}
% \begin{macro}{\VectCP*}
% \begin{macro}{\VectCpE}
% \begin{macro}{\VectCpE*}
% \begin{macro}{\VectCPE}
% \begin{macro}{\VectCPE*}
% Selector de columnas de una matriz
%    \begin{macrocode}
\NewDocumentCommand\VectC{omm}{\ensuremath{\IfNoValueTF{#1}%
        {\elemR  {\Mat    {#2}}{#3}}
        {\elemRp*{\Mat[#1]{#2}}{#3}}    }\xspace}

\NewDocumentCommand\VectCp{somm}{\ensuremath{\IfBooleanTF#1%%
        {\elemRp*{\Mat[#2]{#3}}{#4}}
        {\elemRp {\Mat[#2]{#3}}{#4}}  }\xspace}

\NewDocumentCommand\VectCP{somm}{\ensuremath{\IfBooleanTF#1%%
        {\elemRP*{\Mat[#2]{#3}}{#4}}
        {\elemRP {\Mat[#2]{#3}}{#4}}  }\xspace}

\NewDocumentCommand\VectCpE{somm}{\ensuremath{\IfBooleanTF#1%
       {\elemRpE*{\IfNoValueTF{#2}
                  {\Mat      {#3}}
                  {\MatP*[#2]{#3}}}{#4}}
       {\elemRpE {\IfNoValueTF{#2}
                  {\Mat      {#3}}
                  {\Matp*[#2]{#3}}}{#4}} }\xspace}

\NewDocumentCommand\VectCPE{somm}{\ensuremath{\IfBooleanTF#1%
       {\elemRPE*{\IfNoValueTF{#2}
                  {\Mat      {#3}}
                  {\MatP*[#2]{#3}}}{#4}}
       {\elemRPE {\IfNoValueTF{#2}
                  {\Mat      {#3}}
                  {\MatP*[#2]{#3}}}{#4}} }\xspace}
%    \end{macrocode}
% \end{macro}
% \end{macro}
% \end{macro}
% \end{macro}
% \end{macro}
% \end{macro}
% \end{macro}
% \end{macro}
% \end{macro}
%
%
% \begin{macro}{\VectTC}
% \begin{macro}{\VectTCp}
% \begin{macro}{\VectTCp*}
% \begin{macro}{\VectTCP}
% \begin{macro}{\VectTCP*}
% \begin{macro}{\VectTCpE}
% \begin{macro}{\VectTCpE*}
% \begin{macro}{\VectTCPE}
% \begin{macro}{\VectTCPE*}
% Selector de columnas de una matriz
%    \begin{macrocode}
\NewDocumentCommand\VectTC{omm}{\ensuremath{\IfNoValueTF{#1}%
        {\elemR  {\MatTpE* {#2}}{#3}}
        {\elemRp*{\MatT[#1]{#2}}{#3}}    }\xspace}

\NewDocumentCommand\VectTCp{somm}{\ensuremath{\IfBooleanTF#1%%
        {\elemRp*{\MatT[#2]{#3}}{#4}}
        {\elemRp {\MatT[#2]{#3}}{#4}}  }\xspace}

\NewDocumentCommand\VectTCP{somm}{\ensuremath{\IfBooleanTF#1%%
        {\elemRP*{\MatT[#2]{#3}}{#4}}
        {\elemRP {\MatT[#2]{#3}}{#4}}  }\xspace}

\NewDocumentCommand\VectTCpE{somm}{\ensuremath{\IfBooleanTF#1%
       {\elemRpE*{\IfNoValueTF{#2}
                  {\MatTpE*    {#3}}
                  {\MatTPE*[#2]{#3}}}{#4}}
       {\elemRpE {\IfNoValueTF{#2}
                  {\MatTpE*    {#3}}
                  {\MatTpE*[#2]{#3}}}{#4}} }\xspace}

\NewDocumentCommand\VectTCPE{somm}{\ensuremath{\IfBooleanTF#1%
       {\elemRPE*{\IfNoValueTF{#2}
                  {\MatTpE*    {#3}}
                  {\MatTPE*[#2]{#3}}}{#4}}
       {\elemRPE {\IfNoValueTF{#2}
                  {\MatTpE*    {#3}}
                  {\MatTPE*[#2]{#3}}}{#4}} }\xspace}
%    \end{macrocode}
% \end{macro}
% \end{macro}
% \end{macro}
% \end{macro}
% \end{macro}
% \end{macro}
% \end{macro}
% \end{macro}
% \end{macro}
%
%%%%%%%%%%%%%%%%%%%%%%%%%%%%%%%%%%%%%%%%%%%%%
% \textbf{de elementos de una matriz}
%
% \begin{macro}{\eleM}
% \begin{macro}{\eleMp}
% \begin{macro}{\eleMp*}
% \begin{macro}{\eleMP}
% \begin{macro}{\eleMP*}
% \begin{macro}{\eleMpE}
% \begin{macro}{\eleMpE*}
% \begin{macro}{\eleMPE}
% \begin{macro}{\eleMPE*}
% Selector de elementos de una matriz
%    \begin{macrocode}
\NewDocumentCommand\eleM {ommm}{\ensuremath{\elemLR {\IfNoValueTF{#1}
                                   {\Mat      {#2}}
                                   {\MatP*[#1]{#2}}                }{#3}{#4}}\xspace}

\NewDocumentCommand\eleMp  {sommm}{\ensuremath{\IfBooleanTF#1
                {\elemLRp* {\Mat[#2]{#3}}{#4}{#5}}
                {\elemLRp  {\Mat[#2]{#3}}{#4}{#5}} }\xspace}

\NewDocumentCommand\eleMP  {sommm}{\ensuremath{\IfBooleanTF#1
                {\elemLRP* {\Mat[#2]{#3}}{#4}{#5}}
                {\elemLRP  {\Mat[#2]{#3}}{#4}{#5}} }\xspace}

\NewDocumentCommand\eleMpE{sommm}{\ensuremath{\IfBooleanTF#1
                {\parentesis*{\IfNoValueTF{#1}
                                {\eleM    {#3}{#4}{#5}}
                                {\eleM[#2]{#3}{#4}{#5}}    }}
                {\parentesis {\IfNoValueTF{#1}
                                {\eleM    {#3}{#4}{#5}}
                                {\eleM[#2]{#3}{#4}{#5}}    }}  }\xspace}

\NewDocumentCommand\eleMPE{sommm}{\ensuremath{\IfBooleanTF#1
                {\Parentesis*{\IfNoValueTF{#1}
                                {\eleM    {#3}{#4}{#5}}
                                {\eleM[#2]{#3}{#4}{#5}}    }}
                {\Parentesis {\IfNoValueTF{#1}
                                {\eleM    {#3}{#4}{#5}}
                                {\eleM[#2]{#3}{#4}{#5}}    }}  }\xspace}
%    \end{macrocode}
% \end{macro}
% \end{macro}
% \end{macro}
% \end{macro}
% \end{macro}
% \end{macro}
% \end{macro}
% \end{macro}
% \end{macro}
%
%%%%%%%%%%%%%%%%%%%%%%%%%%%%%%%%%%%%%%%%%%%%%
% \textbf{de elementos de una matriz transpuesta}
%
% \begin{macro}{\eleMT}
% \begin{macro}{\eleMTp}
% \begin{macro}{\eleMTp*}
% \begin{macro}{\eleMTP}
% \begin{macro}{\eleMTP*}
% \begin{macro}{\eleMTpE}
% \begin{macro}{\eleMTpE*}
% \begin{macro}{\eleMTPE}
% \begin{macro}{\eleMTPE*}
% Selector de elementos de una matriz
%    \begin{macrocode}
\NewDocumentCommand\eleMT{ommm}{\ensuremath{\elemLRP*{\MatT[#1]{#2}}{#3}{#4}}\xspace}

\NewDocumentCommand\eleMTp  {sommm}{\ensuremath{\IfBooleanTF#1
                {\elemLRp*{\parentesis*{\Mat[#2]{#3}}^\T}{#4}{#5}}
                {\elemLRp {\parentesis*{\Mat[#2]{#3}}^\T}{#4}{#5}}  }\xspace}

\NewDocumentCommand\eleMTP  {sommm}{\ensuremath{\IfBooleanTF#1
                {\elemLRP*{\parentesis*{\Mat[#2]{#3}}^\T}{#4}{#5}}
                {\elemLRP {\parentesis*{\Mat[#2]{#3}}^\T}{#4}{#5}}  }\xspace}

\NewDocumentCommand\eleMTpE{sommm}{\ensuremath{\IfBooleanTF#1
                {\parentesis*{\IfNoValueTF{#1}
                                {\eleMT    {#3}{#4}{#5}}
                                {\eleMT[#2]{#3}{#4}{#5}}    }}
                {\parentesis {\IfNoValueTF{#1}
                                {\eleMT    {#3}{#4}{#5}}
                                {\eleMT[#2]{#3}{#4}{#5}}    }}  }\xspace}

\NewDocumentCommand\eleMTPE{sommm}{\ensuremath{\IfBooleanTF#1
                {\Parentesis*{\IfNoValueTF{#1}
                                {\eleMT    {#3}{#4}{#5}}
                                {\eleMT[#2]{#3}{#4}{#5}}    }}
                {\Parentesis {\IfNoValueTF{#1}
                                {\eleMT    {#3}{#4}{#5}}
                                {\eleMT[#2]{#3}{#4}{#5}}    }}  }\xspace}
%    \end{macrocode}
% \end{macro}
% \end{macro}
% \end{macro}
% \end{macro}
% \end{macro}
% \end{macro}
% \end{macro}
% \end{macro}
% \end{macro}
%
%
% \subsubsection{Operaciones elementales}
%
% \iffalse
%%%%%%%%%%%%%%%%%%%%%%%%%%%%%%%%%%%%
%% --- Operaciones elementales y permutaciones
%%%%%%%%%%%%%%%%%%%%%%%%%%%%%%%%%%%%
% \fi
%
% \begin{macro}{\TrEl}
% Signo de transformación elemental
%    \begin{macrocode}
\DeclareMathOperator{\TrEl}{\boldsymbol{\tau}}
%    \end{macrocode}
% \end{macro}
%
%
% \begin{macro}{\su}
% Transformación elemental Tipo I
%    \begin{macrocode}
\NewDocumentCommand\su{mmm}{\ensuremath{%
                      \left(#1\right){\boldsymbol{#2}}+{\boldsymbol{#3}}   }\xspace}
%    \end{macrocode}
% \end{macro}
%
%
% \begin{macro}{\pr}
% Transformación elemental Tipo II
%    \begin{macrocode}
\NewDocumentCommand\pr{mm }{\ensuremath{%
                      \left(#1\right){\boldsymbol{#2}}                     }\xspace}
%    \end{macrocode}
% \end{macro}
%
%
% \begin{macro}{\pe}
% Intercambio (permuta de dos elementos)
%    \begin{macrocode}
\NewDocumentCommand\pe{mm }{\ensuremath{%
                     \boldsymbol{#1} \rightleftharpoons \boldsymbol{#2}    }\xspace}
%    \end{macrocode}
% \end{macro}
%
%
% \begin{macro}{\perm}
% Reordenamiento de los elementos (permutación)
%    \begin{macrocode}
\NewDocumentCommand\perm{}{\ensuremath{ \mathfrak{S}                       }\xspace}
%    \end{macrocode}
% \end{macro}
%
%%%%%%%%%%%%%%%%%%%%%%%%%%%%%%%%%%%%%%%%%%%%%%%%%%%%%%%%%%%
%
%
% \begin{macro}{\OpE}
% Operación elemental
%    \begin{macrocode}
\NewDocumentCommand\OpE{m}{\ensuremath{\underset{\left[{#1}\right]}{\TrEl}}\xspace}
%    \end{macrocode}
% \end{macro}
%
%
% \begin{macro}{\OEsu}
% Oper. elem. que suma un múltiplo de una componente a otra
%    \begin{macrocode}
\NewDocumentCommand\OEsu {mmm}{\ensuremath{ \OpE{ \su{#1}{#2}{#3} } }\xspace}
%    \end{macrocode}
% \end{macro}
%
%
% \begin{macro}{\OEpr}
% Oper. elem. que multiplica una componente por un número
%    \begin{macrocode}
\NewDocumentCommand\OEpr  {mm}{\ensuremath{ \OpE{ \pr{#1}{#2}     } }\xspace}
%    \end{macrocode}
% \end{macro}
%
%
% \begin{macro}{\OEin}
% Intercambio de posición entre componentes
%    \begin{macrocode}
\NewDocumentCommand\OEin  {mm}{\ensuremath{ \OpE{ \pe{#1}{#2}     } }\xspace}
%    \end{macrocode}
% \end{macro}
%
%
% \begin{macro}{\OEper}
% Reordenamiento o permutación entre componentes
%    \begin{macrocode}
\NewDocumentCommand\OEper   {}{\ensuremath{ \OpE{ \perm           } }\xspace}
%    \end{macrocode}
% \end{macro}
%
%
% \begin{macro}{\EOEsu}
% Espejo de oper. elem. que suma un múltiplo de una componente a otra
%    \begin{macrocode}
\NewDocumentCommand\EOEsu{mmm}{\ensuremath{ esp\Big(\OEsu{#1}{#2}{#3}\Big)}\xspace}
%    \end{macrocode}
% \end{macro}
%
%
% \begin{macro}{\EOEpr}
% Espejo de oper. elem. que multiplica una componente por un número
%    \begin{macrocode}
\NewDocumentCommand\EOEpr {mm}{\ensuremath{ esp\Big(\OEpr{#1}{#2}    \Big)}\xspace}
%    \end{macrocode}
% \end{macro}
%
%
% \paragraph{Transformaciones elementales generales}
%
% \iffalse
%%%%%%%%%%%%%%%%%%%%%%%%%%%%%%%%%%%%
%% --- Operaciones elementales genéricas
%%%%%%%%%%%%%%%%%%%%%%%%%%%%%%%%%%%%
% \fi
%
% \begin{macro}{\OEg}
% Operación elemental genérica
%    \begin{macrocode}
\NewDocumentCommand  \OEg{O{}O{}}{\ensuremath{ \RidxE{\TrEl}{\!#1}{#2} }\xspace}
%    \end{macrocode}
% \end{macro}
%
%
% \begin{macro}{\EOEg}
% Operación espejo de una operación elemental genérica
%    \begin{macrocode}
\NewDocumentCommand \EOEg{O{}O{}}{\ensuremath{ esp( \OEg[#1][#2]      ) }\xspace}
%    \end{macrocode}
% \end{macro}
%
%
% \begin{macro}{\InvOEg}
% Inversa de una operación elemental genérica
%    \begin{macrocode}
\NewDocumentCommand \InvOEg{O{} }{\ensuremath{      \OEg[#1][\minus1]   }\xspace}
%    \end{macrocode}
% \end{macro}
%
%
% \begin{macro}{\EInvOEg}
% Espejo de la inversa de una operación elemental genérica
%    \begin{macrocode}
\NewDocumentCommand\EInvOEg{O{} }{\ensuremath{ esp( \InvOEg[#1]       ) }\xspace}
%    \end{macrocode}
% \end{macro}
%
%
% \begin{macro}{\SOEg}
% Sucesión de operaciones elementales genéricas
%    \begin{macrocode}
\NewDocumentCommand\SOEg{O{1}O{k}O{}}{\ensuremath{%
                                       \OEg[#1][#3]\cdots\OEg[#2][#3]  }\xspace}
%    \end{macrocode}
% \end{macro}
%
%%%%%%%%%%%%%%%%%%%%%%%%%%%%%%%%%%%%%%%%%%%%%%%%%%%%%%%%%%
%
% \begin{macro}{\dOEgE}
% \begin{macro}{\dOEg}
% Operación elemental genérica con exponente y sin exponente
%    \begin{macrocode}
\NewDocumentCommand\dOEgE  {mm}{\ensuremath{ \RidxE{\TrEl}{\!#1}{#2}    }\xspace}
\NewDocumentCommand\dOEg    {m}{\ensuremath{     \dOEgE{#1}{}           }\xspace}
%    \end{macrocode}
% \end{macro}
% \end{macro}
%
%
% \begin{macro}{\dEOEgE}
% \begin{macro}{\dEOEg}
% Operación espejo de una elemental genérica con exponente y sin exponente
%    \begin{macrocode}
\NewDocumentCommand\dEOEgE {mm}{\ensuremath{ esp(\dOEgE{#1}{#2}     )   }\xspace}
\NewDocumentCommand\dEOEg   {m}{\ensuremath{ esp(\dOEg {#1}         )   }\xspace}
%    \end{macrocode}
% \end{macro}
% \end{macro}
%
%
% \begin{macro}{\dInvOEg}
% Operación inversa de una elemental genérica
%    \begin{macrocode}
\NewDocumentCommand\dInvOEg {m}{\ensuremath{     \dOEgE{#1}{\minus1}    }\xspace}
%    \end{macrocode}
% \end{macro}
%
%
% \begin{macro}{\dEInvOEg}
% Operación espejo de la inversa de una elemental genérica
%    \begin{macrocode}
\NewDocumentCommand\dEInvOEg{m}{\ensuremath{ esp(\dInvOEg{#1} )         }\xspace}
%    \end{macrocode}
% \end{macro}
%
%
% \iffalse
%%%%%%%%%%%%%%%%%%%%%%%%%%%%%%%%%%%%
%% --- Sucesión de operaciones elementales genéricas
%%%%%%%%%%%%%%%%%%%%%%%%%%%%%%%%%%%%
% \fi
%
%
% \begin{macro}{\dSOEgE}
% \begin{macro}{\dSOEg}
% Sucesión de operaciones elementales genéricas con exponente o sin exponente
%    \begin{macrocode}
\NewDocumentCommand\dSOEgE{mmm}{\ensuremath{\dOEgE{#1}{#3}\cdots\dOEgE{#2}{#3}}\xspace}
\NewDocumentCommand\dSOEg  {mm}{\ensuremath{\dOEg {#1}    \cdots\dOEg {#2}    }\xspace}
%    \end{macrocode}
% \end{macro}
% \end{macro}
%
% \subsubsection{Transformaciones elementales particulares}
%
% \iffalse
%%%%%%%%%%%%%%%%%%%%%%%%%%%%%%%%%%%%
%% --- Transformaciones elementales y permutaciones
%%%%%%%%%%%%%%%%%%%%%%%%%%%%%%%%%%%%
% \fi
%
% \paragraph{Transf. elemental aplicada la izquierda o derecha de un objeto}
%
% \iffalse
%%%%%%%%%%%%%%%%%%%%%%%%%%%%%%%%%%%%
%% --- Transf. elemental aplicada la izquierda o derecha de un objeto
%%%%%%%%%%%%%%%%%%%%%%%%%%%%%%%%%%%%
% \fi
%
%%%%%%%%%%%% Tipo I - Fil %%%%%%%%%%%%%%%%%%%%%%%%%%%%%%
%
% \begin{macro}{\TESF}
% \begin{macro}{\TESFp}
% \begin{macro}{\TESFP}
% \begin{macro}{\TESFpE}
% \begin{macro}{\TESFPE}
% Una transformación elemental Tipo I por la izquierda
%    \begin{macrocode}
\NewDocumentCommand\TESF  {mmmm}{\ensuremath{\Lidx{#4}{  \OEsu{#1}{#2}{#3}\!}}\xspace}
\NewDocumentCommand\TESFp {smmmm}{\ensuremath{\IfBooleanTF#1
  {\Lidxp* {#5}{\OEsu{#2}{#3}{#4}\!\!}} {\Lidxp {#5}{\OEsu{#2}{#3}{#4}\!\!}} }\xspace}
\NewDocumentCommand\TESFP {smmmm}{\ensuremath{\IfBooleanTF#1
  {\LidxP* {#5}{\OEsu{#2}{#3}{#4}\!\!}} {\LidxP {#5}{\OEsu{#2}{#3}{#4}\!\!}} }\xspace}
\NewDocumentCommand\TESFpE{smmmm}{\ensuremath{\IfBooleanTF#1
  {\LidxpE*{#5}{\OEsu{#2}{#3}{#4}\!  }} {\LidxpE{#5}{\OEsu{#2}{#3}{#4}\!  }} }\xspace}
\NewDocumentCommand\TESFPE{smmmm}{\ensuremath{\IfBooleanTF#1
  {\LidxPE*{#5}{\OEsu{#2}{#3}{#4}\!  }} {\LidxPE{#5}{\OEsu{#2}{#3}{#4}\!  }} }\xspace}
%    \end{macrocode}
% \end{macro}
% \end{macro}
% \end{macro}
% \end{macro}
% \end{macro}
%
%%%%%%%%%%%% Tipo I - Col %%%%%%%%%%%%%%%%%%%%%%%%%%%%%%
%
% \begin{macro}{\TESC}
% \begin{macro}{\TESC}
% \begin{macro}{\TESC}
% \begin{macro}{\TESC}
% \begin{macro}{\TESC}
% Una transformación elemental Tipo I por la derecha
%    \begin{macrocode}
\NewDocumentCommand\TESC  {mmmm}{\ensuremath{\Ridx{#4}{\!\OEsu{#1}{#2}{#3}  }}\xspace}
\NewDocumentCommand\TESCp {smmmm}{\ensuremath{\IfBooleanTF#1
  {\Ridxp* {#5}{\!\!\OEsu{#2}{#3}{#4}}} {\Ridxp {#5}{\!\!\OEsu{#2}{#3}{#4}}} }\xspace}
\NewDocumentCommand\TESCP {smmmm}{\ensuremath{\IfBooleanTF#1
  {\RidxP* {#5}{\!\!\OEsu{#2}{#3}{#4}}} {\RidxP {#5}{\!\!\OEsu{#2}{#3}{#4}}} }\xspace}
\NewDocumentCommand\TESCpE{smmmm}{\ensuremath{\IfBooleanTF#1
  {\RidxpE*{#5}{\!  \OEsu{#2}{#3}{#4}}} {\RidxpE{#5}{\!  \OEsu{#2}{#3}{#4}}} }\xspace}
\NewDocumentCommand\TESCPE{smmmm}{\ensuremath{\IfBooleanTF#1
  {\RidxPE*{#5}{\!  \OEsu{#2}{#3}{#4}}} {\RidxPE{#5}{\!  \OEsu{#2}{#3}{#4}}} }\xspace}
%    \end{macrocode}
% \end{macro}
% \end{macro}
% \end{macro}
% \end{macro}
% \end{macro}
%
%%%%%%%%%%%% Tipo II - Fil %%%%%%%%%%%%%%%%%%%%%%%%%%%%%%
%
% \begin{macro}{\TEPF}
% \begin{macro}{\TEPFp}
% \begin{macro}{\TEPFP}
% \begin{macro}{\TEPFpE}
% \begin{macro}{\TEPFPE}
% Una transformación elemental Tipo II por la izquierda
%    \begin{macrocode}
\NewDocumentCommand\TEPF   {mmm}{\ensuremath{\Lidx{#3}{  \OEpr{#1}{#2}\!} }\xspace}
\NewDocumentCommand\TEPFp {smmm}{\ensuremath{\IfBooleanTF#1
      {\Lidxp* {#4}{\OEpr{#2}{#3}\!\!}} {\Lidxp {#4}{\OEpr{#2}{#3}\!\!} } }\xspace}
\NewDocumentCommand\TEPFP {smmm}{\ensuremath{\IfBooleanTF#1
      {\LidxP* {#4}{\OEpr{#2}{#3}\!\!}} {\LidxP {#4}{\OEpr{#2}{#3}\!\!} } }\xspace}
\NewDocumentCommand\TEPFpE{smmm}{\ensuremath{\IfBooleanTF#1
      {\LidxpE*{#4}{\OEpr{#2}{#3}\!  }} {\LidxpE{#4}{\OEpr{#2}{#3}\!}   } }\xspace}
\NewDocumentCommand\TEPFPE{smmm}{\ensuremath{\IfBooleanTF#1
      {\LidxPE*{#4}{\OEpr{#2}{#3}\!  }} {\LidxPE{#4}{\OEpr{#2}{#3}\!}   } }\xspace}
%    \end{macrocode}
% \end{macro}
% \end{macro}
% \end{macro}
% \end{macro}
% \end{macro}
%
%%%%%%%%%%%% Tipo II - Col %%%%%%%%%%%%%%%%%%%%%%%%%%%%%%
%
% \begin{macro}{\TEPC}
% \begin{macro}{\TEPCp}
% \begin{macro}{\TEPCP}
% \begin{macro}{\TEPCpE}
% \begin{macro}{\TEPCPE}
% Una transformación elemental Tipo II por la derecha
%    \begin{macrocode}
\NewDocumentCommand\TEPC   {mmm}{\ensuremath{\Ridx{#3}{\!\OEpr{#1}{#2}  } }\xspace}
\NewDocumentCommand\TEPCp {smmm}{\ensuremath{\IfBooleanTF#1
      {\Ridxp* {#4}{\!\!\OEpr{#2}{#3}}} {\Ridxp {#4}{\!\!\OEpr{#2}{#3}} } }\xspace}
\NewDocumentCommand\TEPCP {smmm}{\ensuremath{\IfBooleanTF#1
      {\RidxP* {#4}{\!\!\OEpr{#2}{#3}}} {\RidxP {#4}{\!\!\OEpr{#2}{#3}} } }\xspace}
\NewDocumentCommand\TEPCpE{smmm}{\ensuremath{\IfBooleanTF#1
      {\RidxpE*{#4}{\!  \OEpr{#2}{#3}}} {\RidxpE{#4}{\!\OEpr{#2}{#3}}   } }\xspace}
\NewDocumentCommand\TEPCPE{smmm}{\ensuremath{\IfBooleanTF#1
      {\RidxPE*{#4}{\!  \OEpr{#2}{#3}}} {\RidxPE{#4}{\!\OEpr{#2}{#3}}   } }\xspace}
%    \end{macrocode}
% \end{macro}
% \end{macro}
% \end{macro}
% \end{macro}
% \end{macro}
%
%%%%%%%%%%%% Intercambio - Fil %%%%%%%%%%%%%%%%%%%%%%%%%%%%%%
%
% \begin{macro}{\TEIF}
% \begin{macro}{\TEIFp}
% \begin{macro}{\TEIFP}
% \begin{macro}{\TEIFpE}
% \begin{macro}{\TEIFPE}
% Intercambio por la izquierda
%    \begin{macrocode}
\NewDocumentCommand\TEIF   {mmm}{\ensuremath{\Lidx{#3}{  \OEin{#1}{#2}\!} }\xspace}
\NewDocumentCommand\TEIFp {smmm}{\ensuremath{\IfBooleanTF#1
      {\Lidxp* {#4}{\OEin{#2}{#3}\!\!}} {\Lidxp {#4}{\OEin{#2}{#3}\!\!} } }\xspace}
\NewDocumentCommand\TEIFP {smmm}{\ensuremath{\IfBooleanTF#1
      {\LidxP* {#4}{\OEin{#2}{#3}\!\!}} {\LidxP {#4}{\OEin{#2}{#3}\!\!} } }\xspace}
\NewDocumentCommand\TEIFpE{smmm}{\ensuremath{\IfBooleanTF#1
      {\LidxpE*{#4}{\OEin{#2}{#3}\!  }} {\LidxpE{#4}{\OEin{#2}{#3}\!}   } }\xspace}
\NewDocumentCommand\TEIFPE{smmm}{\ensuremath{\IfBooleanTF#1
      {\LidxPE*{#4}{\OEin{#2}{#3}\!  }} {\LidxPE{#4}{\OEin{#2}{#3}\!}   } }\xspace}
%    \end{macrocode}
% \end{macro}
% \end{macro}
% \end{macro}
% \end{macro}
% \end{macro}
%
%%%%%%%%%%%% Intercambio - Col %%%%%%%%%%%%%%%%%%%%%%%%%%%%%%
%
% \begin{macro}{\TEIC}
% \begin{macro}{\TEICp}
% \begin{macro}{\TEICP}
% \begin{macro}{\TEICpE}
% \begin{macro}{\TEICPE}
% Intercambio por la derecha
%    \begin{macrocode}
\NewDocumentCommand\TEIC   {mmm}{\ensuremath{\Ridx{#3}{\!\OEin{#1}{#2}  } }\xspace}
\NewDocumentCommand\TEICp {smmm}{\ensuremath{\IfBooleanTF#1
      {\Ridxp* {#4}{\!\!\OEin{#2}{#3}}} {\Ridxp {#4}{\!\!\OEin{#2}{#3}} } }\xspace}
\NewDocumentCommand\TEICP {smmm}{\ensuremath{\IfBooleanTF#1
      {\RidxP* {#4}{\!\!\OEin{#2}{#3}}} {\RidxP {#4}{\!\!\OEin{#2}{#3}} } }\xspace}
\NewDocumentCommand\TEICpE{smmm}{\ensuremath{\IfBooleanTF#1
      {\RidxpE*{#4}{\!  \OEin{#2}{#3}}} {\RidxpE{#4}{\!\OEin{#2}{#3}}   } }\xspace}
\NewDocumentCommand\TEICPE{smmm}{\ensuremath{\IfBooleanTF#1
      {\RidxPE*{#4}{\!  \OEin{#2}{#3}}} {\RidxPE{#4}{\!\OEin{#2}{#3}}   } }\xspace}
%    \end{macrocode}
% \end{macro}
% \end{macro}
% \end{macro}
% \end{macro}
% \end{macro}
%
%%%%%%%%%%%%%%%%%%%%%%%%%%%%%%%%%%%%%%%%%%
%
% \begin{macro}{\Mint}
% \begin{macro}{\MintT}
% Matriz intercambio y matriz intercambio (filas)
%    \begin{macrocode}
\NewDocumentCommand\Mint {mm}{\ensuremath{ \TEIC{#1}{#2}{\Mat{I}} }\xspace}
\NewDocumentCommand\MintT{mm}{\ensuremath{ \TEIF{#1}{#2}{\Mat{I}} }\xspace}
%    \end{macrocode}
% \end{macro}
% \end{macro}
%
%
% \begin{macro}{\PF}
% \begin{macro}{\PC}
% Permutación por la izquierda y permutación por la derecha
%    \begin{macrocode}
\NewDocumentCommand\PF    {m}{\ensuremath{ \Lidx{#1}{  \OEper\! } }\xspace}
\NewDocumentCommand\PC    {m}{\ensuremath{ \Ridx{#1}{\!\OEper   } }\xspace}
%    \end{macrocode}
% \end{macro}
% \end{macro}
%
%
% \begin{macro}{\MP}
% \begin{macro}{\MPT}
% Matriz permutación y matriz permutación
%    \begin{macrocode}
\NewDocumentCommand\MP     {}{\ensuremath{ \PC  {\Mat{I}}         }\xspace}
\NewDocumentCommand\MPT    {}{\ensuremath{ \PF  {\Mat{I}}         }\xspace}
%    \end{macrocode}
% \end{macro}
% \end{macro}
%
%
% \paragraph{Sucesiones indiciadas de Transf. elementales}
%
% \iffalse
%%%%%%%%%%%%%%%%%%%%%%%%%%%%%%%%%%%%
%% --- Sucesiones indiciadas de Transf. elementales
%%%%%%%%%%%%%%%%%%%%%%%%%%%%%%%%%%%%
% \fi
%
% \begin{macro}{\SITEF}
% \begin{macro}{\SITEFp}
% \begin{macro}{\SITEFP}
% \begin{macro}{\SITEFpE}
% \begin{macro}{\SITEFPE}
% Sucesión indiciada de transformaciones elementales genéricas por la izquierda (filas)
%    \begin{macrocode}
\NewDocumentCommand\SITEF  {mmm}{\ensuremath{ \Lidx{#3}{\dSOEg{#1}{#2}}    }\xspace}

\NewDocumentCommand\SITEFp{smmm}{\ensuremath{\IfBooleanTF#1
            {\SITEF  {#2}{#3}{\parentesis*{#4}}}
            {\SITEF  {#2}{#3}{\parentesis {#4}}}                       }\xspace}

\NewDocumentCommand\SITEFP{smmm}{\ensuremath{\IfBooleanTF#1
            {\SITEF  {#2}{#3}{\Parentesis*{#4}}}
            {\SITEF  {#2}{#3}{\Parentesis {#4}}}                       }\xspace}

\NewDocumentCommand\SITEFpE{smmm}{\ensuremath{\IfBooleanTF#1
            {\parentesis*{\SITEF {#2}{#3}{#4}}}
            {\parentesis {\SITEF {#2}{#3}{#4}}}                        }\xspace}

\NewDocumentCommand\SITEFPE{smmm}{\ensuremath{\IfBooleanTF#1
            {\Parentesis*{\SITEF {#2}{#3}{#4}}}
            {\Parentesis {\SITEF {#2}{#3}{#4}}}                        }\xspace}
%    \end{macrocode}
% \end{macro}
% \end{macro}
% \end{macro}
% \end{macro}
% \end{macro}
%
% \begin{macro}{\SITEC}
% \begin{macro}{\SITECp}
% \begin{macro}{\SITECP}
% \begin{macro}{\SITECpE}
% \begin{macro}{\SITECPE}
% Sucesión indiciada de transformaciones elementales genéricas por la izquierda (filas)
%    \begin{macrocode}
\NewDocumentCommand\SITEC  {mmm}{\ensuremath{ \Ridx{#3}{\dSOEg{#1}{#2}}    }\xspace}

\NewDocumentCommand\SITECp{smmm}{\ensuremath{\IfBooleanTF#1
            {\SITEC  {#2}{#3}{\parentesis*{#4}}}
            {\SITEC  {#2}{#3}{\parentesis {#4}}}                       }\xspace}

\NewDocumentCommand\SITECP{smmm}{\ensuremath{\IfBooleanTF#1
            {\SITEC  {#2}{#3}{\Parentesis*{#4}}}
            {\SITEC  {#2}{#3}{\Parentesis {#4}}}                       }\xspace}

\NewDocumentCommand\SITECpE{smmm}{\ensuremath{\IfBooleanTF#1
            {\parentesis*{\SITEC {#2}{#3}{#4}}}
            {\parentesis {\SITEC {#2}{#3}{#4}}}                        }\xspace}

\NewDocumentCommand\SITECPE{smmm}{\ensuremath{\IfBooleanTF#1
            {\Parentesis*{\SITEC {#2}{#3}{#4}}}
            {\Parentesis {\SITEC {#2}{#3}{#4}}}                        }\xspace}
%    \end{macrocode}
% \end{macro}
% \end{macro}
% \end{macro}
% \end{macro}
% \end{macro}
%
%
% \begin{macro}{\SITEFC}
% \begin{macro}{\SITEFCp}
% \begin{macro}{\SITEFCP}
% \begin{macro}{\SITEFCpE}
% \begin{macro}{\SITEFCPE}
% Sucesión de transformaciones elementales genéricas a izquierda y derecha
%    \begin{macrocode}
\NewDocumentCommand\SITEFC{mmm}{\ensuremath{
                            {\LRidx{#3}{\dSOEg{#2}{#1}}{\dSOEg{#1}{#2}}}   }\xspace}                   \NewDocumentCommand\SITEFCp{smmm}{\ensuremath{\IfBooleanTF#1
            {\SITEFC {#2}{#3}{\parentesis*{#4}}}
            {\SITEFC {#2}{#3}{\parentesis {#4}}}                       }\xspace}
\NewDocumentCommand\SITEFCP{smmm}{\ensuremath{\IfBooleanTF#1
            {\SITEFC {#2}{#3}{\Parentesis*{#4}}}
            {\SITEFC {#2}{#3}{\Parentesis {#4}}}                       }\xspace}
\NewDocumentCommand\SITEFCpE{smmm}{\ensuremath{\IfBooleanTF#1
            {\parentesis*{\SITEFC {#2}{#3}{#4}}}
            {\parentesis {\SITEFC {#2}{#3}{#4}}}                      }\xspace}
\NewDocumentCommand\SITEFCPE{smmm}{\ensuremath{\IfBooleanTF#1
            {\Parentesis*{\SITEFC {#2}{#3}{#4}}}
            {\Parentesis {\SITEFC {#2}{#3}{#4}}}                      }\xspace}
%    \end{macrocode}
% \end{macro}
% \end{macro}
% \end{macro}
% \end{macro}
% \end{macro}
%
% \begin{macro}{\SITEFCR}
% \begin{macro}{\SITEFCRp}
% \begin{macro}{\SITEFCRP}
% \begin{macro}{\SITEFCRpE}
% \begin{macro}{\SITEFCRPE}
% Sucesión de transformaciones elementales genéricas a izquierda y derecha
%    \begin{macrocode}
\NewDocumentCommand\SITEFCR{mmm}{\ensuremath{
                            {\LRidx{#3}{\dSOEg{#1}{#2}}{\dSOEg{#1}{#2}}}   }\xspace}                   \NewDocumentCommand\SITEFCRp{smmm}{\ensuremath{\IfBooleanTF#1
            {\SITEFC {#2}{#3}{\parentesis*{#4}}}
            {\SITEFC {#2}{#3}{\parentesis {#4}}}                       }\xspace}
\NewDocumentCommand\SITEFCRP{smmm}{\ensuremath{\IfBooleanTF#1
            {\SITEFC {#2}{#3}{\Parentesis*{#4}}}
            {\SITEFC {#2}{#3}{\Parentesis {#4}}}                       }\xspace}
\NewDocumentCommand\SITEFCRpE{smmm}{\ensuremath{\IfBooleanTF#1
            {\parentesis*{\SITEFC {#2}{#3}{#4}}}
            {\parentesis {\SITEFC {#2}{#3}{#4}}}                      }\xspace}
\NewDocumentCommand\SITEFCRPE{smmm}{\ensuremath{\IfBooleanTF#1
            {\Parentesis*{\SITEFC {#2}{#3}{#4}}}
            {\Parentesis {\SITEFC {#2}{#3}{#4}}}                      }\xspace}
%    \end{macrocode}
% \end{macro}
% \end{macro}
% \end{macro}
% \end{macro}
% \end{macro}
%
%%%%%%%%%%%%%%%%%%%%%%%%%%%%%%%%%%%%%%%%%%
%
%
% \begin{macro}{\TrF}
% \begin{macro}{\TrFp}
% \begin{macro}{\TrFp*}
% \begin{macro}{\TrFP}
% \begin{macro}{\TrFP*}
% \begin{macro}{\TrFpE}
% \begin{macro}{\TrFpE*}
% \begin{macro}{\TrFPE}
% \begin{macro}{\TrFPE*}
% Transformaciones elementales por la izquierdaq de un objeto
%    \begin{macrocode}
\NewDocumentCommand\TrF  { O{\SOEg} m }{\ensuremath{ \Lidx{#2}{#1} }\xspace}

\NewDocumentCommand\TrFp {s O{\SOEg} m}{\ensuremath{\IfBooleanTF#1
                       {\TrF[#2]{\parentesis*{#3}}}
                       {\TrF[#2]{\parentesis {#3}}}               }\xspace}

\NewDocumentCommand\TrFP {s O{\SOEg} m}{\ensuremath{\IfBooleanTF#1
                       {\TrF[#2]{\Parentesis*{#3}}}
                       {\TrF[#2]{\Parentesis {#3}}}               }\xspace}

\NewDocumentCommand\TrFpE{s O{\SOEg} m}{\ensuremath{\IfBooleanTF#1
                       {\parentesis*{\TrF[#2]{#3}}}
		       {\parentesis {\TrF[#2]{#3}}}               }\xspace}

\NewDocumentCommand\TrFPE{s O{\SOEg} m}{\ensuremath{\IfBooleanTF#1
                       {\Parentesis*{\TrF[#2]{#3}}}
		       {\Parentesis {\TrF[#2]{#3}}}               }\xspace}

%    \end{macrocode}
% \end{macro}
% \end{macro}
% \end{macro}
% \end{macro}
% \end{macro}
% \end{macro}
% \end{macro}
% \end{macro}
% \end{macro}
%
%
% \begin{macro}{\TrC}
% \begin{macro}{\TrCp}
% \begin{macro}{\TrCp*}
% \begin{macro}{\TrCP}
% \begin{macro}{\TrCP*}
% \begin{macro}{\TrCpE}
% \begin{macro}{\TrCpE*}
% \begin{macro}{\TrCPE}
% \begin{macro}{\TrCPE*}
% Transformaciones elementales por la derecha de un objeto
%    \begin{macrocode}
\NewDocumentCommand\TrC  { O{\SOEg} m }{\ensuremath{ \Ridx{#2}{#1} }\xspace}

\NewDocumentCommand\TrCp {s O{\SOEg} m}{\ensuremath{\IfBooleanTF#1
                       {\TrC[#2]{\parentesis*{#3}}}
                       {\TrC[#2]{\parentesis {#3}}}               }\xspace}

\NewDocumentCommand\TrCP {s O{\SOEg} m}{\ensuremath{\IfBooleanTF#1
                       {\TrC[#2]{\Parentesis*{#3}}}
                       {\TrC[#2]{\Parentesis {#3}}}               }\xspace}

\NewDocumentCommand\TrCpE{s O{\SOEg} m}{\ensuremath{\IfBooleanTF#1
                       {\parentesis*{\TrC[#2]{#3}}}
		       {\parentesis {\TrC[#2]{#3}}}               }\xspace}

\NewDocumentCommand\TrCPE{s O{\SOEg} m}{\ensuremath{\IfBooleanTF#1
                       {\Parentesis*{\TrC[#2]{#3}}}
		       {\Parentesis {\TrC[#2]{#3}}}               }\xspace}

%    \end{macrocode}
% \end{macro}
% \end{macro}
% \end{macro}
% \end{macro}
% \end{macro}
% \end{macro}
% \end{macro}
% \end{macro}
% \end{macro}
%
%
% \begin{macro}{\TrFC}
% \begin{macro}{\TrFCp}
% \begin{macro}{\TrFCp*}
% \begin{macro}{\TrFCP}
% \begin{macro}{\TrFCP*}
% \begin{macro}{\TrFCpE}
% \begin{macro}{\TrFCpE*}
% \begin{macro}{\TrFCPE}
% \begin{macro}{\TrFCPE*}
% Transformaciones elementales por la izquierdaq de un objeto
%    \begin{macrocode}
\NewDocumentCommand\TrFC{ O{\SOEg} O{\SOEg[(k+1)][p]} m }{\ensuremath{ \LRidx{#3}{#1}{#2} }\xspace}

\NewDocumentCommand\TrFCp {s O{\SOEg} O{\SOEg[(k+1)][p]} m}{\ensuremath{\IfBooleanTF#1
                       {\TrFC[#2][#3]{\parentesis*{#4}}}
                       {\TrFC[#2][#3]{\parentesis {#4}}}               }\xspace}

\NewDocumentCommand\TrFCP {s O{\SOEg} O{\SOEg[(k+1)][p]} m}{\ensuremath{\IfBooleanTF#1
                       {\TrFC[#2][#3]{\Parentesis*{#4}}}
                       {\TrFC[#2][#3]{\Parentesis {#4}}}               }\xspace}

\NewDocumentCommand\TrFCpE{s O{\SOEg} O{\SOEg[(k+1)][p]} m}{\ensuremath{\IfBooleanTF#1
                       {\parentesis*{\TrFC[#2][#3]{#4}}}
		       {\parentesis {\TrFC[#2][#3]{#4}}}               }\xspace}

\NewDocumentCommand\TrFCPE{s O{\SOEg} O{\SOEg[(k+1)][p]} m}{\ensuremath{\IfBooleanTF#1
                       {\Parentesis*{\TrFC[#2][#3]{#4}}}
		       {\Parentesis {\TrFC[#2][#3]{#4}}}               }\xspace}

%    \end{macrocode}
% \end{macro}
% \end{macro}
% \end{macro}
% \end{macro}
% \end{macro}
% \end{macro}
% \end{macro}
% \end{macro}
% \end{macro}
%
%
%%%%%%%%%%%%%%%%%%%%%%%%%%%%%%%%%%%%%%%%%%
%
% \paragraph{Transf. elemental aplicada la izquierda de un objeto}
%
% \iffalse
%%%%%%%%%%%%%%%%%%%%%%%%%%%%%%%%%%%%
%% --- Transformaciones elementales genéricas aplicadas a la izquierda de un objeto
%%%%%%%%%%%%%%%%%%%%%%%%%%%%%%%%%%%%
% \fi
%
%
% \begin{macro}{\TEF}
% \begin{macro}{\TEFp}
% \begin{macro}{\TEFp*}
% \begin{macro}{\TEFP}
% \begin{macro}{\TEFP*}
% \begin{macro}{\TEFpE*}
% \begin{macro}{\TEFpE}
% \begin{macro}{\TEFPE}
% \begin{macro}{\TEFPE*}
% Una transformación elemental genérica por la izquierda
%    \begin{macrocode}
\NewDocumentCommand\TEF{O{}O{}m}{\ensuremath{ \Lidx{#3}{\OEg[#1\!][#2\!]} }\xspace}

\NewDocumentCommand\TEFp {sO{}O{\,}m}{\ensuremath{\IfBooleanTF#1
    {\Lidxp* {#4}{\OEg[#2][#3]\!}} {\Lidxp {#4}{\OEg[#2][#3]\!\!}}   }\xspace}

\NewDocumentCommand\TEFP {sO{}O{\,}m}{\ensuremath{\IfBooleanTF#1
    {\LidxP* {#4}{\OEg[#2][#3]\!}} {\LidxP {#4}{\OEg[#2][#3]\!\!}} }\xspace}

\NewDocumentCommand\TEFpE{sO{}O{\,}m}{\ensuremath{\IfBooleanTF#1
    {\LidxpE*{#4}{\OEg[#2][#3]\!}} {\LidxpE{#4}{\OEg[#2][#3]\!}}     }\xspace}

\NewDocumentCommand\TEFPE{sO{}O{\,}m}{\ensuremath{\IfBooleanTF#1
    {\LidxPE*{#4}{\OEg[#2][#3]\!}} {\LidxPE{#4}{\OEg[#2][#3]\!}}     }\xspace}
%    \end{macrocode}
% \end{macro}
% \end{macro}
% \end{macro}
% \end{macro}
% \end{macro}
% \end{macro}
% \end{macro}
% \end{macro}
% \end{macro}
%
%
% \iffalse
%%%%%%%%%%%%%%%%%%%%%%%%%%%%%%%%%%%%
%% --- Transformaciones elementales genéricas aplicadas a la derecha de un objeto
%%%%%%%%%%%%%%%%%%%%%%%%%%%%%%%%%%%%
% \fi
%
% \begin{macro}{\TEC}
% \begin{macro}{\TECp}
% \begin{macro}{\TECp*}
% \begin{macro}{\TECP}
% \begin{macro}{\TECP*}
% \begin{macro}{\TECpE*}
% \begin{macro}{\TECpE}
% \begin{macro}{\TECPE}
% \begin{macro}{\TECPE*}
% Una transformación elemental genérica por la izquierda
%    \begin{macrocode}
\NewDocumentCommand\TEC{O{}O{}m}{\ensuremath{ \Ridx{#3}{\OEg[#1\!][#2\!]} }\xspace}

\NewDocumentCommand\TECp {sO{}O{  }m}{\ensuremath{\IfBooleanTF#1
    {\Ridxp* {#4}{\OEg[#2][#3]}}   {\Ridxp {#4}{\!\OEg[#2][#3]}}     }\xspace}

\NewDocumentCommand\TECP {sO{}O{  }m}{\ensuremath{\IfBooleanTF#1
    {\RidxP* {#4}{\OEg[#2][#3]}}   {\RidxP {#4}{\!\OEg[#2][#3]}}     }\xspace}

\NewDocumentCommand\TECpE{sO{}O{  }m}{\ensuremath{\IfBooleanTF#1
    {\RidxpE*{#4}{\OEg[#2][#3]}}   {\RidxpE{#4}{\OEg[#2][#3]}}       }\xspace}

\NewDocumentCommand\TECPE{sO{}O{  }m}{\ensuremath{\IfBooleanTF#1
    {\RidxPE*{#4}{\OEg[#2][#3]}}   {\RidxPE{#4}{\OEg[#2][#3]}}       }\xspace}
%    \end{macrocode}
% \end{macro}
% \end{macro}
% \end{macro}
% \end{macro}
% \end{macro}
% \end{macro}
% \end{macro}
% \end{macro}
% \end{macro}
%
%%%%%%%%%%%%%%%%%%%%%%%%%%%%%%%%%%%%%%%%%%%%%%%%%%%%%%%%%%%%%%%%%%%%%%%%%%%%%%%%%%
%
% \paragraph{espejo de una transformación elemental por la izquierda de un objeto}
%
% \iffalse
%%%%%%%%%%%%%%%%%%%%%%%%%%%%%%%%%%%%
%% --- Espejo de transformaciones elementales genéricas aplicadas a la izquierda de un objeto
%%%%%%%%%%%%%%%%%%%%%%%%%%%%%%%%%%%%
% \fi
%
%
% \begin{macro}{\ETEF}
% \begin{macro}{\ETEFp}
% \begin{macro}{\ETEFp*}
% \begin{macro}{\ETEFP}
% \begin{macro}{\ETEFP*}
% \begin{macro}{\ETEFpE*}
% \begin{macro}{\ETEFpE}
% \begin{macro}{\ETEFPE}
% \begin{macro}{\ETEFPE*}
% Espejo de una transformación elemental genérica por la izquierda
%    \begin{macrocode}
\NewDocumentCommand\ETEF{O{}O{}m}{\ensuremath{ \Lidx{#3}{\EOEg[#1][#2]} }\xspace}

\NewDocumentCommand\ETEFp {sO{}O{}m}{\ensuremath{\IfBooleanTF#1
    {\Lidxp* {#4}{\EOEg[#2][#3]}} {\Lidxp {#4}{\EOEg[#2][#3]}} }\xspace}

\NewDocumentCommand\ETEFP {sO{}O{}m}{\ensuremath{\IfBooleanTF#1
    {\LidxP* {#4}{\EOEg[#2][#3]}} {\LidxP {#4}{\EOEg[#2][#3]}} }\xspace}

\NewDocumentCommand\ETEFpE{sO{}O{}m}{\ensuremath{\IfBooleanTF#1
    {\LidxpE*{#4}{\EOEg[#2][#3]}} {\LidxpE{#4}{\EOEg[#2][#3]}}     }\xspace}

\NewDocumentCommand\ETEFPE{sO{}O{}m}{\ensuremath{\IfBooleanTF#1
    {\LidxPE*{#4}{\EOEg[#2][#3]}} {\LidxPE{#4}{\EOEg[#2][#3]}}     }\xspace}
%    \end{macrocode}
% \end{macro}
% \end{macro}
% \end{macro}
% \end{macro}
% \end{macro}
% \end{macro}
% \end{macro}
% \end{macro}
% \end{macro}
%
%
% \paragraph{espejo de una transformación elemental por la derecha de un objeto}
%
% \iffalse
%%%%%%%%%%%%%%%%%%%%%%%%%%%%%%%%%%%%
%% --- Espejo de transformaciones elementales genéricas aplicadas a la derecha de un objeto
%%%%%%%%%%%%%%%%%%%%%%%%%%%%%%%%%%%%
% \fi
%
%
% \begin{macro}{\ETEC}
% \begin{macro}{\ETECp}
% \begin{macro}{\ETECp*}
% \begin{macro}{\ETECP}
% \begin{macro}{\ETECP*}
% \begin{macro}{\ETECpE*}
% \begin{macro}{\ETECpE}
% \begin{macro}{\ETECPE}
% \begin{macro}{\ETECPE*}
% Espejo de una transformación elemental genérica por la derecha
%    \begin{macrocode}
\NewDocumentCommand\ETEC{O{}O{}m}{\ensuremath{ \Ridx{#3}{\EOEg[#1][#2]} }\xspace}

\NewDocumentCommand\ETECp {sO{}O{}m}{\ensuremath{\IfBooleanTF#1
    {\Ridxp* {#4}{\EOEg[#2][#3]}} {\Ridxp {#4}{\EOEg[#2][#3]}} }\xspace}

\NewDocumentCommand\ETECP {sO{}O{}m}{\ensuremath{\IfBooleanTF#1
    {\RidxP* {#4}{\EOEg[#2][#3]}} {\RidxP {#4}{\EOEg[#2][#3]}} }\xspace}

\NewDocumentCommand\ETECpE{sO{}O{}m}{\ensuremath{\IfBooleanTF#1
    {\RidxpE*{#4}{\EOEg[#2][#3]}} {\RidxpE{#4}{\EOEg[#2][#3]}}     }\xspace}

\NewDocumentCommand\ETECPE{sO{}O{}m}{\ensuremath{\IfBooleanTF#1
    {\RidxPE*{#4}{\EOEg[#2][#3]}} {\RidxPE{#4}{\EOEg[#2][#3]}}     }\xspace}
%    \end{macrocode}
% \end{macro}
% \end{macro}
% \end{macro}
% \end{macro}
% \end{macro}
% \end{macro}
% \end{macro}
% \end{macro}
% \end{macro}
%
%%%%%%%%%%%%%%%%%%%%%%%%%%%%%%%%%%%%%%%%%%%%%%%%%%%%%%%%%%%%%%%%%%%%%%%%%%%%%%%%%%
%
% \paragraph{Inversa de una transformación elemental por la izquierda de un objeto}
%
% \iffalse
%%%%%%%%%%%%%%%%%%%%%%%%%%%%%%%%%%%%
%% --- Inversa de transformaciones elementales genéricas aplicadas a la izquierda de un objeto
%%%%%%%%%%%%%%%%%%%%%%%%%%%%%%%%%%%%
% \fi
%
%
% \begin{macro}{\InvTEF}
% \begin{macro}{\InvTEFp}
% \begin{macro}{\InvTEFp*}
% \begin{macro}{\InvTEFP}
% \begin{macro}{\InvTEFP*}
% \begin{macro}{\InvTEFpE*}
% \begin{macro}{\InvTEFpE}
% \begin{macro}{\InvTEFPE}
% \begin{macro}{\InvTEFPE*}
% Espejo de una transformación elemental genérica por la izquierda
%    \begin{macrocode}
\NewDocumentCommand\InvTEF{O{}m}{\ensuremath{ \Lidx{#2}{\InvOEg[#1]} }\xspace}

\NewDocumentCommand\InvTEFp {sO{}m}{\ensuremath{\IfBooleanTF#1
    {\Lidxp* {#3}{\InvOEg[#2]}} {\Lidxp {#3}{\InvOEg[#2]}} }\xspace}

\NewDocumentCommand\InvTEFP {sO{}m}{\ensuremath{\IfBooleanTF#1
    {\LidxP* {#3}{\InvOEg[#2]}} {\LidxP {#3}{\InvOEg[#2]}} }\xspace}

\NewDocumentCommand\InvTEFpE{sO{}m}{\ensuremath{\IfBooleanTF#1
    {\LidxpE*{#3}{\InvOEg[#2]}} {\LidxpE{#3}{\InvOEg[#2]}}     }\xspace}

\NewDocumentCommand\InvTEFPE{sO{}m}{\ensuremath{\IfBooleanTF#1
    {\LidxPE*{#3}{\InvOEg[#2]}} {\LidxPE{#3}{\InvOEg[#2]}}     }\xspace}
%    \end{macrocode}
% \end{macro}
% \end{macro}
% \end{macro}
% \end{macro}
% \end{macro}
% \end{macro}
% \end{macro}
% \end{macro}
% \end{macro}
%
%
% \paragraph{Inversa de una transformación elemental por la derecha de un objeto}
%
% \iffalse
%%%%%%%%%%%%%%%%%%%%%%%%%%%%%%%%%%%%
%% --- Inversa de transformaciones elementales genéricas aplicadas a la derecha de un objeto
%%%%%%%%%%%%%%%%%%%%%%%%%%%%%%%%%%%%
% \fi
%
%
% \begin{macro}{\InvTEC}
% \begin{macro}{\InvTECp}
% \begin{macro}{\InvTECp*}
% \begin{macro}{\InvTECP}
% \begin{macro}{\InvTECP*}
% \begin{macro}{\InvTECpE*}
% \begin{macro}{\InvTECpE}
% \begin{macro}{\InvTECPE}
% \begin{macro}{\InvTECPE*}
% Espejo de una transformación elemental genérica por la derecha
%    \begin{macrocode}
\NewDocumentCommand\InvTEC{O{}m}{\ensuremath{ \Ridx{#2}{\InvOEg[#1]} }\xspace}

\NewDocumentCommand\InvTECp {sO{}m}{\ensuremath{\IfBooleanTF#1
    {\Ridxp* {#3}{\InvOEg[#2]}} {\Ridxp {#3}{\InvOEg[#2]}} }\xspace}

\NewDocumentCommand\InvTECP {sO{}m}{\ensuremath{\IfBooleanTF#1
    {\RidxP* {#3}{\InvOEg[#2]}} {\RidxP {#3}{\InvOEg[#2]}} }\xspace}

\NewDocumentCommand\InvTECpE{sO{}m}{\ensuremath{\IfBooleanTF#1
    {\RidxpE*{#3}{\InvOEg[#2]}} {\RidxpE{#3}{\InvOEg[#2]}}     }\xspace}

\NewDocumentCommand\InvTECPE{sO{}m}{\ensuremath{\IfBooleanTF#1
    {\RidxPE*{#3}{\InvOEg[#2]}} {\RidxPE{#3}{\InvOEg[#2]}}     }\xspace}
%    \end{macrocode}
% \end{macro}
% \end{macro}
% \end{macro}
% \end{macro}
% \end{macro}
% \end{macro}
% \end{macro}
% \end{macro}
% \end{macro}
%
%%%%%%%%%%%%%%%%%%%%%%%%%%%%%%%%%%%%%%%%%%%%%%%%%%%%%%%%%%%%%%%%%%%%%%%%%%%%%%%%%%
%
% \paragraph{Espejo de la inversa de una transformación elemental por la izquierda de un objeto}
%
% \iffalse
%%%%%%%%%%%%%%%%%%%%%%%%%%%%%%%%%%%%
%% --- Espejo de la inversa de transformaciones elementales genéricas aplicadas a la izquierda de un objeto
%%%%%%%%%%%%%%%%%%%%%%%%%%%%%%%%%%%%
% \fi
%
%
% \begin{macro}{\EInvTEF}
% \begin{macro}{\EInvTEFp}
% \begin{macro}{\EInvTEFp*}
% \begin{macro}{\EInvTEFP}
% \begin{macro}{\EInvTEFP*}
% \begin{macro}{\EInvTEFpE*}
% \begin{macro}{\EInvTEFpE}
% \begin{macro}{\EInvTEFPE}
% \begin{macro}{\EInvTEFPE*}
% Espejo de la inversa de una transformación elemental genérica por la izquierda
%    \begin{macrocode}
\NewDocumentCommand\EInvTEF{O{}m}{\ensuremath{ \Lidx{#2}{\EInvOEg[#1]} }\xspace}

\NewDocumentCommand\EInvTEFp {sO{}m}{\ensuremath{\IfBooleanTF#1
    {\Lidxp* {#3}{\EInvOEg[#2]}} {\Lidxp {#3}{\EInvOEg[#2]}} }\xspace}

\NewDocumentCommand\EInvTEFP {sO{}m}{\ensuremath{\IfBooleanTF#1
    {\LidxP* {#3}{\EInvOEg[#2]}} {\LidxP {#3}{\EInvOEg[#2]}} }\xspace}

\NewDocumentCommand\EInvTEFpE{sO{}m}{\ensuremath{\IfBooleanTF#1
    {\LidxpE*{#3}{\EInvOEg[#2]}} {\LidxpE{#3}{\EInvOEg[#2]}}     }\xspace}

\NewDocumentCommand\EInvTEFPE{sO{}m}{\ensuremath{\IfBooleanTF#1
    {\LidxPE*{#3}{\EInvOEg[#2]}} {\LidxPE{#3}{\EInvOEg[#2]}}     }\xspace}
%    \end{macrocode}
% \end{macro}
% \end{macro}
% \end{macro}
% \end{macro}
% \end{macro}
% \end{macro}
% \end{macro}
% \end{macro}
% \end{macro}
%
%
% \paragraph{Espejo de la inversa de una transformación elemental por la derecha de un objeto}
%
% \iffalse
%%%%%%%%%%%%%%%%%%%%%%%%%%%%%%%%%%%%
%% --- Espejo de la inversa de transformaciones elementales genéricas aplicadas a la derecha de un objeto
%%%%%%%%%%%%%%%%%%%%%%%%%%%%%%%%%%%%
% \fi
%
%
% \begin{macro}{\EInvTEC}
% \begin{macro}{\EInvTECp}
% \begin{macro}{\EInvTECp*}
% \begin{macro}{\EInvTECP}
% \begin{macro}{\EInvTECP*}
% \begin{macro}{\EInvTECpE*}
% \begin{macro}{\EInvTECpE}
% \begin{macro}{\EInvTECPE}
% \begin{macro}{\EInvTECPE*}
% Espejo de la inversa de una transformación elemental genérica por la derecha
%    \begin{macrocode}
\NewDocumentCommand\EInvTEC{O{}m}{\ensuremath{ \Ridx{#2}{\EInvOEg[#1]} }\xspace}

\NewDocumentCommand\EInvTECp {sO{}m}{\ensuremath{\IfBooleanTF#1
    {\Ridxp* {#3}{\EInvOEg[#2]}} {\Ridxp {#3}{\EInvOEg[#2]}} }\xspace}

\NewDocumentCommand\EInvTECP {sO{}m}{\ensuremath{\IfBooleanTF#1
    {\RidxP* {#3}{\EInvOEg[#2]}} {\RidxP {#3}{\EInvOEg[#2]}} }\xspace}

\NewDocumentCommand\EInvTECpE{sO{}m}{\ensuremath{\IfBooleanTF#1
    {\RidxpE*{#3}{\EInvOEg[#2]}} {\RidxpE{#3}{\EInvOEg[#2]}}     }\xspace}

\NewDocumentCommand\EInvTECPE{sO{}m}{\ensuremath{\IfBooleanTF#1
    {\RidxPE*{#3}{\EInvOEg[#2]}} {\RidxPE{#3}{\EInvOEg[#2]}}     }\xspace}
%    \end{macrocode}
% \end{macro}
% \end{macro}
% \end{macro}
% \end{macro}
% \end{macro}
% \end{macro}
% \end{macro}
% \end{macro}
% \end{macro}
%
%
% \iffalse
%%%%%%%%%%%%%%%%%%%%%%%%%%%%%%%%%%%%
%% --- (Duplicado para Moodle) Transformaciones elementales genéricas aplicadas a la izquierda de un objeto
%%%%%%%%%%%%%%%%%%%%%%%%%%%%%%%%%%%%
% \fi
%
% \begin{macro}{\dTEEF}
% \begin{macro}{\dTEEFp}
% \begin{macro}{\dTEEFP}
% \begin{macro}{\dTEEFpE}
% \begin{macro}{\dTEEFPE}
% Una transformación elemental genérica por la izquierda con exponente o sin
%    \begin{macrocode}
\NewDocumentCommand\dTEEF   {mmm}{\ensuremath{ \Lidx  {#3}{\dOEgE{#1}{#2\!}} }\xspace}
\NewDocumentCommand\dTEEFp  {mmm}{\ensuremath{ \Lidxp {#3}{\dOEgE{#1}{#2\!}}}\xspace}
\NewDocumentCommand\dTEEFP  {mmm}{\ensuremath{ \LidxP {#3}{\dOEgE{#1}{#2\!}}}\xspace}
\NewDocumentCommand\dTEEFpE {mmm}{\ensuremath{ \LidxpE{#3}{\dOEgE{#1}{#2\!}}}\xspace}
\NewDocumentCommand\dTEEFPE {mmm}{\ensuremath{ \LidxPE{#3}{\dOEgE{#1}{#2\!}}}\xspace}
%    \end{macrocode}
% \end{macro}
% \end{macro}
% \end{macro}
% \end{macro}
% \end{macro}
%
%
% \begin{macro}{\dTEF}
% \begin{macro}{\dTEFp}
% \begin{macro}{\dTEFP}
% \begin{macro}{\dTEFpE}
% \begin{macro}{\dTEFPE}
% Una transformación elemental genérica por la izquierda con exponente o sin
%    \begin{macrocode}
\NewDocumentCommand\dTEF  {mm}{\ensuremath{ \Lidx  {#2}{{    \dOEg{#1}}} }\xspace}
\NewDocumentCommand\dTEFp {mm}{\ensuremath{ \Lidxp {#2}{{    \dOEg{#1}}} }\xspace}
\NewDocumentCommand\dTEFP {mm}{\ensuremath{ \LidxP {#2}{{    \dOEg{#1}}} }\xspace}
\NewDocumentCommand\dTEFpE{mm}{\ensuremath{ \LidxpE{#2}{{    \dOEg{#1}}} }\xspace}
\NewDocumentCommand\dTEFPE{mm}{\ensuremath{ \LidxPE{#2}{{    \dOEg{#1}}} }\xspace}
%    \end{macrocode}
% \end{macro}
% \end{macro}
% \end{macro}
% \end{macro}
% \end{macro}
%
%
% \begin{macro}{\dETEF}
% \begin{macro}{\dETEFp}
% \begin{macro}{\dETEFP}
% \begin{macro}{\dETEFpE}
% \begin{macro}{\dETEFPE}
% Una transformación elemental espejo genérica por la izquierda
%    \begin{macrocode}
\NewDocumentCommand\dETEF  {mm}{\ensuremath{ \Lidx  {#2}{{   \dEOEg{#1}}} }\xspace}
\NewDocumentCommand\dETEFp {mm}{\ensuremath{ \Lidxp {#2}{{   \dEOEg{#1}}} }\xspace}
\NewDocumentCommand\dETEFP {mm}{\ensuremath{ \LidxP {#2}{{   \dEOEg{#1}}} }\xspace}
\NewDocumentCommand\dETEFpE{mm}{\ensuremath{ \LidxpE{#2}{{   \dEOEg{#1}}} }\xspace}
\NewDocumentCommand\dETEFPE{mm}{\ensuremath{ \LidxPE{#2}{{   \dEOEg{#1}}} }\xspace}
%    \end{macrocode}
% \end{macro}
% \end{macro}
% \end{macro}
% \end{macro}
% \end{macro}
%
%
% \begin{macro}{\dInvTEF}
% \begin{macro}{\dInvTEFp}
% \begin{macro}{\dInvTEFP}
% \begin{macro}{\dInvTEFpE}
% \begin{macro}{\dInvTEFPE}
% Una transformación elemental inversa genérica por la izquierda
%    \begin{macrocode}
\NewDocumentCommand\dInvTEF  {mm}{\ensuremath{ \Lidx  {#2}{{ \dInvOEg{#1}}} }\xspace}
\NewDocumentCommand\dInvTEFp {mm}{\ensuremath{ \Lidxp {#2}{{ \dInvOEg{#1}}} }\xspace}
\NewDocumentCommand\dInvTEFP {mm}{\ensuremath{ \LidxP {#2}{{ \dInvOEg{#1}}} }\xspace}
\NewDocumentCommand\dInvTEFpE{mm}{\ensuremath{ \LidxpE{#2}{{ \dInvOEg{#1}}} }\xspace}
\NewDocumentCommand\dInvTEFPE{mm}{\ensuremath{ \LidxPE{#2}{{ \dInvOEg{#1}}} }\xspace}
%    \end{macrocode}
% \end{macro}
% \end{macro}
% \end{macro}
% \end{macro}
% \end{macro}
%
%
% \begin{macro}{\dEInvTEF}
% \begin{macro}{\dEInvTEFp}
% \begin{macro}{\dEInvTEFP}
% \begin{macro}{\dEInvTEFpE}
% \begin{macro}{\dEInvTEFPE}
% Una transformación elemental inversa genérica por la izquierda
%    \begin{macrocode}
\NewDocumentCommand\dEInvTEF  {mm}{\ensuremath{ \Lidx  {#2}{{\dEInvOEg{#1}}} }\xspace}
\NewDocumentCommand\dEInvTEFp {mm}{\ensuremath{ \Lidxp {#2}{{\dEInvOEg{#1}}} }\xspace}
\NewDocumentCommand\dEInvTEFP {mm}{\ensuremath{ \LidxP {#2}{{\dEInvOEg{#1}}} }\xspace}
\NewDocumentCommand\dEInvTEFpE{mm}{\ensuremath{ \LidxpE{#2}{{\dEInvOEg{#1}}} }\xspace}
\NewDocumentCommand\dEInvTEFPE{mm}{\ensuremath{ \LidxPE{#2}{{\dEInvOEg{#1}}} }\xspace}
%    \end{macrocode}
% \end{macro}
% \end{macro}
% \end{macro}
% \end{macro}
% \end{macro}
%
%
% \iffalse
%%%%%%%%%%%%%%%%%%%%%%%%%%%%%%%%%%%%
%% --- (Duplicado para Moodle) Transformaciones elementales genéricas aplicadas a la izquierda de un objeto
%%%%%%%%%%%%%%%%%%%%%%%%%%%%%%%%%%%%
% \fi
%
% \begin{macro}{\dTEEC}
% \begin{macro}{\dTEECp}
% \begin{macro}{\dTEECP}
% \begin{macro}{\dTEECpE}
% \begin{macro}{\dTEECPE}
% Una transformación elemental genérica por la izquierda con exponente o sin
%    \begin{macrocode}
\NewDocumentCommand\dTEEC   {mmm}{\ensuremath{ \Ridx  {#3}{\dOEgE{#1}{#2}} }\xspace}
\NewDocumentCommand\dTEECp  {mmm}{\ensuremath{ \Ridxp {#3}{\dOEgE{#1}{#2}}}\xspace}
\NewDocumentCommand\dTEECP  {mmm}{\ensuremath{ \RidxP {#3}{\dOEgE{#1}{#2}}}\xspace}
\NewDocumentCommand\dTEECpE {mmm}{\ensuremath{ \RidxpE{#3}{\dOEgE{#1}{#2}}}\xspace}
\NewDocumentCommand\dTEECPE {mmm}{\ensuremath{ \RidxPE{#3}{\dOEgE{#1}{#2}}}\xspace}
%    \end{macrocode}
% \end{macro}
% \end{macro}
% \end{macro}
% \end{macro}
% \end{macro}
%
%
% \begin{macro}{\dTEC}
% \begin{macro}{\dTECp}
% \begin{macro}{\dTECP}
% \begin{macro}{\dTECpE}
% \begin{macro}{\dTECPE}
% Una transformación elemental genérica por la izquierda con exponente o sin
%    \begin{macrocode}
\NewDocumentCommand\dTEC  {mm}{\ensuremath{ \Ridx  {#2}{{    \dOEg{#1}}} }\xspace}
\NewDocumentCommand\dTECp {mm}{\ensuremath{ \Ridxp {#2}{{    \dOEg{#1}}} }\xspace}
\NewDocumentCommand\dTECP {mm}{\ensuremath{ \RidxP {#2}{{    \dOEg{#1}}} }\xspace}
\NewDocumentCommand\dTECpE{mm}{\ensuremath{ \RidxpE{#2}{{    \dOEg{#1}}} }\xspace}
\NewDocumentCommand\dTECPE{mm}{\ensuremath{ \RidxPE{#2}{{    \dOEg{#1}}} }\xspace}
%    \end{macrocode}
% \end{macro}
% \end{macro}
% \end{macro}
% \end{macro}
% \end{macro}
%
%
% \begin{macro}{\dETEC}
% \begin{macro}{\dETECp}
% \begin{macro}{\dETECP}
% \begin{macro}{\dETECpE}
% \begin{macro}{\dETECPE}
% Una transformación elemental espejo genérica por la izquierda
%    \begin{macrocode}
\NewDocumentCommand\dETEC  {mm}{\ensuremath{ \Ridx  {#2}{{   \dEOEg{#1}}} }\xspace}
\NewDocumentCommand\dETECp {mm}{\ensuremath{ \Ridxp {#2}{{   \dEOEg{#1}}} }\xspace}
\NewDocumentCommand\dETECP {mm}{\ensuremath{ \RidxP {#2}{{   \dEOEg{#1}}} }\xspace}
\NewDocumentCommand\dETECpE{mm}{\ensuremath{ \RidxpE{#2}{{   \dEOEg{#1}}} }\xspace}
\NewDocumentCommand\dETECPE{mm}{\ensuremath{ \RidxPE{#2}{{   \dEOEg{#1}}} }\xspace}
%    \end{macrocode}
% \end{macro}
% \end{macro}
% \end{macro}
% \end{macro}
% \end{macro}
%
%
% \begin{macro}{\dInvTEC}
% \begin{macro}{\dInvTECp}
% \begin{macro}{\dInvTECP}
% \begin{macro}{\dInvTECpE}
% \begin{macro}{\dInvTECPE}
% Una transformación elemental inversa genérica por la izquierda
%    \begin{macrocode}
\NewDocumentCommand\dInvTEC  {mm}{\ensuremath{ \Ridx  {#2}{{ \dInvOEg{#1}}} }\xspace}
\NewDocumentCommand\dInvTECp {mm}{\ensuremath{ \Ridxp {#2}{{ \dInvOEg{#1}}} }\xspace}
\NewDocumentCommand\dInvTECP {mm}{\ensuremath{ \RidxP {#2}{{ \dInvOEg{#1}}} }\xspace}
\NewDocumentCommand\dInvTECpE{mm}{\ensuremath{ \RidxpE{#2}{{ \dInvOEg{#1}}} }\xspace}
\NewDocumentCommand\dInvTECPE{mm}{\ensuremath{ \RidxPE{#2}{{ \dInvOEg{#1}}} }\xspace}
%    \end{macrocode}
% \end{macro}
% \end{macro}
% \end{macro}
% \end{macro}
% \end{macro}
%
%
% \begin{macro}{\dEInvTEC}
% \begin{macro}{\dEInvTECp}
% \begin{macro}{\dEInvTECP}
% \begin{macro}{\dEInvTECpE}
% \begin{macro}{\dEInvTECPE}
% Una transformación elemental inversa genérica por la izquierda
%    \begin{macrocode}
\NewDocumentCommand\dEInvTEC  {mm}{\ensuremath{ \Ridx  {#2}{{\dEInvOEg{#1}}} }\xspace}
\NewDocumentCommand\dEInvTECp {mm}{\ensuremath{ \Ridxp {#2}{{\dEInvOEg{#1}}} }\xspace}
\NewDocumentCommand\dEInvTECP {mm}{\ensuremath{ \RidxP {#2}{{\dEInvOEg{#1}}} }\xspace}
\NewDocumentCommand\dEInvTECpE{mm}{\ensuremath{ \RidxpE{#2}{{\dEInvOEg{#1}}} }\xspace}
\NewDocumentCommand\dEInvTECPE{mm}{\ensuremath{ \RidxPE{#2}{{\dEInvOEg{#1}}} }\xspace}
%    \end{macrocode}
% \end{macro}
% \end{macro}
% \end{macro}
% \end{macro}
% \end{macro}
%
%
%%%%%%%%%%%%%%%%%%%%%%%%%%%%%%%%%%%%%%%%%%%%%%%%%%%%%%%%%%%%%%%%%%%%%%
%%%%%%%%%%%%%%%%%%%%%%%%%%%%%%%%%%%%%%%%%%%%%%%%%%%%%%%%%%%%%%%%%%%%%%
%
% \paragraph{Transformaciones elementales particulares}
%
% \iffalse
%%%%%%%%%%%%%%%%%%%%%%%%%%%%%%%%%%%%
%% --- Transformaciones elementales particulares
%%%%%%%%%%%%%%%%%%%%%%%%%%%%%%%%%%%%
% \fi
%
% \begin{macro}{\dTrF}
% \begin{macro}{\dTrFp}
% \begin{macro}{\dTrFP}
% \begin{macro}{\dTrFpE}
% \begin{macro}{\dTrFPE}
% Transformación o sucesión de transformaciones elementales por la izquierda
%    \begin{macrocode}
\NewDocumentCommand\dTrF  {mm}{\ensuremath{ \Lidx{#2}{#1}              }\xspace}
\NewDocumentCommand\dTrFp {mm}{\ensuremath{ \dTrF{#1}{\parentesis{#2}} }\xspace}
\NewDocumentCommand\dTrFP {mm}{\ensuremath{ \dTrF{#1}{\Parentesis{#2}} }\xspace}
\NewDocumentCommand\dTrFpE{mm}{\ensuremath{ \parentesis{\dTrF{#1}{#2}} }\xspace}
\NewDocumentCommand\dTrFPE{mm}{\ensuremath{ \Parentesis{\dTrF{#1}{#2}} }\xspace}
%    \end{macrocode}
% \end{macro}
% \end{macro}
% \end{macro}
% \end{macro}
% \end{macro}
%
%
% \begin{macro}{\dTrC}
% \begin{macro}{\dTrCp}
% \begin{macro}{\dTrCP}
% \begin{macro}{\dTrCpE}
% \begin{macro}{\dTrCPE}
% Transformación o sucesión de transformaciones elementales por la izquierda
%    \begin{macrocode}
\NewDocumentCommand\dTrC  {mm}{\ensuremath{ \Ridx{#2}{#1}              }\xspace}
\NewDocumentCommand\dTrCp {mm}{\ensuremath{ \dTrC{#1}{\parentesis{#2}} }\xspace}
\NewDocumentCommand\dTrCP {mm}{\ensuremath{ \dTrC{#1}{\Parentesis{#2}} }\xspace}
\NewDocumentCommand\dTrCpE{mm}{\ensuremath{ \parentesis{\dTrC{#1}{#2}} }\xspace}
\NewDocumentCommand\dTrCPE{mm}{\ensuremath{ \Parentesis{\dTrC{#1}{#2}} }\xspace}
%    \end{macrocode}
% \end{macro}
% \end{macro}
% \end{macro}
% \end{macro}
% \end{macro}
%
%
% \begin{macro}{\dTrFC}
% \begin{macro}{\dTrFCp}
% \begin{macro}{\dTrFCP}
% \begin{macro}{\dTrFCpE}
% \begin{macro}{\dTrFCPE}
% Transformación o sucesión de transformaciones elementales por la izquierda
%    \begin{macrocode}
\NewDocumentCommand\dTrFC  {mmm}{\ensuremath{ \LRidx{#3}{#1}{#2}              }\xspace}
\NewDocumentCommand\dTrFCp {mmm}{\ensuremath{ \dTrFC{#1}{#2}{\parentesis{#3}} }\xspace}
\NewDocumentCommand\dTrFCP {mmm}{\ensuremath{ \dTrFC{#1}{#2}{\Parentesis{#3}} }\xspace}
\NewDocumentCommand\dTrFCpE{mmm}{\ensuremath{ \parentesis{\dTrFC{#1}{#2}{#3}} }\xspace}
\NewDocumentCommand\dTrFCPE{mmm}{\ensuremath{ \Parentesis{\dTrFC{#1}{#2}{#3}} }\xspace}
%    \end{macrocode}
% \end{macro}
% \end{macro}
% \end{macro}
% \end{macro}
% \end{macro}
%
%
%
% \subsubsection{Operador que quita un elemento}
%
% \iffalse
%%%%%%%%%%%%%%%%%%%%%%%%%%%%%%%%%%%%
%% --- OPERADOR QUE QUITA UN ELEMENTO
%%%%%%%%%%%%%%%%%%%%%%%%%%%%%%%%%%%%
% \fi
%
% \begin{macro}{\fueraitemL}
% \begin{macro}{\fueraitemR}
% Signo de operador que quita un elemento (por la derecha o por la izquierda)
%    \begin{macrocode}
\NewDocumentCommand\fueraitemL{m}{ \leftidx{_{}}{#1}{^{\!\Lsh  }} }
\NewDocumentCommand\fueraitemR{m}{ \leftidx{^{\Rsh\!\!}}{#1}{_{}} }
%    \end{macrocode}
% \end{macro}
% \end{macro}
%
%
% \begin{macro}{\quitaLR}
% \begin{macro}{\quitaL}
% \begin{macro}{\quitaR}
% Sistema resultante de quitar un elemento por la izquierda y/u otro por la derecha
%    \begin{macrocode}
\NewDocumentCommand\quitaLR{mmm}{\ensuremath{
                          \leftidx {^{\fueraitemL{#2}\!}} {{#1}} {^{\!\fueraitemR{#3}}} }\xspace}
\NewDocumentCommand\quitaL{mm}{\ensuremath{ \leftidx{^{\fueraitemL{#2}\!}}{{#1}}{^{}} }\xspace}
\NewDocumentCommand\quitaR{mm}{\ensuremath{ \leftidx{^{}}{{#1}}{^{\!\fueraitemR{#2}}} }\xspace}
%    \end{macrocode}
% \end{macro}
% \end{macro}
% \end{macro}
%
%
% \subsubsection{Selección de elementos sin emplear el operador selector}
%
% \iffalse
%%%%%%%%%%%%%%%%%%%%%%%%%%%%%%%%%%%%
%% --- Selección de elementos sin emplear el operador selector
%%%%%%%%%%%%%%%%%%%%%%%%%%%%%%%%%%%%
% \fi
%
% \begin{macro}{\elemUUU}
% Selección de un elemento de un sistema
%    \begin{macrocode}
\NewDocumentCommand\elemUUU {mm}{\ensuremath{\textrm{elem}_{#2}\Parentesis*{#1}}\xspace}
%    \end{macrocode}
% \end{macro}
%
%
% \begin{macro}{\VectCCC}
% \begin{macro}{\VectCCCT}
% Selección de una columna de una matriz
%    \begin{macrocode}
\NewDocumentCommand\VectCCC {mm}{\ensuremath{\textrm{col}_{#2}\MatP*  {#1}}\xspace}
\NewDocumentCommand\VectCCCT{mm}{\ensuremath{\textrm{col}_{#2}\MatTPE*{#1}}\xspace}
%    \end{macrocode}
% \end{macro}
% \end{macro}
%
%
% \begin{macro}{\VectFFF}
% \begin{macro}{\VectFFFT}
% Selección de una columna de una matriz
%    \begin{macrocode}
\NewDocumentCommand\VectFFF {mm}{\ensuremath{\textrm{\eng{fila}{row}}_{#2}\MatP*  {#1}}\xspace}
\NewDocumentCommand\VectFFFT{mm}{\ensuremath{\textrm{\eng{fila}{row}}_{#2}\MatTPE*{#1}}\xspace}
%    \end{macrocode}
% \end{macro}
% \end{macro}
%
%
% \begin{macro}{\eleVVV}
% \begin{macro}{\eleVV}
% Selección de un elemento de un vector
%    \begin{macrocode}
\NewDocumentCommand\eleVVV {mm}{\ensuremath{\textrm{elem}_{#2}{\VectP*  {#1}}}\xspace}
\NewDocumentCommand\eleVV  {mm}{\ensuremath{\MakeLowercase{#1}_{{#2}}        }\xspace}
%    \end{macrocode}
% \end{macro}
% \end{macro}
%
%
% \begin{macro}{\eleMMM}
% \begin{macro}{\eleMMMT}
% \begin{macro}{\eleMM}
% Selección de un elemento de una matriz
%    \begin{macrocode}
\NewDocumentCommand\eleMMM {mmm}{\ensuremath{\textrm{elem}_{#2#3}{\MatP*  {#1}}}\xspace}
\NewDocumentCommand\eleMMMT{mmm}{\ensuremath{\textrm{elem}_{#2#3}{\MatTPE*{#1}}}\xspace}
\NewDocumentCommand\eleMM  {mmm}{\ensuremath{\MakeLowercase{#1}_{{#2}{#3}}     }\xspace}
%    \end{macrocode}
% \end{macro}
% \end{macro}
% \end{macro}
%
%
% \iffalse
%%%%%%%%%%%%%%%%%%%%%%%%%%%%%%%%%%%%
%% --- Sistemas genéricos
%%%%%%%%%%%%%%%%%%%%%%%%%%%%%%%%%%%%
% \fi
%
% \subsection{Sistemas genéricos}
%
%%%%%%%%%%%%%%%%%%%%%%%%%%%%%%%%%%%%%%%%%%%%%
%
% \begin{macro}{\SV}
% Sistema de Vectores
%    \begin{macrocode}
\NewDocumentCommand\SV{O{}m}{\ensuremath{\mathsf{\MakeUppercase{#2}}_{#1}}\xspace}
%    \end{macrocode}
% \end{macro}
%
%
% \begin{macro}{\concatSV}
% Concatenación de sistemas
%    \begin{macrocode}
\NewDocumentCommand\concatSV{mm}{\ensuremath{{#1}\concat{#2}}\xspace}
%    \end{macrocode}
% \end{macro}
%
%
% \iffalse
%%%%%%%%%%%%%%%%%%%%%%%%%%%%%%%%%%%%
%% --- Notación matricial
%%%%%%%%%%%%%%%%%%%%%%%%%%%%%%%%%%%%
% \fi
%
% \subsection{Vectores y matrices}
% \subsubsection{Vectores}
%
% \begin{macro}{\vect}
% \begin{macro}{\vectp}
% \begin{macro}{\vectP}
% Vector genérico
%    \begin{macrocode}
\NewDocumentCommand\vect     {om}{\ensuremath{\IfNoValueTF{#1}
           { \vv {\MakeLowercase{#2}}       }
           { \vv*{\MakeLowercase{#2}}{\!#1} }                   }\xspace}

\NewDocumentCommand\vectp {som}{\ensuremath{\IfBooleanTF#1
    {\parentesis*{\IfNoValueTF{#2}{\vect{#3}}{\vect[#2]{#3}}}}
    {\parentesis {\IfNoValueTF{#2}{\vect{#3}}{\vect[#2]{#3}}}}  }\xspace}

\NewDocumentCommand\vectP {som}{\ensuremath{\IfBooleanTF#1
    {\Parentesis*{\IfNoValueTF{#2}{\vect{#3}}{\vect[#2]{#3}}}}
    {\Parentesis {\IfNoValueTF{#2}{\vect{#3}}{\vect[#2]{#3}}}}  }\xspace}
%    \end{macrocode}
% \end{macro}
% \end{macro}
% \end{macro}
%
%
%%%%%%%%%%%%%%%%%%%%%%%%%%%%%%%%%%%%%%%%%%%%%
% \subsubsection{Vectores de $\R[n]$}
%
% \begin{macro}{\Vect}
% \begin{macro}{\Vectp}
% \begin{macro}{\VectP}
% Vector de $\mathbb{R}^n$
%    \begin{macrocode}
\NewDocumentCommand\Vect   {O{}O{}m}{\ensuremath{%
                       \RidxE{\boldsymbol{\MakeLowercase{#3}}}{#1}{\boldsymbol{#2}}   }\xspace}

\NewDocumentCommand\Vectp {sO{}O{}m}{\ensuremath{\IfBooleanTF#1
    {\parentesis*{\Vect[#2][#3]{#4}}}
    {\parentesis {\Vect[#2][#3]{#4}}} }\xspace}

\NewDocumentCommand\VectP {sO{}O{}m}{\ensuremath{\IfBooleanTF#1
    {\Parentesis*{\Vect[#2][#3]{#4}}}
    {\Parentesis {\Vect[#2][#3]{#4}}} }\xspace}

\NewDocumentCommand\VectpKK {som}{\ensuremath{\IfBooleanTF#1
    {\parentesis*{\IfNoValueTF{#2}{\Vect{#3}}{\Vect[#2]{#3}}}}
    {\parentesis {\IfNoValueTF{#2}{\Vect{#3}}{\Vect[#2]{#3}}}} }\xspace}

\NewDocumentCommand\VectPKK {som}{\ensuremath{\IfBooleanTF#1
    {\Parentesis*{\IfNoValueTF{#2}{\Vect{#3}}{\Vect[#2]{#3}}}}
    {\Parentesis {\IfNoValueTF{#2}{\Vect{#3}}{\Vect[#2]{#3}}}} }\xspace}
%    \end{macrocode}
% \end{macro}
% \end{macro}
% \end{macro}
%
%
% \begin{macro}{\irvec}
% Sucesión de vectores de Rn
%    \begin{macrocode}
\NewDocumentCommand\irvec{O{1}O{n}m}{\ensuremath{\Vect[#1]{#3},\ldots,\Vect[#2]{#3} }\xspace}
%    \end{macrocode}
% \end{macro}
%
% \begin{macro}{\irvecC}
% Sucesión de columnas de una matriz
%    \begin{macrocode}
\NewDocumentCommand\irvecC{O{1}O{n}m}{\ensuremath{\VectC{#3}{#1},\ldots,\VectC{#3}{#2} }\xspace}
%    \end{macrocode}
% \end{macro}
%
%%%%%%%%%%%%%%%%%%%%%%%%%%%%%%%%%%%%%%%%%%%%%
% \subsubsection{Matrices}
%
% \begin{macro}{\Mat}
% \begin{macro}{\Matp}
% \begin{macro}{\Matp*}
% \begin{macro}{\MatP}
% \begin{macro}{\MatP*}
% Matriz
%    \begin{macrocode}
\NewDocumentCommand\Mat    {O{}m}{\ensuremath{%
                       \Ridx{\boldsymbol{\mathsf{\MakeUppercase{#2}}}}{#1}   }\xspace}

\NewDocumentCommand\Matp  {som}{\ensuremath{\IfBooleanTF#1
    {\parentesis*{\IfNoValueTF{#2}{\Mat{#3}}{\Mat[#2]{#3}}}}
    {\parentesis {\IfNoValueTF{#2}{\Mat{#3}}{\Mat[#2]{#3}}}}   }\xspace}

\NewDocumentCommand\MatP  {som}{\ensuremath{\IfBooleanTF#1
    {\Parentesis*{\IfNoValueTF{#2}{\Mat{#3}}{\Mat[#2]{#3}}}}
    {\Parentesis {\IfNoValueTF{#2}{\Mat{#3}}{\Mat[#2]{#3}}}}   }\xspace}
%    \end{macrocode}
% \end{macro}
% \end{macro}
% \end{macro}
% \end{macro}
% \end{macro}
%
%%%%%%%%%%%%%%%%%%%%%%%%%%%%%%%%%%%%%%%%%%%%%
% \textbf{Matrices transpuestas}
%
% \begin{macro}{\MatT}
% \begin{macro}{\MatTp}
% \begin{macro}{\MatTp*}
% \begin{macro}{\MatTP}
% \begin{macro}{\MatTP*}
% \begin{macro}{\MatTpE}
% \begin{macro}{\MatTpE*}
% \begin{macro}{\MatTPE}
% \begin{macro}{\MatTPE*}
% Matriz transpuesta
%    \begin{macrocode}
\NewDocumentCommand\MatT{O{}m}{\ensuremath{\RidxE{\Mat{#2}}{#1}{\T}}\xspace}

\NewDocumentCommand\MatTp  {som}{\ensuremath{\IfBooleanTF#1
       {\Transp* {\Mat[#2]{#3}}}
       {\Transp  {\Mat[#2]{#3}}}            }\xspace}

\NewDocumentCommand\MatTP  {som}{\ensuremath{\IfBooleanTF#1
       {\TransP* {\Mat[#2]{#3}}}
       {\TransP  {\Mat[#2]{#3}}}            }\xspace}

\NewDocumentCommand\MatTpE {sO{}m}{\ensuremath{\IfBooleanTF#1
       {\RidxEpE*{\Mat{#3}}{#2}{\T}}
       {\RidxEpE*{\Mat{#3}}{#2}{\T}}     }\xspace}

\NewDocumentCommand\MatTPE {sO{}m}{\ensuremath{\IfBooleanTF#1
       {\RidxEPE*{\Mat{#3}}{#2}{\T}}
       {\RidxEPE*{\Mat{#3}}{#2}{\T}}     }\xspace}
%    \end{macrocode}
% \end{macro}
% \end{macro}
% \end{macro}
% \end{macro}
% \end{macro}
% \end{macro}
% \end{macro}
% \end{macro}
% \end{macro}
%
%%%%%%%%%%%%%%%%%%%%%%%%%%%%%%%%%%%%%%%%%%%%%
% \textbf{\small\quad Matriz transpuesta de la transpuesta}
%
% \begin{macro}{\MatTT}
% \begin{macro}{\MatTT*}
% \begin{macro}{\MatTTPE}
% \begin{macro}{\MatTTPE*}
% Matriz transpuesta
%    \begin{macrocode}
\NewDocumentCommand\MatTT  {som}{\ensuremath{\IfBooleanTF#1
       {\TransP*{\MatT[#2]{#3}}}
       {\Transp {\MatT[#2]{#3}}}           }\xspace}

\NewDocumentCommand\MatTTPE{som}{\ensuremath{\IfBooleanTF#1
       {\Parentesis*{\MatTT*[#2]{#3}}}
       {\Parentesis {\MatTT [#2]{#3}}}     }\xspace}
%    \end{macrocode}
% \end{macro}
% \end{macro}
% \end{macro}
% \end{macro}
%
%%%%%%%%%%%%%%%%%%%%%%%%%%%%%%%%%%%%%%%%%%%%%
% \textbf{Matrices columna}
%
% \begin{macro}{\MVect}
% \begin{macro}{\MVect*}
% Matriz columna creada con un vector
%    \begin{macrocode}
\NewDocumentCommand\MVect{som}{\ensuremath{\IfBooleanTF#1
              {\left[\Vect[#2]{#3}\vphantom{\Big.}\right]}
              { \big[\Vect[#2]{#3}                \big  ]}   }\xspace}
%    \end{macrocode}
% \end{macro}
% \end{macro}
%
%
% \begin{macro}{\MVectT}
% \begin{macro}{\MVectT*}
% Matriz fila creada con un vector
%    \begin{macrocode}
\NewDocumentCommand\MVectT{som}{\ensuremath{\IfBooleanTF#1
              {\Trans{\MVect*[#2]{#3}} }
              {\Trans{\MVect [#2]{#3}} }                     }\xspace}
%    \end{macrocode}
% \end{macro}
% \end{macro}
%
%
% \begin{macro}{\MVectF}
% Matriz columna creada con una fila
%    \begin{macrocode}
\NewDocumentCommand\MVectF{somm}{\ensuremath{\IfBooleanTF#1
              {\left[\VectF[#2]{#3}{#4}\vphantom{\Big.}\right]}
              { \big[\VectF[#2]{#3}{#4}                \big  ]}   }\xspace}
%    \end{macrocode}
% \end{macro}
%
%
% \begin{macro}{\MVectC}
% Matriz columna creada con una columna
%    \begin{macrocode}
\NewDocumentCommand\MVectC{somm}{\ensuremath{\IfBooleanTF#1
              {\left[\VectC[#2]{#3}{#4}\vphantom{\Big.}\right]}
              { \big[\VectC[#2]{#3}{#4}                \big  ]}   }\xspace}
%    \end{macrocode}
% \end{macro}
%
%%%%%%%%%%%%%%%%%%%%%%%%%%%%%%%%%%%%%%%%%%%%%
% \textbf{Matrices fila}
%
% \begin{macro}{\MVectFT}
%    \begin{macrocode}
% Matriz fila creada con una fila y matriz fila creada con una columna
\NewDocumentCommand\MVectFT{somm}{\ensuremath{\IfBooleanTF#1
            {\Trans{\left[\VectF[#2]{#3}{#4}\vphantom{\Big.}\right]}}
            {\Trans{ \big[\VectF[#2]{#3}{#4}                  \big]}}  }\xspace}
%    \end{macrocode}
% \end{macro}
%
%
% \begin{macro}{\MVectCT}
%    \begin{macrocode}
% Matriz fila creada con una columna
\NewDocumentCommand\MVectCT{somm}{\ensuremath{\IfBooleanTF#1
            {\Trans{\left[\VectC[#2]{#3}{#4}\vphantom{\Big.}\right]}}
            {\Trans{ \big[\VectC[#2]{#3}{#4}                  \big]}}  }\xspace}
%    \end{macrocode}
% \end{macro}
%
%%%%%%%%%%%%%%%%%%%%%%%%%%%%%%%%%%%%%%%%%%%%%
%
% \iffalse
%%%%%%%%%%%%%%%%%%%%%%%%%%%%%%%%%%%%
%% --- Miscelánea matrices
%%%%%%%%%%%%%%%%%%%%%%%%%%%%%%%%%%%%
% \fi
%
% \subsubsection{Miscelánea matrices}
%
% \textbf{Características de las matrices}
%
% \begin{macro}{\Traza}
% Operador traza
%    \begin{macrocode}
\DeclareMathOperator{\Traza}{tr}
%    \end{macrocode}
% \end{macro}
%
%
% \begin{macro}{\rg}
% Operador rango
%    \begin{macrocode}
\DeclareMathOperator{\rg}{rg}
%    \end{macrocode}
% \end{macro}
%
%
% \begin{macro}{\traza}
% \begin{macro}{\traza*}
% Traza
%    \begin{macrocode}
\NewDocumentCommand\traza  {sm   }{\ensuremath{\IfBooleanTF#1
            {\Traza{\Parentesis*{#2}}}
            {\Traza{\parentesis {#2}}}         }\xspace}
%    \end{macrocode}
% \end{macro}
% \end{macro}
%
%
% \begin{macro}{\rango}
% \begin{macro}{\rango*}
% Rango
%    \begin{macrocode}\
\NewDocumentCommand\rango  {sm   }{\ensuremath{\IfBooleanTF#1
            {\rg{\Parentesis*{#2}}}
            {\rg{\parentesis {#2}}}            }\xspace}
%    \end{macrocode}
% \end{macro}
% \end{macro}
%
%
% \textbf{Determinante de una matriz}
%
%
% \begin{macro}{\cof}
% Cofactor
%    \begin{macrocode}
\DeclareMathOperator{\cof}{cof}
%    \end{macrocode}
% \end{macro}
%
%
% \begin{macro}{\adj}
% Adjunta
%    \begin{macrocode}
\DeclareMathOperator{\adj}{Adj}
%    \end{macrocode}
% \end{macro}
%
%
% \begin{macro}{\determinante}
% \begin{macro}{\determinante*}
% Determinante con barras
%    \begin{macrocode}
\NewDocumentCommand\determinante{sm}{\ensuremath{\IfBooleanTF#1
            {\modulus*{#2}}
            {\modulus {#2}}                     }\xspace}
%    \end{macrocode}
% \end{macro}
% \end{macro}
%
%
% \begin{macro}{\detp}
% \begin{macro}{\detp*}
% Determinante con la abreviatura det y paréntesis
%    \begin{macrocode}
\NewDocumentCommand\detp{sm}{\ensuremath{\IfBooleanTF#1
            {\det\parentesis*{#2}}
            {\det\parentesis {#2}}                     }\xspace}
%    \end{macrocode}
% \end{macro}
% \end{macro}
%
%
% \begin{macro}{\detP}
% \begin{macro}{\detP*}
% Determinante con la abreviatura det y Paréntesis
%    \begin{macrocode}
\NewDocumentCommand\detP{sm}{\ensuremath{\IfBooleanTF#1
            {\det\Parentesis*{#2}}
            {\det\Parentesis {#2}}                     }\xspace}
%    \end{macrocode}
% \end{macro}
% \end{macro}
%
%
% \begin{macro}{\subMat}
% Determinante con barras
%    \begin{macrocode}
\NewDocumentCommand\subMat{mmm}{\ensuremath{
        \quitaLR{\Mat{#1}}{#2}{#3}             }\xspace}
%    \end{macrocode}
% \end{macro}
%
%
% \begin{macro}{\Menor}
% \begin{macro}{\MenorR}
% Menor de una matriz
%    \begin{macrocode}
\NewDocumentCommand\Menor {mmm}{\ensuremath{
        \det\parentesis{\subMat{#1}{#2}{#3}}   }\xspace}

\NewDocumentCommand\MenoR {mmm}{\ensuremath{
               \big|{\subMat{#1}{#2}{#3}}\big| }\xspace}
%    \end{macrocode}
% \end{macro}
% \end{macro}
%
%
% \begin{macro}{\Cof}
% Cofactor de una matriz
%    \begin{macrocode}
\NewDocumentCommand\Cof{smmm}{\ensuremath{\IfBooleanTF#1
            {\cof_{{#3}{#4}}\Parentesis*{\Mat{#2}}}
            {\cof_{{#3}{#4}}\parentesis {\Mat{#2}}}  }\xspace}
%    \end{macrocode}
% \end{macro}
%
%
% \textbf{Orden de las matrices}
%
% \begin{macro}{\Dim}
% \begin{macro}{\Dimp}
% \begin{macro}{\Dimp*}
% \begin{macro}{\DimP}
% \begin{macro}{\DimP*}
% \begin{macro}{\DimpE}
% \begin{macro}{\DimpE*}
% \begin{macro}{\DimPE}
% \begin{macro}{\DimPE*}
% Orden del objeto
%    \begin{macrocode}
\NewDocumentCommand\Dim{mmm}{\ensuremath{
       \mathop{#1}\limits_{\scriptscriptstyle #2\times#3}  }\xspace}

\NewDocumentCommand\Dimp{smmm}{\ensuremath{\IfBooleanTF#1
       {\Dim{\parentesis*{#2}}{#3}{#4}}
       {\Dim{\parentesis {#2}}{#3}{#4}}      }\xspace}

\NewDocumentCommand\DimP     {smmm}{\ensuremath{\IfBooleanTF#1
       {\Dim{\Parentesis*{#2}}{#3}{#4}}
       {\Dim{\Parentesis {#2}}{#3}{#4}}      }\xspace}

\NewDocumentCommand\DimpE    {smmm}{\ensuremath{\IfBooleanTF#1
       {\parentesis*{\Dim{#2}{#3}{#4}}}
       {\parentesis {\Dim{#2}{#3}{#4}}}      }\xspace}

\NewDocumentCommand\DimPE    {smmm}{\ensuremath{\IfBooleanTF#1
       {\Parentesis*{\Dim{#2}{#3}{#4}}}
       {\Parentesis {\Dim{#2}{#3}{#4}}}      }\xspace}
%    \end{macrocode}
% \end{macro}
% \end{macro}
% \end{macro}
% \end{macro}
% \end{macro}
% \end{macro}
% \end{macro}
% \end{macro}
% \end{macro}
%
%
% \begin{macro}{\Matdim}
% \begin{macro}{\Matdimp}
% \begin{macro}{\Matdimp*}
% \begin{macro}{\MatdimP}
% \begin{macro}{\MatdimP*}
% \begin{macro}{\MatdimpE}
% \begin{macro}{\MatdimpE*}
% \begin{macro}{\MatdimPE}
% \begin{macro}{\MatdimPE*}
% Matriz con el orden por debajo
%    \begin{macrocode}
\NewDocumentCommand\Matdim{ommm}{\ensuremath{ \Dim{\Mat[#1]{#2}}{#3}{#4} }\xspace}

\NewDocumentCommand\Matdimp {sommm}{\ensuremath{\IfBooleanTF#1
       {\Dimp*{\Mat[#2]{#3}}{#4}{#5}}
       {\Dimp {\Mat[#2]{#3}}{#4}{#5}}        }\xspace}

\NewDocumentCommand\MatdimP {sommm}{\ensuremath{\IfBooleanTF#1
       {\DimP*{\Mat[#2]{#3}}{#4}{#5}}
       {\DimP {\Mat[#2]{#3}}{#4}{#5}}        }\xspace}

\NewDocumentCommand\MatdimpE{sommm}{\ensuremath{\IfBooleanTF#1
       {\DimpE*{\Mat[#2]{#3}}{#4}{#5}}
       {\DimpE {\Mat[#2]{#3}}{#4}{#5}}       }\xspace}

\NewDocumentCommand\MatdimPE{sommm}{\ensuremath{\IfBooleanTF#1
       {\DimPE*{\Mat[#2]{#3}}{#4}{#5}}
       {\DimPE {\Mat[#2]{#3}}{#4}{#5}}       }\xspace}
%    \end{macrocode}
% \end{macro}
% \end{macro}
% \end{macro}
% \end{macro}
% \end{macro}
% \end{macro}
% \end{macro}
% \end{macro}
% \end{macro}
%
%
% \begin{macro}{\MatTdim}
% \begin{macro}{\MatTdimp}
% \begin{macro}{\MatTdimp*}
% \begin{macro}{\MatTdimP}
% \begin{macro}{\MatTdimP*}
% \begin{macro}{\MatTdimpE}
% \begin{macro}{\MatTdimpE*}
% \begin{macro}{\MatTdimPE}
% \begin{macro}{\MatTdimPE*}
% Matriz con el orden por debajo
%    \begin{macrocode}
\NewDocumentCommand\MatTdim{ommm}{\ensuremath{ \Dim{\MatT[#1]{#2}}{#3}{#4} }\xspace}

\NewDocumentCommand\MatTdimp {sommm}{\ensuremath{\IfBooleanTF#1
       {\Dimp*{\MatT[#2]{#3}}{#4}{#5}}
       {\Dimp {\MatT[#2]{#3}}{#4}{#5}}        }\xspace}

\NewDocumentCommand\MatTdimP {sommm}{\ensuremath{\IfBooleanTF#1
       {\DimP*{\MatT[#2]{#3}}{#4}{#5}}
       {\DimP {\MatT[#2]{#3}}{#4}{#5}}        }\xspace}

\NewDocumentCommand\MatTdimpE{sommm}{\ensuremath{\IfBooleanTF#1
       {\DimpE*{\MatT[#2]{#3}}{#4}{#5}}
       {\DimpE {\MatT[#2]{#3}}{#4}{#5}}       }\xspace}

\NewDocumentCommand\MatTdimPE{sommm}{\ensuremath{\IfBooleanTF#1
       {\DimPE*{\MatTpE*[#2]{#3}}{#4}{#5}}
       {\DimPE {\MatTpE*[#2]{#3}}{#4}{#5}}       }\xspace}
%    \end{macrocode}
% \end{macro}
% \end{macro}
% \end{macro}
% \end{macro}
% \end{macro}
% \end{macro}
% \end{macro}
% \end{macro}
% \end{macro}
%
% \textbf{Matriz de autovalores}
%
% \begin{macro}{\MDaV}
% Matriz de autovalores
%    \begin{macrocode}
\def\MDaV{D}
%    \end{macrocode}
% \end{macro}
%
%
% \textbf{Matriz triangular superior unitaria} (según denominación de G. H. Golub y C. F. Van Loan)
%
% \begin{macro}{\Umat}
% Matriz triangular superior unitaria
%    \begin{macrocode}
\NewDocumentCommand\UMat{O{}m}{\ensuremath{ \Mat[#1]{\Dot{#2}} }\xspace}
%    \end{macrocode}
% \end{macro}
%
% \begin{macro}{\InvUmat}
% Inversa de matriz triangular superior unitaria
%    \begin{macrocode}
\NewDocumentCommand\InvUMat{O{}m}{\ensuremath{\RidxE{\Mat{\Dot{#2}}}{#1}{\minus1} }\xspace}

%    \end{macrocode}
% \end{macro}
%
% \textbf{Matriz triangular inferior unitaria} (según denominación de G. H. Golub y C. F. Van Loan)
%
% \begin{macro}{\UmatT}
% Matriz triangular inferior unitaria
%    \begin{macrocode}
\NewDocumentCommand\UMatT{O{}m}{\ensuremath{\RidxE{\Mat{\Dot{#2}}}{#1}{\T} }\xspace}
%    \end{macrocode}
% \end{macro}
%
% \textbf{Matriz de eliminación gaussiana (por columnas) y su inversa}
%
% \begin{macro}{\MatGC}
% \begin{macro}{\InvMatGC}
% Matriz de eliminación gaussiana (por columnas)
%    \begin{macrocode}
\NewDocumentCommand\MatGC   {m}{\ensuremath{       \UMat[#1\triangleright]{G}           }\xspace}
\NewDocumentCommand\InvMatGC{m}{\ensuremath{\RidxE{\UMat{G}}{#1\triangleright}{\minus1} }\xspace}
%    \end{macrocode}
% \end{macro}
% \end{macro}
%
%
% \iffalse
%%%%%%%%%%%%%%%%%%%%%%%%%%%%%%%%%%%%
%% --- Productos entre vectores
%%%%%%%%%%%%%%%%%%%%%%%%%%%%%%%%%%%%
% \fi
%
% \subsection{Productos entre vectores}
%
% \subsubsection{Producto escalar}
%
% \begin{macro}{\eSc}
% \begin{macro}{\eSc*}
% Producto escalar
%    \begin{macrocode}
\NewDocumentCommand\eSc{sO{}mm}{\ensuremath{\IfBooleanTF#1
    {\Ridx{\Angulos*{\left.#3 \right| #4}}{\!#2}}
    {\Ridx{\angulos {      #3   \big| #4}}{\!#2}}     }\xspace}
\NewDocumentCommand\eSckk{sO{}mm}{\ensuremath{{\IfBooleanTF#1
             {\left< {#3} , {#4} \right>}
             {\big\langle{{#3} , {#4}\big\rangle}}}_{\!#2}  }\xspace}
%    \end{macrocode}
% \end{macro}
% \end{macro}
%
%
% \begin{macro}{\esc}
% \begin{macro}{\esc*}
% Producto escalar entre vectores genéricos
%    \begin{macrocode}
\NewDocumentCommand\esc{sO{}mm}{\ensuremath{{\IfBooleanTF#1
            {\eSc*{\vect{#3}}{\vect{#4}}}
            {\eSc {\vect{#3}}{\vect{#4}}}}_{\!#2}         }\xspace}
%    \end{macrocode}
% \end{macro}
% \end{macro}
%
%
% \subsubsection{Producto punto}
%
% \begin{macro}{\dotProd}
% \begin{macro}{\dotProdp}
% \begin{macro}{\dotProdp*}
% \begin{macro}{\dotProdP}
% \begin{macro}{\dotProdP*}
% Producto punto
%    \begin{macrocode}
\NewDocumentCommand\dotProd{mm}{\ensuremath{{#1}\cdot{#2}}\xspace}

\NewDocumentCommand\dotProdp{smm}{\ensuremath{\IfBooleanTF#1
            {\parentesis*{{#2}\cdot{#3}}}
            {\parentesis {{#2}\cdot{#3}}}              }\xspace}

\NewDocumentCommand\dotProdP{smm}{\ensuremath{\IfBooleanTF#1
            {\Parentesis*{{#2}\cdot{#3}}}
            {\Parentesis {{#2}\cdot{#3}}}              }\xspace}
%    \end{macrocode}
% \end{macro}
% \end{macro}
% \end{macro}
% \end{macro}
% \end{macro}
%
%
% \begin{macro}{\dotprod}
% \begin{macro}{\dotprodp}
% \begin{macro}{\dotprodp*}
% \begin{macro}{\dotprodP}
% \begin{macro}{\dotprodP*}
% Producto punto
%    \begin{macrocode}
\NewDocumentCommand\dotprod{O{}mO{}m}{\ensuremath{\dotProd{\Vect[#1]{#2}}{\Vect[#3]{#4}}}\xspace}

\NewDocumentCommand\dotprodp{sO{}mO{}m}{\ensuremath{\IfBooleanTF#1
            {\parentesis*{\dotprod[#2]{#3}[#4]{#5}}}
            {\parentesis {\dotprod[#2]{#3}[#4]{#5}}}           }\xspace}

\NewDocumentCommand\dotprodP{sO{}mO{}m}{\ensuremath{\IfBooleanTF#1
            {\Parentesis*{\dotprod[#2]{#3}[#4]{#5}}}
            {\Parentesis {\dotprod[#2]{#3}[#4]{#5}}}           }\xspace}
%    \end{macrocode}
% \end{macro}
% \end{macro}
% \end{macro}
% \end{macro}
% \end{macro}
%
% \subsubsection{Producto punto a punto o \emph{Hadamard}}
%
% \begin{macro}{\prodH}
% \begin{macro}{\prodHp}
% \begin{macro}{\prodHp*}
% \begin{macro}{\prodHP}
% \begin{macro}{\prodHP*}
% Producto punto a punto o \emph{Hadamard}
%    \begin{macrocode}
\NewDocumentCommand\prodH{mm}{\ensuremath{ {#1}\odot{#2} }\xspace}

\NewDocumentCommand\prodHp{smm}{\ensuremath{\IfBooleanTF#1
            {\parentesis*{\prodH{#2}{#3}}}
            {\parentesis {\prodH{#2}{#3}}}           }\xspace}

\NewDocumentCommand\prodHP{smm}{\ensuremath{\IfBooleanTF#1
            {\Parentesis*{\prodH{#2}{#3}}}
            {\Parentesis {\prodH{#2}{#3}}}           }\xspace}
%    \end{macrocode}
% \end{macro}
% \end{macro}
% \end{macro}
% \end{macro}
% \end{macro}
%
%
% \begin{macro}{\prodh}
% \begin{macro}{\prodhp}
% \begin{macro}{\prodhp*}
% \begin{macro}{\prodhP}
% \begin{macro}{\prodhP*}
% Producto punto a punto o \emph{Hadamard}
%    \begin{macrocode}
\NewDocumentCommand\prodh{O{}mO{}m}{\ensuremath{
            \prodH{\Vect[#1]{#2}}{\Vect[#3]{#4}} }\xspace}

\NewDocumentCommand\prodhp{sO{}mO{}m}{\ensuremath{\IfBooleanTF#1
            {\parentesis*{\prodh[#2]{#3}[#4]{#5}}}
            {\parentesis {\prodh[#2]{#3}[#4]{#5}}}           }\xspace}

\NewDocumentCommand\prodhP{sO{}mO{}m}{\ensuremath{\IfBooleanTF#1
            {\Parentesis*{\prodh[#2]{#3}[#4]{#5}}}
            {\Parentesis {\prodh[#2]{#3}[#4]{#5}}}           }\xspace}
%    \end{macrocode}
% \end{macro}
% \end{macro}
% \end{macro}
% \end{macro}
% \end{macro}
%
%
% \iffalse
%%%%%%%%%%%%%%%%%%%%%%%%%%%%%%%%%%%%
%% --- Matriz por vector y vector por matriz
%%%%%%%%%%%%%%%%%%%%%%%%%%%%%%%%%%%%
% \fi
%
% \subsection{Matriz por vector y vector por matriz}
%
% \begin{macro}{\MV}
% \begin{macro}{\MVpE}
% \begin{macro}{\MVpE*}
% \begin{macro}{\MVPE}
% \begin{macro}{\MVPE*}
% Producto de matriz por vector
%    \begin{macrocode}
\NewDocumentCommand\MV { O{}mO{}m}{\ensuremath{ \Mat[#1]{#2}\Vect[#3]{#4} }\xspace}

\NewDocumentCommand\MVpE{sO{}mO{}m}{\ensuremath{\IfBooleanTF#1
            {\parentesis*{\MV[#2]{#3}[#4]{#5}}}
            {\parentesis {\MV[#2]{#3}[#4]{#5}}}   }\xspace}

\NewDocumentCommand\MVPE{sO{}mO{}m}{\ensuremath{\IfBooleanTF#1
            {\Parentesis*{\MV[#2]{#3}[#4]{#5}}}
            {\Parentesis {\MV[#2]{#3}[#4]{#5}}}   }\xspace}
%    \end{macrocode}
% \end{macro}
% \end{macro}
% \end{macro}
% \end{macro}
% \end{macro}
%
%
% \begin{macro}{\VM}
% \begin{macro}{\VMpE}
% \begin{macro}{\VMpE*}
% \begin{macro}{\VMPE}
% \begin{macro}{\VMPE*}
% Producto de vector por matriz
%    \begin{macrocode}
\NewDocumentCommand\VM { O{}mO{}m}{\ensuremath{ \Vect[#1]{#2}\Mat[#3]{#4} }\xspace}

\NewDocumentCommand\VMpE{sO{}mO{}m}{\ensuremath{\IfBooleanTF#1
            {\parentesis*{\VM[#2]{#3}[#4]{#5}}}
            {\parentesis {\VM[#2]{#3}[#4]{#5}}}   }\xspace}

\NewDocumentCommand\VMPE{sO{}mO{}m}{\ensuremath{\IfBooleanTF#1
            {\Parentesis*{\VM[#2]{#3}[#4]{#5}}}
            {\Parentesis {\VM[#2]{#3}[#4]{#5}}}   }\xspace}
%    \end{macrocode}
% \end{macro}
% \end{macro}
% \end{macro}
% \end{macro}
% \end{macro}
%
%
% \begin{macro}{\MTV}
% \begin{macro}{\MTVp}
% \begin{macro}{\MTVp*}
% \begin{macro}{\MTVP}
% \begin{macro}{\MTVP*}
% Producto de matriz por vector
%    \begin{macrocode}
\NewDocumentCommand\MTV{ O{}mO{}m}{\ensuremath{ \MatT[#1]{#2}\Vect[#3]{#4} }\xspace}

\NewDocumentCommand\MTVp{sO{}mO{}m}{\ensuremath{\IfBooleanTF#1
            {\MatTpE*[#2]{#3}\Vect[#4]{#5}}
            {\MatTpE [#2]{#3}\Vect[#4]{#5}}  }\xspace}

\NewDocumentCommand\MTVP{sO{}mO{}m}{\ensuremath{\IfBooleanTF#1
            {\MatTPE*[#2]{#3}\Vect[#4]{#5}}
            {\MatTPE [#2]{#3}\Vect[#4]{#5}}  }\xspace}
%    \end{macrocode}
% \end{macro}
% \end{macro}
% \end{macro}
% \end{macro}
% \end{macro}
%
%
% \begin{macro}{\VMT}
% \begin{macro}{\VMTp}
% \begin{macro}{\VMTp*}
% \begin{macro}{\VMTP}
% \begin{macro}{\VMTP*}
% Producto de vector por matriz
%    \begin{macrocode}
\NewDocumentCommand\VMT{ O{}mO{}m}{\ensuremath{ \Vect[#1]{#2}\MatT[#3]{#4} }\xspace}

\NewDocumentCommand\VMTp{sO{}mO{}m}{\ensuremath{\IfBooleanTF#1
            {\Vect[#2]{#3}\MatTpE*[#4]{#5}}
            {\Vect[#2]{#3}\MatTpE [#4]{#5}}        }\xspace}

\NewDocumentCommand\VMTP{sO{}mO{}m}{\ensuremath{\IfBooleanTF#1
            {\Vect[#2]{#3}\MatTPE*[#4]{#5}}
            {\Vect[#2]{#3}\MatTPE [#4]{#5}}        }\xspace}
%    \end{macrocode}
% \end{macro}
% \end{macro}
% \end{macro}
% \end{macro}
% \end{macro}
%
%
% \iffalse
%%%%%%%%%%%%%%%%%%%%%%%%%%%%%%%%%%%%
%% --- Matriz por matriz
%%%%%%%%%%%%%%%%%%%%%%%%%%%%%%%%%%%%
% \fi
%
% \subsection{Matriz por matriz}
%
% \begin{macro}{\MN}
% Producto de matriz por matriz
%    \begin{macrocode}
\NewDocumentCommand\MN  {O{}mO{}m}{\ensuremath{ \Mat[#1]{#2}\Mat[#3]{#4} }\xspace}
%    \end{macrocode}
% \end{macro}
%
%
% \begin{macro}{\MTN}
% \begin{macro}{\MTNp}
% \begin{macro}{\MTNp*}
% \begin{macro}{\MTNP}
% \begin{macro}{\MTNP*}
% Producto de matriz transpuesta por matriz
%    \begin{macrocode}
\NewDocumentCommand\MTN {O{}mO{}m}{\ensuremath{ \MatT[#1]{#2}\Mat[#3]{#4} }\xspace}

\NewDocumentCommand\MTNp {sO{}mO{}m}{\ensuremath{\IfBooleanTF#1
            {\MatTpE*[#2]{#3}\Mat[#4]{#5}}
            {\MatTpE [#2]{#3}\Mat[#4]{#5}}                         }\xspace}

\NewDocumentCommand\MTNP {sO{}mO{}m}{\ensuremath{\IfBooleanTF#1
            {\MatTPE*[#2]{#3}\Mat[#4]{#5}}
            {\MatTPE [#2]{#3}\Mat[#4]{#5}}                         }\xspace}
%    \end{macrocode}
% \end{macro}
% \end{macro}
% \end{macro}
% \end{macro}
% \end{macro}
%
%
% \begin{macro}{\MNT}
% \begin{macro}{\MNTp}
% \begin{macro}{\MNTp*}
% \begin{macro}{\MNTP}
% \begin{macro}{\MNTP*}
% Producto de matriz por matriz transpuesta
%    \begin{macrocode}
\NewDocumentCommand\MNT {O{}mO{}m}{\ensuremath{ \Mat[#1]{#2}\MatT[#3]{#4} }\xspace}

\NewDocumentCommand\MNTp {sO{}mO{}m}{\ensuremath{\IfBooleanTF#1
            {\Mat[#2]{#3}\MatTpE*[#4]{#5}}
            {\Mat[#2]{#3}\MatTpE [#4]{#5}}                         }\xspace}

\NewDocumentCommand\MNTP {sO{}mO{}m}{\ensuremath{\IfBooleanTF#1
            {\Mat[#2]{#3}\MatTPE*[#4]{#5}}
            {\Mat[#2]{#3}\MatTPE [#4]{#5}}                         }\xspace}
%    \end{macrocode}
% \end{macro}
% \end{macro}
% \end{macro}
% \end{macro}
% \end{macro}
%
%
% \begin{macro}{\MTM}
% \begin{macro}{\MTMp}
% \begin{macro}{\MTMp*}
% \begin{macro}{\MTMP}
% \begin{macro}{\MTMP*}
% Producto de matriz transpuesta por matriz
%    \begin{macrocode}
\NewDocumentCommand\MTM {O{}m }{\ensuremath{ \MTN[#1]{#2}[#1]{#2}    }\xspace}

\NewDocumentCommand\MTMp{som}{\ensuremath{\IfBooleanTF#1
                             {\MTNp*[#2]{#3}[#2]{#3}}
                             {\MTNp [#2]{#3}[#2]{#3}}              }\xspace}

\NewDocumentCommand\MTMP{som}{\ensuremath{\IfBooleanTF#1
                             {\MTNP*[#2]{#3}[#2]{#3}}
                             {\MTNP [#2]{#3}[#2]{#3}}              }\xspace}
%    \end{macrocode}
% \end{macro}
% \end{macro}
% \end{macro}
% \end{macro}
% \end{macro}
%
%
% \begin{macro}{\MMT}
% \begin{macro}{\MMTp}
% \begin{macro}{\MMTp*}
% \begin{macro}{\MMTP}
% \begin{macro}{\MMTP*}
% Producto de matriz por su transpuesta
%    \begin{macrocode}
\NewDocumentCommand\MMT {O{}m }{\ensuremath{ \MNT[#1]{#2}[#1]{#2}    }\xspace}

\NewDocumentCommand\MMTp{som}{\ensuremath{\IfBooleanTF#1
                             {\MNTp*[#2]{#3}[#2]{#3}}
                             {\MNTp [#2]{#3}[#2]{#3}}              }\xspace}

\NewDocumentCommand\MMTP{som}{\ensuremath{\IfBooleanTF#1
                             {\MNTP*[#2]{#3}[#2]{#3}}
                             {\MNTP [#2]{#3}[#2]{#3}}              }\xspace}
%    \end{macrocode}
% \end{macro}
% \end{macro}
% \end{macro}
% \end{macro}
% \end{macro}
%
%%%%%%%%%%%%%%%%%%%%%%%%%%%%%%%
%
% \begin{macro}{\MNMT}
% \begin{macro}{\MNMTp}
% \begin{macro}{\MNMTp*}
% \begin{macro}{\MNMTP}
% \begin{macro}{\MNMTP*}
% Producto de matriz por matriz por matriz transpuesta
%    \begin{macrocode}
\NewDocumentCommand\MNMT{O{}mO{}m}{\ensuremath{ \MN[#1]{#2}[#3]{#4}\MatT[#1]{#2} }\xspace}

\NewDocumentCommand\MNMTp{somom}{\ensuremath{\IfBooleanTF#1
                             {\MN[#2]{#3}[#4]{#5}\MatTpE*[#2]{#3}}
                             {\MN[#2]{#3}[#4]{#5}\MatTpE [#2]{#3}} }\xspace}

\NewDocumentCommand\MNMTP{somom}{\ensuremath{\IfBooleanTF#1
                             {\MN[#2]{#3}[#4]{#5}\MatTPE*[#2]{#3}}
                             {\MN[#2]{#3}[#4]{#5}\MatTPE [#2]{#3}} }\xspace}
%    \end{macrocode}
% \end{macro}
% \end{macro}
% \end{macro}
% \end{macro}
% \end{macro}
%
%
% \begin{macro}{\MTNM}
% \begin{macro}{\MTNMp}
% \begin{macro}{\MTNMp*}
% \begin{macro}{\MTNMP}
% \begin{macro}{\MTNMP*}
% Producto de matriz transpuesta por matriz por matriz
%    \begin{macrocode}
\NewDocumentCommand\MTNM{O{}mO{}m}{\ensuremath{ \MTN[#1]{#2}[#3]{#4}\Mat[#1]{#2} }\xspace}

\NewDocumentCommand\MTNMp{somom}{\ensuremath{\IfBooleanTF#1
                             {\MTNp*[#2]{#3}[#4]{#5}\Mat[#2]{#3}}
                             {\MTNp [#2]{#3}[#4]{#5}\Mat[#2]{#3}}              }\xspace}

\NewDocumentCommand\MTNMP{somom}{\ensuremath{\IfBooleanTF#1
                             {\MTNP*[#2]{#3}[#4]{#5}\Mat[#2]{#3}}
                             {\MTNP [#2]{#3}[#4]{#5}\Mat[#2]{#3}}              }\xspace}
%    \end{macrocode}
% \end{macro}
% \end{macro}
% \end{macro}
% \end{macro}
% \end{macro}
%
%%%%%%%%%%%%%%%%%%%%%%%%%%%%%%%%%%%%%%%%%%%%%
%
% \iffalse
%%%%%%%%%%%%%%%%%%%%%%%%%%%%%%%%%%%%
%% --- Matriz inversa
%%%%%%%%%%%%%%%%%%%%%%%%%%%%%%%%%%%%
% \fi
%
% \paragraph{Matriz inversa}
%
% \begin{macro}{\InvMat}
% \begin{macro}{\InvMatp}
% \begin{macro}{\InvMatp*}
% \begin{macro}{\InvMatP}
% \begin{macro}{\InvMatP*}
% \begin{macro}{\InvMatpE}
% \begin{macro}{\InvMatpE*}
% \begin{macro}{\InvMatPE}
% \begin{macro}{\InvMatPE*}
% Inversa de una matriz
%    \begin{macrocode}
\NewDocumentCommand\InvMat  { O{}m}{\ensuremath{\RidxE{\Mat{#2}}{#1}{\minus1} }\xspace}

\NewDocumentCommand\InvMatp {som}{\ensuremath{\IfBooleanTF#1
            {\Invp*{\Mat[#2]{#3}}}
            {\Invp {\Mat[#2]{#3}}}              }\xspace}

\NewDocumentCommand\InvMatP {som}{\ensuremath{\IfBooleanTF#1
            {\InvP*{\Mat[#2]{#3}}}
            {\InvP {\Mat[#2]{#3}}}              }\xspace}

\NewDocumentCommand\InvMatpE{sO{}m}{\ensuremath{\IfBooleanTF#1
            {\RidxEpE*{\Mat{#3}}{#2}{\minus1}}
            {\RidxEpE*{\Mat{#3}}{#2}{\minus1}}  }\xspace}

\NewDocumentCommand\InvMatPE{sO{}m}{\ensuremath{\IfBooleanTF#1
            {\RidxEPE*{\Mat{#3}}{#2}{\minus1}}
            {\RidxEPE*{\Mat{#3}}{#2}{\minus1}}  }\xspace}

%    \end{macrocode}
% \end{macro}
% \end{macro}
% \end{macro}
% \end{macro}
% \end{macro}
% \end{macro}
% \end{macro}
% \end{macro}
% \end{macro}
%
%
% \begin{macro}{\InvMatT}
% \begin{macro}{\InvMatT*}
% \begin{macro}{\InvMatTpE}
% \begin{macro}{\InvMatTpE*}
% \begin{macro}{\InvMatTPE}
% \begin{macro}{\InvMatTPE*}
% Inversa de una matriz transpuesta
%    \begin{macrocode}
\NewDocumentCommand\InvMatT {som }{\ensuremath{\IfBooleanTF#1
            {\InvP*{ \MatT[#2]{#3} }}
            {\Invp { \MatT[#2]{#3} }}                        }\xspace}

\NewDocumentCommand\InvMatTpE{som }{\ensuremath{\IfBooleanTF#1
            {\parentesis*{\InvP*{ \MatT[#2]{#3} }}}
            {\parentesis {\Invp*{ \MatT[#2]{#3} }}}          }\xspace}

\NewDocumentCommand\InvMatTPE{som }{\ensuremath{\IfBooleanTF#1
            {\Parentesis*{\InvP*{ \MatT[#2]{#3} }}}
            {\Parentesis {\Invp { \MatT[#2]{#3} }}}          }\xspace}
%    \end{macrocode}
% \end{macro}
% \end{macro}
% \end{macro}
% \end{macro}
% \end{macro}
% \end{macro}
%
%
% \begin{macro}{\TInvMat}
% \begin{macro}{\TInvMat*}
% \begin{macro}{\TInvMatpE}
% \begin{macro}{\TInvMatpE*}
% \begin{macro}{\TInvMatPE}
% \begin{macro}{\TInvMatPE*}
% Transpuesta de la inversa de una matriz
%    \begin{macrocode}
\NewDocumentCommand\TInvMat {som }{\ensuremath{\IfBooleanTF#1
            {\Trans{\left.\InvMatpE*[#2]{#3}\!\right.}}
            {\Trans{      \InvMatpE [#2]{#3}}}           }\xspace}

\NewDocumentCommand\TInvMatpE {som }{\ensuremath{\IfBooleanTF#1
            {\parentesis*{  \TInvMat*[#2]{#3}}}
            {\parentesis {\!\TInvMat*[#2]{#3}}}          }\xspace}

\NewDocumentCommand\TInvMatPE {som }{\ensuremath{\IfBooleanTF#1
            {\Parentesis*{\TInvMat*[#2]{#3}}}
            {\Parentesis {\TInvMat [#2]{#3}}}            }\xspace}
%    \end{macrocode}
% \end{macro}
% \end{macro}
% \end{macro}
% \end{macro}
% \end{macro}
% \end{macro}
%
%
% \iffalse
%%%%%%%%%%%%%%%%%%%%%%%%%%%%%%%%%%%%
%% --- Otros productos entre matrices y vectores
%%%%%%%%%%%%%%%%%%%%%%%%%%%%%%%%%%%%
% \fi
%
% \subsection{Otros productos entre matrices y vectores}
%
%
% \begin{macro}{\MTMV}
% \begin{macro}{\MTMVp}
% \begin{macro}{\MTMVp*}
% \begin{macro}{\MTMVP}
% \begin{macro}{\MTMVP*}
% Producto de matriz transpuesta por matriz por vector
%    \begin{macrocode}
\NewDocumentCommand\MTMV { mm }{\ensuremath{ \MTN {#1}{#1}\Vect{#2} }\xspace}

\NewDocumentCommand\MTMVp{smm}{\ensuremath{\IfBooleanTF#1
            {\MTNp*{#2}{#2}\Vect{#3}}
            {\MTNp {#2}{#2}\Vect{#3}}      }\xspace}

\NewDocumentCommand\MTMVP{smm}{\ensuremath{\IfBooleanTF#1
            {\MTNP*{#2}{#2}\Vect{#3}}
            {\MTNP {#2}{#2}\Vect{#3}}      }\xspace}
%    \end{macrocode}
% \end{macro}
% \end{macro}
% \end{macro}
% \end{macro}
% \end{macro}
%
%
% \begin{macro}{\VMW}
% Producto de vector por matriz por vector
%    \begin{macrocode}
\NewDocumentCommand\VMW  { mmm}{\ensuremath{ \VM  {#1}{#2}\Vect{#3} }\xspace}
%    \end{macrocode}
% \end{macro}
%
%
% \begin{macro}{\VMV}
% Producto de vector por matriz por vector
%    \begin{macrocode}
\NewDocumentCommand\VMV  { mm }{\ensuremath{ \VMW {#1}{#2}{#1}      }\xspace}
%    \end{macrocode}
% \end{macro}
%
%
% \begin{macro}{\VMTW}
% \begin{macro}{\VMTWp}
% \begin{macro}{\VMTWp*}
% \begin{macro}{\VMTWP}
% \begin{macro}{\VMTWP*}
% Producto de vector por matriz transpuesta por vector
%    \begin{macrocode}
\NewDocumentCommand\VMTW { mmm}{\ensuremath{ \VMT {#1}{#2}\Vect{#3} }\xspace}

\NewDocumentCommand\VMTWp{smmm}{\ensuremath{\IfBooleanTF#1
            {\VMTp*{#2}{#3}\Vect{#4}}
            {\VMTp {#2}{#3}\Vect{#4}}      }\xspace}

\NewDocumentCommand\VMTWP{smmm}{\ensuremath{\IfBooleanTF#1
            {\VMTP*{#2}{#3}\Vect{#4}}
            {\VMTP {#2}{#3}\Vect{#4}}      }\xspace}
%    \end{macrocode}
% \end{macro}
% \end{macro}
% \end{macro}
% \end{macro}
% \end{macro}
%
%
% \begin{macro}{\VMTV}
% \begin{macro}{\VMTVp}
% \begin{macro}{\VMTVp*}
% \begin{macro}{\VMTVP}
% \begin{macro}{\VMTVP*}
% Producto de vector por matriz transpuesta por vector
%    \begin{macrocode}
\NewDocumentCommand\VMTV { mm }{\ensuremath{ \VMTW{#1}{#2}{#1}      }\xspace}

\NewDocumentCommand\VMTVp{smm}{\ensuremath{\IfBooleanTF#1
            {\VMTp*{#2}{#3}\Vect{#2}}
            {\VMTp {#2}{#3}\Vect{#2}}      }\xspace}

\NewDocumentCommand\VMTVP{smm}{\ensuremath{\IfBooleanTF#1
            {\VMTP*{#2}{#3}\Vect{#2}}
            {\VMTP {#2}{#3}\Vect{#2}}      }\xspace}
%    \end{macrocode}
% \end{macro}
% \end{macro}
% \end{macro}
% \end{macro}
% \end{macro}
%
%
% \begin{macro}{\InvMTM}
% \begin{macro}{\InvMTM*}
% Inversa del producto de una matriz transpuesta por ella misma
%    \begin{macrocode}
\NewDocumentCommand\InvMTM  {som }{\ensuremath{\IfBooleanTF#1
            {\InvP*{ \MTM[#2]{#3} }}
            {\Invp { \MTM[#2]{#3} }}                }\xspace}
%    \end{macrocode}
% \end{macro}
% \end{macro}
%
%
% \begin{macro}{\InvXTX}
% Inversa del producto de la matriz X transpuesta por ella misma
%    \begin{macrocode}
\NewDocumentCommand\InvXTX{}{\ensuremath{\InvMTM{X}}\xspace}
%    \end{macrocode}
% \end{macro}
%
%
% \begin{macro}{\MInvMTMMT}
% \begin{macro}{\MInvMTMMT}
% Matriz proyección sobre el espacio columna de la matriz de rango completo por columnas indicada
%    \begin{macrocode}
\NewDocumentCommand\MInvMTMMT{sO{}m}{\ensuremath{\IfBooleanTF#1
            {\MVect[#2]{#3}\Invp{\VTV[#2]{#3}}\MVectT[#2]{#3}}
            {\Mat[#2]{#3}\InvMTM[#2]{#3}\MatT[#2]{#3}}         }\xspace}
                            
\NewDocumentCommand\MInvMTMMTkk{sO{}m}{\ensuremath{\Mat[#1]{#2}\InvMTM[#1]{#2}\MatT[#1]{#2}}\xspace}
%    \end{macrocode}
% \end{macro}
% \end{macro}
%
%
% \begin{macro}{\VTW}
% Matriz fila por matriz columna
%    \begin{macrocode}
\NewDocumentCommand\VTW{omom}{\ensuremath{\MVectT[#1]{#2}\!\MVect[#3]{#4}}\xspace}
%    \end{macrocode}
% \end{macro}
%
%
% \begin{macro}{\VTV}
% Matriz fila por su transpuesta
%    \begin{macrocode}
\NewDocumentCommand\VTV{om}{\ensuremath{\MVectT[#1]{#2}\!\MVect[#1]{#2}}\xspace}
%    \end{macrocode}
% \end{macro}
%
%
% \begin{macro}{\VWT}
% Matriz columna por matriz fila
%    \begin{macrocode}
\NewDocumentCommand\VWT{omom}{\ensuremath{\MVect[#1]{#2}\!\MVectT[#3]{#4}}\xspace}
%    \end{macrocode}
% \end{macro}
%
%
% \begin{macro}{\VVT}
% Matriz columna por su transpuesta
%    \begin{macrocode}
\NewDocumentCommand\VVT{om}{\ensuremath{\MVect[#1]{#2}\!\MVectT[#1]{#2}}\xspace}
%    \end{macrocode}
% \end{macro}
%
%
%
% \iffalse
%%%%%%%%%%%%%%%%%%%%%%%%%%%%%%%%%%%%
%% --- Sistemas de ecuaciones
%%%%%%%%%%%%%%%%%%%%%%%%%%%%%%%%%%%%
% \fi
%
% \subsection{Sistemas de ecuaciones}
%
% \begin{macro}{\SEL}
% Sistema de ecuaciones lineales con notación matricial
%    \begin{macrocode}
\NewDocumentCommand\SEL  {mmm}{\ensuremath{\MV   {#1}{#2}=\Vect{#3}}\xspace}
%    \end{macrocode}
% \end{macro}
%
%
% \begin{macro}{\SELT}
% \begin{macro}{\SELTP}
% Sistema de ecuaciones lineales con notación matricial (matriz de coeficientes transpuesta)
%    \begin{macrocode}
\NewDocumentCommand\SELT {mmm}{\ensuremath{\MTV  {#1}{#2}=\Vect{#3}}\xspace}
\NewDocumentCommand\SELTP{mmm}{\ensuremath{\MTVP*{#1}{#2}=\Vect{#3}}\xspace}
%    \end{macrocode}
% \end{macro}
% \end{macro}
%
%
% \begin{macro}{\SELF}
% Sistema de ecuaciones lineales con notación matricial (matriz de coeficientes transpuesta)
%    \begin{macrocode}
\NewDocumentCommand\SELF {mmm}{\ensuremath{\VM   {#1}{#2}=\Vect{#3}}\xspace}
%    \end{macrocode}
% \end{macro}
%
%
% \iffalse
%%%%%%%%%%%%%%%%%%%%%%%%%%%%%%%%%%%%
%% --- Espacios vectoriales
%%%%%%%%%%%%%%%%%%%%%%%%%%%%%%%%%%%%
% \fi
%
% \subsection{Espacios vectoriales}
%
%
% \begin{macro}{\EV}
% Sistema de ecuaciones lineales con notación matricial (matriz de coef. transpuesta)
%    \begin{macrocode}
\NewDocumentCommand\EV{O{}O{}m}{\ensuremath{\RidxE{\mathcal{#3}}{\scriptstyle{#1}}{#2}}\xspace}
%    \end{macrocode}
% \end{macro}
%
%
% \begin{macro}{\EspacioNul}
% Letra que denota al Espacio nulo (o núcleo)
%    \begin{macrocode}
\DeclareMathOperator{\EspacioNul}{\EV{N}}
%    \end{macrocode}
% \end{macro}
%
%
% \begin{macro}{\EspacioCol}
% Letra que denota al Espacio Columna
%    \begin{macrocode}
\DeclareMathOperator{\EspacioCol}{\EV{C}}
%    \end{macrocode}
% \end{macro}
%
%
% \begin{macro}{\Nulls}
% \begin{macro}{\Nulls*}
% Espacio nulo (o núcleo) de un objeto
%    \begin{macrocode}
\NewDocumentCommand\Nulls{sm}{\ensuremath{\IfBooleanTF#1
    {\EspacioNul\Parentesis*{#2}}
    {\EspacioNul\parentesis {#2}}        }\xspace}
%    \end{macrocode}
% \end{macro}
% \end{macro}
%
%
% \begin{macro}{\nulls}
% \begin{macro}{\nulls*}
% Espacio nulo (o núcleo) de una matriz
%    \begin{macrocode}
\NewDocumentCommand\nulls{sm}{\ensuremath{\IfBooleanTF#1
    {\Nulls*{\Mat{#2}}}
    {\Nulls {\Mat{#2}}}                     }\xspace}
%    \end{macrocode}
% \end{macro}
% \end{macro}
%
%
% \begin{macro}{\Cols}
% \begin{macro}{\Cols*}
% Espacio columna de un objeto
%    \begin{macrocode}
\NewDocumentCommand\Cols{sm}{\ensuremath{\IfBooleanTF#1
    {\EspacioCol\Parentesis*{#2}}
    {\EspacioCol\parentesis {#2}}        }\xspace}
%    \end{macrocode}
% \end{macro}
% \end{macro}
%
%
% \begin{macro}{\cols}
% \begin{macro}{\cols*}
% Espacio columna de una matriz
%    \begin{macrocode}
\NewDocumentCommand\cols{sm}{\ensuremath{\IfBooleanTF#1
    {\Cols*{\Mat{#2}}}
    {\Cols {\Mat{#2}}}                     }\xspace}
%    \end{macrocode}
% \end{macro}
% \end{macro}
%
%
% \begin{macro}{\Span}
% \begin{macro}{\Span*}
% Espacio generado por un sistema generador
%    \begin{macrocode}
\NewDocumentCommand\Span{sm}{\ensuremath{\IfBooleanTF#1
        {\EV{L}\Parentesis*{#2}}
        {\EV{L}\parentesis {#2}}             }\xspace}
%    \end{macrocode}
% \end{macro}
% \end{macro}
%
%
% \begin{macro}{\PSpanNew}
% \begin{macro}{\PSpanNew*}
% Espacio semi-euclídeo de probabilidad generado por un sistema
%    \begin{macrocode}
\NewDocumentCommand\PSpanNew{sm}{\ensuremath{\,\IfBooleanTF#1
        {\topinset{\tiny\EV{P}}{\EV{L}}{2pt}{2pt}\Parentesis*{#2}}
        {\topinset{\tiny\EV{P}}{\EV{L}}{2pt}{2pt}\parentesis {#2}} }\xspace}
%    \end{macrocode}
% \end{macro}
% \end{macro}
%
%
% \begin{macro}{\coord}
% \begin{macro}{\coordP}
% \begin{macro}{\coordP*}
% \begin{macro}{\coordPE}
% \begin{macro}{\coordPE*}
% Coordenadas respecto de una base
%    \begin{macrocode}
\NewDocumentCommand\coord  {m m}{\ensuremath{
        \Ridx{#1}{\!\Ridx{\mathbin{/}}{\!#2}} }\xspace}

\NewDocumentCommand\coordP  {smm}{\ensuremath{\IfBooleanTF#1
        {\coord{\Parentesis*{#2}}{#3}}
        {\coord{\parentesis {#2}}{#3}}        }\xspace}

\NewDocumentCommand\coordPE {smm}{\ensuremath{\IfBooleanTF#1
        {\Parentesis*{\coord{#2}{#3}}}
        {\parentesis {\coord{#2}{#3}}}        }\xspace}
%    \end{macrocode}
% \end{macro}
% \end{macro}
% \end{macro}
% \end{macro}
% \end{macro}
%
%
% \iffalse
%%%%%%%%%%%%%%%%%%%%%%%%%%%%%%%%%%%%
%% --- Notación funcional
%%%%%%%%%%%%%%%%%%%%%%%%%%%%%%%%%%%%
% \fi
%
% \subsection{Notación funcional}
%
% \begin{macro}{\dom}
% Dominio de una función
%    \begin{macrocode}
\DeclareMathOperator{\dom}{dom}
%    \end{macrocode}
% \end{macro}
%
%
% \begin{macro}{\imagen}
% Imagen de una función
%    \begin{macrocode}
\DeclareMathOperator{\imagen}{imag}
%    \end{macrocode}
% \end{macro}
%
%
% \begin{macro}{\mifun}
% \begin{macro}{\mifun*}
% Breve descripción de una función
%    \begin{macrocode}
\NewDocumentCommand\mifun {smmm}{\ensuremath{\IfBooleanTF#1
        {#3\xrightarrow{#2}#4}
        {#2 \colon #3 \to  #4}        }\xspace}
%    \end{macrocode}
% \end{macro}
% \end{macro}
%
%
% \begin{macro}{\deffun}
% Breve descripción de una función
%    \begin{macrocode}
\NewDocumentCommand\deffun {m m m m m}{
  \ensuremath{
    \begingroup
    {\setlength{\arraycolsep}{0pt}
      \begin{array}[t]{r@{\,}c@{\,}c@{\,}l}
        #1\colon & #2 & \longrightarrow & #3\\
                 & #4 & \longmapsto & #5
      \end{array}}
    \endgroup}\xspace}
%    \end{macrocode}
% \end{macro}
%
%
% \begin{macro}{\imrec}
% Imagen inversa
%    \begin{macrocode}
\NewDocumentCommand\imrec {mm}{\ensuremath{%
    \RidxE{#1}{}{{\lfloor{\scriptscriptstyle\!#2}}} }\xspace}
%    \end{macrocode}
% \end{macro}
%
%
% \begin{macro}{\sproy}
% Operador proyección ortogonal
%    \begin{macrocode}
\DeclareMathOperator{\sproy}{Prj}
%    \end{macrocode}
% \end{macro}
%
%
% \begin{macro}{\proy}
% \begin{macro}{\proy*}
% Proyección ortogonal
%    \begin{macrocode}
\NewDocumentCommand\proy{sO{}m}{\ensuremath{\IfBooleanTF#1
                         {\Ridx{\sproy}{#2}\!\Parentesis*{#3}} {\Ridx{\sproy}{#2}\!\parentesis{#3}}
                            }\xspace}
%    \end{macrocode}
% \end{macro}
% \end{macro}
%
%%%%%%%%%%%%%%%%%%%%%%%%%%%%%%%%%%%%%%%%%%%%%
%
% \iffalse
%%%%%%%%%%%%%%%%%%%%%%%%%%%%%%%%%%%%
%% --- Probabilidad
%%%%%%%%%%%%%%%%%%%%%%%%%%%%%%%%%%%%
% \fi
%
% \subsection{Probabilidad}
%
%    \begin{macrocode}
% % %\DeclareMathAlphabet{\mathbbmsl}{U}{bbm}{m}{sl}
\DeclareMathAlphabet{\mymathbb}{U}{BOONDOX-ds}{m}{n}
%    \end{macrocode}
%
%
% \begin{macro}{\Cero}
% Función constante 0
%    \begin{macrocode}
\NewDocumentCommand\Cero{}{\ensuremath{\mymathbb{0}}\xspace}
%    \end{macrocode}
% \end{macro}
%
%
% \begin{macro}{\Uno}
% Función constante 1
%    \begin{macrocode}
\NewDocumentCommand\Uno{}{\ensuremath{\mymathbb{1}}\xspace}
%    \end{macrocode}
% \end{macro}
%
%
% \begin{macro}{\indCero}
% Función indicatriz nula
%    \begin{macrocode}
\NewDocumentCommand\indCero{}{\ensuremath{\Cero}\xspace}
%    \end{macrocode}
% \end{macro}
%
%
% \begin{macro}{\indUno}
% Función indicatriz constante uno
%    \begin{macrocode}
\NewDocumentCommand\indUno{} {\ensuremath{\Uno}\xspace}
%    \end{macrocode}
% \end{macro}
%
%
% \begin{macro}{\Ind}
% Función indicatriz constante uno
%    \begin{macrocode}
\NewDocumentCommand\Ind{} {\ensuremath{\Uno}\xspace}
%    \end{macrocode}
% \end{macro}
%
%
% \begin{macro}{\ind}
% Función indicatriz de un suceso (argumento obligatorio)
%    \begin{macrocode}
\NewDocumentCommand\ind{m}{\ensuremath{{\Uno}_{{#1}}}\xspace}
%    \end{macrocode}
% \end{macro}
%
%
% \begin{macro}{\sspi}
% Símbolo del semi-producto interior
%    \begin{macrocode}
\NewDocumentCommand\sspi{O{}O{}}{\ensuremath{\RidxE{\eta}{\scriptstyle{#1}}{#2}}\xspace}
%    \end{macrocode}
% \end{macro}
%
%
% \begin{macro}{\SPI}
% Símbolo del semi-producto interior
%    \begin{macrocode}
\NewDocumentCommand\SPI{sO{}O{}mm}{\ensuremath{\IfBooleanTF#1
    {\Ridx{\Angulos*{\left.#4 \right| #5}}{\!\sspi[#2][#3]}}
    {\Ridx{\angulos {      #4   \big| #5}}{\!\sspi[#2][#3]}}     }\xspace}
%    \end{macrocode}
% \end{macro}
%
%
% \begin{macro}{\sesp}
% Símbolo de la esperanza (integral de Lebesgue)
%    \begin{macrocode}
\NewDocumentCommand\sesp{O{}}{\ensuremath{\Ridx{\mathbb{S}}{{#1\!}}}\xspace}
%    \end{macrocode}
% \end{macro}
%
%
% \begin{macro}{\ESP}
% \begin{macro}{\ESP*}
% Esperanza (integral de Lebesgue) de un objeto
%    \begin{macrocode}
\NewDocumentCommand\ESP{sO{}m}{\ensuremath{\IfBooleanTF#1
                        {\sesp[#2\!]\Parentesis*{#3}} {\sesp[#2]\parentesis{#3}}
                          } \xspace}
%    \end{macrocode}
% \end{macro}
% \end{macro}
%
%
% \begin{macro}{\domesp}
% Dominio de la esperanza (integral de Lebesgue)
%    \begin{macrocode}
\NewDocumentCommand\domesp{m}{\ensuremath{\Ridx{L}{\scriptstyle#1}}\xspace}
%    \end{macrocode}
% \end{macro}
%
%
% \begin{macro}{\spro}
% Símbolo de la probabilidad
%    \begin{macrocode}
\NewDocumentCommand\spro{O{}}{\ensuremath{\Ridx{\mathbb{P}}{{\!#1}}}\xspace}
%    \end{macrocode}
% \end{macro}
%
%
% \begin{macro}{\PRO}
% \begin{macro}{\PRO*}
% Probabilidad de un suceso
%    \begin{macrocode}
\NewDocumentCommand\PRO{sO{}m}{\ensuremath{\IfBooleanTF#1
                         {\spro[#2]\Parentesis*{#3}} {\spro[#2]\parentesis{#3}}
                            }\xspace}
%    \end{macrocode}
% \end{macro}
% \end{macro}
%
%
% \begin{macro}{\PRObh}
% \begin{macro}{\PRObh*}
% Probabilidad de un suceso bajo hipótesis
%    \begin{macrocode}
\NewDocumentCommand\PRObh{smm}{\ensuremath{\IfBooleanTF#1
                         {\spro[_{#3}\!]\Parentesis*{#2}} {\spro[_{#3}\!]\parentesis{#2}}
                            }\xspace}
%    \end{macrocode}
% \end{macro}
% \end{macro}
%
% 
% \begin{macro}{\pindep}
% Símbolo de independencia probabilística 
%    \begin{macrocode}
\newcommand{\pindep}{\mathbin{\mathpalette\PindeP@t\relax}}
\newcommand{\PindeP@t}[2]{%
  \vcenter{\hbox{%
    \sbox\z@{$\m@th#1-$}%
    \setlength{\unitlength}{\wd\z@}%
    \begin{picture}(.7,1)
    \roundcap
    \put(0.1,0.2){\line(5,0){0.5}}
    \put(0.4,0.6){\line(5,0){0.3}}
    \put(0.1,0.2){\line(0,1){0.7}}
    \put(0.4,0.6){\line(0,1){0.5}}
    \put(0.1,0.2){\line(5,6.5){0.3}}
    \end{picture}%
  }} }
%    \end{macrocode}
% \end{macro}
%
%
% \begin{macro}{\dperp}
% Símbolo alternativo de independencia probabilística 
%    \begin{macrocode}
\newcommand{\dperp}{\mathbin{\mathpalette\Dperp@t\relax}}
\newcommand{\Dperp@t}[2]{%
  \vcenter{\hbox{%
    \sbox\z@{$\m@th#1-$}%
    \setlength{\unitlength}{\wd\z@}%
    \begin{picture}(1,1)
    \roundcap
    \put(0.1,0.2){\line(1,0){0.8}}
    \put(0.40,0.2){\line(0,1){0.8}}
    \put(0.60,0.2){\line(0,1){0.8}}
    \end{picture}%
  }} }
%    \end{macrocode}
% \end{macro}
%
%
% \begin{macro}{\ndperp}
% Símbolo para negar la independencia probabilística 
%    \begin{macrocode}
\newcommand{\ndperp}{\mathbin{\mathpalette\nDperp@t\relax}}
\newcommand{\nDperp@t}[2]{%
  \vcenter{\hbox{%
    \sbox\z@{$\m@th#1-$}%
    \setlength{\unitlength}{\wd\z@}%
    \begin{picture}(1,1)
    \roundcap
    \put(0.1,0.2){\line(1,0){0.8}}
    \put(0.40,0.2){\line(0,1){0.8}}
    \put(0.60,0.2){\line(0,1){0.8}}
    \put(0.2,-.05){\line(0.6,1.15){0.65}}
    \end{picture}%
  }} }
%    \end{macrocode}
% \end{macro}
%
%
% \begin{macro}{\PSpan}
% \begin{macro}{\PSpan*}
% Espacio semi-euclídeo de probabilidad generado por un sistema
%    \begin{macrocode}
\NewDocumentCommand\PSpan{sm}{\ensuremath{\,\IfBooleanTF#1
        {\EV{L{\!\!{\scriptstyle{{}^\mathbb{_P}}}}}\Parentesis*{#2}}
        {\EV{L{\!\!{\scriptstyle{{}^\mathbb{_P}}}}}\parentesis {#2}} }\xspace}
%    \end{macrocode}
% \end{macro}
% \end{macro}
%
%    \begin{macrocode}
\DeclareFontFamily{U}{matha}{\hyphenchar\font45}
\DeclareFontShape{U}{matha}{m}{n}{ <-6> matha5 <6-7> matha6 <7-8>
matha7 <8-9> matha8 <9-10> matha9 <10-12> matha10 <12-> matha12 }{}
\DeclareSymbolFont{matha}{U}{matha}{m}{n}
%
\DeclareFontFamily{U}{mathx}{\hyphenchar\font45}
\DeclareFontShape{U}{mathx}{m}{n}{ <-6> mathx5 <6-7> mathx6 <7-8>
mathx7 <8-9> mathx8 <9-10> mathx9 <10-12> mathx10 <12-> mathx12 }{}
\DeclareSymbolFont{mathx}{U}{mathx}{m}{n}
%
\DeclareMathDelimiter{\ldbrack} {4}{matha}{"76}{mathx}{"30}
\DeclareMathDelimiter{\rdbrack} {5}{matha}{"77}{mathx}{"38}
%
\DeclareSymbolFont{mathx}{U}{mathx}{m}{n}
\DeclareMathSymbol{\bigtimes}{\mathop}{mathx}{"91}
%    \end{macrocode}
%
% \begin{macro}{\Clase}
% Clase de equivalencia
%    \begin{macrocode}
\NewDocumentCommand\Clase {m}{\ensuremath{ \ldbrack #1 \rdbrack }\xspace}
%    \end{macrocode}
% \end{macro}
%
%
% \begin{macro}{\Media}
% \begin{macro}{\Mediap}
% \begin{macro}{\MediaP}
% Media (proyección ortogonal sobre los vectores contantes)
%    \begin{macrocode}
\NewDocumentCommand\Media {m}{\ensuremath{ \widebar{#1} }\xspace}

\NewDocumentCommand\Mediap{sm}{\ensuremath{\IfBooleanTF#1
                             {\Media{\parentesis*{#2}}}
                             {\Media{\parentesis {#2}}} }\xspace}

\NewDocumentCommand\MediaP{sm}{\ensuremath{\IfBooleanTF#1
                             {\Media{\Parentesis*{#2}}}
                             {\Media{\Parentesis {#2}}} }\xspace}
%    \end{macrocode}
% \end{macro}
% \end{macro}
% \end{macro}
%
%
% \begin{macro}{\Smedia}
% Símbolo para el valor medio
%    \begin{macrocode}
\NewDocumentCommand\Smedia {}{\mu}
%    \end{macrocode}
% \end{macro}
%
%
% \begin{macro}{\SmediaM}
% Símbolo para la media muestral
%    \begin{macrocode}
\NewDocumentCommand\SmediaM {}{m}
%    \end{macrocode}
% \end{macro}
%
%
% \begin{macro}{\Scov}
% Símbolo para covarianza
%    \begin{macrocode}
\NewDocumentCommand\Scov   {}{\sigma}
%    \end{macrocode}
% \end{macro}
%
%
% \begin{macro}{\ScovM}
% Símbolo para covarianza muestral
%    \begin{macrocode}
\NewDocumentCommand\ScovM  {}{s}
%    \end{macrocode}
% \end{macro}
%
%
% \begin{macro}{\Svar}
% Símbolo para varianza
%    \begin{macrocode}
\NewDocumentCommand\Svar   {}{\Scov^2}
%    \end{macrocode}
% \end{macro}
%
%
% \begin{macro}{\SvarM}
% Símbolo para varianza muestral
%    \begin{macrocode}
\NewDocumentCommand\SvarM  {}{\ScovM^2}
%    \end{macrocode}
% \end{macro}
%
%
% \begin{macro}{\ScvarM}
% Símbolo para cuasivarianza muestral
%    \begin{macrocode}
\NewDocumentCommand\ScvarM  {}{\mathfrak{s}^2}
%    \end{macrocode}
% \end{macro}
%
%
% \begin{macro}{\Scorr}
% Símbolo para correlación
%    \begin{macrocode}
\NewDocumentCommand\Scorr   {}{\rho}
%    \end{macrocode}
% \end{macro}
%
%
% \begin{macro}{\ScorrM}
% Símbolo para correlación muestral
%    \begin{macrocode}
\NewDocumentCommand\ScorrM   {}{r}
%    \end{macrocode}
% \end{macro}
%
%
% \begin{macro}{\media}
% \begin{macro}{\mediap}
% \begin{macro}{\mediaP}
% Valor medio
%    \begin{macrocode}
\NewDocumentCommand\media {m}{\ensuremath{\IfNoValueTF{#1}%
     {  \Smedia       }
     { {\Smedia}_{#1} } }\xspace}

\NewDocumentCommand\mediap{sm}{\ensuremath{\IfBooleanTF#1
                     {\media{\parentesis*{#2}}}
                     {\media{\parentesis {#2}}} }\xspace}

\NewDocumentCommand\mediaP{sm}{\ensuremath{\IfBooleanTF#1
                     {\media{\Parentesis*{#2}}}
                     {\media{\Parentesis {#2}}} }\xspace}
%    \end{macrocode}
% \end{macro}
% \end{macro}
% \end{macro}
%
%
% \begin{macro}{\mediaM}
% \begin{macro}{\mediaMp}
% \begin{macro}{\mediaMP}
% Media muestral
%    \begin{macrocode}
\NewDocumentCommand\mediaM {m}{\ensuremath{\IfNoValueTF{#1}%
     {  \SmediaM       }
     { {\SmediaM}_{#1} } }\xspace}

\NewDocumentCommand\mediaMp{sm}{\ensuremath{\IfBooleanTF#1
                     {\mediaM{\parentesis*{#2}}}
                     {\mediaM{\parentesis {#2}}} }\xspace}

\NewDocumentCommand\mediaMP{sm}{\ensuremath{\IfBooleanTF#1
                     {\mediaM{\Parentesis*{#2}}}
                     {\mediaM{\Parentesis {#2}}} }\xspace}
%    \end{macrocode}
% \end{macro}
% \end{macro}
% \end{macro}
%
%
% \begin{macro}{\dt}
% \begin{macro}{\dtp}
% \begin{macro}{\dtP}
% Desviación típica
%    \begin{macrocode}
\NewDocumentCommand\dt {m}{\ensuremath{\IfNoValueTF{#1}%
     {  \Scov       }
     { {\Scov_{#1}} } }\xspace}

\NewDocumentCommand\dtp{sm}{\ensuremath{\IfBooleanTF#1
                     {\dt{\parentesis*{#2}}}
                     {\dt{\parentesis {#2}}} }\xspace}

\NewDocumentCommand\dtP{sm}{\ensuremath{\IfBooleanTF#1
                     {\dt{\Parentesis*{#2}}}
                     {\dt{\Parentesis {#2}}} }\xspace}
%    \end{macrocode}
% \end{macro}
% \end{macro}
% \end{macro}
%
%
% \begin{macro}{\dtM}
% \begin{macro}{\dtMp}
% \begin{macro}{\dtMP}
% Desviación típica muestral
%    \begin{macrocode}
\NewDocumentCommand\dtM {m}{\ensuremath{\IfNoValueTF{#1}%
     {  \ScovM       }
     { {\ScovM_{#1}} } }\xspace}

\NewDocumentCommand\dtMp{sm}{\ensuremath{\IfBooleanTF#1
                     {\dtM{\parentesis*{#2}}}
                     {\dtM{\parentesis {#2}}} }\xspace}

\NewDocumentCommand\dtMP{sm}{\ensuremath{\IfBooleanTF#1
                     {\dtM{\Parentesis*{#2}}}
                     {\dtM{\Parentesis {#2}}} }\xspace}
%    \end{macrocode}
% \end{macro}
% \end{macro}
% \end{macro}
%
%
% \begin{macro}{\var}
% \begin{macro}{\varp}
% \begin{macro}{\varP}
% Varianza
%    \begin{macrocode}
\NewDocumentCommand\var {m}{\ensuremath{\IfNoValueTF{#1}%
     {  \Svar       }
     { {\Svar_{#1}} } }\xspace}

\NewDocumentCommand\varp{sm}{\ensuremath{\IfBooleanTF#1
                     {\var{\parentesis*{#2}}}
                     {\var{\parentesis {#2}}} }\xspace}

\NewDocumentCommand\varP{sm}{\ensuremath{\IfBooleanTF#1
                     {\var{\Parentesis*{#2}}}
                     {\var{\Parentesis {#2}}} }\xspace}
%    \end{macrocode}
% \end{macro}
% \end{macro}
% \end{macro}
%
%
% \begin{macro}{\varM}
% \begin{macro}{\varMp}
% \begin{macro}{\varMP}
% Varianza muestral
%    \begin{macrocode}
\NewDocumentCommand\varM {m}{\ensuremath{\IfNoValueTF{#1}%
     {  \SvarM       }
     { {\SvarM_{#1}} } }\xspace}

\NewDocumentCommand\varMp{sm}{\ensuremath{\IfBooleanTF#1
                     {\varM{\parentesis*{#2}}}
                     {\varM{\parentesis {#2}}} }\xspace}

\NewDocumentCommand\varMP{sm}{\ensuremath{\IfBooleanTF#1
                     {\varM{\Parentesis*{#2}}}
                     {\varM{\Parentesis {#2}}} }\xspace}
%    \end{macrocode}
% \end{macro}
% \end{macro}
% \end{macro}
%
%
% \begin{macro}{\cvarM}
% \begin{macro}{\cvarMp}
% \begin{macro}{\cvarMP}
% Cuasi-varianza muestral
%    \begin{macrocode}
\NewDocumentCommand\cvarM {m}{\ensuremath{\IfNoValueTF{#1}%
     {  \ScvarM       }
     { {\ScvarM_{#1}} } }\xspace}

\NewDocumentCommand\cvarMp{sm}{\ensuremath{\IfBooleanTF#1
                     {\cvarM{\parentesis*{#2}}}
                     {\cvarM{\parentesis {#2}}} }\xspace}

\NewDocumentCommand\cvarMP{sm}{\ensuremath{\IfBooleanTF#1
                     {\cvarM{\Parentesis*{#2}}}
                     {\cvarM{\Parentesis {#2}}} }\xspace}
%    \end{macrocode}
% \end{macro}
% \end{macro}
% \end{macro}
%
%
% \begin{macro}{\cov}
% \begin{macro}{\covp}
% \begin{macro}{\covP}
% Covarianza
%    \begin{macrocode}
\NewDocumentCommand\cov {mm}{\ensuremath{\IfNoValueTF{#1}%
     {  \Scov       }
     { {\Scov_{#1#2}} } }\xspace}

\NewDocumentCommand\covp{smm}{\ensuremath{\IfBooleanTF#1
                     {\cov{\parentesis*{#2#3}}}
                     {\cov{\parentesis {#2#3}}} }\xspace}

\NewDocumentCommand\covP{smm}{\ensuremath{\IfBooleanTF#1
                     {\cov{\Parentesis*{#2#3}}}
                     {\cov{\Parentesis {#2#3}}} }\xspace}
%    \end{macrocode}
% \end{macro}
% \end{macro}
% \end{macro}
%
%
% \begin{macro}{\covM}
% \begin{macro}{\covMp}
% \begin{macro}{\covMP}
% Covarianza muestral
%    \begin{macrocode}
\NewDocumentCommand\covM {mm}{\ensuremath{\IfNoValueTF{#1}%
     {  \ScovM       }
     { {\ScovM_{#1#2}} } }\xspace}

\NewDocumentCommand\covMp{smm}{\ensuremath{\IfBooleanTF#1
                     {\covM{\parentesis*{#2#3}}}
                     {\covM{\parentesis {#2#3}}} }\xspace}

\NewDocumentCommand\covMP{smm}{\ensuremath{\IfBooleanTF#1
                     {\covM{\Parentesis*{#2#3}}}
                     {\covM{\Parentesis {#2#3}}} }\xspace}
%    \end{macrocode}
% \end{macro}
% \end{macro}
% \end{macro}
%
%
% \begin{macro}{\corr}
% \begin{macro}{\corrp}
% \begin{macro}{\corrP}
% Correlación
%    \begin{macrocode}
\NewDocumentCommand\corr {mm}{\ensuremath{\IfNoValueTF{#1}%
     {  \Scorr       }
     { {\Scorr_{#1#2}} } }\xspace}

\NewDocumentCommand\corrp{smm}{\ensuremath{\IfBooleanTF#1
                     {\corr{\parentesis*{#2#3}}}
                     {\corr{\parentesis {#2#3}}} }\xspace}

\NewDocumentCommand\corrP{smm}{\ensuremath{\IfBooleanTF#1
                     {\corr{\Parentesis*{#2#3}}}
                     {\corr{\Parentesis {#2#3}}} }\xspace}
%    \end{macrocode}
% \end{macro}
% \end{macro}
% \end{macro}
%
%
% \begin{macro}{\corrM}
% \begin{macro}{\corrMp}
% \begin{macro}{\corrMP}
% Correlación muestral
%    \begin{macrocode}
\NewDocumentCommand\corrM {mm}{\ensuremath{\IfNoValueTF{#1}%
     {  \ScorrM       }
     { {\ScorrM_{#1#2}} } }\xspace}

\NewDocumentCommand\corrMp{smm}{\ensuremath{\IfBooleanTF#1
                     {\corrM{\parentesis*{#2#3}}}
                     {\corrM{\parentesis {#2#3}}} }\xspace}

\NewDocumentCommand\corrMP{smm}{\ensuremath{\IfBooleanTF#1
                     {\corrM{\Parentesis*{#2#3}}}
                     {\corrM{\Parentesis {#2#3}}} }\xspace}
%    \end{macrocode}
% \end{macro}
% \end{macro}
% \end{macro}
%
%%%%%%%%%%%%%%%%%%%%%%%%%%%%%%%%%%%%%%%%%%%%%
%
% \iffalse
%%%%%%%%%%%%%%%%%%%%%%%%%%%%%%%%%%%%
%% --- Econometría
%%%%%%%%%%%%%%%%%%%%%%%%%%%%%%%%%%%%
% \fi
%
% \subsection{Econometría}
%
%
% \begin{macro}{\TM}
% Tamaño muestral
%    \begin{macrocode}
\NewDocumentCommand\TM{} {\ensuremath{N}\xspace}
%    \end{macrocode}
% \end{macro}
%
%
% \begin{macro}{\Serror}
% Símbolo del error de ajuste
%    \begin{macrocode}
\NewDocumentCommand\Serror{} {\ensuremath{e}\xspace}
%    \end{macrocode}
% \end{macro}
%
%
% \begin{macro}{\resi}
% Error de ajuste MCO
%    \begin{macrocode}
\NewDocumentCommand\resi{m} {\ensuremath{ \Estmc{\Serror}_{#1} }\xspace}
%    \end{macrocode}
% \end{macro}
%
%
% \begin{macro}{\res}
% Vector de errores de ajuste MCO
%    \begin{macrocode}
\NewDocumentCommand\res{} {\ensuremath{ \Estmc{\Vect{\Serror}} }\xspace}
%    \end{macrocode}
% \end{macro}
%
%
% \begin{macro}{\SRC}
% Suma de residuos al cuadrado
%    \begin{macrocode}
\NewDocumentCommand\SRC{} {\ensuremath{ \dotprod{\res}{\res} }\xspace}
%    \end{macrocode}
% \end{macro}
%
%
% \begin{macro}{\ColorA}
% Color objeto aleatorio (vector de un espacio euclídeo probabilístico)
%    \begin{macrocode}
\NewDocumentCommand\ColorA {m}{\ensuremath{ {\color{violet}{#1}} }\xspace}
%    \end{macrocode}
% \end{macro}
%
%
% \begin{macro}{\VColorA}
% Vector de aleatorio (vector de un espacio euclídeo probabilístico)
%    \begin{macrocode}
\NewDocumentCommand\VColorA {m}{\ensuremath{ \Vect{\ColorA{#1}} }\xspace}
%    \end{macrocode}
% \end{macro}
%
%
% \begin{macro}{\VAn}
% Variable aleatoria con subíndice
%    \begin{macrocode}
\NewDocumentCommand\VAn{mm}{\ensuremath{ \ColorA{\MakeUppercase{#1}_{#2}} }\xspace}
%    \end{macrocode}
% \end{macro}
%
%
% \begin{macro}{\VAi}
% Variable aleatoria (con subíndice opcional)
%    \begin{macrocode}
\NewDocumentCommand\VAi{O{}m}{\ensuremath{ \VAn{#2}{#1} }\xspace}
%    \end{macrocode}
% \end{macro}
%
%
% \begin{macro}{\VA}
% Variable aleatoria
%    \begin{macrocode}
\NewDocumentCommand\VA{O{}m}{\ensuremath{ \VAn{#2}{#1} }\xspace}
%    \end{macrocode}
% \end{macro}
%
%
% \begin{macro}{\VAind}
% Variable aleatoria
%    \begin{macrocode}
\NewDocumentCommand\VAind{m}{\ensuremath{ \VA{\ind{#1}} }\xspace}
%    \end{macrocode}
% \end{macro}
%
%
% \begin{macro}{\VAindCero}
% Variable aleatoria
%    \begin{macrocode}
\NewDocumentCommand\VAindCero{}{\ensuremath{ \VA{\indCero} }\xspace}
%    \end{macrocode}
% \end{macro}
%
%
% \begin{macro}{\VAindUno}
% Variable aleatoria
%    \begin{macrocode}
\NewDocumentCommand\VAindUno{}{\ensuremath{ \VA{\indUno} }\xspace}
%    \end{macrocode}
% \end{macro}
%
%
% \begin{macro}{\cteVA}
% Variable aleatoria
%    \begin{macrocode}
\NewDocumentCommand\cteVA{m}{\ensuremath{ \VA{{\mathit{#1}}} }\xspace}
%    \end{macrocode}
% \end{macro}
%
%
% \begin{macro}{\VVA}
% Vector aleatorio
%    \begin{macrocode}
\NewDocumentCommand\VVA{O{} m}{\ensuremath{ \Vect[\ColorA{#1}]{\VA{#2}} }\xspace}
%    \end{macrocode}
% \end{macro}
%
%
% \begin{macro}{\MVA}
% \begin{macro}{\MVAp}
% \begin{macro}{\MVAp*}
% \begin{macro}{\MVAP}
% \begin{macro}{\MVAP*}
% Matriz aleatoria
%    \begin{macrocode}
\NewDocumentCommand\MVA    {O{}m}{\ensuremath{ %
              \ColorA{\Ridx{\mathbf{\MakeUppercase{#2}}}{#1}}  }\xspace}

\NewDocumentCommand\MVAp  {som}{\ensuremath{\IfBooleanTF#1
    {\parentesis*{\IfNoValueTF{#2}{\MVA{#3}}{\MVA[#2]{#3}}}}
    {\parentesis {\IfNoValueTF{#2}{\MVA{#3}}{\MVA[#2]{#3}}}}   }\xspace}

\NewDocumentCommand\MVAP  {som}{\ensuremath{\IfBooleanTF#1
    {\Parentesis*{\IfNoValueTF{#2}{\MVA{#3}}{\MVA[#2]{#3}}}}
    {\Parentesis {\IfNoValueTF{#2}{\MVA{#3}}{\MVA[#2]{#3}}}}   }\xspace}
%    \end{macrocode}
% \end{macro}
% \end{macro}
% \end{macro}
% \end{macro}
% \end{macro}
%
%
% \begin{macro}{\MVAT}
% \begin{macro}{\MVATp}
% \begin{macro}{\MVATp*}
% \begin{macro}{\MVATP}
% \begin{macro}{\MVATP*}
% \begin{macro}{\MVATpE}
% \begin{macro}{\MVATpE*}
% \begin{macro}{\MVATPE}
% \begin{macro}{\MVATPE*}
% Matriz transpuesta
%    \begin{macrocode}
\NewDocumentCommand\MVAT{O{\vphantom{k}}m}{\ensuremath{\RidxE{\MVA{#2}}{\ColorA{#1}}{\T}}\xspace}

\NewDocumentCommand\MVATp  {som}{\ensuremath{\IfBooleanTF#1
       {\Transp* {\MVA[#2]{#3}}}
       {\Transp  {\MVA[#2]{#3}}}            }\xspace}

\NewDocumentCommand\MVATP  {som}{\ensuremath{\IfBooleanTF#1
       {\TransP* {\MVA[#2]{#3}}}
       {\TransP  {\MVA[#2]{#3}}}            }\xspace}

\NewDocumentCommand\MVATpE {sO{}m}{\ensuremath{\IfBooleanTF#1
       {\RidxEpE*{\MVA{#3}}{\ColorA{#2}}{\T}}
       {\RidxEpE*{\MVA{#3}}{\ColorA{#2}}{\T}}     }\xspace}

\NewDocumentCommand\MVATPE {sO{}m}{\ensuremath{\IfBooleanTF#1
       {\RidxEPE*{\MVA{#3}}{\ColorA{#2}}{\T}}
       {\RidxEPE*{\MVA{#3}}{\ColorA{#2}}{\T}}     }\xspace}
%    \end{macrocode}
% \end{macro}
% \end{macro}
% \end{macro}
% \end{macro}
% \end{macro}
% \end{macro}
% \end{macro}
% \end{macro}
% \end{macro}
%
%
% \begin{macro}{\VVAKK}
% Vector aleatorio
%    \begin{macrocode}
\NewDocumentCommand\VVAKK{O{} m}{\ensuremath{ \VA{\Vect[{\MakeLowercase {#1}}]{#2}} }\xspace}
%    \end{macrocode}
% \end{macro}
%
%
% \begin{macro}{\MVAKK}
% Matriz aleatoria
%    \begin{macrocode}
\NewDocumentCommand\MVAKK{O{} m}{\ensuremath{ \VA{\Mat[{\MakeLowercase {#1}}]{#2}} }\xspace}
%    \end{macrocode}
% \end{macro}
%
%
% \begin{macro}{\SVA}
% Sistema de variables aleatorias
%    \begin{macrocode}
\NewDocumentCommand\SVA{O{}m}{\ensuremath{ \ColorA{\MakeUppercase{\mathsf{#2}}_{#1}} }\xspace}
%    \end{macrocode}
% \end{macro}
%
%
% \begin{macro}{\SVAT}
% Sistema de variables aleatorias transpuesto
%    \begin{macrocode}
\NewDocumentCommand\SVAT{O{}m}{\ensuremath{ \Trans{{\SVA[#1]{#2}}} }\xspace}
%    \end{macrocode}
% \end{macro}
%
%
% \begin{macro}{\perturbacion}
% Símbolo para el término de perturbación
%    \begin{macrocode}
\def\perturbacion{\MakeUppercase{u}}
%    \end{macrocode}
% \end{macro}
%
%
% \begin{macro}{\per}
% Perturbación de un modelo
%    \begin{macrocode}
\NewDocumentCommand\per{}{\ensuremath{\VA{\perturbacion}}\xspace}
%    \end{macrocode}
% \end{macro}
%
%
% \begin{macro}{\peri}
% Perturbación con subíndice de un modelo
%    \begin{macrocode}
\NewDocumentCommand\peri{O{n}}{\ensuremath{\VAi[#1]{\perturbacion}}\xspace}
%    \end{macrocode}
% \end{macro}
%
%
% \begin{macro}{\Vper}
% Vector de perturbaciones
%    \begin{macrocode}
\NewDocumentCommand\Vper{}{\ensuremath{\VVA{\perturbacion}}\xspace}
%    \end{macrocode}
% \end{macro}
%
%
% \begin{macro}{\esperanza}
% Símbolo de la esperanza matemática
%    \begin{macrocode}
\DeclareMathOperator{\esperanza}{E}
%    \end{macrocode}
% \end{macro}
%
%
% \begin{macro}{\E}
% \begin{macro}{\E*}
% Esperanza de una variable aleatoria
%    \begin{macrocode}
\NewDocumentCommand\E{sm}{\ensuremath{\IfBooleanTF#1
                        {\esperanza\Parentesis*{#2}} {\esperanza\parentesis{#2}}
                          } \xspace}
%    \end{macrocode}
% \end{macro}
% \end{macro}
%
%
% \begin{macro}{\desviaciontipica}
% Símbolo de la desviación típica
%    \begin{macrocode}
\DeclareMathOperator{\desviaciontipica}{Dt}
%    \end{macrocode}
% \end{macro}
%
%
% \begin{macro}{\Dt}
% \begin{macro}{\Dt*}
% Desviación típica de una variable aleatoria
%    \begin{macrocode}
\NewDocumentCommand\Dt{sm}{\ensuremath{\IfBooleanTF#1
                        {\desviaciontipica\Parentesis*{#2}} {\desviaciontipica\parentesis{#2}}
                          } \xspace}
%    \end{macrocode}
% \end{macro}
% \end{macro}
%
%
% \begin{macro}{\varianza}
% Símbolo de la varianza
%    \begin{macrocode}
\DeclareMathOperator{\varianza}{Var}
%    \end{macrocode}
% \end{macro}
%
%
% \begin{macro}{\Var}
% \begin{macro}{\Var*}
% Varianza de una variable aleatoria
%    \begin{macrocode}
\NewDocumentCommand\Var{sm}{\ensuremath{\IfBooleanTF#1
                        {\varianza\Parentesis*{#2}} {\varianza\parentesis{#2}}
                          } \xspace}
%    \end{macrocode}
% \end{macro}
% \end{macro}
%
%
% \begin{macro}{\covarianza}
% Símbolo de la covarianza
%    \begin{macrocode}
\DeclareMathOperator{\covarianza}{Cov}
%    \end{macrocode}
% \end{macro}
%
%
% \begin{macro}{\Cov}
% \begin{macro}{\Cov*}
% Covarianza de dos variables aleatorias
%    \begin{macrocode}
\NewDocumentCommand\Cov{smm}{\ensuremath{\IfBooleanTF#1
                        {\covarianza\Parentesis*{#2,#3}} {\covarianza\parentesis{#2,#3}}
                          } \xspace}
%    \end{macrocode}
% \end{macro}
% \end{macro}
%
%
% \begin{macro}{\correlacion}
% Símbolo de la correlacion
%    \begin{macrocode}
\DeclareMathOperator{\correlacion}{Corr}
%    \end{macrocode}
% \end{macro}
%
%
% \begin{macro}{\Corr}
% \begin{macro}{\Corr*}
% Correlación ente dos variables aleatorias
%    \begin{macrocode}
\NewDocumentCommand\Corr{smm}{\ensuremath{\IfBooleanTF#1
                        {\correlacion\Parentesis*{#2,#3}} {\correlacion\parentesis{#2,#3}}
                          } \xspace}
%    \end{macrocode}
% \end{macro}
% \end{macro}
%
%
% \begin{macro}{\ECond}
% \begin{macro}{\ECond*}
% Esperanza condicionada
%    \begin{macrocode}
\NewDocumentCommand\ECond{smm}{\ensuremath{ \ColorA{\mathbb{E}} \IfBooleanTF#1
                     {\Parentesis*{\left.#2\,\right|#3}}
                     {\parentesis {#2 \mid   #3       }} }\xspace}
%    \end{macrocode}
% \end{macro}
% \end{macro}
%
%
% \begin{macro}{\ECondYX}
% \begin{macro}{\ECondYX*}
% Esperanza condicionada a un sistema de variables aleatorias
%    \begin{macrocode}
\NewDocumentCommand\ECondYX{smm}{\ensuremath{ \IfBooleanTF#1
                     {\ECond*{#2}{\SVA{#3}}} 
                     {\ECond {#2}{\SVA{#3}}} }\xspace}
%    \end{macrocode}
% \end{macro}
% \end{macro}
%
%
% \begin{macro}{\DtCond}
% \begin{macro}{\DtCond*}
% Desviación típica condicionada
%    \begin{macrocode}
\NewDocumentCommand\DtCond{smm}{\ensuremath{ \ColorA{\mathbb{D}t} \IfBooleanTF#1
                     {\Parentesis*{#2\left|\,#3\right.}}
                     {\parentesis {#2 \mid   #3       }} }\xspace}
%    \end{macrocode}
% \end{macro}
% \end{macro}
%
%
% \begin{macro}{\VarCond}
% \begin{macro}{\VarCond*}
% Varianza condicionada
%    \begin{macrocode}
\NewDocumentCommand\VarCond{smm}{\ensuremath{ \ColorA{\mathbb{V}\!ar} \IfBooleanTF#1
                     {\Parentesis*{#2\left|\,#3\right.}}
                     {\parentesis {#2 \mid   #3       }} }\xspace}
%    \end{macrocode}
% \end{macro}
% \end{macro}
%
%
% \begin{macro}{\VarCondYX}
% \begin{macro}{\VarCondYX*}
% Varianza condicionada a un sistema de variables aleatorias 
%    \begin{macrocode}
\NewDocumentCommand\VarCondYX{smm}{\ensuremath{ \IfBooleanTF#1
                     {\VarCond*{#2}{\SVA{#3}}} 
                     {\VarCond {#2}{\SVA{#3}}} }\xspace}
%    \end{macrocode}
% \end{macro}
% \end{macro}
%
%
% \begin{macro}{\CovCond}
% \begin{macro}{\CovCond*}
% Covarianza condicionada
%    \begin{macrocode}
\NewDocumentCommand\CovCond{smmm}{\ensuremath{ \ColorA{\mathbb{C}\!ov} \IfBooleanTF#1
                     {\Parentesis*{#2,#3\left|\,#4\right.}}
                     {\parentesis {#2,#3 \mid   #4       }} }\xspace}
%    \end{macrocode}
% \end{macro}
% \end{macro}
%
%
% \begin{macro}{\CovCondXYZ}
% \begin{macro}{\CovCondXYZ*}
% Covarianza condicionada a un sistema de variables aleatorias 
%    \begin{macrocode}
\NewDocumentCommand\CovCondXYZ{smmm}{\ensuremath{ \IfBooleanTF#1
                     {\CovCond*{#2}{#3}{\SVA{#4}}} 
                     {\CovCond {#2}{#3}{\SVA{#4}}} }\xspace}
%    \end{macrocode}
% \end{macro}
% \end{macro}
%
%
% \begin{macro}{\Estmc}
% \begin{macro}{\VEstmc}
% Ajuste por MCO
%    \begin{macrocode}
\NewDocumentCommand\Estmc {m}{\ensuremath{ \widehat{#1}      }\xspace}
\NewDocumentCommand\VEstmc{O{}m}{\ensuremath{ \Estmc{\Vect[#1]{#2}} }\xspace}
%    \end{macrocode}
% \end{macro}
% \end{macro}
%
%
% \begin{macro}{\Estmd}
% \begin{macro}{\VEstmd}
% Estimador MCO
%    \begin{macrocode}
\NewDocumentCommand\Estmd {m}{\ensuremath{ \ColorA{\Estmc{#1}} }\xspace}
\NewDocumentCommand\VEstmd{O{}m}{\ensuremath{ \Estmd{\Vect[#1]{#2}}   }\xspace}
%    \end{macrocode}
% \end{macro}
% \end{macro}
%
%
% \begin{macro}{\MCO}
% Ajuste por MCO
%    \begin{macrocode}
\NewDocumentCommand\MCO{mm}{\ensuremath{ \InvMTM*{#2}\MTV{#2}{#1} }\xspace}
%    \end{macrocode}
% \end{macro}
%
%
% \begin{macro}{\MCOc}
% Parametros del ajuste por MCO del regresor de Rn y sobre X
%    \begin{macrocode}
\NewDocumentCommand\MCOc{}{\ensuremath{ \MCO{y}{X} }\xspace}
%    \end{macrocode}
% \end{macro}
%
%
% \begin{macro}{\MCOd}
% Estimador parámetros ajuste por MCO con muestra Y y X
%    \begin{macrocode}
\NewDocumentCommand\MCOd{}{\ensuremath{ \MCO{\VVA{y}}{\MVA{X}} }\xspace}
%    \end{macrocode}
% \end{macro}
%
%
% \begin{macro}{\MLT}
% \begin{macro}{\MLS}
% \begin{macro}{\MLG}
% Modelo lineal trivial, simple y general
%    \begin{macrocode}
\NewDocumentCommand\MLT{}{\ensuremath{ \VA{Y} = \beta_1\VAindUno + \VA{U} }\xspace}
\NewDocumentCommand\MLS{}{\ensuremath{ \VA{Y} = \beta_1\VAindUno + \beta_2\VA{X} + \VA{U} }\xspace}
\NewDocumentCommand\MLG{}{\ensuremath{ \VA{Y} = \SVA{X}\Vect{\beta} + \VA{U} }\xspace}
%    \end{macrocode}
% \end{macro}
% \end{macro}
% \end{macro}
%
%
% \begin{macro}{\masMLT}
% \begin{macro}{\masMLS}
% \begin{macro}{\masMLG}
% Modelos muestrales lineal trivial, simple y general
%    \begin{macrocode}
\NewDocumentCommand\masMLT{}{\ensuremath{ \VVA{Y} = \beta_1\VVA{1} + \VVA{U} }\xspace}
\NewDocumentCommand\masMLS{}{\ensuremath{ \VVA{Y} = \beta_1\VVA{1} + \beta_2\VVA{X} + \VVA{U} }\xspace}
\NewDocumentCommand\masMLG{}{\ensuremath{ \VVA{Y} = \MVA{X}\Vect{\beta} + \VVA{U} }\xspace}
%    \end{macrocode}
% \end{macro}
% \end{macro}
% \end{macro}
%
%
% \begin{macro}{\ajusteMLT}
% \begin{macro}{\ajusteMLS}
% \begin{macro}{\ajusteMLG}
% Ajueste modelos lineal trivial, simple y general
%    \begin{macrocode}
\NewDocumentCommand\ajusteMLT{}{\ensuremath{ \Vect{y} = \Estmc{\beta}\Vect{1} + \res }\xspace}
\NewDocumentCommand\ajusteMLS{}{\ensuremath{ \Vect{y} = \Estmc{\beta_1}\Vect{1} + \Estmc{\beta_2}\Vect{x} + \res }\xspace}
\NewDocumentCommand\ajusteMLG{}{\ensuremath{ \Vect{y} = \MV{x}{\Estmc{\beta}} + \res }\xspace}
%    \end{macrocode}
% \end{macro}
% \end{macro}
% \end{macro}
%
%
% \begin{macro}{\SupI}
% Primer supuesto del Modelo Lineal General
%    \begin{macrocode}
\NewDocumentCommand\SupI{}{\ensuremath{ \MLG }\xspace}
%    \end{macrocode}
% \end{macro}
%
%
% \begin{macro}{\SupII}
% Segundo supuesto del Modelo Lineal General
%    \begin{macrocode}
\NewDocumentCommand\SupII{}{\ensuremath{ \ECondYX*{\per}{X}=\VAindCero }\xspace}
%    \end{macrocode}
% \end{macro}
%
%
% \begin{macro}{\SupIII}
% Tercer supuesto del Modelo Lineal General
%    \begin{macrocode}
\NewDocumentCommand\SupIII{}{\ensuremath{ \ECondYX*{\per^2}{X}=\sigma^2\VAindUno }\xspace}
%    \end{macrocode}
% \end{macro}
%
%
% \begin{macro}{\SupIV}
% Cuarto supuesto del Modelo Lineal General
%    \begin{macrocode}
\NewDocumentCommand\SupIV{}{\ensuremath{ \E*{\SVAT{X}\SVA{X}} \textrm{ es invertible} }\xspace}
%    \end{macrocode}
% \end{macro}
%
%
% \begin{macro}{\SupIImas}
% Segundo supuesto muestral del Modelo Lineal General
%    \begin{macrocode}
\NewDocumentCommand\SupIImas{}{\ensuremath{ \ECond*{\VVA{\per}}{\MVA{X}}=\VVA{0} }\xspace}
%    \end{macrocode}
% \end{macro}
%
%
% \begin{macro}{\SupIIImas}
% Tercer supuesto muestral del Modelo Lineal General
%    \begin{macrocode}
\NewDocumentCommand\SupIIImas{}{\ensuremath{ \VarCond*{\VVA{\per}}{\MVA{X}}=\sigma^2\MVA{I} }\xspace}
%    \end{macrocode}
% \end{macro}
%
%
% \begin{macro}{\SupIVmas}
% Cuarto supuesto muestral del Modelo Lineal General
%    \begin{macrocode}
\NewDocumentCommand\SupIVmas{}{\ensuremath{ \E*{\MVAT{X}\MVA{X}} \textrm{ es invertible} }\xspace}
%    \end{macrocode}
% \end{macro}
%
%
% \begin{macro}{\SupVmas}
% Quinto supuesto muestral del Modelo Lineal General
%    \begin{macrocode}
\NewDocumentCommand\SupVmas{}{\ensuremath{ \VVA{\per}\sim\Normal{\Vect{0}}{\sigma^2\Mat{I}} }\xspace}
%    \end{macrocode}
% \end{macro}
%
%
% \begin{macro}{\MVAR}
% Matriz de varianzas y covarianzas
%    \begin{macrocode}
\NewDocumentCommand\MVAR{m}{\ensuremath{ {\pmb{\Sigma}}_{\scriptscriptstyle\Mat{#1#1}} }\xspace}
%    \end{macrocode}
% \end{macro}
%
%
% \begin{macro}{\VCOV}
% Vector de covarianzas
%    \begin{macrocode}
\NewDocumentCommand\VCOV{mm}{\ensuremath{ {\pmb{\sigma}}_{\scriptscriptstyle\MV{#1}{#2}} }\xspace}
%    \end{macrocode}
% \end{macro}
%
%
% \begin{macro}{\MVARM}
% Matriz de varianzas y covarianzas muestrales
%    \begin{macrocode}
\NewDocumentCommand\MVARM{m}{\ensuremath{ \Mat[_{\Mat{#1#1}}]{S} }\xspace}
%    \end{macrocode}
% \end{macro}
%
%
% \begin{macro}{\VCOVM}
% Vector de covarianzas muestrales
%    \begin{macrocode}
\NewDocumentCommand\VCOVM{mm}{\ensuremath{ \Vect[_{\MV{#1}{#2}}]{s} }\xspace}
%    \end{macrocode}
% \end{macro}
%
%
% \begin{macro}{\normal}
% Símbolo de la distribución normal
%    \begin{macrocode}
\DeclareMathOperator{\normal}{\it N\/}
%    \end{macrocode}
% \end{macro}
%
%
% \begin{macro}{\tstudent}
% Símbolo de la distribución t de student
%    \begin{macrocode}
\DeclareMathOperator{\tstudent}{\it t\/}
%    \end{macrocode}
% \end{macro}
%
%
% \begin{macro}{\fsnedecor}
% Símbolo de la distribución F de Snedecor
%    \begin{macrocode}
\DeclareMathOperator{\fsnedecor}{\it F\/}
%    \end{macrocode}
% \end{macro}
%
%
% \begin{macro}{\Normal}
% Distribución Normal
%    \begin{macrocode}
\NewDocumentCommand\Normal{mm}{\ensuremath{ \normal\left(#1,\,#2\right) }\xspace}
%    \end{macrocode}
% \end{macro}
%
%
% \begin{macro}{\TStudent}
% Distribución t de Student
%    \begin{macrocode}
\NewDocumentCommand\TStudent{m}{\ensuremath{ \tstudent_{#1} }\xspace}
%    \end{macrocode}
% \end{macro}
%
%
% \begin{macro}{\FSnedecor}
% Distribución t de FSnedecor
%    \begin{macrocode}
\NewDocumentCommand\FSnedecor{mm}{\ensuremath{ \fsnedecor_{\!{#1,#2}} }\xspace}
%    \end{macrocode}
% \end{macro}
%
%
% \begin{macro}{\ChiCuadrado}
% Distribución Chi cuadrado
%    \begin{macrocode}
\NewDocumentCommand\ChiCuadrado{m}{\ensuremath{ {\chi^2_{#1}} }\xspace}
%    \end{macrocode}
% \end{macro}
%
%
% \begin{macro}{\ValorC}
% Valor Critico
%    \begin{macrocode}
\NewDocumentCommand\ValorC{mmm}{\ensuremath{ {\it #1\/}_{_{#2}}^{\langle#3\rangle} }\xspace}
%    \end{macrocode}
% \end{macro}
%
%
% \begin{macro}{\EstmcE}
% Estimación de la esperanza
%    \begin{macrocode}
\NewDocumentCommand\EstmcE  {sm   }{\ensuremath{\IfBooleanTF#1
            {\Estmc{\esperanza}\Parentesis*{#2}}
            {\Estmc{\esperanza}\parentesis {#2}}            }\xspace}
%    \end{macrocode}
% \end{macro}
%
%
% \begin{macro}{\EstmdE}
% Estimador de la esperanza
%    \begin{macrocode}
\NewDocumentCommand\EstmdE  {sm   }{\ensuremath{\IfBooleanTF#1
            {\Estmd{\esperanza}\Parentesis*{#2}}
            {\Estmd{\esperanza}\parentesis {#2}}            }\xspace}
%    \end{macrocode}
% \end{macro}
%
%
% \begin{macro}{\EstmcECond}
% Estimación de la esperanza condicionada
%    \begin{macrocode}
\NewDocumentCommand\EstmcECond  {smm  }{\ensuremath{\IfBooleanTF#1
            {\Estmc{\esperanza}\Parentesis*{#2\left|\,#3\right.}}
            {\Estmc{\esperanza}\parentesis {#2 \mid   #3       }}            }\xspace}
%    \end{macrocode}
% \end{macro}
%
%
% \begin{macro}{\EstmdECond}
% Estimador de la esperanza condicionada
%    \begin{macrocode}
\NewDocumentCommand\EstmdECond  {smm  }{\ensuremath{\IfBooleanTF#1
            {\Estmd{\esperanza}\Parentesis*{#2\left|\,#3\right.}}
            {\Estmd{\esperanza}\parentesis {#2 \mid   #3       }}            }\xspace}
%    \end{macrocode}
% \end{macro}
%
%
% \begin{macro}{\EstmcDt}
% Estimación de la desviación típica
%    \begin{macrocode}
\NewDocumentCommand\EstmcDt  {sm   }{\ensuremath{\IfBooleanTF#1
            {\Estmc{\desviaciontipica}\Parentesis*{#2}}
            {\Estmc{\desviaciontipica}\parentesis {#2}}            }\xspace}
%    \end{macrocode}
% \end{macro}
%
%
% \begin{macro}{\EstmdDt}
% Estimador de la desviación típica
%    \begin{macrocode}
\NewDocumentCommand\EstmdDt  {sm   }{\ensuremath{\IfBooleanTF#1
            {\Estmd{\desviaciontipica}\Parentesis*{#2}}
            {\Estmd{\desviaciontipica}\parentesis {#2}}            }\xspace}
%    \end{macrocode}
% \end{macro}
%
%
% \begin{macro}{\EstmcDtCond}
% Estimación de la desviación típica condicionada
%    \begin{macrocode}
\NewDocumentCommand\EstmcDtCond  {smm  }{\ensuremath{\IfBooleanTF#1
            {\Estmc{\desviaciontipica}\Parentesis*{#2\left|\,#3\right.}}
            {\Estmc{\desviaciontipica}\parentesis {#2 \mid   #3       }}            }\xspace}
%    \end{macrocode}
% \end{macro}
%
%
% \begin{macro}{\EstmdDtCond}
% Estimador de la desviación típica condicionada
%    \begin{macrocode}
\NewDocumentCommand\EstmdDtCond  {smm  }{\ensuremath{\IfBooleanTF#1
            {\Estmd{\desviaciontipica}\Parentesis*{#2\left|\,#3\right.}}
            {\Estmd{\desviaciontipica}\parentesis {#2 \mid   #3       }}            }\xspace}
%    \end{macrocode}
% \end{macro}
%
%
% \begin{macro}{\EstmcVar}
% Estimación de la varianza
%    \begin{macrocode}
\NewDocumentCommand\EstmcVar  {sm   }{\ensuremath{\IfBooleanTF#1
            {\Estmc{\varianza}\Parentesis*{#2}}
            {\Estmc{\varianza}\parentesis {#2}}            }\xspace}
%    \end{macrocode}
% \end{macro}
%
%
% \begin{macro}{\EstmdVar}
% Estimador de la varianza
%    \begin{macrocode}
\NewDocumentCommand\EstmdVar  {sm   }{\ensuremath{\IfBooleanTF#1
            {\Estmd{\varianza}\Parentesis*{#2}}
            {\Estmd{\varianza}\parentesis {#2}}            }\xspace}
%    \end{macrocode}
% \end{macro}
%
%
% \begin{macro}{\EstmcVarCond}
% Estimación de la varianza condicionada
%    \begin{macrocode}
\NewDocumentCommand\EstmcVarCond  {smm  }{\ensuremath{\IfBooleanTF#1
            {\Estmc{\varianza}\Parentesis*{#2\left|\,#3\right.}}
            {\Estmc{\varianza}\parentesis {#2 \mid   #3       }}            }\xspace}
%    \end{macrocode}
% \end{macro}
%
%
% \begin{macro}{\EstmdVarCond}
% Estimador de la varianza condicionada
%    \begin{macrocode}
\NewDocumentCommand\EstmdVarCond  {smm  }{\ensuremath{\IfBooleanTF#1
            {\Estmd{\varianza}\Parentesis*{#2\left|\,#3\right.}}
            {\Estmd{\varianza}\parentesis {#2 \mid   #3       }}            }\xspace}
%    \end{macrocode}
% \end{macro}
%
%
% \begin{macro}{\EstmcCov}
% Estimación de la covarianza
%    \begin{macrocode}
\NewDocumentCommand\EstmcCov  {smm   }{\ensuremath{\IfBooleanTF#1
            {\Estmc{\covarianza}\Parentesis*{#2,#3}}
            {\Estmc{\covarianza}\parentesis {#2,#3}}            }\xspace}
%    \end{macrocode}
% \end{macro}
%
%
% \begin{macro}{\EstmdCov}
% Estimador de la covarianza
%    \begin{macrocode}
\NewDocumentCommand\EstmdCov  {smm   }{\ensuremath{\IfBooleanTF#1
            {\Estmd{\covarianza}\Parentesis*{#2,#3}}
            {\Estmd{\covarianza}\parentesis {#2,#3}}            }\xspace}
%    \end{macrocode}
% \end{macro}
%
%
% \begin{macro}{\EstmcCovCond}
% Estimación de la covarianza condicionada
%    \begin{macrocode}
\NewDocumentCommand\EstmcCovCond  {smmm  }{\ensuremath{\IfBooleanTF#1
            {\Estmc{\covarianza}\Parentesis*{#2,#3\left|\,#4\right.}}
            {\Estmc{\covarianza}\parentesis {#2,#3 \mid   #4       }}            }\xspace}
%    \end{macrocode}
% \end{macro}
%
%
% \begin{macro}{\EstmdCovCond}
% Estimador de la covarianza condicionada
%    \begin{macrocode}
\NewDocumentCommand\EstmdCovCond  {smmm  }{\ensuremath{\IfBooleanTF#1
            {\Estmd{\covarianza}\Parentesis*{#2,#3\left|\,#4\right.}}
            {\Estmd{\covarianza}\parentesis {#2,#3 \mid   #4       }}            }\xspace}
%    \end{macrocode}
% \end{macro}
%
%
% \begin{macro}{\EstmcCorr}
% Estimación de la correlación
%    \begin{macrocode}
\NewDocumentCommand\EstmcCorr  {smm   }{\ensuremath{\IfBooleanTF#1
            {\Estmc{\correlacion}\Parentesis*{#2,#3}}
            {\Estmc{\correlacion}\parentesis {#2,#3}}            }\xspace}
%    \end{macrocode}
% \end{macro}
%
%
% \begin{macro}{\EstmdCorr}
% Estimador de la correlación
%    \begin{macrocode}
\NewDocumentCommand\EstmdCorr  {smm   }{\ensuremath{\IfBooleanTF#1
            {\Estmd{\correlacion}\Parentesis*{#2,#3}}
            {\Estmd{\correlacion}\parentesis {#2,#3}}            }\xspace}
%    \end{macrocode}
% \end{macro}
%
%
% \begin{macro}{\EstmcCorrCond}
% Estimación de la correlación condicionada
%    \begin{macrocode}
\NewDocumentCommand\EstmcCorrCond  {smmm  }{\ensuremath{\IfBooleanTF#1
            {\Estmc{\correlacion}\Parentesis*{#2,#3\left|\,#4\right.}}
            {\Estmc{\correlacion}\parentesis {#2,#3 \mid   #4       }}            }\xspace}
%    \end{macrocode}
% \end{macro}
%
%
% \begin{macro}{\EstmdCorrCond}
% Estimador de la correlación condicionada
%    \begin{macrocode}
\NewDocumentCommand\EstmdCorrCond  {smmm  }{\ensuremath{\IfBooleanTF#1
            {\Estmd{\correlacion}\Parentesis*{#2,#3\left|\,#4\right.}}
            {\Estmd{\correlacion}\parentesis {#2,#3 \mid   #4       }}            }\xspace}
%    \end{macrocode}
% \end{macro}
%
%
% \begin{macro}{\estimEcond}
% Estimación de la esperanza condicionada 
%    \begin{macrocode}
\NewDocumentCommand\estimEcond{mm}{\ensuremath{ \EstmcE{#1 \mid #2} }\xspace}
%    \end{macrocode}
% \end{macro}
%
%
% \begin{macro}{\Hnula}
% \begin{macro}{\Halt}
% \begin{macro}{\Rcritica}
% \begin{macro}{\Racept}
% Hipótesis nula, hipótesis alternativa, región crítica y regiónde aceptación
%    \begin{macrocode}
\NewDocumentCommand\Hnula   {}{\ensuremath{ H_0 }\xspace}
\NewDocumentCommand\Halt    {}{\ensuremath{ H_1 }\xspace}
\NewDocumentCommand\Rcritica{}{\ensuremath{ RC  }\xspace}
\NewDocumentCommand\Racept  {}{\ensuremath{ RA  }\xspace}
%    \end{macrocode}
% \end{macro}
% \end{macro}
% \end{macro}
% \end{macro}
%
%
% \begin{macro}{\fdppar}
% Función de densidad paramétrica
%    \begin{macrocode}
\NewDocumentCommand\fdppar{O{\theta}m}{\ensuremath{ f_{\VA{#2}}(\MakeLowercase{#2};\Vect{#1}) }\xspace}
%    \end{macrocode}
% \end{macro}
%
%
% \begin{macro}{\testad}
% \begin{macro}{\testadistico}
% \begin{macro}{\Testadistico}
% Estadístico t de student
%    \begin{macrocode}
\NewDocumentCommand\testad{}{\ensuremath{ \mathcal{T} }\xspace}
\NewDocumentCommand\testadistico{}{\ensuremath{ \Estmc{\testad} }\xspace}
\NewDocumentCommand\Testadistico{}{\ensuremath{ \ColorA{\testad} }\xspace}
%    \end{macrocode}
% \end{macro}
% \end{macro}
% \end{macro}
%
%
% \begin{macro}{\festad}
% \begin{macro}{\festadistico}
% \begin{macro}{\Festadistico}
% Estadístico t de student
%    \begin{macrocode}
\NewDocumentCommand\festad{}{\ensuremath{ \mathcal{F} }\xspace}
\NewDocumentCommand\festadistico{}{\ensuremath{ \Estmc{\festad} }\xspace}
\NewDocumentCommand\Festadistico{}{\ensuremath{ \ColorA{\festad} }\xspace}
%    \end{macrocode}
% \end{macro}
% \end{macro}
% \end{macro}
%
%
% \begin{macro}{\simBajoCond}
% \begin{macro}{\simnula}
% \begin{macro}{\simNula}
% Distribución bajo hipótesis nula
%    \begin{macrocode}
\NewDocumentCommand\simBajoCond{m}{\ensuremath{ \mathop{\sim}\limits_{#1}}\xspace}
\NewDocumentCommand\simnula{ }{\ensuremath{ \simBajoCond{\Hnula}         }\xspace}
\NewDocumentCommand\simNula{m}{\ensuremath{ \simBajoCond{\Hnula:\, #1}     }\xspace}
%    \end{macrocode}
% \end{macro}
% \end{macro}
% \end{macro}
%
%
% \begin{macro}{\IConfc}
% \begin{macro}{\IConfd}
% Intervalo de confianza
%    \begin{macrocode}
\NewDocumentCommand\IConfc{mm}{\ensuremath{ \Estmc{\text{IC}}_{#1}^{#2} }\xspace}
\NewDocumentCommand\IConfd{mm}{\ensuremath{ \Estmd{\text{IC}}_{#1}^{#2} }\xspace}
%    \end{macrocode}
% \end{macro}
% \end{macro}
%
%%%%%%%%%%%%%%%%%%%%%%%%%%%%%%%%%%%%%%%%%%%%%
%
% \iffalse
%%%%%%%%%%%%%%%%%%%%%%%%%%%%%%%%%%%%
%% --- Sucesiones
%%%%%%%%%%%%%%%%%%%%%%%%%%%%%%%%%%%%
% \fi
%
% \subsection{Sucesiones}
%
%
% \begin{macro}{\suc}
% \begin{macro}{\suc*}
% Sucesión
%    \begin{macrocode}
\NewDocumentCommand\suc {sO{n}O{\N}m }{\ensuremath{\IfBooleanTF#1
                         {{\{\esuc*[#2]{#4}\}}_{#2\in#3}} {\boldsymbol{\ddot{#4}}}
                           }\xspace}     
%    \end{macrocode}
% \end{macro}
% \end{macro}
%
%
% \begin{macro}{\esuc}
% \begin{macro}{\esuc*}
% Elemento de una sucesión
%    \begin{macrocode}
\NewDocumentCommand\esuc{sO{n}m      }{\ensuremath{\IfBooleanTF#1
                         {{#3}_{#2}} {\elemR{\boldsymbol{\ddot{#3}}}{#2}}
                           }\xspace}
%    \end{macrocode}
% \end{macro}
% \end{macro}
%
%
%
% \Finale
\endinput
