%%%%%%%%%% PREÁMBULO %%%%%%%%%%%%%%%%

\documentclass{article}

\usepackage{nacal}

%%%%%%%%%%%%%%%%%%%%%%%%%%%%%%%%%%%%%

\begin{document}

Como se puede comprobar con este documento, basta cargar el paquete
\texttt{nacal} en el preámbulo.
\bigskip

\begin{itemize}
\item Esto es un vector genérico:\; \vect{x}
\item Esto es un vector de \R[n]:\; \Vect{x}
\item Esto es un vector de \R[m\times n]:\; \Mat{A}
\end{itemize}

El operador selector de la columna $j$-ésima por la derecha de una matriz es
\begin{displaymath}
  \deffun{\_\getitemR{j}}{\R[m\times n]}{\R[m]}{\Mat{A}}{\VectC{A}{j}}
\end{displaymath}


Si $\TrC{\Mat{A}}=\InvMat{A}$, entonces $\TrC{\Mat{I}}=\Mat{A}^{\minus2}$.
\bigskip

La norma al cuadrado de un vector del espacio euclídeo \R[n] es
\[
  \norma[\R[n]]{\Vect{x}}^2=
  \dotprod{x}{x}=
  \dotProd{\prodhp{x}{x}}{\Vect{1}}=
  \sum_{i=1}^n\eleVRpE{x}{i}^2.
\]

\end{document}
